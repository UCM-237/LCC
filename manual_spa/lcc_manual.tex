\documentclass[a4paper,10pt]{book}
\usepackage[pdftex]{graphicx}
\usepackage{epstopdf}
\usepackage{subfigure}
\usepackage{amsmath,amsthm}
\usepackage{tikz}
\usetikzlibrary{shapes.geometric}
\usetikzlibrary{shapes.multipart}
\textwidth= 15cm
\evensidemargin=0cm
\usepackage[spanish]{babel}
\usepackage[latin1]{inputenc}
\usepackage{textcomp}
\usepackage{amstext}
\usepackage{amsfonts}
\usepackage{amssymb}
\usepackage{multicol}
\usepackage[hyperindex=true,breaklinks=true,colorlinks=true,linkcolor=blue]{hyperref}
\renewcommand{\tablename}{Tabla}
\renewcommand{\listtablename}{\'Indice de Tablas}

\usepackage{listings}
\usepackage{color} %red, green, blue, yellow, cyan, magenta, black, white
\definecolor{mygreen}{RGB}{28,172,0} % color values Red, Green, Blue
\definecolor{mylilas}{RGB}{170,55,241}

\usepackage{multirow}
\usepackage{imakeidx}

%\usepackage{draftwatermark}
%\SetWatermarkText{Borrador,juan.jimenez@fis.ucm.es}
%\SetWatermarkScale{2}


\graphicspath{{./figuras/}}
\makeindex
\begin{document}
\title{
\begin{flushleft}
\includegraphics[width=2.5cm]{ucm2.pdf}
Universidad Complutense de Madrid\\
---------------------------------------------------------------------\
\end{flushleft}
Laboratorio de Computaci\'on Cient\'ifica}
\author{ Juan Jim\'enez\\ H\'ector Garc\'ia de Marina}
\maketitle
\vspace*{\fill}

\includegraphics[scale=1]{by-sa.eps}\\
El contenido de estos apuntes est\'e1 bajo licencia Creative Commons Atribution-ShareAlike 4.0\\
\href{http://creativecommons.org/licenses/by-sa/4.0/}{http://creativecommons.org/licenses/by-sa/4.0/}\\
\copyright Juan Jim\'enez

\bigskip
\tableofcontents
\listoffigures
\listoftables

%\section*{Matlab Code}
\lstset{language=Matlab,%
    %basicstyle=\color{red},
    breaklines=true,%
    morekeywords={matlab2tikz},
    keywordstyle=\color{blue},%
    morekeywords=[2]{1}, keywordstyle=[2]{\color{black}},
    identifierstyle=\color{black},%
    stringstyle=\color{mylilas},
    commentstyle=\color{mygreen},%
    showstringspaces=false,%without this there will be a symbol in the places where there is a space
    numbers=left,%
    numberstyle={\tiny \color{black}},% size of the numbers
    numbersep=9pt, % this defines how far the numbers are from the text
    emph=[1]{for,end,break},emphstyle=[1]\color{red}, %some words to emphasise
    %emph=[2]{word1,word2}, emphstyle=[2]{style},    
}
%\lstinputlisting{../codigo/matlab/1-intro/ejemplo1.m}
\include{preface}
\chapter{Introducci�n al software cient�fico}
En la actualidad, el ordenador se ha convertido en una herramienta imprescindible para el trabajo de cualquier investigador cient�fico. Su uso ha permitido realizar tareas que sin su ayuda resultar�an sencillamente imposibles de acometer. Entre otras, distinguiremos las tres siguientes:
\begin{itemize}
\item Adquisici�n de datos de dispositivos experimentales.
\item An�lisis y tratamiento de datos experimentales. \index{Datos, an�lisis}
\item C�lculo Cient�fico.
\end{itemize}

La primera de �stas tareas queda fuera de los contenidos de esta asignatura. Su objetivo es emplear el ordenador para recoger datos autom�ticamente de los sensores empleados en un dispositivo experimental. El procedimiento habitual es emplear dispositivos electr�nicos que traducen las lecturas de un sensor (un term�metro, un man�metro, un caudal�metro, una c�mara etc.) a un voltaje. El voltaje es digitalizado, es decir, convertido a una secuencia de ceros y unos, y almacenado en un ordenador para su posterior an�lisis o/y directamente monitorizado, es decir, mostrado en la pantalla del ordenador. En muchos casos el ordenador es a su vez capaz de interactuar con el dispositivo experimental: iniciar o detener un experimento, regular las condiciones en que se realiza, disparar alarmas si se producen errores, etc.

De este modo, el investigador cient�fico, queda dispensado de la tarea de adquirir por s� mismo los datos experimentales. Tarea que en algunos casos resultar�a imposible, por ejemplo si necesita medir muchas variables a la vez o si debe medirlas a gran ritmo; y en la que, en general, es relativamente f�cil cometer errores.

El an�lisis y tratamiento de datos experimentales, constituye una tarea fundamental dentro del trabajo de investigaci�n cient�fica. Los ordenadores permiten realizar dichas tareas, de una forma eficiente y segura con cantidades de datos que resultar�an imposibles de manejar hace 50 a�os. Como veremos m�s adelante, una simple hoja de c�lculo puede ahorrarnos una cuantas horas de c�lculos tediosos. El an�lisis estad�stico de un conjunto de datos experimentales, el c�lculo --la estimaci�n-- de los errores experimentales cometidos, la posterior regresi�n de los datos obtenidos a una funci�n matem�tica que permita establecer una ley o al menos una relaci�n entre los datos obtenidos, formar parte del trabajo cotidiano del investigador, virtualmente en todos los campos de la ciencia.

Por �ltimo el c�lculo. \index{C�lculo num�rico} Cabr�a decir que constituye el n�cleo del trabajo de investigaci�n. El cient�fico trata de explicar la realidad que le rodea, mediante el empleo de una descripci�n matem�tica. Dicha descripci�n suele tomar la forma de un modelo matem�tico m�s o menos complejo. La validez de un modelo est� ligada a que sea capaz de reproducir los resultados experimentales obtenidos del fen�meno que pretende explicar. Si el modelo es bueno ser� capaz de obtener mediante c�lculo unos resultados similares a los obtenido mediante el experimento. De este modo, el modelo queda validado y es posible emplearlo para predecir c�mo se comportar� el sistema objeto de estudio en otras condiciones.

\section{Introducci�n a los computadores} \index{Computador} \index{Ordenador}
M�s o menos todos estamos familiarizados con lo que es un computador, los encontramos a diario continuamente  y, de hecho, hay muchos aspectos de nuestra vida actual que ser�an inimaginables sin los computadores.  En t�rminos muy generales, podemos definir un computador como una m�quina que es capaz de recibir instrucciones y realizar operaciones (c�lculos) a partir de las instrucciones recibidas. Precisamente es la capacidad de recibir instrucciones lo que hace del ordenador una herramienta vers�til; seg�n las instrucciones recibidas y de acuerdo tambi�n a sus posibilidades como m�quina,  el ordenador puede realizar tareas muy distintas, entre las que cabe destacar como m�s generales, las siguientes:
\begin{itemize}
\item Procesamiento de datos 
\item Almacenamiento de datos
\item Transferencias de datos entre el computador y el exterior
\item Control de las anteriores operaciones
\end{itemize}

El computador se dise�a para realizar funciones generales que se especifican cuando se programa. La programaci�n es la que concreta las tareas que efectivamente realiza un ordenador concreto.

\subsection{Niveles de descripci�n de un ordenador}

La figura \ref{fig:nivel} muestra un modelo general de un computador descrito por niveles. Cada nivel, supone y se apoya en el nivel anterior. 
\begin{figure}[h]
	\centering
		\includegraphics[width=10cm]{nivel_descripcion.pdf}
	\caption{Descripci�n por niveles de un computador}
	\label{fig:nivel}
\end{figure}
\begin{enumerate}
\item \textbf{Nivel F�sico.} Constituye la base del \emph{hardware} del computador. Est� constituido por los componentes electr�nicos b�sicos, diodos, transistores, resistencias, etc.  En un computador moderno, no es posible separar o tan siquiera observar dichos componentes: Se han fabricado directamente sobre un cristal semiconductor, y forman parte de un dispositivo electr�nico conocido con el nombre de circuito integrado.

\item \textbf{Circuito Digital.}
Los componentes del nivel f�sico se agrupan formando circuitos digitales, (En nuestro caso circuitos digitales integrados). Los circuitos digitales trabajan solo con dos niveles de tensi�n ($V_1, V_0$) lo que permite emplearlos para establecer relaciones l�gicas: $V_1$=verdadero, $V_2$=falso. Estas relaciones l�gicas establecidas empleando los valores de la tensi�n de los circuitos digitales constituyen el soporte de todos los c�lculos que el computador puede realizar.

\item \textbf{Organizaci�n Hardware del sistema.}\index{Computador ! \emph{hardware}} 
Los circuitos digitales integrados se agrupan y organizan para formar el \emph{Hardware} del ordenador.  Los m�dulos b�sicos que constituyen el \emph{Hardware} son la unidad central de procesos (CPU), La unidad de memoria y las unidades de entrada y salida de datos. Dichos componentes est�n conectados entre s� mediante un bus, que transfiere datos de una unidad a otra.

\item \textbf{Arquitectura del computador.} \index{Computador ! arquitectura}
La arquitectura define c�mo trabaja el computador. Por tanto, est� estrechamente relacionada con la organizaci�n hardware del sistema, pero opera a un nivel de abstracci�n superior. Establece c�mo se accede a los registros de memoria, arbitra el uso de los buses que comunican unos componentes con otros, y regula el trabajo de la CPU.  

Sobre la arquitectura se establece el lenguaje b�sico en el que trabaja el ordenador, conocido c�mo lenguaje m�quina. Es un lenguaje que emplea todav�a niveles l�gicos binarios (ceros o unos) y por tanto no demasiado apto para ser interpretado por los seres humanos. Este lenguaje permite al ordenador realizar operaciones b�sicas como copiar el contenido de un registro de memoria en otro, sumar el contenido de dos registros de memoria, etc. 

El lenguaje m�quina es adecuado para los computadores, pero no para los humanos, por eso, los fabricantes suministran junto con el computador un repertorio b�sico de instrucciones que su m�quina puede entender y realizar en un lenguaje algo m�s asequible. Se trata del lenguaje ensamblador. Los comandos de �ste lenguaje son f�cilmente traducibles en una o varias instrucciones de lenguaje m�quina.   A�n as� se trata de un lenguaje en el que programar directamente resulta una tarea tediosa y proclive a cometer errores. 

\item \textbf{Compiladores y Sistemas Operativos} \index{Sistema operativo} \index{Compilador}
Los Compiladores constituyen un tipo de programas especiales que permiten convertir un conjunto de instrucciones, escritas en un lenguaje de alto nivel en lenguaje m�quina. El programador escribe sus instrucciones en un fichero de texto normal, perfectamente legible para el ser humano, y el compilador convierte las instrucciones contenidas en dicho fichero en secuencias binarias comprensibles por la m�quina.

Los computadores primitivos solo eran capaces de ejecutar un programa a la vez. A medida que se fueron fabricando ordenadores mas sofisticados, surgi� la idea de crear programas que se encargaran de las tareas b�sicas: gestionar el flujo de informaci�n, manejar perif�ricos, etc. Estos programas reciben el nombre de sistemas operativos. Los computadores modernos cargan al arrancar un sistema operativo que controla la ejecuci�n del resto de las aplicaciones. Ejemplos de sistemas operativos son DOS (Disk Operating System), Unix y su versi�n para ordenadores personales Linux.

\item \textbf{Lenguajes de alto nivel.} \index{Programaci�n! lenguajes}
Los lenguajes de alto nivel est�n pensados para facilitar la tarea del programador, desentendi�ndose de los detalles de implementaci�n del hardware del ordenador.  Est�n compuestos por un conjunto de comandos y unas reglas sint�cticas, que permiten describir las instrucciones para el computador en forma de texto.

De una manera muy general, se pueden dividir los lenguajes de alto nivel en lenguajes compilados y lenguajes interpretados. Los lenguajes compilados emplean un compilador para convertir los comandos del lenguaje de alto nivel en lenguaje m�quina. Ejemplos de lenguajes compilados son C , C++ y Fortran. Los lenguajes interpretados a diferencia de los anteriores no se traducen a lenguaje m�quina antes de ejecutarse. Si no que utilizan otro programa --el interprete-- que va leyendo los comandos del lenguaje y convirti�ndolos en instrucciones m�quina a la vez que el programa se va ejecutando. Ejemplos de programas interpretado son Basic, Python y Java.

\item \textbf{Aplicaciones.} \index{Programaci�n! aplicaciones} Se suele entender por aplicaciones programas orientados a tareas espec�ficas, disponibles para un usuario final. Habitualmente se trata de programas escritos en un lenguaje de alto nivel y presentados en un formato f�cilmente comprensible para quien los usa.

Existen multitud de aplicaciones, entre las m�s conocidas cabe incluir los navegadores para Internet, como Explorer, Mocilla o Google Crome, los editores de texto, como Word, las hojas de c�lculo como Excel o los clientes de correo como Outlook. En realidad, la lista de aplicaciones disponibles en el mercado ser�a interminable. 
\end{enumerate}
\subsection{El modelo de computador de Von Neumann} \index{Von Neumann}
Los computadores modernos siguen, en lineas generales, el modelo propuesto por Von Newmann.  La figura \ref{fig:vonn} muestra un esquema de dicho modelo. 

\begin{figure}[h]
	\centering
		\includegraphics[width=10cm]{von.pdf}
	\caption{Modelo de Von Neumann}
	\label{fig:vonn}
\end{figure}

En el modelo de Von Newman se pueden  distinguir tres m�dulos b�sicos y una serie de elementos de interconexi�n.  Los m�dulos b�sicos son: 

\begin{itemize}
\item \textbf{La Unidad Central de Procesos.} CPU \index{CPU} (\emph{Central process unit)}) , esta unidad constituye el n�cleo en el que el ordenador realiza las operaciones. 

Dentro de la CPU pueden a su vez distinguirse las siguientes partes

\begin{itemize}

\item La unidad de proceso � ruta de datos: Est� formada por La Unidad Aritm�tico L�gica (ALU), \index{ALU} capaz de realizar las operaciones aritm�ticas y l�gicas que indican las instrucciones del programa. En general las ALUs se construyen para realizar aritm�tica entre enteros, y realizar las operaciones l�gicas b�sicas del algebra de Boole (AND, OR, etc). Habitualmente, las operaciones para n�meros no enteros, representados en \emph{punto flotante} se suelen realizar empleando un procesador espec�fico que se conoce con el nombre de Coprocesador matem�tico. La velocidad de procesamiento suele medirse en millones de operaciones por segundo (MIPS) o millones de operaciones en punto flotante por segundo (MFLOPS).

\item El banco de registros: Conjunto de registros en los que se almacenan los datos con los que trabaja la ALU y los resultados obtenidos.
 
\item La unidad de control (UC) o ruta de control: se encarga de buscar las instrucciones en la memoria principal y guardarlas en el registro de instrucciones, las decodifica, las ejecuta empleando la ALU, guarda los resultados en el registro de datos, y guarda las condiciones derivadas de la operaci�n realizada en el registro de estado.  El registro de datos de memoria, contiene los datos que se est�n leyendo de la memoria principal o van a escribirse en la misma. El registro de direcciones de memoria, guarda la direcci�n de la memoria principal a las que esta accediendo la ALU, para leer o escribir. El contador del programa, tambi�n conocido como puntero de instrucciones, es un registro que guarda la posici�n en la que se encuentra la CPU dentro de la secuencia de instrucciones de un programa.
\end{itemize}
 

\item \textbf{La unidad de memoria.} Se trata de la memoria principal o primaria del computador.  Est� dividida en bloques de memoria que se identifican mediante una direcci�n. La CPU tiene acceso directo a dichos bloques de memoria.

La unidad elemental de informaci�n digital es el bit \index{bit} (0,1). La capacidad de almacenamiento de datos se mide en Bytes \index{Byte} y en sus m�ltiplos, calculados  como potencias de 2\footnote{Los prefijos $K$(Kilo),$M$(Mega),$G$ (Giga), etc., se reservan en el sistema internacional para indicar potencias de 10. Para su equivalente en potencias de 2 se deben emplear los t�rminos $Ki$ (Kibi), $Mi$ (Mebi),$Gi$ (Gibi),$Ti$,(Tebi). Por tanto, deber�a decirse  Kibibyte, Mebibyte, etc. Sin embargo, esta notaci�n no est� muy extendida y se habla de KB (KiloBytes), MB (Megabytes), etc aunque se realice el c�lculo en potencias de 2 }

\begin{align} \nonumber
1\  Byte = &\ 8\ bits\ &\  \\ \nonumber
1\  Word = &\ 16\ bits=2B &\  \\ \nonumber
1\  KiB  = &\ 2^{10}\ bits=1024\ B&\ \\  \nonumber
1\  MiB = &\ 2^{20}\ bits=1024\ KB&\ \\  \nonumber
1\  GiB = &\ 2^{30}\ bits &\ \\  \nonumber
1\  TiB  = &\ 2^{40}\ bits\ &\
\end{align} 

\item \textbf{Unidad de Entrada/Salida.} Transfiere informaci�n entre el computador y los dispositivos perif�ricos.
\end{itemize}

Los elementos de interconexi�n se conocen con el nombre de \emph{Buses}. Se pueden distinguir tres: En bus de datos, por el que se transfieren datos entre la CPU y la memoria � la unidad de entrada/salida. El bus de direcciones, par especificar una direcci�n de memoria o del registro de E/S. Y el bus de Control, por el que se env�an se�ales de control, tales como la se�al de reloj, la se�al de control de lectura/escrituras entre otras.  

\subsection{Representaci�n binaria} \index{Base 2}
Veamos con algo m�s de detalle, c�mo representa la informaci�n un computador. Como se explic� anteriormente, La electr�nica que constituye la parte f�sica del ordenador, trabaja con dos niveles de voltaje. Esto permite definir dos estados, --alto, bajo-- que pueden representarse dos s�mbolos  $0$ y $1$. Habitualmente, empleamos $10$ s�mbolos ${0,1,2,3,4,5,6,7,8,9}$, es decir, empleamos una representaci�n decimal. Cuando queremos representar n�meros mayores que nueve, dado que hemos agotado el n�mero de d�gitos disponibles, lo que hacemos es combinarlos, agrupando cantidades de diez en diez. As� por ejemplo, el numero $16$, representa seis unidades m�s un grupo de diez unidades y el n�mero $462$ representa dos unidades m�s seis grupos de diez unidades m�s cuatro grupos de 10 grupos de 10 unidades.  Matem�ticamente, esto es equivalentes a emplear sumas de d�gitos por potencias de diez:
\begin{equation*}
13024 = 1\times10^4+3\times10^3+0\times10^2+2\times10^1+4\times10^0 
\end{equation*}

Si recorremos los d�gitos que componen el n�mero de izquierda derecha, cada uno de ellos representa una potencia de diez superior, porque cada uno representa la cantidad de grupos de 10 grupos, de grupos ... de diez grupos de unidades. Esto hace que potencialmente podamos representar cantidades tan grandes como queramos, empleando tan solo diez s�mbolos. Esta representaci�n, a la que estamos habituados recibe el nombre de representaci�n en base 10 \index{Base 10}. Pero no es la �nica posible.

Volvamos a la representaci�n empleada por el computador. En este caso solo tenemos dos s�mbolos distintos el $0$ y el $1$. Si queremos emplear una representaci�n an�loga a la representaci�n en base diez, deberemos agrupar ahora las cantidad en grupos de dos. As� los �nicos n�meros que admiten ser representados con un solo d�gito son el uno y el cero. Para representar el n�mero dos, necesitamos agrupar: tendremos $0$ unidades y $1$ grupo de dos, con lo que la representaci�n del n�mero dos en base dos ser� $10$. Para representar el n�mero tres, tendremos una unidad m�s un grupo de dos, por lo que la representaci�n ser� $11$, y as� sucesivamente. Matem�ticamente esto es equivalente emplear sumas de d�gitos por potencias de 2:

\begin{equation*}
10110 = 1\times2^4+0\times2^3+1\times2^2+1\times2^1+0\times2^0 
\end{equation*}

Esta representaci�n recibe el nombre de representaci�n binaria o en base 2.
\index{Conversi�n! binario a decimal}La expansi�n de un n�mero representado en binario en potencias de 2, nos da un m�todo directo de obtener su representaci�n decimal. As�, para el ejemplo anterior, si calculamos las potencias de dos y sumamos los resultados obtenemos:

  \begin{equation*}
  1\times2^4+0\times2^3+1\times2^2+1\times2^1+0\times2^0=16+0+4+2+0=22 
\end{equation*}

que es la representaci�n en base 10 del n�mero binario $10110$.

Para n�meros no enteros, la representaci�n tanto en decimal como en binario, se extiende de modo natural empleando potencias negativas de 10 y de 2 respectivamente. As�,
 
\begin{equation} \nonumber
835.41 = 8\times10^2+3\times10^1+5\times10^0+4\times10^{-1}+1\times10^{-2} 
\end{equation}

y para un n�mero en binario,

\begin{equation} \nonumber
101.01 = 1\times2^2+0\times2^1+1\times2^0+0\times2^{-1}+1\times2^{-2} 
\end{equation}

De nuevo, basta calcular el t�rmino de la derecha de la expresi�n anterior para obtener la representaci�n decimal del n�mero $101.01$.

\index{Conversi�n! decimal a binario. n�meros enteros}�C�mo transformar la representaci�n de un n�mero de decimal a binario? De nuevo nos da la clave la representaci�n en sumas de productos de d�gitos por potencias de dos. Empecemos por el caso de un n�mero entero. Supongamos un n�mero D, representado en decimal. Queremos expandirlo en una suma de potencias de dos. Si dividimos el n�mero por 2, podr�amos representarlo c�mo:
 
\begin{equation*}
\label{eq:1}
D=2\cdot C_1+R_1
\end{equation*}

donde $C_1$ representa el cociente de la divisi�n y $R_1$ el resto. Como estamos dividiendo por dos, el resto solo puede valer cero o uno. Supongamos ahora que volvemos a dividir el cociente obtenido por dos,

 \begin{equation*}
 \label{eq:2}
C_1=2\cdot C_2+R_2 \
\end{equation*}

Si sustituimos el valor obtenido para $C_1$ en la ecuaci�n inicial obtenemos,   
\begin{equation*}
D=2\cdot(2\cdot C_2+R_2)+R_1= 2^2\cdot C_2+R_2\cdot 2^1+R_1\cdot 2^0 
\end{equation*}

Si volvemos a dividir el nuevo cociente obtenido $C_2$ por dos, y volvemos a sustituir,

 \begin{align*}
C_2&=2\cdot C_3+R_3 \\
D&=2^2\cdot(2\cdot C_3+R_3)+R_2\cdot 2^1+R_1\cdot 2^0=2^3\cdot C_3+R_3\cdot 2^2 +R_2\cdot 2^1+R_1\cdot 2^0
\end{align*}

Supongamos que tras repetir este proceso $n$ veces, obtenemos un conciente $C_n=1$. L�gicamente no tiene sentido seguir dividiendo ya que a partir de este punto, cualquier divisi�n posterior que hagamos nos dar� cociente $0$ y resto igual a $C_n$. Por tanto, 

 \begin{align*}
D&=1\cdot 2^n+R_n\cdot 2^{n-1}\cdots +R_3\cdot 2^2 +R_2\cdot 2^1+R_1\cdot 2^0
\end{align*}

La expresi�n obtenida, coincide precisamente con la expansi�n en potencias de dos del n�mero binario $1R_n \cdots R_3R_2R_1$.


Como ejemplo,podemos obtener la representaci�n en binario del n�mero $234$, empleando el m�todo descrito: vamos dividiendo el n�mero y los cocientes sucesivos entre dos, hasta obtener un cociente igual a uno y a continuaci�n, construimos la representaci�n binaria del n�mero colocando por orden, de derecha a izquierda.  los restos  obtenidos de las sucesivas divisiones y a�adiendo un uno m�s a la izquierda de la cifra construida con los restos:

\begin{table}[h]
\begin{tabular}{|r|r|r|r|}
Dividendo& &Cociente $\div 2$&Resto\\
\hline
234& &117&0\\
117& &58&1\\
58& &29&0\\
29& &14&1\\
14& &7&0\\
7& &3&1\\
3& &1&1
\end{tabular}
\end{table}
 
 Por tanto, la representaci�n en binario de 234 es 11101010.
 
\index{Conversi�n! decimal a binario, n�meros no entero}Supongamos ahora un n�mero no entero, representado en decimal, de la forma $0,d$ . Si lo multiplicamos por dos:

\begin{equation}
E_1,d_1=0,d\cdot 2
\end{equation}
Donde $E_1$ representa la parte entera y $d_1$ la parte decimal del n�mero calculado.
Podemos entonces representar $0,d$ como,
\begin{equation}
\label{eq:5}
0,d=(E_1,d_1)\cdot 2^{-1}=E_1\cdot 2^{-1}+0,d_1\cdot 2^{-1}
\end{equation}  

Si volvemos a multiplicar $0,d_1$ por dos,

\begin{equation}
E_2,d_2 = 0,d_1\cdot 2
\end{equation}

\begin{equation}
0,d_1=E_2\cdot 2^{-1}+0,d_2\cdot 2^{-1}
\end{equation}  

y sustituyendo en \ref{eq:5}

\begin{equation}
0,d=E_1\cdot 2^{-1}+E_2\cdot 2^{-2}+0,d_2\cdot 2^{-2}
\end{equation}

�Hasta cuando repetir el proceso? En principio hasta que obtengamos un valor cero para la parte decimal, $0,d_n=0$. Pero esta condici�n puede no cumplirse nunca. Puede darse el caso --de hecho es lo m�s probable-- de que un n�mero que tiene una representaci�n exacta en decimal, no la tenga en binario. El criterio para detener el proceso ser� entonces obtener un determinado n�mero de decimales o bien seguir el proceso hasta que la parte decimal obtenida vuelva a repetirse. Puesto que los ordenadores tienen un tama�o de registro limitado, tambi�n est� limitado el n�mero de d�gitos con el que pueden representar un n�mero decimal. Por eso, lo habitual ser� truncar el n�mero asumiendo el error que se comete al proceder as�.  De este modo, obtenemos la expansi�n del n�mero original en potencias de dos,

\begin{equation}
0,d\cdot 2=E_1\cdot 2^{-1}+E_2\cdot 2^{-2}+\cdots+ E_n\cdot 2^{-3}+\cdots
\end{equation} 

Donde los valores $E_1\cdots E_n$ son precisamente los d�gitos correspondientes a la representaci�n del n�mero en binario: $0.E_1E_2\cdots E_n$. (Es trivial comprobar que solo pueden valer $0$ � $1$).


Veamos un ejemplo de cada caso, obteniendo la representaci�n binaria del n�mero $0,625$, que tiene representaci�n exacta, y la del n�mero $0,626$, que no la tiene. En este segundo caso, calcularemos una representaci�n aproximada, tomando 8 decimales.

\begin{table}[h]
\begin{tabular}{|r|r|r|r|r r|r|r|r|r|}
P decimal& &$\times 2$& P entera& &&P decimal& &$\times 2$& P entera\\
\cline{1-4}
\cline{7-10}
0,625& &1,25&1& &&0,623& &1,246&1\\
0,25  & &0,5  &0& &&0,246& &0,492&0\\
0,5    & &1,0  &1& &&0,492& &0,984&0\\
         & &       &  & &&0,984& &1,968&1\\
         & &       &  & &&0,968& &1,936&1\\
         & &       &  & &&0,936& &1,872&1\\
         & &       &  & &&0.872& &1.744&1\\
         & &       &  & &&0.744& &1.488&1\\
\end{tabular}
\end{table}

Para construir la representaci�n binaria del primero de los n�meros, nos basta tomar las partes enteras obtenidas, por orden, escribirlas de de izquierda a derecha y a�adir un $0$ y la coma decimal a la izquierda. Por tanto  la representaci�n binaria de $0,625$ es $0,101$.  Si expandimos su valor en potencias de dos, volvemos a recuperar el n�mero original en su representaci�n decimal.

 En el segundo caso, la representaci�n binaria, tomando nueve decimales de $0,623$ es $0.10011111$. Podemos calcular el error que cometemos al despreciar el resto de los decimales, volviendo a convertir el resultado obtenido a su representaci�n en base diez,

 \begin{equation*}
0\cdot 2^{0}+1\cdot 2^{-1}+0\cdot 2^{-2}+ 0\cdot 2^{-3}+1\cdot 2^{-4}+1\cdot 2^{-5}+ 1\cdot 2^{-6}+1\cdot 2^{-7}+1\cdot 2^{-8}=0,62109375
\end{equation*} 

El error cometido es, en este caso: $\text{Error}=0,623-0,62109375=0,00190625$.
  
 \section{Aplicaciones de Software Cient�fico}
Dentro del mundo de las aplicaciones, merecen una menci�n aparte las dedicadas al c�lculo cient�fico, por su conexi�n con la asignatura. 

Es posible emplear lenguajes de alto nivel para construir rutinas y programas que permitan resolver directamente un determinado problema de c�lculo. En este sentido, el lenguaje FORTRAN se ha empleado durante a�os para ese fin, y todav�a sigue emple�ndose en muchas disciplinas cient�ficas y de la Ingenier�a.  Sin embargo, hay muchos aspectos no triviales del c�lculo con un computador, que obligar�an al cient�fico que tuviera que programar sus propios programas a ser a la vez un experto en computadores.  Por esta raz�n, se han ido desarrollando aplicaciones espec�ficas para c�lculo cient�fico que permiten al investigador centrarse en la resoluci�n de su problema y no en el desarrollo de la herramienta adecuada para resolverlo.  
 
En algunos casos, se trata de aplicaciones a medida, relacionadas directamente con alg�n �rea cient�fica concreta. En otros, consisten en paquetes de funciones espec�ficos para realizar de forma eficiente determinados c�lculos, como por ejemplo
el paquete SPSS para c�lculo estad�stico.

Un grupo especialmente interesante lo forman algunos paquetes de software que podr�amos situar a mitad de camino entre los lenguajes de alto nivel y las aplicaciones: Contienen extensas librer�as de funciones, que pueden ser empleadas de una forma directa para realizar c�lculos y adem�s permiten realizar programas espec�ficos empleando su propio lenguaje. Entre estos podemos destacar Mathematica, Maple , Matlab, Octave y Scilab y Python. El uso de estas herramientas se ha extendido enormemente en la comunidad cient�fica. Algunas como Matlab  constituyen casi un est�ndar en determinadas �reas de conocimiento.

 


    

 

  

\include{introprog}
\include{aritmetica}
\chapter{C�lculo de ra�ces de una funci�n}

\section{Ra�ces de una funci�n}
Se entiende por ra�ces de una funci�n real $f(x):\mathbb{R} \rightarrow \mathbb{R}$. los valores $x=r$ que satisfacen, $f(r)=0$

El c�lculo de las ra�ces de una funci�n, tiene una gran importancia en la ciencia, donde un n�mero significativo de problemas pueden reducirse a obtener la ra�z o ra�ces de una ecuaci�n.

La obtenci�n de la ra�z de una ecuaci�n es inmediata en aquellos casos en que se conoce la forma anal�tica de su funci�n inversa $f^{-1}$, ($f(x)=y\Rightarrow f^{-1}(y)=x$). En este caso, $r=f^{-1}(0)$. Por ejemplo,
\begin{align*}
f(x)&=x^2-4\\
f^{-1}(y)&=\pm\sqrt{y+4}\Rightarrow r=f^{-1}(0)=\pm 2\
\end{align*}

Sin embargo, en muchos casos de inter�s las funciones no pueden invertirse.  Un ejemplo, extra�do de la f�sica es la ecuaci�n de Kepler para el c�lculo de las �rbitas planetarias,
\begin{equation*}
x-a\sin(x)=b
\end{equation*}

Donde $a$ y $b$ son par�metros conocidos y se desea conocer el valor de $x$. La soluci�n de la ecuaci�n de Kepler es equivalente a obtener las ra�ces de la funci�n $f(x)=x-a\sin(x)-b$. (La figura \ref{fig:kepler} muestra un ejemplo de dicha funci�n.) En este caso, no se conoce la funci�n inversa, y solo es posible conocer el valor de la ra�z, aproximadamente, empleando m�todos num�ricos.
\begin{figure}[h]
\centering
		\includegraphics[width=12cm]{kepler.pdf}
	\caption{Ejemplo de ecuaci�n de Kepler para $a=40$ y $b=2$}
	\label{fig:kepler}
\end{figure}

\paragraph*{M�todos iterativos}Todos los m�todos que se describen en este cap�tulo, se basan en procedimientos iterativos. La idea es estimar un valor inicial para la ra�z $r_0$, y a partir de �l ir refinando paso a paso la soluci�n, de modo que el resultado se acerque cada vez m�s al valor real de la ra�z. Cada nueva aproximaci�n a la ra�z se obtiene a partir de las aproximaciones anteriores. 
\begin{align*}
r_0\ \ \  \rightarrow \  \ r_1  \ \ \rightarrow \ \ r_2 \ \ \rightarrow \cdots \rightarrow \ \ r_k \ \rightarrow \cdots\\
\vert f(r_0)\vert \ge \vert f(r_1)\vert \ge \vert f(r_2)\vert \ge \cdots \ge \vert f(r_k)\vert \ge \cdots
\end{align*}

El proceso que lleva de una soluci�n aproximada a la siguiente se conoce con el nombre de \emph{iteraci�n}. Lo habitual es que en cada iteraci�n se realicen las mismas operaciones matem�ticas una y otra vez. 

El proceso se detiene cuando la soluci�n alcanzada se estima lo suficientemente pr�xima a la soluci�n real como para darla por buena. Para ello, se suele establecer un valor (\emph{tolerancia}) que act�a como criterio de convergencia. De este modo, las iteraciones se repiten hasta que se llega a un valor $r_n$ 	que cumple,
\begin{equation*}
\vert f(r_n) \vert \leq \text(tol)
\end{equation*}

Se dice entonces que el algoritmo empleado para obtener la ra�z ha convergido en \emph{n} iteraciones. Por otro lado, es importante se�alar que los algoritmos para el c�lculo de ra�ces de una funci�n no siempre convergen. Hay veces en que no es posible aproximarse cada vez m�s al valor de la ra�z bien por la naturaleza de la funci�n o bien por que el algoritmo no es adecuado para obtenerla.

\paragraph*{B�squeda local.} Una funci�n puede tener cualquier n�mero de ra�ces, incluso infinitas, basta pensar por ejemplo en funciones trigonom�tricas como $\cos(x)$. Una caracter�stica importante de los m�todos descritos en este cap�tulo es que solo son capaces de aproximar una ra�z. La ra�z de la funci�n a la que el m�todo converge depende de el valor inicial $r_0$ con el que se comienza la b�squeda iterativa\footnote{En ocasiones, como veremos m�s adelante no se suministra al algoritmo un valor inicial, sino un intervalo en el que buscar la ra�z}. Por ello reciben el nombre de m�todos locales. Si queremos encontrar varias (o todas) las ra�ces de una determinada funci�n, es preciso emplear el m�todo para cada una de las ra�ces por separado, cambiando cada vez el punto de partida.

\section{Metodos iterativos locales}
\subsection{M�todo de la bisecci�n}
\paragraph*{Teorema de Bolzano.}
\begin{quote}
Si una funcion $f(x)$, continua en el intervalo $[a, b]$, cambia de signo en los extremos del intervalo: $f(a)\cdot f(b) \le 0$, debe tener una ra�z en el intervalo [a, b]. (figura: \ref{fig:bolzano}) 
\end{quote}

\begin{figure}[h]
\centering
\includegraphics[width=14cm]{bolzano.pdf}
\caption{Ilustraci�n del teorema de Bolzano}
\label{fig:bolzano}
\end{figure}

El conocido teorema de Bolzano, suministra el m�todo m�s sencillo de aproximar la ra�z de una funci�n: Se parte de un intervalo inicial en el que se cumpla el teorema; y se va acotando sucesivamente el intervalo que contiene la ra�z, reduci�ndolo a la mitad en cada iteraci�n, de forma que en cada nuevo intervalo se cumpla siempre el teorema de Bolzano.

\begin{figure}[h]
\centering
\begin{tikzpicture}
%\usetikzlibrary{shapes.geometric}
\path (5,0) node(a) [rectangle,draw=blue, very thick,align=center,rounded corners]{Partimos de $[a,b]$\\ con\\ $f(a)\cdot f(b)<0$}
(5,-2) node(b)[rectangle,draw=blue, thick,rounded corners]{Calculamos $c=\frac{a+b}{2}, f(c)$}
(5,-4) node(c)[diamond,aspect=3,draw=red,thick]{es $\vert f(c) \vert \le \text{tol}$?}
(9,-4) node(d)[rectangle,draw=blue,align=center,very thick, rounded corners]{convergencia:\\ terminar}
(5,-6) node(e)[diamond,aspect=3,draw=red,thick]{es $f(a)\cdot f(c) < 0$?}
(9.5,-6) node(f)[rectangle,draw=blue,thick,rounded corners,align=center]{$b=c$\\$f(b)=f(c)$}
(5,-8) node(g)[rectangle,draw=blue,thick,rounded corners,align=center]{$a=c$\\$f(a)=f(c)$};
\draw[blue,-latex](a.south)--(b);
\draw[blue,-latex](b.south)--(c);
\draw[blue,-latex](c.east)--(d);
\draw (7.5,-4)node[above]{S�};
\draw[blue,-latex](c.south)--(e);
\draw (5,-5)node[right]{No};
\draw[blue,-latex](e.east)--(f);
\draw (8,-6)node[above]{S�};
\draw[blue,-latex](e.south)--(g);
\draw (5,-7.2)node[right]{No};
\draw[blue,-latex](g.south)|-(2,-9)|-(b);
\draw[blue,-latex](f.east)-|(11,-2)--(b);
\end{tikzpicture}
\caption{Diagrama de flujo del m�todo de la bisecci�n}
\label{fig:dfbisec}
\end{figure}
En la figura \ref{fig:dfbisec} se muestra un diagrama de flujo correspondiente al m�todo de la bisecci�n. El punto de partida es un intervalo $[a,b]$ en el que se cumple el teorema de Bolzano, y que contiene por tanto al menos una ra�z. Es interesante hacer notar que el teorema de Bolzano se cumple siempre que la funci�n sea continua en el intervalo $[a,b]$ y existan un n�mero impar de ra�ces. Por esto es importante realizar cuidadosamente la elecci�n del intervalo $[a,b]$, si hay m�s de una ra�z, el algoritmo puede no converger.

 Una vez que se tiene el intervalo se calcula el punto medio $c$. A continuaci�n se compara el valor que toma la funci�n en $c$, es decir $f(c)$ con la tolerancia. Si el valor es menor que �sta, el algoritmo ha encontrado un valor aproximado de la ra�z con la tolerancia requerida, con lo que $c$ es la ra�z y no hace falta seguir buscando. Si por el contrario, $f(c)$ est� por encima de la tolerancia requerida, comparamos su signo con el que toma la funci�n en uno cualquiera de los extremos del intervalo, En el diagrama de flujo se ha elegido el extremo $a$, pero el algoritmo funcionar�a igualmente si eligi�ramos $b$. Si el signo de $f(c)$ coincide con el que toma la funci�n en el extremo del intervalo elegido, $c$ sustituye al extremo, (hacemos $a=c$ y $f(a)=f(c)$) si por el contrario el signo es distinto, hacemos que $c$ sustituya al otro extremo del intervalo. (hacemos $b=c$ y $f(b)=f(c)$). Este proceso se repetir� hasta que se cumpla que $f(c)\le \text{tol}$ 

El proceso se muestra gr�ficamente en la figura \ref{fig:bisec}, para un caso particular. Se trata de obtener la ra�z de la funci�n mostrada en la figura \ref{fig:bolzano}, $f(x)=e^x-x^2$. esta funci�n tiene una �nica ra�z: $r\approx -0.7035$. Para iniciar el algoritmo se ha elegido un intervalo $[a=-2,b=2]$. La figura \ref{fig:bisec}, muestra tres iteraciones sucesivas,y la soluci�n final, que se obtiene al cabo de ocho iteraciones en �ste ejemplo, para el que se a empleado una tolerancia $tol=0.01$. En la secuencia de gr�ficas se puede observar tambi�n la evoluci�n del intervalo de b�squeda, $[-2, 2]\rightarrow [-2, 0] \rightarrow [-1, 0] \rightarrow [-1, -0.5] \cdots$; as� como el cambio alternativo del l�mite derecho o izquierdo, para asegurar que la ra�z queda siempre dentro de los sucesivos intervalos de b�squeda obtenidos. 
\begin{figure}
\centering
\subfigure[intervalo inicial]{\includegraphics[width=7cm]{rint0.pdf}} \qquad
\subfigure[iteracion 1]{\includegraphics[width=7cm]{rint1.pdf}}\\
\subfigure[iteracion 2]{\includegraphics[width=7cm]{rint2.pdf}}\qquad
\subfigure[iteracion 3]{\includegraphics[width=7cm]{rint3.pdf}}\\
\subfigure[iteracion 6: ra�z alcanzada]{\includegraphics[width=7cm]{rint4.pdf}}

\caption{proceso de obtenci�n de la ra�z de una funci�n por el m�todo de la bisecci�n }
\label{fig:bisec}
\end{figure}

\subsection{M�todo de interpolaci�n lineal o (\emph{Regula falsi})}
Este m�todo supone una mejora del anterior ya que, en general,  converge m�s r�pidamente. La idea es modificar el modo en que calculamos el punto $c$. En el caso del m�todo de la bisecci�n el criterio consist�a en ir tomando en cada iteraci�n el punto medio del intervalo que contiene la ra�z. El m�todo de interpolaci�n lineal, elige como punto $c$ el punto de corte con el eje x, de la recta que pasa por los puntos $\left(a,f(a)\right)$ y $\left(b,f(b)\right)$. Es decir la recta que corta a la funci�n $f(x)$ en ambos l�mites del intevalo que contiene a la ra�z buscada. La recta que pasa por ambos puntos puede construirse a partir de ellos como,
\begin{equation*}
y=\frac{f(a)-f(b)}{a-b}\cdot(x-b)+f(b)
\end{equation*}
el punto de corte con el eje $x$, que ser� el valor que tomaremos para $c$, se obtiene cuando $y=0$,
\begin{equation*}
0=\frac{f(a)-f(b)}{a-b}\cdot(x-b)+f(b)
\end{equation*}

y despejando $c\equiv x$ en la ecuaci�n anterior obtenemos,
\begin{equation*}
c=b-\frac{f(b)}{f(b)-f(a)}\cdot(b-a)
\end{equation*}

La figura \ref{fig:regulaf} muestra  gr�ficamente la posici�n del punto $c$ obtenido mediante el m�todo de interpolaci�n. 

\begin{figure}[h]
\centering
\includegraphics[width=14cm]{rinter00.pdf}

\caption{Obtenci�n de la recta que une los extremos de un intervalo $[a,b]$  que contiene una ra�z de la funci�n}
\label{fig:regulaf}
\end{figure}

Por lo dem�s, el procedimiento es el mismo que en el caso del m�todo de la bisecci�n. Se empieza con un intervalo $[a,b]$ que  contenga una ra�z, se obtiene el punto $c$ por el procedimiento descrito y se intercambia $c$ con el extremo del intervalo cuya imagen $f(a)$ o $f(b)$ tenga el mismo signo que $f(c)$ el procedimiento se repite iterativamente hasta que f(c) sea menor que el valor de tolerancia preestablecido. 

\begin{figure}[h]
\centering
\begin{tikzpicture}
%\usetikzlibrary{shapes.geometric}
\path (5,0) node(a) [rectangle,draw=blue, very thick,align=center,rounded corners]{Partimos de $[a,b]$\\ con\\ $f(a)\cdot f(b)<0$}
(5,-2) node(b)[rectangle,draw=blue, thick,rounded corners,align=center]{Calculamos\\ $c=b-\frac{f(b)}{f(b)-f(a)}\cdot(b-a), f(c)$}
(5,-4) node(c)[diamond,aspect=3,draw=red,thick]{es $\vert f(c) \vert \le \text{tol}$?}
(9,-4) node(d)[rectangle,draw=blue,align=center,very thick, rounded corners]{convergencia:\\ terminar}
(5,-6) node(e)[diamond,aspect=3,draw=red,thick]{es $f(a)\cdot f(c) < 0$?}
(9.5,-6) node(f)[rectangle,draw=blue,thick,rounded corners,align=center]{$b=c$\\$f(b)=f(c)$}
(5,-8) node(g)[rectangle,draw=blue,thick,rounded corners,align=center]{$a=c$\\$f(a)=f(c)$};
\draw[blue,-latex](a.south)--(b);
\draw[blue,-latex](b.south)--(c);
\draw[blue,-latex](c.east)--(d);
\draw (7.5,-4)node[above]{S�};
\draw[blue,-latex](c.south)--(e);
\draw (5,-5)node[right]{No};
\draw[blue,-latex](e.east)--(f);
\draw (8,-6)node[above]{S�};
\draw[blue,-latex](e.south)--(g);
\draw (5,-7.2)node[right]{No};
\draw[blue,-latex](g.south)|-(2,-9)|-(b);
\draw[blue,-latex](f.east)-|(11,-2)--(b);
\end{tikzpicture}
\caption{Diagrama de flujo del m�todo de interpolaci�n lineal}
\label{fig:regula}
\end{figure}

En la figura \ref{fig:regula} se muestra el diagrama de flujo para el m�todo de interpolaci�n lineal. Como puede verse, es id�ntico al de la bisecci�n excepto en el paso en que se obtiene el valor de $c$, donde se ha sustituido el c�lculo del punto medio del intervalo de b�squeda, por el c�lculo del punto de corte con el eje de abscisas  de la recta que une los extremos del intervalo.

La figura \ref{fig:iterr2} Muestra gr�ficamente el proceso iterativo seguido para obtener la ra�z de una funci�n en un intervalo mediante el m�todo de interpolaci�n lineal. Se ha empleado la misma funci�n y el mismo intervalo inicial que en el caso de la bisecci�n. 

Es f�cil ver, sin embargo, que los puntos intermedios que obtiene el algoritmo hasta converger a la ra�z son distintos. De hecho, el algoritmo emplea ahora tan solo siete iteraciones para obtener la ra�z, empleando el mismo valor para la tolerancia, 0.01, que se emple� en el m�todo de la bisecci�n.

Una observaci�n final, se ha dicho al principio que �ste m�todo supone una mejora al m�todo anterior de la bisecci�n. Esto no siempre es cierto. El m�todo de la bisecci�n tiene una tasa de convergencia constante, cada iteraci�n divide el espacio de b�squeda por la mitad. Sin embargo la convergencia del m�todo de interpolaci�n  lineal depende de la funci�n $f(x)$ y de la posici�n relativa de los puntos iniciales $(a, f(a))$  y $(b, f(b))$ con respecto al la ra�z. Por esto no es siempre cierto que converja m�s r�pido que el m�todo de  la bisecci�n. Por otro lado, el c�lculo de los sucesivos valores del punto $c$, requiere m�s operaciones aritm�ticas en el m�todo de interpolaci�n, con lo que cada iteraci�n resulta m�s lenta que en el caso de la bisecci�n.
\begin{figure}
\centering
\subfigure[intervalo inicial]{\includegraphics[width=7cm]{rinter0.pdf}} \qquad
\subfigure[iteracion 1]{\includegraphics[width=7cm]{rinter1.pdf}}\\
\subfigure[iteracion 2]{\includegraphics[width=7cm]{rinter2.pdf}}\qquad
\subfigure[iteracion 3]{\includegraphics[width=7cm]{rinter3.pdf}}\\
\subfigure[iteracion 6: ra�z alcanzada]{\includegraphics[width=7cm]{rinter4.pdf}}

\caption{Proceso de obtenci�n de la ra�z de una funci�n por el m�todo de interpolaci�n lineal}
\label{fig:iterr2}
\end{figure}

\subsection{M�todo de Newton-Raphson}
El m�todo de Newton se basa en la expansi�n de una funci�n $f(x)$ en serie de Taylor en el entorno de un punto $x_0$,
\begin{equation*}
f(x)\approx f(x_0)+f'(x_0)(x-x_0)+\frac{1}{2}f''(x_0)(x-x_0)^2+\cdots+\frac{1}{n!}f^{(n)}(x_0)(x-x_0)^n+\cdots
\end{equation*}
 Pertenece a una familia de m�todos ampliamente empleados en c�lculo num�rico. La idea en el caso del m�todo de Newton es aproximar la funci�n para la que se desea obtener la ra�z, mediante el primer t�rmino de la serie de Taylor. Es decir aproximar localmente $f(x)$, en el entorno de $x_0$ por la recta,
\begin{equation*}
 f(x_0)+f'(x_0)(x-x_0)
\end{equation*}
Esta recta, es precisamente la recta tangente a la curva $f(x)$ en el punto $x_0$ (figura \ref{fig:newton1})
\begin{figure}[h]
\centering
\includegraphics[width=14cm]{newt0.pdf}[h]
\caption{Recta tangente a la funci�n $f(x)$ en el punto $x_0$}
\label{fig:newton1}
\end{figure}

El m�todo consiste en obtener el corte de esta recta tangente con el eje de abscisas,
\begin{equation*}
0= f(x_0)+f'(x_0)(x-x_0)
\end{equation*}

y despejando x,

\begin{equation*}
x=x_0-\frac{f(x_0)}{f'(x_0)}
\end{equation*}

A continuaci�n se eval�a la funci�n en el punto obtenido $x\rightarrow f(x)$. Como en los m�todos anteriores, se compara el valor de $f(x)$ con una cierta tolerancia preestablecida. Si es menor, el valor de $x$ se toma como ra�z de la funci�n. Si no, se vuelve aplicar el algoritmo, empleando ahora el valor de x que acabamos de obtener como punto de partida. Cada c�lculo constituye una nueva iteraci�n y los sucesivos valores obtenidos para $x$, convergen a la ra�z,

\begin{equation*}
x_0\rightarrow x_1=x_0-\frac{f(x_0)}{f'(x_0)}\rightarrow x_2=x_1-\frac{f(x_1)}{f'(x_1)}\rightarrow  \cdots \rightarrow x_n=x_{n-1}-\frac{f(x_{n-1})}{f'(x_{n-1})}\rightarrow \cdots
\end{equation*}

La figura \ref{fig:newton} muestra un diagrama de flujo correspondiente al m�todo de Newton. Si se compara con los diagramas de flujo de los algoritmos anteriores, el algoritmo de Newton resulta algo m�s simple de implementar. Sin embargo es preciso evaluar en cada iteraci�n el valor de la funci�n y el de su derivada. 

El c�lculo de la derivada, es el punto d�bil de este algoritmo, ya que para valores $x_0$ pr�ximos a un m�nimo o m�ximo local obtendremos valores de la derivada pr�ximos a cero, lo que puede causar un error de desbordamiento al calcular el punto de corte de la recta tangente con el eje de abscisas o hacer que el algoritmo converja a una ra�z alejada del punto inicial de b�squeda.
 
\begin{figure}[h]
\centering
\begin{tikzpicture}
%\usetikzlibrary{shapes.geometric}
\path (5,0) node(a) [rectangle,draw=blue, very thick,align=center,rounded corners]{Partimos de un punto inicial $x_0$}
(5,-2) node(b)[rectangle,draw=blue, thick,rounded corners,align=center]{Calculamos\\ $x=x_0-\frac{f(x_0)}{f'(x_0)}, f(x)$}
(5,-4) node(c)[diamond,aspect=3,draw=red,thick]{es $\vert f(x) \vert \le \text{tol}$?}
(9,-4) node(d)[rectangle,draw=blue,align=center,very thick, rounded corners]{convergencia:\\ terminar}
(5,-6) node(g)[rectangle,draw=blue,thick,rounded corners,align=center]{$x_0=x$};
\draw[blue,-latex](a.south)--(b);
\draw[blue,-latex](b.south)--(c);
\draw[blue,-latex](c.east)--(d);
\draw (7.5,-4)node[above]{S�};
\draw[blue,-latex](c.south)--(g);
\draw (5,-5)node[right]{No};
\draw[blue,-latex](g.south)|-(2,-7)|-(b);

\end{tikzpicture}
\caption{Diagrama de flujo del m�todo de Newton-Raphson}
\label{fig:newton}
\end{figure}

La figura \ref{fig:newton2} muestra un ejemplo de obtenci�n de la ra�z de una funci�n mediante el m�todo de Newton. El m�todo es m�s r�pido que los dos anteriores, es decir, partiendo de una distancia comparable a la ra�z, es el que converge en menos iteraciones. 

En el ejemplo de la figura se ha obtenido la ra�z para la misma funci�n que en los ejemplos del m�todo de la bisecci�n e interpolaci�n lineal. Se ha empezado sin embargo en un punto m�s alejado de la ra�z, para que pueda observarse mejor en la figura la evoluci�n del algoritmo. En cada uno de los gr�ficos que componen la figura pueden observarse  los pasos del algoritmo: dado el punto  $x_i$, se calcula  la recta tangente a la funci�n $f(x)$ en el punto y se obtiene un nuevo punto $x_{i+1}$,  como el corte de dicha recta tangente con el eje de abscisas.

En este ejemplo el algoritmo converge en las cinco iteraciones que se muestran en la figura, para la misma tolerancia empleada en los m�todos anteriores, $tol=0.01$. El punto de inicio empleado fue $x_0=2.5$, por tanto esta fuera del intervalo $[-2, 2]$ y m�s alejado de la ra�z que en el caso de los m�todos anteriores.   

\begin{figure}
\centering
\subfigure[intervalo inicial]{\includegraphics[width=7cm]{newt01.pdf}} \qquad
\subfigure[iteracion 1]{\includegraphics[width=7cm]{newt02.pdf}}\\
\subfigure[iteracion 2]{\includegraphics[width=7cm]{newt1.pdf}}\qquad
\subfigure[iteracion 3]{\includegraphics[width=7cm]{newt2.pdf}}\\
\subfigure[iteracion 4]{\includegraphics[width=7cm]{newt3.pdf}}\qquad
\subfigure[iteracion 5: ra�z de la funci�n]{\includegraphics[width=7cm]{newt4.pdf}}

\caption{Proceso de obtenci�n de la ra�z de una funci�n por el m�todo de Newton}
\label{fig:newton2}
\end{figure}
\subsection{M�todo de la secante}
El m�todo de la secante podr�a considerarse una variante del m�todo de newton en el que se sustituye la recta tangente al punto $x0$ por la recta secante que une dos puntos obtenidos en iteraciones sucesivas. La idea es \emph{aproximar} la derivada a la funci�n $f$ en el punto $x_n$ por la pendiente de una recta secante, es decir de una recta que corta a la funci�n en dos puntos, 
\begin{equation*}
f'(x_n)\approx \frac{f(x_n)-f(x_{n-1})}{x_n-x_{n-1}}
\end{equation*}

Las sucesivas aproximaciones a la ra�z de la funci�n se obtienen de modo similar a las del m�todo de Newton, simplemente sustituyendo la derivada de la funci�n por su valor aproximado,

\begin{equation*}
x_{n+1}=x_n-\frac{f(x_n)}{f'(x_n)}\approx x_n-\frac{(x_n-x_{n-1})\cdot f(x_n)}{f(x_n)-f(x_{n-1})}
\end{equation*}

Para iniciar el algoritmo, es preciso emplear en este caso dos puntos iniciales. La figura \ref{fig:secante} muestra un ejemplo.

\begin{figure}[h]
\includegraphics[width=14cm]{secante0.pdf}
\caption{Recta secante a la  funci�n $f(x)$ en los puntos $x_0$ y $x_1$}
\label{fig:secante}
\end{figure}

El m�todo podr�a en este punto confundirse con el de interpolaci�n, sin embargo tiene dos diferencias importantes: En primer lugar, la elecci�n de los dos puntos iniciales $x_0$ e $x_1$, no tienen por qu� formar un intervalo que contenga a la ra�z. Es decir, podr�an estar ambos situados al mismo lado de la ra�z. En segundo lugar, los puntos obtenidos se van sustituyendo por orden, de manera que la nueva recta secante se construye siempre a partir de los dos �ltimos puntos obtenidos, sin prestar atenci�n a que el valor de la ra�z est� contenido entre ellos. (No se comparan los signos de la funci�n en los puntos para ver cual se sustituye, como en el caso del m�todo de interpolaci�n). 

La figura \ref{fig:secante2} muestra un diagrama de flujo para el m�todo de la secante. El diagrama es b�sicamente el mismo que el empleado para el m�todo de Newton. Las dos diferencias fundamentales son, que ahora en lugar de evaluar la funci�n y la derivada en cada iteraci�n, se calcula  el valor del punto de corte de la recta que pasa por los dos �ltimos puntos obtenidos (es decir, empleamos una recta secante, que corta a la curva en dos puntos, en lugar de emplear una recta tangente). 

Adem�s es preciso actualizar, en cada iteraci�n, el valor de los dos �ltimos puntos obtenidos: el m�s antiguo se desecha, el punto reci�n obtenido sustituye al anterior y �ste al obtenido dos iteraciones antes. 

\begin{figure}[h]
\centering
\begin{tikzpicture}
%\usetikzlibrary{shapes.geometric}
\path (5,0) node(a) [rectangle,draw=blue, very thick,align=center,rounded corners]{Partimos de dos puntos inicial $x_0$, $x_1$}
(5,-2) node(b)[rectangle,draw=blue, thick,rounded corners,align=center]{Calculamos\\ $x=x_1-\frac{(x_1-x_0)\cdot f(x_1)}{f(x_1)-f(x_0)}, f(x)$}
(5,-4) node(c)[diamond,aspect=3,draw=red,thick]{es $\vert f(x) \vert \le \text{tol}$?}
(9,-4) node(d)[rectangle,draw=blue,align=center,very thick, rounded corners]{convergencia:\\ terminar}
(5,-6) node(g)[rectangle,draw=blue,thick,rounded corners,align=center]{$x_0=x_1$\\ $x_1=x$};
\draw[blue,-latex](a.south)--(b);
\draw[blue,-latex](b.south)--(c);
\draw[blue,-latex](c.east)--(d);
\draw (7.5,-4)node[above]{S�};
\draw[blue,-latex](c.south)--(g);
\draw (5,-5)node[right]{No};
\draw[blue,-latex](g.south)|-(2,-7)|-(b);

\end{tikzpicture}
\caption{Diagrama de flujo del m�todo de la secante}
\label{fig:secante2}
\end{figure}

La figura \ref{fig:secante3} muestra un ejemplo de la obtenci�n de una ra�z por el m�todo de la secante. Se ha empleado de nuevo la misma funci�n que en los ejemplos anteriores, tomando como valores iniciales, $x_0=-2.5$ y $x_1=0.5$. La tolerancia se ha fijado en $tol=0.01$ tambi�n como en los anteriores algoritmos descritos. En este caso, el algoritmo encuentra la ra�z en cinco iteraciones. Cada uno de los gr�ficos que compone la figura \ref{fig:secante3}, muestra la obtenci�n de un nuevo punto a partir de los dos anteriores. 

En la iteraci�n 2, puede observarse como el nuevo punto se obtiene a partir de dos puntos que est�n ambos situados a la derecha de la ra�z, es decir, no forman un intervalo que contenga a la ra�z.  Aqu� se pone claramente de manifiesto la diferencia con el m�todo de interpolaci�n lineal. De hecho, com ya se ha dicho, el m�todo de la secante puede iniciarse tomando los dos primeros puntos a uno de los lados de la ra�z.

 El m�todo es, en principio, m�s eficiente que el de la bisecci�n y el de interpolaci�n lineal, y menos eficiente que el de Newton.

La ventaja de este m�todo respecto al de Newton es que evita tener que calcular expl�citamente la derivada de la funci�n para la que se quiere calcular la ra�z. El algunos casos, la obtenci�n de la forma anal�tica de dicha derivada puede ser compleja.   

\begin{figure}
\centering
\subfigure[intervalo inicial]{\includegraphics[width=7cm]{secante0.pdf}} \qquad
\subfigure[iteracion 1]{\includegraphics[width=7cm]{secante1.pdf}}\\
\subfigure[iteraci�n 2]{\includegraphics[width=7cm]{secante2.pdf}}\qquad
\subfigure[iteraci�n 3]{\includegraphics[width=7cm]{secante3.pdf}}\\
\subfigure[iteraci�n 4]{\includegraphics[width=7cm]{secante31.pdf}}\qquad
\subfigure[iteraci�n 5]{\includegraphics[width=7cm]{secante4.pdf}}

\caption{proceso de obtenci�n de la ra�z de una funci�n por el m�todo de la secante}
\label{fig:secante3}
\end{figure}

\subsection{M�todo de las aproximaciones sucesivas o del punto fijo}\label{pfijo}

El m�todo del punto fijo es, como se ver� a lo largo de esta secci�n, el m�s sencillo de programar de todos. Desafortunadamente, presenta el problema de que no podemos aplicarlo a todas las funciones. Hay casos en los que el m�todo no converge, con lo que no es posible emplearlo para encontrar la ra�z o ra�ces de una funci�n.
 
\paragraph{Punto fijo de una funci�n.} \index{Punto fijo! de una funci�n}Se dice que un punto $x_f$ es un punto fijo de una funci�n $g(x)$ si se cumple,
\begin{equation*}
g(x_f)=x_f
\end{equation*}

Es decir, la imagen del punto fijo $x_f$ es de nuevo el punto fijo. As� por ejemplo la funci�n,
\begin{equation*}
g(x)=-\sqrt{e^x}
\end{equation*}

Tiene un punto fijo en $x_f=-0.703467$, porque $g(-0.703467)=-0.703467$. La existencia de un punto fijo puede obtenerse gr�ficamente, representando en un mismo gr�fico la funci�n $y=g(x)$ y la recta $y=x$. Si existe un punto de corte entre ambas gr�ficas, se trata de un punto fijo.  La figura \ref{fig:pfijo0}, muestra gr�ficamente el punto fijo de la funci�n $g(x)=-\sqrt{e^x}$ del ejemplo anterior.

Una funci�n puede tener uno o m�s puntos fijos o no tener ninguno. Por ejemplo, la funci�n $y=\sqrt{e^x}$ no tiene ning�n punto fijo.


\begin{figure}[h]
\includegraphics[width=14cm]{pfijo0.eps}
\caption{Obtenci�n gr�fica del punto fijo de la funci�n, $g(x)=-\sqrt{e^x}$}
\label{fig:pfijo0}
\end{figure}

\paragraph{Punto fijo atractivo.} \index{Punto fijo! atractivo}Supongamos ahora que, a partir de la funci�n $g(x)$ creamos la siguiente sucesi�n,
\begin{equation*}
x_{n+1}=g(x_n)
\end{equation*}

Es decir, empezamos tomando un punto inicial $x_0$ y a partir de �l vamos obteniendo los siguientes valores de la sucesi�n como,
\begin{equation*}
x_0\rightarrow x_1=g(x_0)\rightarrow x_2=g(x_1)=g\left(g(x_0)\right) \rightarrow \cdots \rightarrow x_{n+1}=g(x_{n})=g\left( g\left( \cdots\left( g(x_0)\right)\right)\right) \rightarrow \cdots
\end{equation*}
Decimos que un punto fijo $x_f$ de la funci�n $g(x)$ es un punto fijo atractivo si la sucesi�n $x_{n+1}=g(x_n)$ converge al valor $x_f$, siempre que $x_0$ se tome \emph{suficientemente} cercano a $x_f$. C�mo de cerca tienen que estar $x_0$ y $x_f$ para que la serie converja, es una cuesti�n delicada. De entrada, es importante descartar que hay funciones que tienen puntos fijos no atractivos, por ejemplo, la funci�n $g(x)=x^2$ tiene dos puntos fijos $x=0$ y $x=1$. El primero es el l�mite de la sucesi�n $x_{n+1}=g(x_n)$ para cualquier valor inicial $x_0$ contenido en el intervalo abierto $(-1,  1)$. El punto $x=1$ resulta inalcanzable para cualquier sucesi�n excepto que el punto de inicio sea �l mismo $x_0=x_f=1$.

Hay algunos casos en los que  es posible, para determinadas funciones, saber cuando uno de sus puntos fijos es atractivo,

\paragraph{Teorema de existencia y unicidad del punto fijo.}\index{Punto fijo! Teorema} Dada una funci�n $g(x)$,  continua y diferen-\- ciable en un intervalo $[a, b]$, si se cumple que, $\forall x \in [a, b] \Rightarrow g(x)\in [a,b]$,  entonces $g(x)$ tiene un punto fijo en el intervalo $[a, b]$. 

Si adem�s existe una constante positiva $k < 1$  y se  cumple que  la derivada $\vert g'(x) \vert \leq k, \  \forall x \in (a, b)$, entonces el punto fijo contenido en $[a,b]$ es �nico. 

Para demostrar la primera parte del teorema, se puede emplear el teorema de Bolzano. Si se cumple que $g(a)=a$ o que  $g(b)=b$, entonces $a$ o $b$ ser�an el punto fijo. Supongamos que no es as�;  entonces tiene que cumplirse que $g(a)>a$ y que $g(b)<b$. Si construimos una funci�n, $f(x)=g(x)-x$ esta funci�n, que es continua por construcci�n, cumple que $f(a)=g(a)-a>0$ y $f(b)=g(b)-b<0$. Pero entonces, debe existir un punto, $x_f \in [a, b]$ para el cual $f(x_f)=0$ y, por tanto, $f(x_f)=g(x_f)-x_f=0 \Rightarrow g(x_f)=x_f$. Es decir, $x_f$ es un punto fijo de $g(x)$.

La segunda parte del teorema puede demostrarse empleando el teorema de valor medio. Si suponemos  que existen dos puntos fijos distintos $x_{f1} \neq x_{f2}$ en el intervalo $[a,b]$, seg�n el teorema del valor medio, existe un punto $\xi$ comprendido entre $x_{f1}$ y $ x_{f2}$ para el que se cumple,

\begin{equation*}
\frac{g(x_{f1})-g(x_{f2})}{x_{f1}-x_{f2}}=g'(\xi)
\end{equation*}

Por tanto,

\begin{equation*}
\vert g(x_{f1})-g(x_{f2}) \vert =\vert x_{f1}-x_{f2} \vert\cdot \vert g'(\xi) \vert \leq \vert x_{f1}-x_{f2} \vert \cdot k < \vert x_{f1}-x_{f2} \vert 
\end{equation*}

Pero como se trata de puntos fijos $\vert g(x_{f1})-g(x_{f2}) \vert =\vert x_{f1}-x_{f2}\vert $. con lo que llegar�amos al resultado contradictorio, 

 \begin{equation*}
\vert x_{f1}-x_{f2}\vert=\vert g(x_{f1})-g(x_{f2}) \vert  \leq \vert x_{f1}-x_{f2} \vert\cdot k < \vert x_{f1}-x_{f2} \vert 
\end{equation*}

Salvo que, en contra de la hip�tesis inicial, se cumpla que  $ x_{f1}=x_{f2}$. En cuyo caso, solo puede existir un �nico punto fijo en el intervalo $[a, b]$ bajo las condiciones impuestas por el teorema.

\paragraph{Teorema de punto fijo (atractivo).} \footnote{Hay varios teoremas de punto fijo definidos en distintos contextos matem�ticos. Aqu� se da una versi�n reducida a funciones $f(x):\mathbb{R} \rightarrow \mathbb{R}$} Dada una funci�n $g(x)$,  continua y diferenciable en un intervalo $[a, b]$, que  cumple que, $\forall x \in [a, b] \Rightarrow g(x)\in [a,b]$ y que  $\vert g'(x) \vert \leq k, \  \forall x \in (a, b)$, con $0<k<1$, entonces se cumple que, para cualquier punto inicial $x_0$, contenido en el intervalo $[a, b]$, la sucesi�n  $x_{n+1}=g(x_n)$ converge al �nico punto fijo del intervalo $[a, b]$.

La demostraci�n puede obtenerse de nuevo a partir del teorema del valor medio.  Si lo aplicamos al valor inicial $x_0$ y al punto fijo $x_f$, obtenemos,

\begin{equation*}
\vert g(x_0)-g(x_f) \vert =\vert x_0-x_f \vert \cdot \vert g'(\xi) \vert \leq \vert x_0-x_f \vert \cdot k 
\end{equation*}

Para la siguiente iteraci�n tendremos,

\begin{equation*}
\vert g(x_1)-g(x_f) \vert \leq \vert x_1-x_f \vert \cdot k \leq \vert x_0-x_f \vert \cdot k^2 
\end{equation*}

puesto que,  $x_1=g(x_0)$ y $x_f = g(x_f)$, puesto que $x_f$ es el punto fijo. 

Por simple inducci�n tendremos que para el t�rmino en�simo de la sucesi�n,

\begin{equation*}
\vert g(x_n)-g(x_f) \vert \leq \vert x_{n-1}-x_f \vert \cdot k \leq \vert x_{n-2}-x_f \vert \cdot k^2 \leq \cdots \leq  \vert x_0-x_f \vert \cdot k^n 
\end{equation*}

Pero
\begin{equation*}
\underset{n\rightarrow \infty}{\text{lim}}k^n=0 \Rightarrow \underset{n\rightarrow \infty}{\text{lim}} \vert x_n-x_f \vert \leq \underset{n\rightarrow \infty}{\text{lim}}\vert x_0-x_f \vert k^n =0
\end{equation*} 

Es decir, la sucesi�n  $x_{n+1}=g(x_n)$ converge al punto fijo $x_f$.


\paragraph{El m�todo del punto fijo.}\index{Punto fijo! M�todo} Como ya hemos visto, obtener una ra�z de una funci�n $f(x)$, consiste en resolver la ecuaci�n $f(x)=0$. Supongamos que podemos descomponer la funci�n $f(x)$ como la diferencia de dos t�rminos, una funci�n auxiliar, $g(x)$, y la propia variable $x$
\begin{equation*}
f(x)=g(x)-x
\end{equation*}

Encontrar una ra�z de $f(x)$ resulta entonces equivalente a buscar un punto fijo de $g(x)$. 

\begin{equation*}
f(x)=0 \rightarrow g(x)-x=0 \rightarrow g(x)=x
\end{equation*}

En general, a partir de una funci�n dada $f(x)$, es posible encontrar distintas funciones $g(x)$ que cumplan que $f(x)=g(x)-x$. No podemos garantizar que cualquiera de las descomposiciones que hagamos nos genere una funci�n $g(x)$ que tenga un punto fijo. 
Adem�s, para funciones que tengan m�s de una ra�z, puede suceder que distintas descomposiciones de la funci�n converjan a distintas ra�ces.
Si podemos encontrar una que cumpla las condiciones del teorema de punto fijo que acabamos de enunciar, en un entorno de una ra�z de $f(x)$, podemos desarrollar un m�todo que obtenga iterativamente los valores de la sucesi�n   $x_{n+1}=g(x_n)$, a partir de un valor inicial $x_0$.  El resultado se aproximar� al punto fijo de $g(x)$, y por tanto a la ra�z de $f(x)$ tanto como queramos. Bastar�, como   en los m�todos anteriores, definir un valor (tolerancia), por debajo del cual consideramos que el valor obtenido es suficientemente pr�ximo a  la ra�z como para darlo por v�lido. 

La figura \ref{fig:pfijo1} muestra un diagrama de flujo del m�todo del punto fijo. 


\begin{figure}[h]
\centering
\begin{tikzpicture}
%\usetikzlibrary{shapes.geometric}
\path (5,0) node(a) [rectangle,draw=blue, very thick,align=center,rounded corners]{Partimos de un punto inicial $x_0$}
(5,-2) node(b)[rectangle,draw=blue, thick,rounded corners,align=center]{Calculamos\\ $x=g(x_0)$}
(5,-4) node(c)[diamond,aspect=3,draw=red,thick]{es $\vert  x-x_0 \vert \le \text{tol}$?}
(9,-4) node(d)[rectangle,draw=blue,align=center,very thick, rounded corners]{convergencia:\\ terminar}
(5,-6) node(g)[rectangle,draw=blue,thick,rounded corners,align=center]{$x_0=x$};
\draw[blue,-latex](a.south)--(b);
\draw[blue,-latex](b.south)--(c);
\draw[blue,-latex](c.east)--(d);
\draw (7.5,-4)node[above]{S�};
\draw[blue,-latex](c.south)--(g);
\draw (5,-5)node[right]{No};
\draw[blue,-latex](g.south)|-(2,-7)|-(b);

\end{tikzpicture}
\caption{Diagrama de flujo del m�todo del punto fijo. N�tese que la ra�z obtenida corresponde a la funci�n $f(x)=g(x)-x$}
\label{fig:pfijo1}
\end{figure}

 La idea es elegir cuidadosamente el punto inicial $x_0$, para asegurar que se encuentra dentro del intervalo de convergencia del punto fijo.  A continuaci�n, calculamos el valor de $g(x_0)$, el resultado ser� un nuevo valor  $x$ .  Comprobamos la diferencia entre el punto obtenido y el anterior y si es menor que una cierta tolerancia,  consideramos que el m�todo ha convergido, dejamos de iterar, y devolvemos el valor de $x$ obtenido como resultado. Si no, copiamos $x$ en $x_0$ y volvemos a empezar todo el proceso. Es interesante hacer notar que que el algoritmo converge cuando la diferencia entre dos puntos consecutivos es menor que un cierto valor.  De acuerdo con la \emph{condici�n } de punto fijo $g(x_0)=x_0$, dicha distancia, ser�a equivalente a la que media entre $f(x_0)=g(x_0)-x_0$, la funci�n para la que queremos obtener la ra�z,   y $0$.

Veamos un ejemplo. Supongamos  que queremos calcular por el m�todo del punto fijo la ra�z de la  
funci�n $y=e^x-x^2$, que hemos empleado en los ejemplos de los m�todos anteriores.

En primer lugar, debemos obtener a partir de ella una nueva funci�n que cumpla que $f(x)=g(x)-x$. Podemos hacerlo de varias maneras despejando una '$x$', de la ecuaci�n $e^x-x^2=0$.  Para ilustrar los distintos casos de convergencia, despejaremos $x$ de tres maneras distintas .

\begin{equation*}
e^x-x^2=0 \Rightarrow \left\{
\begin{aligned}
x&=\pm  \sqrt{e^x}\\
x&= ln(x^2)=2\cdot ln(\vert x \vert)\\
x&=\frac{e^x}{x} 
\end{aligned} 
\right.
\end{equation*}

\begin{figure}[h]
\includegraphics[width=14cm]{pfijo1.pdf}
\caption{$g(x)=\pm \sqrt{e^x}$, Solo la rama negativa tiene un punto fijo.}
\label{fig:pfijo01}
\end{figure}

En nuestro ejemplo hemos obtenido tres formas distintas de \emph{despejar} la variable $x$. La cuesti�n que surge inmediatamente es, si todas las funciones obtenidas, tienen un punto fijo y, en caso de tenerlo, si es posible alcanzarlo iterativamente.

En el primer caso, $x=\pm \sqrt{e^x}$, obtenemos las dos ramas de la ra�z cuadrada. Cada una de ellas constituye a los efectos de nuestro c�lculo una funci�n distinta. Si las dibujamos junto a la recta $y=x$ (figura \ref{fig:pfijo01}), observamos que solo la rama negativa la corta. Luego ser�  esta rama $g(x)=-\sqrt{e^x}$,  la que podremos utilizar para obtener la ra�z de la funci�n original por el m�todo del punto fijo. La rama positiva, al no cortar a la recta $y=x$ en ning�n punto, es una funci�n que carece de punto fijo.

No es dif�cil demostrar, que la funci�n $g(x)=-\sqrt{e^x} $ cumple las condiciones del teorema de punto fijo descrito m�s arriba para el intervalo $(-\infty, 0]$. Luego el algoritmo del punto fijo deber�a converger para cualquier punto de inicio $x_0$ contenido en dicho intervalo. De hecho, para esta funci�n, el algoritmo converge desde cualquier punto de inicio (Si empezamos en punto positivo, el siguiente punto, $x_1$ ser� negativo, y por tanto estar� dentro del intervalo de convergencia). Esta funci�n es un ejemplo de que el teorema suministra una condici�n suficiente, pero no necesaria para que un punto fijo sea atractivo. 

La figura \ref{fig:pfijo2} muestra un ejemplo del c�lculo de la ra�z de la funci�n $f(x)=e^x-x^2$ empleando la funci�n $g(x)=-\sqrt{e^x}$, para obtener el punto fijo. Se ha tomado como punto de partida $x_0=2.5$, un valor fuera del intervalo en el que se cumple el teorema. Como puede observarse en \ref{fig:pfijo21}. A pesar de ello el algoritmo converge r�pidamente, y tras 5 iteraciones, \ref{fig:pfijo25}, ha alcanzado el punto fijo ---y por tanto la ra�z buscada---, con la tolerancia impuesta
  
\begin{figure}
\centering
\subfigure[valor inicial \label{fig:pfijo21}]{\includegraphics[width=6.6cm]{pfijo3.pdf}} \qquad
\subfigure[iteracion 1]{\includegraphics[width=6.5cm]{pfijo4.pdf}}\\
\subfigure[iteraci�n 2]{\includegraphics[width=6.5cm]{pfijo5.pdf}}\qquad
\subfigure[iteraci�n 3]{\includegraphics[width=6.5cm]{pfijo6.pdf}}\\
\subfigure[iteraci�n 4]{\includegraphics[width=6.5cm]{pfijo7.pdf}}\qquad
\subfigure[iteraci�n 5 \label{fig:pfijo25}]{\includegraphics[width=6.5cm]{pfijo8.pdf}}
\caption{proceso de obtenci�n de la ra�z de la funci�n $f(x)=e^x-x^2$ aplicando el m�todo del punto fijo sobre la funci�n $g(x)=-\sqrt{e^x}$}.
\label{fig:pfijo2}
\end{figure}

Si tratamos de emplear la funci�n $g(x)=ln(x^2)$ para obtener la ra�z, observamos que la funci�n no cumple el teorema para ning�n intervalo que contenga la ra�z. 

La figura \ref{fig:pfijo03} muestra la funci�n $g(x)$, la recta $y=x$ y la evoluci�n del algoritmo tras cuatro evaluaciones. Es f�cil deducir que el algoritmo saltar� de la rama positiva a la negativa y de �sta volver� a saltar de nuevo a la positiva. 

\begin{figure}[h]
\includegraphics[width=14cm]{p1fijo0.pdf}
\caption{primeras iteraciones de la obtenci�n de la ra�z de la funci�n $f(x)=e^x-x^2$ aplicando el m�todo del punto fijo sobre la funci�n $g(x)=ln(x^2)$.}
\label{fig:pfijo03}
\end{figure}

La funci�n presenta una as�ntota vertical en el $0$. Si se empieza desde $x_0=0$, $x_0=1$ 0 $x_0=-1$ el algoritmo no converge, puesto que la funci�n diverge hacia $-\infty$. Para el resto de los valores, la funci�n oscila entre una rama y otra. Si en alguna de las oscilaciones acierta a pasar suficientemente cerca del punto fijo, $x_n-x_{n-} \leq tol$, el algoritmo habr� aproximado la ra�z, aunque propiamente no se puede decir que converja.

 La figura \ref{fig:pfijo41}, muestra la evoluci�n del algoritmo, tomando como punto inicial $x_0=-0.2$.  Tras 211 iteraciones el algoritmo 'atrapa la ra�z'. En este caso la tolerancia se fij� en $tol=0.01$.  
 
 La gr�fica \ref{fig:pfijo42} muestra una ampliaci�n de \ref{fig:pfijo41} en la que pueden observarse en detalles los valores obtenidos para las dos �ltimas iteraciones. Las dos l�neas horizontales de puntos marcan los l�mites $\text{ra�z}\pm tol$. 
 
 El algoritmo se detiene porque la diferencia entre los valores obtenidos en las dos �ltimas iteraciones caen dentro de la tolerancia. El valor obtenido en la pen�ltima iteraci�n, que proviene de la rama positiva de la funci�n $g(x)$ cae muy cerca del punto fijo. El �ltimo valor obtenido, se aleja de hecho del valor de la ra�z, respecto al obtenido en la iteraci�n anterior, pero no lo suficiente como para salirse de los l�mites de la banda marcada por la tolerancia. Como resultado, se cumple la condici�n de terminaci�n y el algoritmo se detiene.  
 
 Si disminuimos el valor de la tolerancia, no podemos garantizar que el algoritmo converja. De hecho, si trazamos cuales habr�an sido los valores siguientes que habr�a tomado la soluci�n del algoritmo, caso de no haberse detenido, es f�cil ver que se alejan cada vez m�s de la ra�z.  De nuevo habr� que esperar a que cambie de rama y vuelva  a pasar otra vez cerca del punto fijo para que haya otra oportunidad de que el algoritmo \emph{atrape} la soluci�n.
 
  La gr�fica \ref{fig:pfijo43} muestra la evoluci�n del error en funci�n del n�mero de iteraci�n. Como puede observarse, el error oscila de forma ca�tica de una iteraci�n a la siguiente. De hecho, el estudio de las sucesiones de la forma $x_{n+1}=g(x_n)$ constituyen uno de los puntos de partidas para la descripci�n y el an�lisis de los llamados sistemas ca�ticos. 

Uno sencillo, pero muy interesante es el de la ecuaci�n log�stica discreta, $x_{n+1}=R\cdot (1-x_n)\cdot x_n$. Esta ecuaci�n muestra un comportamiento muy distinto, seg�n cual sea el valor de $R$ y el valor inicial $x_0$ con el que empecemos a iterar.
 
 

\begin{figure}[h]
\centering
\subfigure[Evoluci�n del algoritmo durante 211 iteraciones \label{fig:pfijo41}]{\includegraphics[width=7cm]{p2fijo1.pdf}} \qquad
\subfigure[Vista detallada de las ultimas iteraciones de \ref{fig:pfijo41} \label{fig:pfijo42}]{\includegraphics[width=7cm]{p2fijo1d2.eps}}\\
\subfigure[Evoluci�n del error \label{fig:pfijo43}]{\includegraphics[width=7.2cm]{p2fijo1e.pdf}}

\caption{proceso de obtenci�n de la ra�z de la funci�n $f(x)=e^x-x^2$ aplicando el m�todo del punto fijo sobre la funci�n $g(x)=ln(x^2)$, el m�todo oscila sin converger a la soluci�n.}
\label{fig:pfijo4}
\end{figure}

Por �ltimo, si empleamos la funci�n $g(x)=\frac{e^x}{x}$, no se cumple el teorema de punto fijo en ning�n punt. En este caso, el algoritmo diverge siempre.  La figura \ref{fig:pfijo5} muestra la evoluci�n del algoritmo del punto fijo para esta funci�n. Se ha elegido un punto de inicio $x_0=-0.745$, muy pr�ximo al valor de la ra�z, para poder observar la divergencia de las soluciones obtenidas con respecto al punto fijo. Como puede verse, el valor de $x_n$ cada vez se aleja m�s de la ra�z. LA soluci�n oscila entre un valor que cada vez se aproxima m�s a cero y otro que tiende hacia $-\infty$. Si se deja aumentar suficientemente el n�mero de iteraciones, llegar� un momento en que se producir� un error de desbordamiento. 

A diferencia de lo que suced�a en la elecci�n de $g(x)=ln(x^2)$, en este caso, el algoritmo no oscila entre las dos ramas. Si empezamos en la rama de la derecha, eligiendo un valor positivo para $x_0$, el algoritmo diverge llevando las soluciones hacia $+\infty$. Es un resultado esperable, ya que dicha rama no tiene ning�n punto fijo.

\begin{figure}
\centering
\subfigure[valor inicial]{\includegraphics[width=7cm]{p3fijo0.pdf}} \qquad
\subfigure[iteracion 1]{\includegraphics[width=7cm]{p3fijo1.pdf}}\\
\subfigure[iteraci�n 2]{\includegraphics[width=7cm]{p3fijo2.pdf}}\qquad
\subfigure[iteraci�n 3]{\includegraphics[width=7cm]{p3fijo3.pdf}}\\
\subfigure[iteraci�n 4]{\includegraphics[width=7cm]{p3fijo4.pdf}}\qquad
\subfigure[iteraci�n 5]{\includegraphics[width=7cm]{p3fijo5.pdf}}

\caption{proceso de obtenci�n de la ra�z de la funci�n $f(x)=e^x-x^2$ aplicando el m�todo del punto fijo sobre la funci�n $g(x)=\frac{e^x}{x}$, el m�todo diverge r�pidamente.}
\label{fig:pfijo5}
\end{figure}

\section{C�lculo de ra�ces de funciones con Matlab.}

Matlab suministra funciones propias para calcular ra�ces de funciones.  Las dividiremos en dos grupos. En primer lugar estudiaremos la funci�n de Matlab \texttt{fzero} y despu�s veremos un conjunto de funciones espec�ficas para manejar polinomios.

\subsection{La funci�n de Matlab \texttt{fzero.}}

La funci�n \texttt{fzero} permite obtener la ra�z de una funci�n cualquiera real $f(x):\mathbb{R} \rightarrow \mathbb{R}$. \texttt{fzero}, es una funci�n especial, ya que opera sobre otras funciones, podemos considerarla como una \emph{funci�n de funciones}. Las funciones ordinarias act�an sobre variables, a lo largo de los cap�tulos anteriores hemos visto como asignar valores y \emph{variables de entrada} a las funciones y tambi�n c�mo guardar los resultados obtenidos de las funciones en \emph{variables de salida}.

Matlab suministra varios mecanismos, para indicar a \texttt{fzero} ---y en general a cualquier \emph{funci�n de funciones}--- la funci�n sobre la que queremos que act�e. Veremos a continuaci�n algunos de los m�s comunes.

\paragraph{\emph{handle} de una funci�n.}\index{"@ |emph{handle} de una funci�n} El primer mecanismo, es asociar a una funci�n un nombre de variable especial. Hasta ahora, siempre hemos empleado las variables para guardar en ellas valores num�ricos o caracteres. Sin embargo Matlab permite guardar tambi�n funciones en una variable. Estas variables, reciben el nombre de \emph{handles}. Veamos un ejemplo. Si escribimos en la ventana de comandos, 

\begin{verbatim}
>> sn =@sin
sn = 
    @sin
\end{verbatim}

Matlab asocia a la nueva variable \texttt{sn} la funci�n seno (\texttt{sin}). Para indicar a Matlab que la nueva variable es el \emph{handle} de una funci�n es imprescindible emplear el s�mbolo @, despu�s del s�mbolo de asignaci�n $=$.

Si pedimos a Matlab que nos muestre qu� variables tiene en el \emph{workspace},

\begin{verbatim}
>> whos
  Name      Size            Bytes  Class          Attributes

  sn      1x1                32  function_handle              

\end{verbatim}

Matlab nos indica que se ha creado una variable \texttt{sn}, cuya clase es \texttt{function\_handle}. Esta variable tiene propiedades muy interesantes: por una parte, podemos manejarla como si se tratara de la funci�n seno, asignando valores de entrada, y guardando el resultado en una variable de salida, 
\begin{verbatim}
>> x=sn(pi/2)
x =
     1
\end{verbatim}

Pero adem�s podemos usarla como variable de entrada para otra funci�n, tal y como se muestra en el siguiente c�digo,

\begin{lstlisting}
function pinta_funcion(fun,intervalo)
% Esta funci�n dibuja la gr�fica de una funcion cualquiera (fun) en un
% itervalo dado (intervalo). fun debe ser un handle de la funci�n que se
% quiere pintar. intervalo debe ser un vector que contenga los extremos del
% intervalo que se desea pintar.

% Construimos cien puntos en el intevalo dado,
x=linspace(intervalo(1),intervalo(2),100);

% calculamos el valor de la funcion en los puntos del intervalo,
y=fun(x);

% dibujamos la gr�fica
plot(x,y)
\end{lstlisting}

La funci�n \texttt{pinta\_funcion} nos dibujar� la gr�fica de cualquier funci�n en el intervalo indicado. p realizar ara ello bastar� crear un \emph{handle} de la funci�n que se quiere dibujar y pasarlo a la funci�n \texttt{pinta\_fun} como una variable de entrada. As� por ejemplo, si escribimos en Matlab,

\begin{verbatim}
>>sn=@sin
>>pinta_funcion(sn,[-pi/2,pi/2])
\end{verbatim}

Se obtendr� la gr�fica de la funci�n seno en el intervalo pedido.

Podemos asignar \emph{handles} no solo a las funciones internas de Matlab sino a cualquier funci�n que escribamos. Por ejemplo, en los m�todos descritos m�s arriba para obtener ra�ces de funciones, usamos la funci�n $f(x)=e^x-x^2$ como funci�n de prueba. podemos crear un fichero que implemente esta funci�n,

\begin{lstlisting}
function y=prueba(x)
% esta es una funcioncilla de prueba para los algoritmos de obtenci�n de
% ra�ces
y=exp(x)-x.^2;
\end{lstlisting}

Si guardamos el fichero, con el nombre prueba.m en el directorio de trabajo, podemos ahora asignar un \emph{handle} a nuestra funci�n,
\begin{verbatim}
mifuncion=@prueba 
\end{verbatim}

Y a continuaci�n, podemos emplear el programa \texttt{pinta\_fun} para representa la funci�n en un intervalo, por ejemplo $[-2 2]$ que contenga la ra�z,

\begin{verbatim}
pinta_funcion(mifuncion,[-2 2])
\end{verbatim} 

El resultado se muestra en la figura \ref{fig:handle}

\begin{figure}[h]
\centering
\includegraphics[width=8cm]{handle.eps}
\caption{Gr�fica de la funci�n $f(x)=e^x-x^2$, obtenida mediante \texttt{pinta\_funcion}.}
\label{fig:handle}
\end{figure}

\paragraph{la funci�n \texttt{feval} de Matlab.} Esta funci�n suministra un m�todo indirecto para calcular los resultados de una funci�n cualquiera.  Su sintaxis es la siguiente,
\begin{verbatim} 
[y1, y2, ..., ym]=feval('fun', x1, x2, ...,xn)
\end{verbatim}

donde \texttt{fun} representa el nombre de la funci�n que se desea evaluar, \texttt{x1, x2,...,xn}, son la variables de entrada empleadas por la funci�n \texttt{fun}, y \texttt{y1, y2, ...,ym} representan sus variables de salida. Es importante destacar que el nombre de la funci�n que se desea evaluar hay que introducirlo entre comillas simples. As� por ejemplo si escribimos,

 \begin{verbatim}
>> y=feval('sin',x)
y =
     1
>> 
\end{verbatim}
obtenemos el mismo resultado que empleando la funci�n \texttt{sin} directamente para calcular  el valor del seno de $\pi/2$,
\begin{verbatim}
>> x=pi/2;
>> y=sin(x)
y =
     1
>> 
\end{verbatim}
 realizar 



\texttt{feval} suministra un m�todo alternativo al uso de \emph{handles} para crear y manejar \emph{funciones de funciones}. Para ver un ejemplo, el siguiente c�digo muestra una versi�n alternativa del programa \texttt{pinta\_funcion}, empleando la funci�n \texttt{feval},

\begin{lstlisting}
function pinta_funcion2(fun,intervalo)
% Esta funci�n dibuja la gr�fica de una funcion cualquiera (fun) en un
% itervalo dado (intervalo). fun debe ser una cadena de caract�res que contengan ex�ctamente el nombre de la funci�n (fun) que se 
% quiere pintar. intervalo debe ser un vector que contenga los extremos del
% intervalo que se desea pintar.

% Construimos cien puntos en el intevalo dado,
x=linspace(intervalo(1),intervalo(2),100);

% calculamos el valor de la funcion en los puntos del intervalo, EMPLEANDO LA FUNCION feval
y=feval(fun,x);

% dibujamos la gr�fica
plot(x,y)
\end{lstlisting}

Para representar la funci�n seno en el intervalo $-\frac{\pi}{2}, \frac{\pi}{2}$, empleando esta nueva funci�n, introducimos en Matlab,

\begin{verbatim}
>> pinta_funcion2('sin',[-pi/2,pi/2])
\end{verbatim}
 realizar 
Tambi�n podemos crear una variable alfa-num�rica con el nombre de la funci�n seno, y pasar directamente la variable creada,

\begin{verbatim}
>> funcion='sin'
>> pinta_funcion2(funcion,[-pi/2,pi/2])
\end{verbatim}

Al igual que en el caso del uso de \texttt{handles} podemos emplear la funci�n \texttt{feval} con funciones creadas por el usuario, por ejemplo podemos representar nuestra funci�n \texttt{prueba}, introducida anteriormente, 

\begin{verbatim}
>>pinta_funcion2('prueba',[-2 2])
\end{verbatim}

El resultado ser�a de nuevo la figura \ref{fig:handle}.  Una �ltima propiedad importante de la funci�n \texttt{feval} es que tambi�n admite que indiquemos la funci�n a evaluar mediante un \emph{handle}. S� volvemos al �ltimo ejemplo, podr�amos haber construido un \emph{handle} para la funci�n \texttt{prueba},
\begin{verbatim}
>>mf=@prueba
>>pinta_funcion2(mf,[-2 2])
\end{verbatim} 

Obtendr�amos una vez m�s el mismo resultado.

\paragraph{Funciones \emph{inline}.} \index{Funciones! \emph{inline}} Las funciones \emph{inlin realizar e} suministran un tercer mecanismo en Matlab para manejar una funci�n de modo que sirva de \emph{variable}  a otra funci�n. Las funciones \emph{inline} tienen una peculiaridad con respecto a las funciones que hemos visto hasta ahora; no se guardan en ficheros .m sino directamente en el \emph{Workspace}. Las funciones \emph{inline} solo existen mientras dura la sesi�n de Matlab en que se crearon, aunque es posible guardarlas en ficheros .mat y volver a cargarlas en Matlab, como se har�a con cualquier otra variable.

Para crear una funci�n \emph{inline} se emplea el comando \texttt{inline}. En su forma m�s sencilla, el comando debe emplear como variable de entrada una expresi�n entre comillas simples que represente la expresi�n matem�tica de la funci�n. Por ejemplo si queremos hacer una versi�n \emph{inline} de la funci�n \texttt{prueba},
\begin{verbatim}
>> fun=inline('exp(x)-x.^2')
fun =
     Inline function:
     fun(x) = exp(x)-x^2
\end{verbatim}

Para calcular el valor de la funci�n en un punto, la funci�n \emph{inline} se maneja de  modo an�logo a cualquier otra funci�n ordinaria.

\begin{verbatim}
>> y=fun(2)
y =
   3.3891
\end{verbatim}

Como en el caso del uso de \texttt{handles}, la variable creada mediante una funci�n \emph{inline} realizar , hace referencia a una funci�n y puede ser empleada como variable  de entrada por otras funciones. Por ejemplo, podr�amos emplear directamente nuestra primera versi�n del programa para pintar funciones, \texttt{pitan\_funcion} para obtener la gr�fica de nuestra funci�n de prueba $f(x)=e^x-x^2$,

\begin{verbatim}
>>funcion=inline('exp(x)-x.^2')
>>pinta_funcion(funcion,[-2 2])
\end{verbatim}

Una vez que hemos visto distintos m�todos para manejar una funci�n como variable de entrada de otra funci�n, volvamos a la funci�n \texttt{fzero}. En su forma m�s sencilla, \texttt{fzero} admite como variable de entrada, una funci�n expresada mediante un \texttt{handle}, mediante su nombre escrito entrecomillas o bien construida como funci�n \emph{inline}. Adem�s es preciso introducir una segunda variable que puede ser un punto $x_0$ pr�ximo a la ra�z de la funci�n o bien un vector $[a b]$ que defina un intervalo que contenga una ra�z. La funci�n \texttt{fzero}, devuelve como variable de salida el valor aproximado de la ra�z. Si \texttt{fzero} no es capaz de encontrar la ra�z de la funci�n, devolver� NaN. Veamos un ejemplo con la funci�n contenida en el fichero, prueba.m, descrito m�s arriba,

\begin{enumerate}
\item Empleando un \emph{handle} y un punto pr�ximo a la ra�z,
\begin{verbatim}
>> hndl=@prueba
hndl = 
    @prueba
>> raiz=fzero(hndl,2)
raiz =
  -0.703467422498392
\end{verbatim}
\item Empleando un \emph{handle} y un intervalo que contenga la ra�z,
\begin{verbatim}
>> raiz=fzero(hndl,[-2 2])
raiz =
  -0.703467422498392
\end{verbatim}

\item Empleando el nombre de la funci�n entre comillas y un punto cercano a la ra�z,
\begin{verbatim}
>> raiz=fzero('prueba',2)
raiz =
  -0.703467422498392
\end{verbatim}

\item  Empleando el nombre de la funci�n entre comillas y un intervalo que contenga la ra�z,
\begin{verbatim}
>> raiz=fzero('prueba',[-2 2])
raiz =
  -0.703467422498392
\end{verbatim}

\item Usando una funci�n \emph{inline} y un punto cercano a la ra�z,
\begin{verbatim} realizar 
>> finl=inline('exp(x)-x.^2')
finl =
     Inline function:
     finl(x) = exp(x)-x.^2

>> raiz=fzero(finl,2)
raiz =
  -0.703467422498392
\end{verbatim}

\item Usando una funci�n \emph{inline} y un intervalo que contenga la ra�z, 
\begin{verbatim}

>> raiz=fzero(finl,[-2 2])
raiz =
  -0.703467422498392
\end{verbatim}
\end{enumerate} 

La funci�n \texttt{fzero}, tiene muchas posibilidades de ajuste de la precisi�n, del m�todo que emplea internamente para buscar la ra�z, etc. Para obtener una visi�n mas completa de su uso, consultar la ayuda de Matlab.

\subsection{C�lculo de ra�ces de polinomios.} \index{Polinomios}

Matlab tiene un conjunto de funciones especialmente pensadas para  manejar polinomios. En primer lugar, en Matlab es habitual representar los polinomios mediante vectores cuyos elementos, son los coeficientes del polinomio ordenados de mayor a menor grado. As� por ejemplo, el polinomio. $y=2x^3+3x^2+4x+1$ se representa mediante el vector, $p1=[2\ 3\ 4\ 1]$,  el polinomio $y=3x^4+2x^2+6x$ se representa mediante el vector,  $p2=[3\ 0\ 2\ 6\ 0]$ y, en general, el polinomio $y(x)=a_nx^n+a_{n-1}x^{n-1}+\cdots+a_2x^2+a_1x+a_0$  se representa mediante el vector $p=[a_n\ a_{n-1}\ \cdots\ a_2\ a_1\ a_0]$. Si al polinomio le falta alg�n o algunos t�rminos, el elemento correspondiente toma el valor $0$ en el vector que representa el polinomio.

Veamos a continuaci�n un conjunto de funciones de Matlab, especialmente pensadas para manejar polinomios,

\paragraph{La funci�n \texttt{roots}.} \index{Polinomios!Ra�ces de un polinomio}Esta funci�n calcula las ra�ces de un polinomio de grado $n$ a partir de los coeficientes del polinomio, contenidos en un vector como los que acabamos de describir. La sintaxis es: \texttt{raices=roots([vector de coeficientes])}. veamos un ejemplo. Dado el polinomio $y(x)=x^3-6^2+11x-6$ lo expresar�amos en Matlab como,

\begin{verbatim}
>> p=[1 -6 11 -6]
\end{verbatim}

Y obtendr�amos sus ra�ces como,

\begin{verbatim}
>> raices=roots(p)
raices =
    3.0000
    2.0000
    1.0000

\end{verbatim}

Matlab devuelve todas las ra�ces del polinomio en un �nico vector, tanto las reales como las complejas. Como por ejemplo en el caso del polinomio $y(x)=x^2+2x+1$

\begin{verbatim}
>> p=[1 2 3]
p =
     1     2     3
>> raices=roots(p)
raices =
  -1.0000 + 1.4142i
  -1.0000 - 1.4142i
\end{verbatim}

\paragraph{la funci�n \texttt{poly}.} Esta funci�n podr�a considerarse la opuesta a la anterior; dado un vector que contiene las ra�ces de un polinomio, nos devuelve los coeficientes del polinomio correspondiente, Por ejemplo si definimos el vector de ra�ces, 
\begin{verbatim}
>>raices=[3 2 1]
\end{verbatim}
podemos obtener los coeficientes del polinomio que posee esas ra�ces como,
\begin{verbatim}
>> raices=[1 2 3]
raices =
     1     2     3
>> coef=poly(raices)
coef =
     1    -6    11    -6
\end{verbatim}

Es decir, las ra�ces pertenecen al polinomio, $y(x)=x^3-6x^2+11x-6$.

\paragraph{la funci�n \texttt{polyval}.}\index{Polinomios!valor de un polinomio en un punto} Esta funci�n calcula el valor de un polinomio en un punto.  Para ello es preciso darle un vector con los coeficientes del polinomio ---definido igual que en los casos anteriores--- y un segundo vector con los puntos para los que se quiere calcular el valor del polinomio,

\begin{verbatim}
>> coef=[1 2 3 4]
coef =
     1     2     3     4
>> x=2
x =
     2
>>y= polyval(coef,x)
y =
    26
>> x=[1:10]
x =
     1     2     3     4     5     6     7     8     9    10
>> y=polyval(coef,x)
y =
  Columns 1 through 6
          10          26          58         112         194         310
  Columns 7 through 10
         466         668         922        1234
\end{verbatim}  
 

En este ejemplo se ha obtenido con \texttt{polyval} el valor del polinomio $y(x)=x^3+2x^2+2x+4$ primero para el punto $x=2$ y despu�s para los puntos $x=[1\ 2\ 3\ 4\ 5\ 6\ 7\ 8\ 9\ 10]$.

\paragraph{La funci�n \texttt{conv}.} \index{Polinomios!Producto de polinomios}Calcula el producto de dos polinomios. Cada polinomio se representa mediante un vector de coeficientes y la funci�n \texttt{conv} devuelve un vector con los coeficientes del polinomio producto. Por ejemplo, si multiplicamos   el polinomio $y_1=x+2$ por el polinomio $y_2=x-1$  obtendremos como resultado, $p=y_1\cdot y_2=x^2+x-2$,  

\begin{verbatim}
>> y1=[1 2]
y1 =
     1     2
>> y2=[1 -1]
y2 =
     1    -1
>> p=conv(y1,y2)
p =
     1     1    -2
\end{verbatim}

\section{Ejercicios}
\begin{enumerate}
\item Crea una funci�n en matlab que implemente el m�todo de  la bisecci�n para obtener la ra�z de una funci�n cualquiera $y = f(x)\vert\  x, y \in \mathbb{R}$. La funci�n deber� admitir como variables de entrada, la funci�n $f(x)$, un intervalo de b�squeda $[a,b]$, un n�mero m�ximo de iteraciones, $nmax$, a realizar y la tolerancia $tol$ o error m�ximo admisible para la soluci�n obtenida. As� mismo la funci�n deber� devolver como como variables de salida, el valor aproximado obtenido para la ra�z $r$, el error cometido $f(r)$ y el n�mero de iteraciones empleado para alcanzar la soluci�n.

Aplica el programa creado a la funci�n de ejemplo empleada empleada en el manual, $f(x) = e^x-x^2$, para comprobar que funciona correctamente.

\item Crea una funci�n en matlab que implemente el m�todo de  interpolaci�n lineal para obtener la ra�z de una funci�n cualquiera $y = f(x)\vert\  x, y \in \mathbb{R}$. La funci�n deber� admitir como variables de entrada, la funci�n $f(x)$, un intervalo de b�squeda $[a,b]$, un n�mero m�ximo de iteraciones, $nmax$, a realizar y la tolerancia $tol$ o error m�ximo admisible para la soluci�n obtenida. As� mismo la funci�n deber� devolver como como variables de salida, el valor aproximado obtenido para la ra�z $r$, el error cometido $f(r)$ y el n�mero de iteraciones empleado para alcanzar la soluci�n.

Aplica el programa creado a la funci�n de ejemplo empleada empleada en el manual, $f(x) = e^x-x^2$, para comprobar que funciona correctamente.

\item Crea una funci�n en matlab que implemente el m�todo de  Newton-Raphson para obtener la ra�z de una funci�n cualquiera $y = f(x)\vert\  x, y \in \mathbb{R}$. La funci�n deber� admitir como variables de entrada, la funci�n $f(x)$ y su funci�n derivada $f'(x)$, un punto inicial de b�squeda $x_0$, un numero m�ximo de iteraciones, $nmax$, a realizar y la tolerancia $tol$ o error m�ximo admisible para la soluci�n obtenida. As� mismo la funci�n deber� devolver como como variables de salida, el valor aproximado obtenido para la ra�z $r$, el error cometido $f(r)$ y el n�mero de iteraciones empleado para alcanzar la soluci�n.

Aplica el programa creado a la funci�n de ejemplo empleada empleada en el manual, $f(x) = e^x-x^2$, para comprobar que funciona correctamente.

\item Crea una funci�n en matlab que implemente el m�todo de la secante para obtener la ra�z de una funci�n cualquiera $y = f(x)\vert\  x, y \in \mathbb{R}$. La funci�n deber� admitir como variables de entrada, la funci�n $f(x)$ y su funci�n derivada $f'(x)$, dos punto iniciales de b�squeda $x_0,\ x_1$, un numero m�ximo de iteraciones, $nmax$, a realizar y la tolerancia $tol$ o error m�ximo admisible para la soluci�n obtenida. As� mismo la funci�n deber� devolver como como variables de salida, el valor aproximado obtenido para la ra�z $r$, el error cometido $f(r)$ y el n�mero de iteraciones empleado para alcanzar la soluci�n.

Aplica el programa creado a la funci�n de ejemplo empleada empleada en el manual, $f(x) = e^x-x^2$, para comprobar que funciona correctamente.
\item Crea una funci�n en matlab que implemente el m�todo del punto fijo para obtener la ra�z de una funci�n cualquiera $y = f(x)\vert\  x, y \in \mathbb{R}$. La funci�n deber� admitir como variables de entrada, la funci�n auxiliar $g(x)$, necesaria para aplicar el m�todo, un punto inicial de b�squeda $x_0$, un numero m�ximo de iteraciones, $nmax$, a realizar y la tolerancia $tol$ o error m�ximo admisible para la soluci�n obtenida. As� mismo la funci�n deber� devolver como como variables de salida, el valor aproximado obtenido para la ra�z $r$, el error cometido $f(r)$ y el n�mero de iteraciones empleado para alcanzar la soluci�n.

Aplica el programa creado a la funci�n de ejemplo empleada empleada en el manual, $f(x) = e^x-x^2$, empleando para ello los tres ejemplos de funciones auxiliare $g(x)$ propuestos en la secci�n \ref{pfijo}. Comprueba que se cumple en cada caso lo expuesto en dicha secci�n.

\item Se quiere aproximar el valor de $\pi$ utilizando las dos siguientes funciones:
\begin{equation*}
y=\cos(x)+1
\end{equation*}
\begin{equation*}
y=\cos(x/2)
\end{equation*}
\begin{enumerate}
\alph{enumii}
\item \label{it1}Dibuja las dos funciones y sus derivadas en el intervalo $[\frac{\pi}{2},\frac{3\pi}{2}]$.
\item Aplica el m�todo de la secante para cada funci�n, tomando como punto de partida  $x_0=3$ y obteniendo en ambos caso la soluci�n con una tolerancia de 0.01. Elige para $x_1$ un valor que consideres adecuado.\\
El programa utilizado deber� a�adir al gr�fico obtenido en  \ref{it1}), un punto en la posici�n $(x_i,y_i)$ que represente el valor obtenido de la ra�z en cada iteraci�n $i$.
\item Compara el n�mero de iteraciones que se necesitan en cada caso para obtener la soluci�n.
\end{enumerate}
\item El m�todo de Steffensen, permite disminuir el n�mero de iteraciones empleadas para obtener una ra�z empleando el m�todo del punto fijo.

El algoritmo calcula cada iteraci�n usando dos pasos intermedios de acuerdo con el siguiente procedimiento,

\begin{lstlisting}
while{condici�n} 
  x0 = x
  x1 = g(x0)
  x2 = g(x1)
  x  = x0 - (x1 -x0)^2/(x2 - 2x1 + x0)
end
\end{lstlisting}

Donde $g(x)$ representa la funci�n auxiliar de la que se busca el punto fijo.


Dada la funci�n,

\begin{equation*}
y= x^2-2x-3
\end{equation*} 

se le puede aplicar el m�todo del punto fijo a  partir de dos reordenaciones distintas:
\begin{equation}
x^2-2x-3=0 \Rightarrow x=\sqrt{2x+3}
\end{equation}
y
\begin{equation}
x^2-2x-3=0 \Rightarrow x=\frac{3}{x-2}
\end{equation}
\begin{enumerate}
\item Aplica el m�todo del punto fijo, tomando como punto de partida $x_0=4$, mediante ambas reordenaciones. Calcula en ambos casos la ra�z con una tolerancia de $0.0001$. Indica en cada caso: el valor de la ra�z y el n�mero de interaciones empleadas.  �Son razonables los resultados?

\item Repite el calculo empleando ahora el m�todo de Steffensen y comprueba si emplea o no menos iteraciones que el punto fijo.
\end{enumerate}

\item Un objeto se mueve de acuerdo con la siguiente ley:
\begin{equation}
x_1 = 3t^6 - 2t^5 + t + 25
\end{equation}
Donde  $t$ representa el tiempo en segundos y $x_1$ la posici�n del objeto en metros y \\
 Un segundo objeto sale en persecuci�n del primero en el instante de tiempo $t=0$. Su ley del movimiento es:
\begin{equation}
x_2 = 4t^6
\end{equation}
\begin{enumerate}
\item Dibuja en un mismo gr�fica la posici�n en funci�n del tiempo para los dos m�viles, de modo que se observe el punto de alcance. (1 punto) 
\item Calcula, empleando el m�todo de la bisecci�n o el de \emph{regula falsi},  el instante de tiempo y la posici�n en que el segundo m�vil alcanza al primero (2 puntos)
\end{enumerate}

\item La distribuci�n de Maxwell, 
\begin{equation}
f(v) = \sqrt{\frac{2}{\pi}}\left( \frac{m}{kT}\right)^{3/2}v^2e^{-\dfrac{\scriptstyle mv^2}{\scriptstyle 2kT}},
\end{equation}
da la probabilidad de que una part�cula de un gas se mueva con velocidad $v$. Donde $m$ es la masa de la part�cula en Kg, $T$ la temperatura en grados Kelving del gas y k $=1.380649\times 10^{-23} \text{JK}^{-1}$ la constante de Boltzmann. 
\begin{enumerate}
\item  \label{eje91}Dibuja una gr�fica de la distribuci�n de Maxwell a temperatura ambiente T $= 300$K para el Nitr�geno N$_2$  cuya masa molecular vale $m = 4.65 \times 10^{-26}	\text{Kg}$.
\item \label{eje92}Calcula a dicha temperatura cual es el valor m�s probable de la velocidad de una part�cula.
\item Calcula cu�nto hay que bajar la temperatura para que la probabilidad de encontrar una part�cula con la velocidad calculada en el apartado \ref{eje92} valga 0.001. Dibuja sobre la misma gr�fica del ejercicio \ref{eje91} la distribuci�n de Maxwell del Nitr�geno para esta nueva temperatura. 

\end{enumerate}

\end{enumerate}

\section{Test del curso 2020/2021}
\begin{enumerate}
\item Una se�al de FM, viene representada por la funci�n
\begin{equation}
	x(t) = 2\cos\left(\pi \, t+ \frac{3}{2}\sin\left(\frac{\pi}{2} \, t\right)\right).
\end{equation}
\begin{enumerate}
	\item \textbf{1 punto. }Dibuja la funci�n en el intervalo $[0,\pi]$ Emplea para ello valores equiespaciados $0.01\pi$.
	\item \textbf{3 puntos. }Calcula los tres primeros instantes de tiempo en los que la se�al vale cero, empleando para ello el m�todo de la bisecci�n. Elige en cada caso los intervalos adecuados. Obt�n los resultados empleando una tolerancia $tol=0.001$ e indica el n�mero de iteraciones empleadas.
\end{enumerate}

\item  Han Solo ha situado en reposo al Halc�n Milenario en un punto del universo alejado de toda galaxia. Tras dejar a Chewbacca en una nave auxiliar, Han pone en marcha los motores (modelo \emph{Girodyne SRB42 Sublight Engine Thrusters}) y el Halc�n acelera con una aceleraci�n uniforme $a_0 = 3\times10^3\ m/s^2$. Chewabaca --que no estudi� f�sicas--  desconoce la Teor�a de la Relatividad, y por lo tanto piensa que puede calcular la posici�n $x_{cl}$ con respecto del Halc�n empleando la funci�n cl�sica del movimiento rectil�neo uniformemente acelerado
\begin{equation}
	x_{cl}(t) =\frac{1}{2}a_0t^2.
\end{equation}
		En realidad el Halc�n no puede acelerar indefinidamente (sobrepasar�a la velocidad de la luz, cosa imposible con los \emph{Sublight Engine Thrusters}), por lo que la posici�n real $x_{rel}$ a la que se encuentra viene expresada por
\begin{equation}
	x_{rel}(t) = \frac{c^2}{a_0}\sqrt{1 + \frac{a_0^2\,t^2}{c^2}} -\frac{c^2}{a_0}, 
\end{equation}
		donde $c \approx 3\times 10^8 m/s$ representa la velocidad de la luz en el vac�o. 

		En los siguientes pasos vamos a calcular cuanto tiempo transcurre hasta que Chewbacca comete un error de un miliparsec\footnote{$1\,\text{Parsec} \approx 3\times 10^{16}m$.} en el c�lculo de la posici�n del Halc�n.

\begin{enumerate}
	\item {\bf 2 puntos.} Define la funci�n de error $e(t) = x_{cl}(t) - x_{rel}(t)$ y simplif�cala con los datos num�ricos dados. Acto seguido, reescribe la funci�n error como $e(t) = t - g(t)$ para poder aplicar m�s adelante el algoritmo del punto fijo. Con una sola descomposici�n de $e(t)$ es suficiente. De hecho, qu�date �nicamente con aquella que tenga una $g(t)$ m�s simple seg�n tu criterio.

	\item {\bf 1.5 puntos.} Dibuja en una misma gr�fica la funci�n $y = g(t)$ y la bisectriz del primer cuadrante $y = t$, y comprueba que $g(t)$ tiene efectivamente un punto fijo. Para la gr�fica escoge el intervalo de tiempo $[0, \quad 3 \times 10^5]$ segundos con intervalos de $1\times 10^4$ segundos.
	\item {\bf 2.5 puntos.} Emplea el m�todo del punto fijo para obtener el tiempo en el que el error en posici�n alcanza un valor igual a un miliParsec. Da el resultado en d�as terrestres. Toma como valor inicial $t_0 = 1$ y da el resultado con una tolerancia $tol = 0.1$. Indica el n�mero de iteraciones que ha empleado el m�todo para obtener el resultado.
	\item {\bf Bonus track: 1 punto.} Un Parsec equivale a la distancia desde el Sol a una estrella que, vista desde la Tierra, subtiende un �ngulo de paralaje de 1 segundo de grado. Da dos razones por las que el Halc�n Milenario nunca pudo atravesar el corredor Kessel en menos de 12 Parsecs, como afirma Han Solo.

\end{enumerate}
\end{enumerate}

\chapter{Aplicaciones del c�lculo cient�fico al �lgebra lineal}
\section{Matrices y vectores}
En esta secci�n vamos a repasar algunos conceptos fundamentales de  �lgebra lineal y c�mo pueden manejarse empleando Matlab. No daremos definiciones precisas ni tampoco demostraciones, ya que tanto unas como otras se ver�n en detalle en la asignatura de �lgebra. 
\paragraph{matrices.} Desde un punto de vista funcional definiremos una matriz como una tabla bidimensional de n�meros ordenados en filas y columnas,
\begin{equation*}
A=
\begin{pmatrix}
1& \sqrt(2)& 3.5& 0\\
-2& \pi& -4.6& 4\\
7& -19& 2.8& 0.6
\end{pmatrix}
\end{equation*}

Cada l�nea horizontal de n�meros constituye una \emph{fila} de la matriz y cada l�nea horizontal una \emph{columna} de la misma.

A una matriz con $m$ filas y $n$ columnas se la denomina matriz de orden $m\times n$. $m$ y $n$ son la dimensiones de la matriz y se dan siempre en el mismo orden: primero el n�mero de filas y despu�s el de columnas. As�, la matriz $A$ del ejemplo anterior es una matriz $3\times 4$. El orden de una matriz expresa el tama�o de la matriz.

Dos matrices son iguales si tienen el mismo orden, y los elementos que ocupan en ambas matrices los mismo lugares son iguales.

Una matriz es cuadrada, si tiene el mismo n�mero de filas que de columnas. Es decir es de orden $n\times n$.

Mientras no se diga expresamente lo contrario, emplearemos letras may�sculas $A, B, \cdots$ para representar matrices. La expresi�n $A_{m\times n}$ indica que la matriz $A$ tiene dimensiones $m \times n$. Para denotar los elementos de una matriz, emplearemos la misma letra en min�sculas empleada para nombrar la matriz, indicando mediante sub�ndices, y siempre por este orden, la fila y la columna a la que pertenece el elemento. As� por ejemplo $a_{ij}$ representa al elemento de la matriz $A$, que ocupa la fila $i$ y la columna $j$.

\begin{equation*}
A=
\begin{pmatrix}
1& \sqrt(2)& 3.5& 0\\
-2& \pi& -4.6& 4\\
7& -19& 2.8& 0.6
\end{pmatrix}
\rightarrow a_{23}=-4.6
\end{equation*}

\paragraph{vectores}
A una matriz compuesta por una sola fila, la denominaremos vector fila. A una matriz compuesta por una sola columna la denominaremos vector columna. Siempre que hablemos de un vector, sin especificar m�s, entenderemos que se trata de un vector columna.\footnote{Esta identificaci�n de los vectores como vectores columna no es general. La introducimos porque simplifica las explicaciones posteriores.} Para representar vectores, emplearemos letras min�sculas. Para representar sus elementos a�adiremos a la letra que representa al vector un sub�ndice indicando la fila a la que pertenece el elemento.

\begin{equation*}
a=
\begin{pmatrix}
a_1\\
a_2\\
\vdots \\
a_i\\
\vdots \\
a_n
\end{pmatrix}
\end{equation*}

Podemos asociar los puntos del plano con los vectores de dimensi�n dos. Para ello, usamos una representaci�n cartesiana, en la que los elementos del vector son los valores de las coordenadas $(x,y)$ del punto del plano que representan. Cada vector se representa gr�ficamente mediante una flecha que parte del origen de coordenadas y termina en el punto $(x,y)$ representado por el vector. La figura \ref{fig:vectores} representa los vectores,
\begin{equation*}
a=
\begin{pmatrix}
1\\
2
\end{pmatrix},
b=
\begin{pmatrix}
2\\
-3
\end{pmatrix},
c=
\begin{pmatrix}
0\\
-2
\end{pmatrix}
\end{equation*}


\begin{figure}[h]
\centering
\includegraphics[width=7cm]{vectores.eps}
\caption{Representaci�n gr�fica de vectores en el plano}
\label{fig:vectores}
\end{figure}

De modo an�logo, podemos asociar vectores de dimensi�n tres con puntos en el espacio tridimensional. En este caso, los valores de los elementos del vector corresponden con la coordenadas $(x,y,z)$ de los puntos en el espacio. La figura \ref{fig:vectores3} muestra la representaci�n gr�fica en espacio tridimensional de los vectores, 
\begin{equation*}
a=
\begin{pmatrix}
1\\
2\\
1
\end{pmatrix},
b=
\begin{pmatrix}
2\\
-3\\
-1
\end{pmatrix},
c=
\begin{pmatrix}
0\\
-2\\
1
\end{pmatrix}
\end{equation*}

\begin{figure}[h]
\centering
\includegraphics[width=9cm]{vectores3.eps}
\caption{Representaci�n gr�fica de vectores en el espacio}
\label{fig:vectores3}
\end{figure}

Evidentemente para vectores de mayor dimensi�n, no es posible obtener una representaci�n gr�fica. Si embargo muchas de las propiedades geom�tricas, observables en los vectores bi y tridimensionales, pueden extrapolarse a vectores de cualquier dimensi�n.



\section{Operaciones matriciales}\label{opmatr}

A continuaci�n definiremos las operaciones matem�ticas m�s comunes, definidas sobre matrices. 

\paragraph{suma.} La suma de dos matrices, se define como la matriz resultante de sumar los elementos que ocupan en ambas la misma posici�n. Solo est� definida para matrices del mismo orden,

\begin{gather*}
C=A+B\\
c_{ij}=a_{ij}+b_{ij}\\
\\
\begin{pmatrix}
1& 2& 3\\
4& 5& 6\\
7& 8& 9\\
\end{pmatrix} =
\begin{pmatrix}
1& 3& 5\\
3& 5& 7\\
5& 7& 9\\
\end{pmatrix} +
\begin{pmatrix}
0& -1& -2\\
1& 0& -1\\
2& 1& 0\\
\end{pmatrix}
\end{gather*}

La suma de matrices cumple las siguientes propiedades,
\begin{enumerate}
\item Asociativa: $(A+B)+C=A+(B+C)$
\item Conmutativa: $A+B=B+A$
\item Elemento neutro: $O_{n\times m}+A_{n\times m}=A_{m\times m}$ El elemento neutro $O_{n\times m}$ de la suma de matrices de orden $n\times m$ es la matriz nula de dicho orden, ---compuesta exclusivamente por ceros--- . 

En Matlab, podemos crear una matriz de cualquier orden, compuesta exclusivamente por ceros mediante el comando \texttt{zeros(m,n)}, donde $m$ es el n�mero de filas y $n$ el de columnas de la matriz de ceros resultante,

\begin{verbatim}
>> O=zeros(2,3)
O =
     0     0     0
     0     0     0
>> A=[1 2 3; 4 3 6]
A =
     1     2     3
     4     3     6
>> B=A+O
B =
     1     2     3
     4     3     6
>> 
\end{verbatim}
\item Elemento opuesto: La opuesta a una matriz se obtiene cambiando de signo todos sus elementos, $A_{op}=-A$

\begin{verbatim}
>> A
A =
     1     2     3
     4     3     6
>> Aop=-A
Aop =
    -1    -2    -3
    -4    -3    -6
>> S=A+Aop
S =
     0     0     0
     0     0     0
\end{verbatim}
\end{enumerate}
En Matlab el signo $+$ representa por defecto la suma de matrices, por lo que la suma de dos matrices puede obtenerse directamente como, 

\begin{verbatim}
>> A=[1 2 3; 4 3 6]
A =
     1     2     3
     4     3     6
>> B=[1 2 3; 4 -3 2]
B =
     1     2     3
     4    -3     2
>> S=A+B
S =
     2     4     6
     8     0     8
\end{verbatim}



\paragraph{Transposici�n}
Dada una matriz $A$, su transpuesta $A^T$ se define como la matriz que se obtiene intercambiando sus filas con sus columnas,

\begin{gather*}
A \rightarrow  A^T\\
a_{ij} \rightarrow  a_{ji}\\
A=
\begin{pmatrix}
1& -3& 2 \\
2& 7& -1
\end{pmatrix}  \rightarrow 
A^T=
\begin{pmatrix}
1& 2 \\
-3& 7\\
2 & -1
\end{pmatrix}
\end{gather*}

En Matlab, la operaci�n de transposici�n se indica mediante un ap�strofo ('),

\begin{verbatim}
>> A=[1 2 3; 4 3 6]
A =
     1     2     3
     4     3     6
>> B=A'
B =
     1     4
     2     3
     3     6
\end{verbatim}

Para vectores, la transposici�n convierte un vector fila en un vector columna y viceversa. 
\begin{gather*}
a \rightarrow a^T\\
a=
\begin{pmatrix}
1 \\
-3\\
2 
\end{pmatrix}
 \rightarrow 
a^T=
\begin{pmatrix}
1& -3 & 2 
\end{pmatrix} 
\end{gather*}

Una matriz cuadrada se dice que es sim�trica si coincide con su transpuesta,

\begin{gather*}
A=A^T\\
a_{ij}=a{ji}\\
A=A^T=
\begin{pmatrix}
\ 1&\ 3&-3\\
\ 3&\ 0&-2\\
-3&-2&\ 4
\end{pmatrix}
\end{gather*}

Una matriz cuadrada es antisim�trica cuando cumple que $A=-A^T$. Cualquier matriz cuadrada se puede descomponer en la suma de una matriz sim�trica m�s otra antisim�trica.

La parte sim�trica puede definirse como,
\begin{equation*}
A_S=\frac{1}{2} \left( A+A^T \right)
\end{equation*}

y la parte antisim�trica como,

\begin{equation*}
A_A=\frac{1}{2}\left( A-A^T \right)
\end{equation*}

As�, por ejemplo,

\begin{equation*}
 A=A_S+A_A \rightarrow
\begin{pmatrix}
1& 2& 3\\
4& 5& 6\\
7& 8& 9\\
\end{pmatrix} =
\begin{pmatrix}
1& 3& 5\\
3& 5& 7\\
5& 7& 9\\
\end{pmatrix} +
\begin{pmatrix}
0& -1& -2\\
1& 0& -1\\
2& 1& 0\\
\end{pmatrix}
\end{equation*}

Por �ltimo, la transpuesta de la suma de matrices cumple,
\begin{equation*}
(A+B)^T=A^T+B^T
\end{equation*}


\paragraph{Producto de una matriz por un escalar.} El producto de una matriz $A$ por un n�mero $b$ es una matriz del mismo orden que $A$, cuyos elementos se obtienen multiplicando los elementos de $A$ por el n�mero $b$,

\begin{gather*}
C=b\cdot A \rightarrow c_{ij}=b\cdot a_{ij}\\
3\cdot
\begin{pmatrix}
1& -2& 0\\
2& 3& -1&
\end{pmatrix}=
\begin{pmatrix}
3& -6& 0\\
6& 9& -3&
\end{pmatrix} 
\end{gather*}
En Matlab, el s�mbolo \texttt{*} se emplea para representar el producto entre escalares (n�meros), entre escalares y matrices y el producto entre matrices, como veremos en los siguientes p�rrafos.

\paragraph{Producto escalar de dos vectores.} Dados vectores de la misma dimensi�n $m$ se define su producto escalar como, 

\begin{gather*}
a\cdot b=\sum_{i=1}^na_ib_i\\
\begin{pmatrix}
1\\
3\\
4
\end{pmatrix}\cdot
\begin{pmatrix}
1\\
-2\\
0
\end{pmatrix}
=1\cdot 1+3 \cdot (-2)+ 4 \cdot 0= -5
\end{gather*}

El resultado de producto escalar de dos vectores, es siempre un n�mero; se multiplican los entre s� los elementos de los vectores que ocupan id�nticas posiciones y se suman los productos resultantes. 


\paragraph{Producto matricial}
El producto de una matriz de orden $n\times m$ por una matriz $m\times l$, es una nueva matriz de orden $n\times l$, cuyos elementos se obtiene de acuerdo con la siguiente expresi�n,

\begin{equation*}
P=A\cdot B \rightarrow a_{ij}=\sum_{t=1}^m a_{it}b_{tj}
\end{equation*}

Por tanto, el elemento de la matriz producto que ocupa la fila i y la columna j, se obtiene multiplicando por orden los elementos de la fila i de la matriz A con los elementos correspondientes de la columna j de la matriz B, y sumando los productos resultantes.

Para que dos matrices puedan multiplicarse es imprescindible que el n�mero de columnas de la primera matriz coincida con el n�mero de filas de la segunda.

Podemos entender la mec�nica del producto de matrices de una manera m�s f�cil si consideramos  la primera matriz como un grupo de vectores fila,
\begin{equation*}
\begin{aligned}
A_1=\begin{pmatrix}
a_{11}& a_{12}& \cdots a_{1n}
\end{pmatrix}\\
A_2=\begin{pmatrix}
a_{21}& a_{22}& \cdots a_{2n}
\end{pmatrix}\\
\vdots \  \ \   \  \  \  \ \ \ \ \\
A_m=\begin{pmatrix}
a_{m1}& a_{m2}& \cdots a_{mn}
\end{pmatrix}
\end{aligned} \ \rightarrow \ 
A=\begin{pmatrix}
a_{11}& a_{12}& \cdots a_{1n}\\
a_{21}& a_{22}& \cdots a_{2n}\\
\vdots& \vdots& \cdots& \vdots \\
a_{m1}& a_{m2}& \cdots a_{mn}
\end{pmatrix}
\end{equation*}
y la segunda matriz como un grupo de vectores columna,

\begin{equation*}
\begin{aligned}
B_1=\begin{pmatrix}
b_{11}\\ b_{21}\\ \vdots \\ b_{n1}
\end{pmatrix}&
B_2=\begin{pmatrix}
b_{12}\\ b_{22}\\ \vdots\\ b_{n2}
\end{pmatrix} &
\cdots  \  \  &
B_3=\begin{pmatrix}
b_{1m}\\ b_{2m}\\ \vdots  b_{nm}
\end{pmatrix}
\end{aligned} \ \rightarrow \ 
B=\begin{pmatrix}
b_{11}& b_{12}& \cdots b_{1n}\\
b_{21}& b_{22}& \cdots b_{2n}\\
\vdots& \vdots& \cdots& \vdots \\
b_{m1}& b_{m2}& \cdots b_{mn}
\end{pmatrix}
\end{equation*}

Podemos ahora considerar  cada elemento $p_{ij}$ de la matriz producto $P=A\cdot B$ como el producto escalar del vector fila $A_i$ for el vector columna $B_j$, $p_{ij}=A_i\cdot B_j$. 
Es ahora relativamente f�cil, deducir algunas de las propiedad del producto matricial,

\begin{enumerate}
\item Para que dos matrices puedan multiplicarse, es preciso que el n�mero de columnas de la primera coincida con el numero de filas de la segunda. Adem�s la matriz producto tiene tantas filas como la primera matriz y tantas columnas como la segunda.

\item El producto matricial no es conmutativo. En general $A\cdot B \neq B \cdot A$

\item $(A\cdot B)^T=B^T\cdot A^T$
\end{enumerate}

\paragraph{Matriz identidad} La matriz identidad de orden $n\times n$ se 
define como:
\begin{equation*}
I_n= \left\{ 
\begin{aligned}
i_{ll}&=1\\
i_{kj}&=0, \ k\neq j
\end{aligned}
\right.
\end{equation*}

Es decir, una matriz en la que todos los elementos que no pertenecen a la diagonal principal son $0$ y los elementos de la diagonal principal son $1$. Por ejemplo,

\begin{equation*}
I_3=\begin{pmatrix}
1& 0& 0\\
0& 1& 0\\
0& 0& 1
\end{pmatrix}
\end{equation*}
La matriz identidad $I_n$ es el elemento neutro del producto de matrices cuadradas de orden $n\times n$,
\begin{equation*}
A_{n\times n}\cdot I_n=I_n\cdot A_{n\times n}
\end{equation*}

Adem�s,
\begin{gather*}
A_{n\times m}\cdot I_m=A_{n \times m}\\
I_n\cdot A_{n\times m}=A_{n\times m}
\end{gather*}

En Matlab se emplea el comando \texttt{eye(n)} para construir la matriz identidad de orden $n\times n$,

\begin{verbatim}
>> I4=eye(4)
I4 =
     1     0     0     0
     0     1     0     0
     0     0     1     0
     0     0     0     1
\end{verbatim}

Una matriz cuadrada se dice que es ortogonal si cumple,
\begin{equation*}
A^T\cdot A=I
\end{equation*}

\paragraph{Producto escalar de dos vectores y producto matricial} Por conveniencia,  representaremos el producto escalar de dos vectores como un producto matricial,

\begin{gather*}
a\cdot b= a^Tb=b^Ta=b\cdot a\\
\begin{pmatrix}
1& 3& 4
\end{pmatrix}
\begin{pmatrix}
1\\
-2\\
0
\end{pmatrix}
=
\begin{pmatrix}
1& -1& 0
\end{pmatrix}
\begin{pmatrix}
1\\
3\\
4
\end{pmatrix}
=1\cdot 1+3 \cdot (-2)+ 4 \cdot 0= -5
\end{gather*}

Es decir, transponemos el primer vector del producto, convirti�ndolo en un vector fila.

\paragraph{Norma de un vector.} La longitud eucl�dea, m�dulo,  norma 2 o simplemente norma  de un vector se define como,

\begin{equation*}
\Vert x \Vert_2 =\Vert x \Vert =\sqrt{x\cdot x}=\sqrt{x^Tx}=\sqrt{x_1^2+x_2^2+\cdots x_n^2}=\left( \sum_{i=1}^nx_i^2 \right)^\frac{1}{2}
\end{equation*}

Constituye la manera usual de medir la longitud de un vector. Tiene una interpretaci�n geom�trica inmediata a trav�s del teorema de Pit�goras: nos da la longitud del segmento que representa al vector. La figura \ref{fig:pitag} muestra dicha interpretaci�n, para un vector bidimensional.  

\begin{figure}[h]
\centering
\includegraphics[width=12cm]{pitag.eps}
\caption{interpretaci�n geom�trica de la norma de un vector}
\label{fig:pitag}
\end{figure}

A partir de la norma de un vector es posible obtener una expresi�n alternativa para el producto escalar de dos vectores,
\begin{equation*}
a\cdot b=\Vert a \Vert \Vert b \Vert \cos \theta
\end{equation*}
Donde $\theta$ representa el �ngulo formado por los dos vectores.

Aunque se trate de la manera m�s com�n de definir la norma de un vector, la norma 2 no es la �nica definici�n posible,

\begin{itemize}
\item Norma 1: Se define como la suma de los valores absolutos de los elementos de un vector,
\begin{equation*}
\Vert x \Vert_1 =\vert x_1\vert +\vert x_2 \vert\cdots \vert x_n\vert
\end{equation*}
\item Norma p, Es una generalizaci�n de la norma 2,
\begin{equation*}
\Vert x \Vert_p =\sqrt[p]{\vert x_1^p\vert+\vert x_2^p\vert+\cdots \vert x_n^p\vert}=\left( \sum_{i=1}^n\vert x_i^p \vert \right)^\frac{1}{p}
\end{equation*}
\item norma $\infty$, se define como el mayor elemento del vector valor absoluto,

\begin{equation*}
\Vert x \Vert_\infty =max \left\lbrace \vert x_i\vert\right \rbrace 
\end{equation*}

\item Norma $-\infty$, el menor elemento del vector en valor absoluto,
 
\begin{equation*}
\Vert x \Vert_{-\infty} =min \left\lbrace \vert x_i\vert\right \rbrace 
\end{equation*}
\end{itemize}

En Matlab la norma de un vector puede obtenerse mediante el comando \texttt{norm(v,p)} La primera variable de entrada debe ser un vector y la segunda el tipo de norma que se desea calcular. Si se omite la segunda variable de entrada, el comando devuelve la norma 2. A continuaci�n se incluyen varios ejemplo de utilizaci�n,
\begin{verbatim}
>> x=[1;2;-3;0;-1]
x =
     1
     2
    -3
     0
    -1
>> norma_2=norm(x,2)
norma_2 =
     3.872983346207417e+00
>> norma=norm(x)
norma =
     3.872983346207417e+00
>> norma_1=norm(x,1)
norma_1 =
     7
>> norma_4=norm(x,4)
norma_4 =
     3.154342145529904e+00
>> norma_inf=norm(x,inf)
norma_inf =
     3
>> norma_minf=norm(x,-inf)
norma_minf =
     0
\end{verbatim} 

En general, una norma se define como una funci�n de $\mathbb{R}^n \rightarrow \mathbb{R}$, que cumple,
\begin{align*}
&\Vert x\Vert \geq 0,\  \Vert x\Vert =0 \Rightarrow x=0\\
&\Vert x+y\Vert \leq \Vert x\Vert +\Vert y\Vert \\
&\Vert \alpha x\Vert = \vert \alpha \vert \Vert x\Vert ,\ \alpha \in \mathbb{R} 
\end{align*}

Llamaremos vectores unitarios $u$, a aquellos para los que se cumple que $\Vert u \Vert=1$.

Dos vectores $a$ y $b$ son ortogonales si cumplen que su producto escalar es nulo, $a^Tb=0 \Rightarrow  a\bot b$. Si adem�s ambos vectores tienen m�dulo unidad, se dice entonces que los vectores son ortonormales.  Desde el punto de vista de su representaci�n geom�trica, dos vectores ortogonales, forman entre s� un �ngulo recto.

\paragraph{Traza de una matriz.} La traza de una matriz cuadrada , se define como la suma de los elementos que ocupan su diagonal principal, 
\begin{gather*}
Tr(A)=\sum_{i=1}^na_{ii}\\
Tr\left(
\begin{pmatrix}
1& 4 & 4\\
2& -2 & 2\\
0& 3 & 6
\end{pmatrix}\right)=1-2+6=5
\end{gather*}

La traza de la suma de dos matrices cuadradas $A$ y $B$ del mismo orden, coincide con la suma de las trazas de $A$ y $B$,
\begin{equation*}
tr(A+B)=tr(A)+tr(B)
\end{equation*}

Dada una  matriz $A$ de dimensi�n $m\times n$  y una matriz $B$ de dimensi�n $n \times m$, se  cumple que,
\begin{equation*}
tr(AB)=tr(BA)
\end{equation*}

En Matlab, puede obtenerse directamente el valor de la traza de una matriz, mediante el comando \texttt{trace},
\begin{verbatim}
>> A=[1 3 4
3  5 2
2 -1 -2]
A =
     1     3     4
     3     5     2
     2    -1    -2
>> t=trace(A)
t =
     4
\end{verbatim}

\paragraph{Determinante de una matriz.} El determinante de una matriz $A$, se representa habitualmente como $\vert A \vert$ o, en ocasiones como $det(A)$. Para poder definir el determinante de una matriz, necesitamos antes introducir una serie de conceptos previos. En primer lugar, si consideramos un escalar como una matriz de un solo elemento, el determinante  ser�a precisamente el valor de ese �nico elemento,
\begin{equation*}
A=\begin{pmatrix}
a_{11}
\end{pmatrix} \rightarrow \vert A \vert =a_{11}
\end{equation*}

Se denomina menor complementario o simplemente menor, $M_{ij}$ del elemento $a_{ij}$ de una matriz $A$, a la matriz que resulta de eliminar de la matriz $A$ la fila $i$ y la columna $j$ a las que pertenece el elemento $a_{ij}$. Por ejemplo,
\begin{align*}
A=\begin{pmatrix}
1& 0& -2\\
3& -2& 3\\
0& 6& 5
\end{pmatrix},& \ 
M_{23}=
\begin{pmatrix}
1& 0\\
0& 6
\end{pmatrix}\\
M_{32}=
\begin{pmatrix}
1& -2\\
3& 3
\end{pmatrix},& \ 
M_{33}=
\begin{pmatrix}
1& 0\\
3& -2
\end{pmatrix}\cdots
\end{align*}

 El  cofactor $C_{ij}$ de un elemento $a_{ij}$ de la matriz $A$, se define a partir del determinante del menor complementario del elemento $a_{ij}$ como,

\begin{equation*}
C_{ij}=(-1)^{i+j}\vert M_{ij} \vert
\end{equation*}

Podemos ahora definir el determinante de una matriz $A$ cuadrada de orden $n$, empleando la f�rmula de Laplace, 

\begin{equation*}
\vert A \vert = \sum_{j=1}^n a_{ij}C_{ij}
\end{equation*}

o alternativamente,


\begin{equation*}
\vert A \vert = \sum_{i=1}^n a_{ij}C_{ij}
\end{equation*}

En el primer caso, se dice que se ha desarrollado el determinante a lo largo de la fila $i$. En el segundo caso, se dice que se ha desarrollo el determinante a lo largo de la columna $j$.

 La f�rmula de Laplace, obtiene el determinante de una matriz de orden $n\times n$ a partir del c�lculo de los determinantes de los menores complementarios de los elementos de una fila; $n$ matrices de orden $(n-1)\times (n-1)$. A su vez, podr�amos calcular el determinante de cada menor complementario, aplicando la formula de Laplace y as� sucesivamente hasta llegar a matrices de orden $2\times 2$. Para una matriz $2\times 2$, si desarrollamos por la primera fila obtenemos su determinante como,

\begin{align*}
A&=\begin{pmatrix}
a_{11}& a_{12}\\
a_{21}& a_{22}
\end{pmatrix}\\
\vert A \vert & =\sum_{j=1}^2a_{1j}C_{1j} =a_{11}C_{11}+a_{12}C_{12}\\
 &=a_{11}(-1)^{1+1}\vert M_{11}\vert +a_{12}(-1)^{1+2}\vert M_{12}\vert \\
 &=-a_{11}a_{22}+a_{12}a_{21}\\
\end{align*}

y si desarrollamos por la segunda columna,

\begin{align*}
A&=\begin{pmatrix}
a_{11}& a_{12}\\
a_{21}& a_{22}
\end{pmatrix}\\
\vert A \vert & =\sum_{j=1}^2a_{i2}C_{i2} =a_{12}C_{12}+a_{22}C_{22}\\
 &=a_{12}(-1)^{1+2}\vert M_{12}\vert +a_{22}(-1)^{2+2}\vert M_{22}\vert \\
 &=-a_{12}a_{21}+a_{22}a_{12}\\
\end{align*}


Para una matriz de dimensi�n arbitraria $n\times n$, el determinante se obtiene aplicando recursivamente la f�rmula de Laplace,

\begin{align*}
&\vert A  \vert = \sum_{j=1}^na_{ij}C_{ij} =\sum_{j=1}^na_{ij}(-1)^{i+j} \left \vert M_{ij}^{(n-1)\times(n-1)} \right \vert \\
&\left \vert M_{ij}^{(n-1)\times(n-1)} \right \vert = \sum_{k=1}^{n-1}m_{lk}C_{lk} =\sum_{k=1}^{n-1}m_{lk}(-1)^{l+k} \left \vert M_{lk}^{(n-2)\times (n-2)} \right \vert\\
&\vdots \\
&\left \vert M_{st}^{1\times 1}\right \vert=(-1)^{s+t}m_{st} 
\end{align*}

Asi, por ejemplo, podemos calcular el determinante de la matriz,
\begin{equation*}
A=\begin{pmatrix}
1& 0& -2\\
3& -2& 3\\
0& 6& 5
\end{pmatrix}
\end{equation*}
desarroll�ndolo por los elementos de la primera columna, como,
\begin{align*}
\left\vert A \right\vert =& 1\cdot (-1)^2\cdot 
\left\vert \begin{matrix}
-2& 3\\ 
6& 5
\end{matrix} \right\vert + 3\cdot (-1)^3\cdot
\left\vert \begin{matrix}
0& -2\\ 
6& 5
\end{matrix} \right\vert+ 0\cdot (-1)^4 \cdot 
\left\vert \begin{matrix}
0& -2\\ 
-2& -3
\end{matrix} \right\vert \\
=& 1\cdot (-1)^2\cdot \left[ (-2)\cdot 5 - 6\cdot 3 \right] +3\cdot (-1)^3\cdot  \left[ 0\cdot 5 - 6\cdot (-2)\right] + 0\cdot (-1)^4 \cdot \left[ 0\cdot 3 - (-2)\cdot (-2)\right]= -64
\end{align*}

Podemos programar en Matlab una funci�n recurrente que calcule el determinante de una matriz de rango $n\times n$. (El m�todo no es especialmente eficiente pero ilustra el uso de funciones recursivas.

\begin{lstlisting}
function d=determinante(M)
% este programa, calcula el determinante de una matriz empleando la formula
% de Laplace. La funci�n es recursiva, (se llama a si misma sucesivamente
% para calcular los cofactores necesarios). Desarrolla siempre por los
% elementos de la primera columna. (Es un prodigio de ineficiencia numerica,
% pero permite manejar bucles y funciones recursivas, asi que supongo que
% puede ser �til para los que empiezan a programar).
% un posible ejercicio para ver lo malo que es el m�todo, consiste ir
% aumentando la dimension de la matriz y comparar lo que lo tarde en
% calcular el determinante con lo que tarda la funci�n de Matlab det...

% primero comprobamos que la matriz suministrada es cuadrada:
d=0;
[a,b]=size(M);
if a~=b
    disp('la matriz no es cuadrada, Campe�n')
    d=[];
else
    for i=1:a
        if a==1
            d=M;
        else
            % Eliminamos la fila y columna que toque
            N=M([1:i-1 i+1:a],2:b);
            % A�adimos el calculo correspondiente al cofactor
            d=(-1)^(i+1)*M(i,1)*determinante(N)+d;
            % pause
        end
    end
end

\end{lstlisting}

En Matlab, el determinante de una matriz se calcula directamente empleando la funci�n \texttt{det}. As�, para calcular el determinante de la matriz $A$ del ejemplo anterior,
\begin{verbatim}
>> A=[1 0 -2; 3 -2 3; 0 6 5]
A =

     1     0    -2
     3    -2     3
     0     6     5
>> da=det(A)
da =
   -64
\end{verbatim}

Entre las propiedades de los determinantes, destacaremos las siguientes,
\begin{enumerate}
\item El determinante del producto de un escalar $a$ por una matriz $A$ de dimensi�n $n\times n$ cumple,
\begin{equation*}
\left\vert a\cdot A \right\vert =a^n\cdot \vert A \vert
\end{equation*}
\item El determinante de una matriz es igual al de su traspuesta,
\begin{equation*}
\vert A \vert =\left\vert A^T \right\vert
\end{equation*}

\item El determinante del producto de dos matrices es igual al producto de los determinantes,
\begin{equation*}
\left\vert A_{n\times n} \cdot  B_{n\times n} \right\vert = \left\vert A_{n\times n} \right\vert \cdot \left\vert B_{n\times n} \right\vert 
\end{equation*}
\end{enumerate}

Una matriz es singular si su determinante es cero.

El rango de una matriz se define como el tama�o de la submatriz m�s grande dentro de $A$, cuyo determinante es distinto de cero. As� por ejemplo la matriz,
\begin{equation*}
A=\begin{pmatrix}
1& 2& 3\\
4& 5& 6\\
7& 8& 9
\end{pmatrix} \rightarrow \vert A \vert =0
\end{equation*}
Es una matriz singular y su rango es dos,
\begin{equation*}
\begin{matrix}
1& 2\\
4& 5
\end{matrix}=-3 \neq 0 \Rightarrow r(A)=2
\end{equation*}

Para una matriz no singular, su rango coincide con su orden.

En Matlab se puede estimar el rango de una  matriz mediante el comando \texttt{rank},
\begin{verbatim}
>> A=[1 2 3
4 5 6
7 8 9]
A =

     1     2     3
     4     5     6
     7     8     9
>> r=rank(A)
r =
     2
\end{verbatim}

\paragraph{Inversi�n.} Dada una matriz cuadrada no singular $A$ existe una �nica matriz $A^{-1}$ que cumple,
\begin{equation*}
A_{n\times n}\cdot A_{n\times n}^{-1}=I_{n\times n}
\end{equation*}

La matriz $A^{-1}$ recibe el nombre de matriz inversa de $A$, y puede calcularse a partir de $A$ como,
\begin{equation*}
A^{-1}=\frac{1}{\vert A \vert}[adj(A)]^T
\end{equation*}

Donde $adj(A)$ es la matriz adjunta de $A$, que se obtiene sustituyendo cada elemento $a_{ij}$ de $A$, por su cofactor $C_{ij}$. A continuaci�n incluimos el c�digo en Matlab de una funci�n \texttt{inversa} que calcula la inversa de una matriz. La funci�n \texttt{inversa} llama a su vez a la funci�n \texttt{determinante} descrita m�s arriba. Lo ideal es crear un fichero \texttt{inversa.m} que incluya las dos funciones una detr�s de la otra tal y como aparecen escritas a continuaci�n. De este modo, si llamamos desde el \emph{workspace} de Matlab a la funci�n \texttt{inversa}, esta encuentra siempre el c�digo de \texttt{deteminante} ya que est� contenido en el mimo fichero.

\begin{lstlisting}
function B=inversa(A)
% este programa calcula la inversa de una matriz a partir de definici�n
% t�pica: A^(-1)=[adj(A)]'/det(A)
% Se ha includo al final del programa una funcion (determinante) para 
% calcular determinantes
% Todo el programa es MUY INEFICIENTE. El �nico interes de esto es ense�ar que
% las estructuras basicas de programacion funcionan, y como se manejan las
% llamadas a funciones en Matlab etc.


% Lo primero que hacemos es comprobar si la matriz es cuadrada
% primero comprobamos que la matriz suministrada es cuadrada:
[a,b]=size(A);
if a~=b
    disp('la matriz no es cuadrada, Campe�n')
    B=[];
else
    % calculamos el determinante de A, si es cero hemos terminado
    dA=determinante(A)
    if dA==0
        % deber�amos condicionar en lugar de comparar con cero, los errores
        % de redondeo pueden matarnos.... Si el determinante es proximo a
        % cero
         disp('la matriz es singular, la pobre')
      B=[]  
    else
    
    % Calculamos el cofactor de cada t�rmino de A mediante un doble bucle.
    for i=1:a
        for j=1:b
            % Construimos el menor correspondiente al elemento (i,j)
            m=A([1:i-1 i+1:a],[1:j-1 j+1:b])
            % calculamos el cofactor llamando a la funci�n determinante
            % lo ponemos ya en la posici�n que corresponderia a la matriz 
            % transpuesta de la adjunta.
            B(j,i)=(-1)^(i+j)*determinante(m)
        end
    end
    % Terminamos la operacion dividiendo por el determinante de A
    B=B/dA
    end
end
end

%%%%%%%%%%%%%%%%%%%%%%%%%%%%%%%%%%%%%%%%%%
% Aqu� incluimos la funcion determinante, as� la funcion inversa, no tiene
% que ir a buscarla a ningun sitio ya que esta incluida en su mismo % fichero
%%%%%%%%%%%%%%%%%%%%%%%%%%%%%%%%%%%%%%%%%%
function d=determinante(M)
% este programa, calcula el determinante de una matriz empleando la formula
% de Laplace. La funci�n es recursiva, (se llama a si misma sucesivamente
% para calcular los cofactores necesarios). Desarrolla siempre por los
% elementos de la primera columna. (Es un prodigio de ineficiencia numerica,
% pero permite manejar bucles y funciones recursivas, asi que supongo que
% puede ser �til para los que empiezan a programar).
% un posible ejercicio para ver lo malo que es el m�todo, consiste ir
% aumentando la dimension de la matriz y comparar lo que lo tarde en
% calcular el determinante con lo que tarda la funci�n de Matlab det...

% primero comprobamos que la matriz suministrada es cuadrada:
d=0;
[a,b]=size(M);
if a~=b
    print('la matriz no es cuadrada, Campe�n')
    d=[];
else
    for i=1:a
        if a==1
            d=M;
        else
            % Elimminamos la fila y columna que toque
            N=M([1:i-1 i+1:a],2:b);
            % A�adimos el calculo correspondiente al cofactor
            d=(-1)^(i+1)*M(i,1)*determinante(N)+d;
            % pause
        end
    end
end
end
\end{lstlisting} 

Como siempre, Matlab incluye una funci�n propia \texttt{inv} para calcular la inversa de una matriz,
\begin{verbatim}
A =

     1     0    -2
     3    -2     3
     0     6     5
>> AI=inv(A)
AI =
    0.4375    0.1875    0.0625
    0.2344   -0.0781    0.1406
   -0.2813    0.0938    0.0313
>> A*AI
ans =
    1.0000         0         0
         0    1.0000         0
   -0.0000    0.0000    1.0000
\end{verbatim}
 
Alternativamente, podemos calcular la inversa, directamente como \texttt{A\^\ -1},
\begin{verbatim}
>> AI=A^-1
AI =
    0.4375    0.1875    0.0625
    0.2344   -0.0781    0.1406
   -0.2813    0.0938    0.0313
\end{verbatim} 
Algunas propiedades relacionadas con la inversi�n de matrices son,
\begin{enumerate}
\item Inversa del producto de dos matrices,
\begin{equation*}
(A\cdot B)^{-1}=B^{-1}\cdot A^{-1}
\end{equation*}

\item Determinante de la inversa,
\begin{equation*}
\left\vert A^{-1} \right\vert = \vert A \vert ^{-1}
\end{equation*}

\item Una matriz es ortogonal si su inversa coincide con su transpuesta,
\begin{equation*}
A^{-1}=A^T
\end{equation*}
\end{enumerate}

\section{Operadores vectoriales}\label{opvect}
En esta secci�n vamos a estudiar el efecto de las operaciones matriciales, descritas en la secci�n anterior, sobre los vectores. Empecemos por considerar el producto por un escalar $\alpha \cdot a$. El efecto fundamental es modificar el m�dulo del vector,
\begin{equation*}
\alpha \cdot \begin{pmatrix}
a_1\\
a_2\\
a_3
\end{pmatrix}=
\begin{pmatrix}
\alpha a_1\\
\alpha a_2\\
\alpha a_3
\end{pmatrix}\rightarrow \vert \vert \alpha \cdot a \vert \vert =\sqrt{\alpha ^2a_1^2+\alpha ^2a_2^2+\alpha ^2a_3^2}=\vert \alpha \vert  \sqrt{a_1^2+a_2^2+a_3^2}=\vert \alpha \vert \cdot \vert \vert a \vert \vert
\end{equation*}

Gr�ficamente, si $alpha$ es un n�mero positivo y mayor que la unidad, el resultado del producto ser� un vector m�s largo que $a$ con la misma direcci�n y sentido. Si por el contrario, $\alpha$ es menor que la unidad, el vector resultante ser� m�s corto que $a$. Por �ltimo si se trata de un n�mero negativo, a los resultados anteriores se a�ade el cambio de sentido con respecto a $a$. La figura \ref{fig:vmod} muestra gr�ficamente un ejemplo  del producto de un vector por un escalar.

\begin{figure}[h]
\centering
\includegraphics[width=7cm]{vmod.eps}
\caption{efecto del producto de un escalar por un vector}
\label{fig:vmod}
\end{figure}

\paragraph{Combinaci�n lineal.} Combinando la suma de vectores, con el producto por un escalar, podemos generar nuevos vectores, a partir de otros, el proceso se conoce como combinaci�n lineal,
\begin{equation*}
c=\alpha \cdot a + \beta \cdot b + \cdots +\theta z
\end{equation*}

As� el vector $c$ ser�a el resultado de una combinaci�n  lineal de los vectores $a, b \cdots z$. 
Dado un conjunto de vectores, se dice que son linealmente independientes entre s�, si no es posible poner a unos como combinaci�n lineal de otros,

\begin{equation*}
\alpha \cdot a + \beta \cdot b + \cdots +\theta z=0 \Rightarrow \alpha =\beta =\cdots =\theta =0
\end{equation*}



Es posible expresar cualquier vector de dimensi�n $n$ como una combinaci�n lineal de $n$ vectores linealmente independientes.

Supongamos $n=2$, cualquier par de vectores que no est�n alineados, pueden generar todos los vectores de dimensi�n $2$ por ejemplo,

\begin{equation*}
\begin{pmatrix}
x_1\\
x_2
\end{pmatrix}=
\alpha \begin{pmatrix}
1\\
2
\end{pmatrix}+\beta \begin{pmatrix}
-1\\
1
\end{pmatrix}
\end{equation*}

La figura \ref{fig:clineal} muestra gr�ficamente estos dos vectores y algunos de los vectores resultantes de combinarlos linealmente.

\begin{figure}[h]
\centering
\includegraphics[width=10cm]{clineal.eps}
\caption{Representaci�n gr�fica de los vectores $a=(1,2)
$, $b=(-1,1)
$ y algunos vectores, combinaci�n lineal de $a$ y $b$.}
\label{fig:clineal}
\end{figure}
Si tomamos como ejemplo $n=3$, cualquier conjunto de vectores que no est�n contenidos en el mismo plano, pueden generar cualquier otro vector de dimensi�n $3$. Por ejemplo,
\begin{equation*}
\begin{pmatrix}
x_1\\
x_2\\
x_3
\end{pmatrix}=\alpha \begin{pmatrix}
1\\
-2\\
1
\end{pmatrix}+ \beta \begin{pmatrix}
2\\
0\\
-1
\end{pmatrix}+ \gamma \begin{pmatrix}
-1\\
1\\
1
\end{pmatrix}
\end{equation*}

La figura \ref{fig:clin3} muestra gr�ficamente estos tres vectores y el vector resultante de su combinaci�n lineal, con $\alpha=1$, $\beta=-0.5$ y $\gamma=1$.  Es f�cil ver a partir de la figura que cualquier otro vector de dimensi�n $3$ que queramos construir puede obtenerse a partir de los vectores $a$, $b$ y $c$.
\begin{figure}[h]
\centering
\includegraphics[width=12cm]{clin3.eps}
\caption{Representaci�n gr�fica de los vectores $a=(1,-2,1)
$, $b=(2,0,-1)$, $c=(-1,1,1)$ y del vector $a-b+c$.}
\label{fig:clin3}
\end{figure}

\paragraph{Espacio vectorial y  bases del espacio vectorial.} El conjunto de los vectores de dimensi�n $n$ (matrices de orden $n\times 1$), junto con la suma vectorial y el producto por un escalar, constituye  un  \emph{espacio vectorial} de dimensi�n $n$.

 Como acabamos de ver, es posible obtener cualquier vector de dicho espacio vectorial a partir de $n$ vectores linealmente independientes del mismo. Un conjunto de $n$ vectores linealmente independientes de un espacio vectorial de dimensi�n $n$ recibe el nombre de base del espacio vectorial. En principio es posible encontrar infinitas bases distintas para un espacio vectorial de dimensi�n $n$. Hay algunas particularmente interesantes,
 
 \subparagraph{Bases ortogonales.} Una base ortogonal es aquella en que todos sus vectores son ortogonales entre s� , es decir cumple que su producto escalar es $b^i\cdot b^j=0, i\neq j$. Donde  $b^i$ representa el \emph{i-�simo} vector de la base, $\mathcal{B}=\left\lbrace b^1, b^2, \cdots, b^n  \right\rbrace $ .
 
 \subparagraph{Bases ortonormales.} Una base ortonormal, es una base ortogonal en la que, adem�s, los vectores de la base tienen m�dulo $1$. Es decir, $b^i\cdot b^j=0, i\neq j$ y  $b^i\cdot b^j=1, i = j$. Un caso particularmente �til de base ortonormal es la base can�nica, formada por los vectores, 
\begin{equation*}
\mathcal{C}=\left\lbrace c^1=\begin{pmatrix}
1\\
0\\
0\\
\vdots \\
0
\end{pmatrix}, c^2=\begin{pmatrix}
0\\
1\\
0\\
\vdots \\
0
\end{pmatrix},
\cdots
c^{n-1}=\begin{pmatrix}
0\\
0\\
\vdots \\
1\\
0
\end{pmatrix},
c^n=\begin{pmatrix}
0\\
0\\
\vdots \\
0\\
1
\end{pmatrix} \right\rbrace
\end{equation*} 

Podemos considerar las componentes de cualquier vector como los coeficientes de la combinaci�n lineal de la base can�nica que lo representa,

\begin{equation*}
a=\begin{pmatrix}
a_1\\
a_2\\
\cdots  \\
a_{n-1}\\
a_n
\end{pmatrix} =a_1\cdot \begin{pmatrix}
1\\
0\\
0\\
\vdots \\
0
\end{pmatrix}+a_2 \cdot  \begin{pmatrix}
0\\
1\\
0\\
\vdots \\
0
\end{pmatrix}+
\cdots +
a_{n-1}\cdot \begin{pmatrix}
0\\
0\\
\vdots \\
1\\
0
\end{pmatrix}+
a_n\cdot \begin{pmatrix}
0\\
0\\
\vdots \\
0\\
1
\end{pmatrix}
\end{equation*}

Por extensi�n, podemos generalizar este resultado a cualquier otra base, es decir podemos agrupar en un vector los coeficientes de la combinaci�n lineal de los vectores de la base que lo generan. Por ejemplo, si construimos, para los vectores de dimensi�n $3$ la base,
\begin{equation*}
\mathcal{B}=\left\lbrace \begin{pmatrix}
1\\
2\\
0
\end{pmatrix}, \begin{pmatrix}
-1\\
0\\
2
\end{pmatrix}, \begin{pmatrix}
1\\
-1\\
1
\end{pmatrix} \right\rbrace
\end{equation*} 

Podemos entonces representar un vector en la base $\mathcal{B}$ como,
\begin{equation*}
\alpha \cdot \begin{pmatrix}
1\\
2\\
0
\end{pmatrix}+\beta \cdot \begin{pmatrix}
-1\\
0\\
2
\end{pmatrix}+ \gamma \cdot \begin{pmatrix}
1\\
-1\\
1
\end{pmatrix} \rightarrow a^{\mathcal{B}}=\begin{pmatrix}
\alpha \\
\beta \\
\gamma
\end{pmatrix}
\end{equation*} 
 
Donde  estamos empleando el super�ndice $^{\mathcal{B}}$, para indicar que las componentes del vector $a$ est�n definidas con respecto a la base $\mathcal{B}$.

As� por ejemplo el vector,
\begin{equation*}
a^{\mathcal{B}}=\begin{pmatrix}
1.125\\
0.375\\
0.75
\end{pmatrix}
\end{equation*}

Tendr�a en la base can�nica las componentes, 
\begin{equation*}
a^{\mathcal{B}}=\begin{pmatrix}
1.125\\
0.375\\
0.75
\end{pmatrix} \rightarrow a= 1.125 \cdot \begin{pmatrix}
1\\
2\\
0
\end{pmatrix}+0.375 \cdot \begin{pmatrix}
-1\\
0\\
2
\end{pmatrix}+ 0.75 \cdot \begin{pmatrix}
1\\
-1\\
1
\end{pmatrix} =\begin{pmatrix}
1.5\\
1.5 \\
1.5
\end{pmatrix}
\end{equation*}

La figura \ref{fig:base1}, muestra gr�ficamente la relaci�n entre los vectores de la base can�nica $\mathcal{C}$, los vectores de la base $\mathcal{B}$, y el vector $a$, cuyas componentes se han representado en ambas bases.

Podemos aprovechar el producto de matrices para obtener las componentes en la base can�nica $\mathcal{C}$ de un vector representado en una base cualquiera $\mathcal{B}$. Si agrupamos los vectores de la base $\mathcal{B}$, en una matriz $B$,

\begin{figure}[h]
\centering
\includegraphics[width=15cm]{base1.eps}
\caption{Representaci�n gr�fica del vector $a$, en las base can�nica $\mathcal{C}$ y en la base $\mathcal{B}$}
\label{fig:base1}
\end{figure}

\begin{equation*}
\mathcal{B}=\left\lbrace b^1=\begin{pmatrix}
b_{11}\\
b_{21}\\
b_{31}\\
\vdots \\
b_{n1}
\end{pmatrix},b^2=\begin{pmatrix}
b_{12}\\
b_{22}\\
b_{32}\\
\vdots \\
b_{n2}
\end{pmatrix},
\cdots
b^n=\begin{pmatrix}
b_{1n}\\
b_{2n}\\
\vdots \\
b_{(n-1)n}\\
b_{nn}
\end{pmatrix} \right\rbrace \rightarrow
B=\begin{pmatrix}
b_{11}&b_{12}& \cdots & b_{1n}\\
b_{21}&b_{22}& \cdots & b_{2n}\\
b_{31}&b_{32}& \cdots & \vdots \\
\vdots & \vdots & \cdots &  b_{(n-1)n}\\
b_{n1}&b_{n2}& \cdots & b_{nn}\\
\end{pmatrix}
\end{equation*}

Supongamos que tenemos un vector $a$ cuyas componentes en la base $\mathcal{B}$ son,
\begin{equation*}
a^{\mathcal{B}}=\begin{pmatrix}
a_1^{\mathcal{B}}\\
a_2^{\mathcal{B}}\\
\vdots \\
a_n^{\mathcal{B}}
\end{pmatrix}
\end{equation*}

Para obtener las componentes en la base can�nica, basta entonces multiplicar la matriz $B$, por el vector $a^{\mathcal{B}}$. As� en el ejemplo que acabamos de ver,

\begin{equation*}
a=B\cdot a^{\mathcal{B}} \rightarrow a=\begin{pmatrix}
1& -1& 1\\
2& 0& -1\\
0& 2& 1
\end{pmatrix}\cdot \begin{pmatrix}
1.125\\
0.375\\
0.75
\end{pmatrix}
= \begin{pmatrix}
1.5\\
1.5\\
1.5
\end{pmatrix}
\end{equation*}

Por �ltimo, una podemos combinar el producto de matrices y la matriz inversa, para obtener las componentes de un vector en una base cualquiera a partir de sus componentes en otra base. Supongamos que tenemos dos bases $\mathcal{B}_1$ y $\mathcal{B}_2$ y un vector $a$. Podemos obtener las componentes de $a$ en la base can�nica, a partir de las componentes en la base $\mathcal{B}_1$ como, $a=B_1\cdot a^{\mathcal{B}_1}$ y a partir de sus componentes en la base $\mathcal{B}_2$ como $a=B_2\cdot a^{\mathcal{B}_2}$. Haciendo uso de la matriz inversa,

\begin{align*}
a&=B_1\cdot a^{\mathcal{B}_1} \Rightarrow a^{\mathcal{B}_1}=B_1^{-1} \cdot a \\
 a&=B_2\cdot a^{\mathcal{B}_2} \Rightarrow a^{\mathcal{B}_2}=B_2^{-1} \cdot a
\end{align*}

Y sustituyendo obtenemos,

\begin{align*}
a^{\mathcal{B}_1}&=B_1^{-1}\cdot B_2 \cdot a^{\mathcal{B}_2}\\
a^{\mathcal{B}_2}&=B_2^{-1} \cdot B_1\cdot a^{\mathcal{B}_1} 
\end{align*}

El siguiente c�digo permite cambiar de base un vector y representa gr�ficamente tanto el vector como las bases antigua y nueva.

\begin{lstlisting}
function aB2=cambia_vb(aB1,B1,B2)
% este programa cabia de base un vector de dimensi�n 3 y lo representa en
% relaci�n con las bases antigua y nueva.
% variables de entrada:
% aB1, componentes del vector en la base 1
% B1 base representada como una matriz, (cada columna contiene un vector de
% la base) En la que est� representado el vector aB1
% B2, base representada como una matriz, (cada columna contiene un vector de
% la base) en la que se quiere representar el vector aB1
% Si solo se incluye un vector y una base, el programa asume que la segunda
% base es la canonica B2=[1 0 0;0 1 0; 0 0 1]
% variables de salida:
% aB2, Componentes del vector aB1 en la nueva base B2.
% la funci�n hace uso de una funci�n auxiliar (pintavec) incluida al final
% del fichero para dibujar los vectores.

if nargin==2
    % asumimos que queremos cambiar el vector de B1 a la base canonica
    % �Es B1 una base sensanta?
    if det(B1)<=eps
        error('los vectores de la base no son l. independientes')
    end
    aB2=B1*aB1;
    B2=eye(3); %creamos la base para luego pintarla...
elseif nargin==3
    % cambio de base B1 a B2
    if (det(B1)<=eps)||(det(B2)<=eps)
        error('los vectores de al menos una de las bases no son l. independientes')
    end
    % invertimos la base nueva y multiplicamos por la antigua y por el
    % vector para obtener las componentes del vector en la base nueva.
    aB2=inv(B2)*B1*aB1;
else
    error('el numero de variables de entrada es menor de dos o mayor de tres')
end

% Dibujo de los vectores con la funci�n pintavec....

pintavec(B1,'r') % vectores de la base original
xlabel('x')
ylabel('y')
zlabel('z')
grid on
pintavec(B2,'b') % vectores de la base nueva
for i=1:3
    pintavec(aB1(i)*B1(:,i),'k') % componentes de aB1
    pintavec(aB2(i)*B2(:,i),'k') % componentes de aB2
end
aBc=B1*aB1; % representacion del vector en la base can�nica
pintavec(aBc,'g') % vector representado



function pintavec(a,par)
% funci�n auxiliar para pintar vectores... con origen en el origen de
% coordenadas (0,0,0).
% la variable a puede ser un vector o una matriz. y par, es una cadena que contiene los
% t�picos par�metros(color, tipo de l�nea`etc'). El programa considera que
% los vectores est�n siempre definidos como vectores columnas...
d=size(a,2); %miramos cuantas columnas tiene a, cada columna representar� un
% vector distinto
if nargin==2
    for i=1:d
        quiver3(0,0,0,a(1,i),a(2,i),a(3,i),0,par)
        hold on
    end
else
    for i=1:d
        quiver3(0,0,0,a(1,i),a(2,i),a(3,i),0)
    end
end

\end{lstlisting}

\paragraph{Operadores lineales.} A partir de los visto en las secciones anteriores, sabemos que el producto de una matriz de $A$ de orden $n\times n$  multiplicada por un vector $b$ de dimension $n$ da como resultado un nuevo vector $c=A\cdot b$  de dimensi�n $n$. Podemos considerar cada matriz $n\times n$ como un \emph{operador lineal}, que transforma unos vectores en otros.  Decimos que se trata de un operador lineal porque las componentes del vector resultante, est�n relacionadas linealmente con las del vector original, por ejemplo para $n=3$,
\begin{equation*}
\begin{pmatrix}
y_1\\
y_2\\
y_3
\end{pmatrix}=\begin{pmatrix}
a_{11}& a_{12}& a_{13}\\
a_{21}& a_{22}& a_{23}\\
a_{31}& a_{32}& a_{33}
\end{pmatrix} \cdot
\begin{pmatrix}
x_1\\
x_2\\
x_3
\end{pmatrix} \rightarrow \begin{matrix}
y_1=a_{11}x_1+a_{12}x_2+a_{13}x_3\\
y_2=a_{21}x_1+a_{22}x_2 +a_{23}x_3\\
y_3=a_{31}x_1+a_{32}x_2+a_{33}x_3
\end{matrix} 
\end{equation*}

Entre los operadores lineales, es posible destacar aquellos que producen transformaciones geom�tricas sencillas. Veamos algunos ejemplos para vectores bidimensionales,

\begin{enumerate}
\item Dilataci�n: aumenta el m�dulo de un vector en un factor $\alpha>1$. Contracci�n: disminuye el m�dulo de un vector en un factor $0<\alpha<1$. En ambos casos, se conserva la direcci�n y el sentido del vector original.
 
\begin{equation*}
R=\begin{pmatrix}
\alpha& 0\\
0& \alpha
\end{pmatrix} \rightarrow R\cdot a = \begin{pmatrix}
\alpha& 0\\
0& \alpha
\end{pmatrix} \cdot \begin{pmatrix}
a_1\\
a_2
\end{pmatrix}= \begin{pmatrix}
\alpha \cdot a_1\\
\alpha \cdot a_2
\end{pmatrix}
\end{equation*}

\item Reflexi�n de un vector respecto al eje x, conservando su m�dulo,
\begin{equation*}
R_x=\begin{pmatrix}
1& 0\\
0& -1
\end{pmatrix} \rightarrow R_x\cdot a = \begin{pmatrix}
1& 0\\
0& -1
\end{pmatrix} \cdot \begin{pmatrix}
a_1\\
a_2
\end{pmatrix}= \begin{pmatrix}
a_1\\
-a_2
\end{pmatrix}
\end{equation*}

\item Reflexi�n de un vector respecto al eje y, conservando su m�dulo,
\begin{equation*}
R_y=\begin{pmatrix}
-1& 0\\
0& 1
\end{pmatrix} \rightarrow R_y\cdot a = \begin{pmatrix}
-1& 0\\
0& 1
\end{pmatrix} \cdot \begin{pmatrix}
a_1\\
a_2
\end{pmatrix}= \begin{pmatrix}
-a_1\\
a_2
\end{pmatrix}
\end{equation*}

\item Reflexi�n respecto al origen: Invierte el sentido de un vector, conservando su m�dulo y direcci�n,
\begin{equation*}
R=\begin{pmatrix}
-1& 0\\
0& -1
\end{pmatrix} \rightarrow R\cdot a = \begin{pmatrix}
-1& 0\\
0& -1
\end{pmatrix} \cdot \begin{pmatrix}
a_1\\
a_2
\end{pmatrix}= \begin{pmatrix}
-a_1\\
-a_2
\end{pmatrix}
\end{equation*}
Ser�a equivalente a aplicar una reflexi�n respecto al eje x y luego respecto al eje y o viceversa,
$R=R_x\cdot R_y= R_y\cdot R_x$.

\item Rotaci�n en torno al origen un �ngulo $\theta$,
\begin{equation*}
R_{\theta}=\begin{pmatrix}
cos(\theta)& -sin(\theta)\\
sin(\theta)& cos(\theta)
\end{pmatrix} \rightarrow R_{\theta}\cdot a = \begin{pmatrix}
cos(\theta)& -sin(\theta)\\
sin(\theta)& cos(\theta)
\end{pmatrix} \cdot \begin{pmatrix}
a_1\\
a_2
\end{pmatrix}= \begin{pmatrix}
a_1cos(\theta)-a_2sin(\theta)\\
a_1sin(\theta)+a_2cos(\theta)
\end{pmatrix}
\end{equation*}
\end{enumerate}

La figura \ref{fig:ltrans} muestra los vectores resultantes de aplicar las transformaciones lineales que acabamos de describir al vector,  $ a=\bigl( \begin{smallmatrix}
1\\
2
\end{smallmatrix} \bigr)$,

\begin{figure}[h]
\centering
\includegraphics[width=12cm]{ltrans.eps}
\caption{Transformaciones lineales del vector $a=[1;2]$. $D$, dilataci�n/contracci�n en un factor $1.5$/$0.5$. $R_x$, reflexi�n respecto al eje x. $R_y$, reflexi�n respecto al eje y. $R_{\theta}$ rotaciones respecto al origen para �ngulos $\theta=\pi /6$ y $\theta=\pi /3$}
\label{fig:ltrans}
\end{figure}

\paragraph{Norma de una matriz.} La norma de una matriz se puede definir a partir del efecto que produce al actuar, como un operador lineal, sobre un vector. En este caso, se les llama normas \emph{inducidas}. Para una matriz $A$ de orden $m\times n$, $y_{(m)}=A_{(m\times n)}x_{(n)}$, La norma inducida de $A$ se define en funci�n de las normas de los vectores $x$ de su dominio y de las normas de los vectores $y$ de su rango como,
\begin{equation*}
\Vert A \Vert =\max_{x \neq 0} \frac{\Vert y \Vert}{\Vert x \Vert}=\max_{x \neq 0} \frac{\Vert Ax \Vert}{\Vert x \Vert}
\end{equation*}

Se puede interpretar como el factor m�ximo con que el que la matriz $A$ puede \emph{alargar} un vector cualquiera. Es posible definir la norma inducida en funci�n de los vectores unitarios del dominio,
\begin{equation*}
\Vert A \Vert =\max_{x \neq 0} \frac{\Vert Ax \Vert}{\Vert x \Vert}= \max_{x \neq 0} \left\Vert A\frac{x}{\Vert x \Vert} \right\Vert= \max_{\Vert x \Vert =1}\Vert Ax \Vert
\end{equation*}

Junto a la norma inducida que acabamos de ver, se definen las siguientes normas,

\begin{enumerate}
\item Norma 1: Se suman los elementos de cada columna de la matriz, y se toma como norma el valor m�ximo de dichas sumas,
\begin{equation*}
\Vert A_{m,n} \Vert _{1} = \max_j \sum_{i=1}^m a_{ij}
\end{equation*}
 \item Norma $\infty$: Se suman los elementos de cada fila y se toma como norma $\infty$ el valor m�ximo de dichas sumas.
\begin{equation*}
\Vert A_{m,n} \Vert _{\infty} = \max_i \sum_{j=1}^m a_{ij}
\end{equation*}

\item Norma 2: Se define como el mayor de los valores singulares de una matriz. (Ver secci�n \ref{sec:SVD}).
\begin{equation*}
\Vert A_{m,n} \Vert _2 = \sigma_{1}
\end{equation*}
\item Norma de Frobenius. Se define como la ra�z cuadrada de la suma de los cuadrados de todos los elementos de la matriz,
\begin{equation*}
\Vert A_{m,n} \Vert _F =\sqrt{\sum_{i=1}^m \sum_{j=1}^m a_{ij}^2}
\end{equation*}
Que tambi�n puede expresarse de forma mas directa como,
\begin{equation*}
\Vert A_{m,n} \Vert _F =\sqrt{tr(A^T\cdot A)}
\end{equation*}
\end{enumerate}
En Matlab, es posible calcular las distintas normas de una matriz, de modo an�logo a como se calculan para el caso de vectores,  mediante el comando \texttt{norm(A,p)}. Donde \texttt{A}, es ahora un a matriz y \texttt{p} especifica el tipo de norma que se quiere calcular. En el caso de una matriz, el par�metro \texttt{p} solo puede tomar los valores, \texttt{1} (norma 1), \texttt{2} (norma 2), \texttt{inf} (norma $\infty$), y \texttt{'fro'} (norma de frobenius). El siguiente ejemplo muestra el c�lculo de las normas 1, 2, $\infty$ y de Frobenius de la misma matriz.
\begin{verbatim}
>> A=[1 3 4 5
2 5 6 -3
1 0 4 3]
A =
     1     3     4     5
     2     5     6    -3
     1     0     4     3
>> n1=norm(A,1)
n1 =
    14
>> n2=norm(A,2)
n2 =
     1.022217214669622e+01
>> ninf=norm(A,inf)
ninf =
    16
>> nfro=norm(A,'fro')
nfro =
     1.228820572744451e+01
\end{verbatim} 

\paragraph{Formas cuadr�ticas.} Se define como forma cuadr�tica a la siguiente operaci�n entre una matriz cuadrada $A$ de orden $n \times n$ y un vector $x$ de dimensi�n $n$,
\begin{equation*}
\alpha=x^T\cdot A \cdot x, \ \alpha \in \mathbb{R}
\end{equation*}

El resultado es un escalar. As� por ejemplo,
\begin{equation*}
A=\begin{pmatrix}
1& 2& -1\\
2& 0& 2\\
3& 2& -2
\end{pmatrix}, \ x=\begin{pmatrix}
1\\
2\\
3
\end{pmatrix} \rightarrow \begin{pmatrix}
1& 2& 3
\end{pmatrix} \cdot \begin{pmatrix}
1& 2& -1\\
2& 0& 2\\
3& 2& -2
\end{pmatrix} \cdot \begin{pmatrix}
1\\
2\\
3
\end{pmatrix}= 21
\end{equation*}

Para dimensi�n $n=2$, 
\begin{equation*}
\alpha =\begin{pmatrix}
x_1& x_2
\end{pmatrix}\cdot \begin{pmatrix}
a_{11}& a_{12}\\
a_{21}& a_{22}
\end{pmatrix}\cdot \begin{pmatrix}
x_1\\
x_2
\end{pmatrix} \rightarrow x_3\equiv \alpha=a_{11}x_1^2+(a_{12}+a_{21})x_1x_2+a_{22}x_2^2
\end{equation*}

Lo que obtenemos, dependiendo de los signos de $a_{11}$ y $a_{12}$, es la ecuaci�n de un paraboloide o un hiperboloide. En la figura \ref{fig:parabol} Se muestra un ejemplo,
\begin{figure}[h]
\centering
\includegraphics[width=14cm]{parabol.eps}
\caption{Formas cuadr�ticas asociadas a las cuatro matrices diagonales: $\vert a_{11}\vert=\vert a_{22}\vert=1$, $a_{12}=a_{21}=0$}
\label{fig:parabol}
\end{figure}

Veamos brevemente, algunas propiedades relacionadas con  las formas cuadr�ticas,
\begin{enumerate}
\item Una matriz $A$ de orden $ n\times n$ se dice que es definida positiva si da lugar a una forma cuadr�tica que es siempre mayor que cero para cualquier vector no nulo,
\begin{equation*}
x^T \cdot A \cdot x > 0, \forall x \neq 0 
\end{equation*}

\item Una matriz \emph{sim�trica} es definida positiva si todos sus \emph{valores propios} (ver secci�n \ref{sec:diag}) son positivos.

\item Una matriz no sim�trica $A$ es definida positiva si su parte sim�trica $A_s=(A+A^T)/2$ lo es.
\begin{equation*}
x\cdot A_s\cdot >0, \forall x \neq 0 \Rightarrow x\cdot A\cdot >0, \forall x \neq 0
\end{equation*}
\end{enumerate}

\section{Tipos de matrices empleados frecuentemente}\label{tiposm}
Definimos a continuaci�n algunos tipos de matrices frecuentemente empleados en �lgebra, algunos ya han sido introducidos en secciones anteriores. Los reunimos todos aqu� para facilitar su consulta

\begin{enumerate}
\item Matriz ortogonal: Una matriz $A_{n\times n}$ es ortogonal cuando su inversa coincide con su traspuesta.
\begin{equation*}
A^T=A^{-1}
\end{equation*}
ejemplo,

\begin{equation*}
A=\begin{pmatrix}
1/3& 2/3& 2/3\\
2/3& -2/3& 1/3\\
2/3& 1/3& -2/3\\
\end{pmatrix} \rightarrow A\cdot A^T =A^T\cdot A= \begin{pmatrix}
1& 0& 0\\
0& 1& 0\\
0& 0& 1
\end{pmatrix}
\end{equation*}
\item Matriz sim�trica: Una matriz $A_{n\times n}$ es sim�trica cuando es igual que su traspuesta,
\begin{equation*}
A=A^T \rightarrow a_{ij}=a_{ji}
\end{equation*}
ejemplo,
\begin{equation*}
A=\begin{pmatrix}
1& -2& 3\\
-2& 4& 0\\
3& 0& -5
\end{pmatrix}
\end{equation*} 
\item Matriz Diagonal: Una matriz $A$  es diagonal si solo son distintos de ceros los elementos de su diagonal principal,
\begin{equation*}
\begin{pmatrix}
a_{11}& 0& \cdots & 0\\
0& a_{22}& \cdots & 0\\
\vdots & \vdots & \ddots & 0\\
0& 0& \cdots & a_{nn}
\end{pmatrix} \rightarrow
a_{ij}=0,\ \forall i\neq j
\end{equation*}

\item Matriz triangular superior: Una matriz es triangular superior cuando todos los elementos situados por debajo de la diagonal son cero. Es estrictamente diagonal superior si adem�s los elementos de la diagonal tambi�n son cero,
\begin{align*}
TRS& \rightarrow a_{ij} = 0, \ \forall i\geq j \\
ETRS& \rightarrow a_{ij} = 0, \ \forall i > j 
\end{align*}
ejemplos,
\begin{align*}
TRS&=\begin{pmatrix}
1 & 3 & 7\\
0 & 2 & -1\\
0 & 0 & 4
\end{pmatrix}\\
ETRS&=\begin{pmatrix}
0 & 3 & 7\\
0 & 0 & -1\\
0 & 0 & 0
\end{pmatrix}\\
\end{align*}
\item Matriz triangular inferior: Una matriz es triangular inferior si todos los elementos pro encima de su diagonal son cero. Es estrictamente triangular inferior si adem�s los elementos de su diagonal son tambi�n cero,
\begin{align*}
TRI& \rightarrow a_{ij} = 0, \ \forall i\leq j \\
ETRI& \rightarrow a_{ij} = 0, \ \forall i < j 
\end{align*}
ejemplos,
\begin{align*}
TRI&=\begin{pmatrix}
1 & 0 & 0\\
3 & 2 & 0\\
7 & -1 & 4
\end{pmatrix}\\
ETRI&=\begin{pmatrix}
0 & 0 & 0\\
3 & 0 & 0\\
7 & -1 & 0
\end{pmatrix}\\
\end{align*}
\item Matriz definida Positiva. Una Matriz $A_{n \times n}$ es definida positiva si dado un vector $x$ no nulo cumple,
\begin{equation*}
x^T\cdot A \cdot x > 0, \ \forall x\neq 0
\end{equation*}
si,
\begin{equation*}
x^T\cdot A \cdot x \geq 0, \ \forall x\neq 0
\end{equation*}

entonces la matriz $A$ es semidefinida positiva.

\item Una matriz es Diagonal dominante si cada uno de los elementos de la diagonal en valor absoluto es mayor que la suma de los valores absolutos de los elementos de fila a la  que pertenece.

\begin{equation*}
\lvert a_{ii} \rvert > \sum_{j\neq i} \lvert a_{ij} \rvert, \ \forall i
\end{equation*}
ejemplo,
\begin{equation*}
A=\begin{pmatrix}
10& 2 & 3\\
2& -5 & 1\\
4& -2 & 8
\end{pmatrix}\rightarrow \left\{ \begin{aligned}
10& > 2+3\\
5&> 2+1\\
8& > 4+2
\end{aligned} \right. 
\end{equation*}

\end{enumerate}



\section{Factorizaci�n de matrices}\label{sec:fact}
La factorizaci�n de matrices, consiste en la descomposici�n de una matriz en el producto de dos o m�s matrices. Las matrices resultantes de la factorizaci�n se eligen de modo que simplifiquen, o hagan m�s robustas num�ricamente determinadas operaciones matriciales: C�lculos de determinantes, inversas, etc. A continuaci�n se describen las m�s comunes.

\subsection{Factorizacion LU}\label{sec:LU}
Consiste en factorizar una matriz como el producto de una matriz triangular inferior $L$ por una  matriz triangular superior $U$, $A=L\cdot U$. Por ejemplo,
\begin{equation*}
\begin{pmatrix}
3& 4& 2\\
2& 0& 1\\
3& 2& 1
\end{pmatrix} = \begin{pmatrix}
1& 0& 0\\
^2/_3 & 1& 0\\
1& ^3/_4& 1
\end{pmatrix}\cdot \begin{pmatrix}
3& 4& 2\\
0& ^{-8}/_3& ^{-1}/_3\\
0& 0& ^{-3}/_4
\end{pmatrix}
\end{equation*}
Una aplicaci�n inmediata, es el calculo del determinante. Puesto que el determinante de una matriz triangular, es directamente el producto de los elementos de la diagonal.

En el ejemplo anterior,
\begin{equation*}
\vert A \vert = 6 \equiv \vert L \vert\cdot\vert U\vert =1 \cdot 1 \cdot 1 \cdot 3 \cdot (-\frac{8}{3})\cdot (-\frac{3}{4})=6    
\end{equation*}

Uno de los m�todos m�s empleados para calcular la factorizaci�n LU de una matriz, se basa en el m�todo conocido como eliminaci�n gaussiana. La idea es convertir en ceros los elementos situados por debajo de la diagonal de la matriz. Para ello, se sustituyen progresivamente las filas de la matriz, exceptuando la primera, por combinaciones  formadas con la fila que se sustituye y la fila anterior.
 veamos en qu� consiste con un ejemplo. Supongamos que tenemos la siguiente matriz de orden $4 \times 4$, 

\begin{equation*}
A=\begin{pmatrix}
3& 4& 2&5\\
2& 0& 1& -2\\
3& 2& 1& 8\\
5& 2& 3& 2
\end{pmatrix} 
\end{equation*}

Si sustituimos la segunda fila por el resultado de restarle la primera multiplicada por $2$ y dividida por $3$ obtendr�amos la matriz,

\begin{equation*}
A=\begin{pmatrix}
3& 4& 2&5\\
2& 0& 1& -2\\
3& 2& 1& 8\\
5& 2& 3& 2
\end{pmatrix} \rightarrow [2\ 0\ 1\ -2]-\frac{2}{3} [3\ 4\ 2\ 5] \rightarrow U_1=\begin{pmatrix}
3& 4& 2&5\\
0& -2.6& -0.33& -5.33\\
3& 2& 1& 8\\
5& 2& 3& 2
\end{pmatrix}
\end{equation*}

De modo an�logo, si sustituimos ahora la tercera fila por el resultado de restarle la primera multiplicada por $3$ y dividida $3$,
 
\begin{equation*}
U_1=\begin{pmatrix}
3& 4& 2&5\\
0& -2.6& -0.33& -5.33\\
3& 2& 1& 8\\
5& 2& 3& 2
\end{pmatrix} \rightarrow [3\ 2\ 1\ 8]-\frac{3}{3} [3\ 4\ 2\ 5] \rightarrow U_1=\begin{pmatrix}
3& 4& 2&5\\
0& -2.6& -0.33& -5.33\\
0& -2& -1& -3\\
5& 2& 3& 2
\end{pmatrix}
\end{equation*}

Por �ltimo si sustituimos la �ltima fila por el resultado de restarle la primera multiplicada por $5$ y dividida por $3$,

\begin{equation*}
U_1=\begin{pmatrix}
3& 4& 2&5\\
0& -2.6& -0.33& -5.33\\
0& -2& -1& -3\\
5& 2& 3& 2
\end{pmatrix} \rightarrow [5\ 2\ 3\ 2]-\frac{5}{3} [3\ 4\ 2\ 5] \rightarrow U_1=\begin{pmatrix}
3& 4& 2&5\\
0& -2.6& -0.33& -5.33\\
0& -2& -1& -3\\
0& -4,66& -0.33& -6.33
\end{pmatrix}
\end{equation*}

El resultado que hemos obtenido, tras realizar esta transformaci�n , es una nueva matriz $U$ en la que todos los elementos de su primera columna, por debajo de la diagonal, son ceros.

Podemos proceder de modo an�logo para \emph{eliminar} ahora los elementos de la segunda columna situados por debajo de la diagonal. Para ellos sustituimos la tercera fila  por la diferencia entre ella y las segunda fila multiplicada por 	$-2$  y dividida por $-2.6$.

\begin{align*}
U_1 &=\begin{pmatrix}
3& 4& 2&5\\
0& -2.6& -0.33& -5.33\\
0& -2& -1& -3\\
0& -4,66& -0.33& -6.33
\end{pmatrix} \rightarrow [0\ -2\ -1\ -3]-\frac{-2}{-2.6} [0\ -2.6\ -0.33\ -5.33\ ] \rightarrow \\
 U_2 &=\begin{pmatrix}
3& 4& 2&5\\
0& -2.6& -0.33& -5.33\\
0& 0& -0.75& 7\\
0& -4.66& -0.33& -6.33
\end{pmatrix}
\end{align*}

Y sustituyendo la �ltima fila por  la diferencia entre ella y la segunda multiplicada por $-4.66$ y dividida por $-2.6$,

\begin{align*}
U_2 &=\begin{pmatrix}
3& 4& 2&5\\
0& -2.6& -0.33& -5.33\\
0& -2& -1& -3\\
0& -4,66& -0.33& -6.33
\end{pmatrix} \rightarrow [0\ -4.66\ -0.33\ -6.33]-\frac{-4.66}{-2.6} [0\ -2.6\ -0.33\ -5.33\ ] \rightarrow \\
 U_2 &=\begin{pmatrix}
3& 4& 2&5\\
0& -2.6& -0.33& -5.33\\
0& 0& -0.75& 7\\
0& 0& 0.25& 3
\end{pmatrix}
\end{align*}

De este modo, los elementos de la segunda columna situados debajo de la diagonal, han sido sustituidos por ceros. Un �ltimo paso, nos llevar� hasta una matriz triangular superior; sustituimos la �ltima fila por la diferencia entre ella y la tercera fila multiplicada por $0.25$ y dividida por $-0.75$,

\begin{align*}
 U_2 &=\begin{pmatrix}
3& 4& 2&5\\
0& -2.6& -0.33& -5.33\\
0& 0& -0.75& 7\\
0& 0& 0.25& 3
\end{pmatrix} \rightarrow [0\ 0\ 0.25\ 3]-\frac{0.25}{-0.75} [0\ 0\ -0.75\ 7\ ] \rightarrow \\
 U_3 &=\begin{pmatrix}
3& 4& 2&5\\
0& -2.6& -0.33& -5.33\\
0& 0& -0.75& 7\\
0& 0& 0& 5.33
\end{pmatrix}=U
\end{align*}

Podemos ahora, a partir del ejemplo, deducir un procedimiento general. Para \emph{eliminar} ---convertir en 0--- el elemento $a_{ij}$ situado por debajo de la diagonal principal,  $i>j$:
\begin{enumerate}
\item Dividimos los elementos de la fila $j$ por el elemento de dicha fila que a su vez pertenece a la diagonal, $a_{jj}$
\begin{equation*}
\begin{bmatrix}
0/ a_{jj}& 0/ a_{jj}& \cdots & a_{jj}/ a_{jj}& a_{jj+1}/ a_{jj}& \cdots
\end{bmatrix}
\end{equation*}

\item Multiplicamos el resultado de la operaci�n anterior por el elemento $a_{ij}$,
\begin{equation*}
\begin{bmatrix}
a_{ij} \cdot 0/ a_{jj}& a_{ij} \cdot 0/ a_{jj}& \cdots & a_{ij} \cdot a_{jj}/ a_{jj}& a_{ij} \cdot a_{jj+1}/ a_{jj}& \cdots
\end{bmatrix}
\end{equation*}
\item Finalmente, sustituimos la fila $i$ de la matriz de partida por la diferencia  entre ella y el resultado de la operaci�n anterior.
\begin{equation*}
\begin{bmatrix}
0& 0& \cdots& a_{ij}& a_{ij+1}& \cdots
\end{bmatrix}-\begin{bmatrix}
a_{ij} \cdot 0/ a_{jj}& a_{ij} \cdot 0/ a_{jj}& \cdots & a_{ij} \cdot a_{jj}/ a_{jj}& a_{ij} \cdot a_{jj+1}/ a_{jj}& \cdots
\end{bmatrix}
\end{equation*}
\end{enumerate}

Este procedimiento se aplica iterativamente empezando en por el elemento  $a_{21}$ de la matriz y desplazando el c�mputo hacia abajo, hasta llegar a la �ltima fila y hacia la derecha hasta llegar en cada fila al elemento anterior a la diagonal.

El siguiente c�digo aplica el procedimiento descrito a una matriz de cualquier orden,\label{elig}

\begin{lstlisting}
function U=eligauss(A)
% Esta funci�n obtiene una matriz triangular superior, a partir de una
% matriz dada, aplicando el m�todo de eliminaci�n gaussiana.
% No realiza piboteo de filas, as� que si alg�n elemento de la diagonal de A
% queda cero o proximo a cero al ir eliminado dar� problemas...

% Obtenemos el n�mero de filas de la matriz..
nf=size(A,1);
U=A
%
for j=1:nf-1 % recorro todas la columnas menos la �ltima
    for i=j+1:nf % Recorro las filas desde debajo de la diagonal hasta la �ltima
        % en Matlab tengo la suerte de poder manejar cada fila de un solo
        % golpe.
        U(i,:)=U(i,:)-U(j,:)*U(i,j)/U(j,j)
    end
end

\end{lstlisting}

Hasta ahora, hemos descrito un procedimiento para transformar una matriz cualquiera en una matriz triangular superior. Nuestro objetivo era obtener la descomposici�n de un matriz en el producto de dos,  una triangular inferior y otra triangular superior.  En primer lugar, podemos asociar    el procedimiento descrito de eliminaci�n gaussiana al producto de matrices. Volviendo al ejemplo anterior, si construimos la matriz $\lambda_1$
\begin{equation*}
\lambda_1=\begin{pmatrix} 
1& 0& 0& 0\\
-2/3& 1& 0& 0\\
-3/3& 0& 1& 0\\
-5/3& 0& 0& 1
\end{pmatrix}
\end{equation*},
El producto $\lambda_1 \cdot A$ da como resultado la matriz,

\begin{equation*} 
 U_1=\begin{pmatrix}
3& 4& 2&5\\
0& -2.66& -0.33& -5.33\\
0& -2& -1& -3\\
0& -4.66& -0.33& -6.33
\end{pmatrix}=\begin{pmatrix} 
1& 0& 0& 0\\
-2/3& 1& 0& 0\\
-3/3& 0& 1& 0\\
-5/3& 0& 0& 1
\end{pmatrix}\cdot \begin{pmatrix}
3& 4& 2&5\\
2& 0& 1& -2\\
3& 2& 1& 8\\
5& 2& 3& 2
\end{pmatrix}
\end{equation*}

De modo an�logo, $U_2=\lambda_2 \cdot U_1$
 
\begin{equation*}
\lambda_2=\begin{pmatrix} 
1& 0& 0& 0\\
0& 1& 0& 0\\
0& -2/2.66& 1& 0\\
0& -4.66/2.66& 0& 1
\end{pmatrix}
\end{equation*}

\begin{equation*}
U_2= \begin{pmatrix}
3& 4& 2&5\\
0& -2.6& -0.33& -5.33\\
0& 0& -0.75& 7\\
0& 0& 0.25& 3
\end{pmatrix}
\end{equation*}

Por �ltimo, $U=\lambda_3 \cdot U_2$

\begin{equation*}
 \lambda_3 =\begin{pmatrix}
1& 0& 0&0\\
0& 1& 0& 0\\
0& 0& 1& 0\\
0& 0& 0.25/0.75 & 1
\end{pmatrix}
\end{equation*}

\begin{equation*}
 U =\begin{pmatrix}
3& 4& 2&5\\
0& -2.66& -0.33& -5.33\\
0& 0& -0.75& 7\\
0& 0& 0& 5.33
\end{pmatrix}
\end{equation*}

De nuevo, podemos generalizar el procedimiento empleado; cada matriz $\lambda_j$ \emph{elimina} todos los elementos de la columna $n$ de una matriz $A$ , situados por debajo de la diagonal. La  matriz $\lambda_j$  toma la forma general,

\begin{equation*}
\lambda_j=\begin{pmatrix}
1& \cdots & 0& 0& 0& \cdots & 0\\
 \vdots &  &  \vdots & \vdots &  \vdots & & \vdots\\
0& \cdots & 1& 0& 0& \cdots & 0\\
0& \cdots & 0& 1& 0& \cdots & 0\\
0& \cdots &0 & -a_{j+1,j}/a_{jj}& 0& \cdots & 0\\
0& \cdots &0 & -a_{j+2,j}/a_{jj}& 1& \cdots & 0\\
 \vdots &  &  \vdots & \vdots &  \vdots & & \vdots\\
0& \cdots & 0& -a_{n,j}/a_{jj}& 0&\cdots & 1\\
\end{pmatrix}
\end{equation*} 
Solo los elementos de la diagonal, que toman el valor $1$, y los elementos de la columna $j$ situados por debajo de la diagonal son distintos de cero.

A partir de la definici�n de las matrices $\lambda_j$, podemos obtener una relaci�n entre la matriz triangular superior $U$  obtenida al final del proceso de eliminaci�n, y la matriz inicial $A$ de orden $n\times n$ para ello, en cada paso multiplicamos tanto por $\lambda_j$ como por su inversa $\lambda_j^{-1}$,
\begin{align*}
A&=\lambda_1^{-1}\cdot \overbrace{\lambda_1\cdot A}^{U_1}\\
A&=\lambda_1^{-1}\cdot\lambda_2^{-1} \cdot \overbrace{\lambda_2  \cdot \lambda_1 A}^{U_2}\\
A&=\lambda_1^{-1}\cdot \lambda_2^{-1}\cdot \overbrace{ \lambda_3^{-1}\cdot  \lambda_3 \cdot  \lambda_2\cdot \lambda_1 \cdot A}^{U_3}\\
\vdots \\
A&=\lambda_1^{-1} \cdot  \lambda_2^{-1}\cdot  \lambda_3^{-1} \cdots  \lambda_n ^{-1}\cdot \overbrace{\lambda_n \cdots \lambda_3 \cdot \lambda_2\cdot \lambda_1  \cdot A}^{U}
\end{align*}

Las matrices $\lambda_j^{-1}$ tienen dos propiedades que las hacen particularmente f�ciles de manejar:  la primera es que cumplen que  su inversa  $\lambda_j^{-1}$ puede obtenerse a partir de $\lambda_j$, sin m�s que cambiar de signo los elementos distintos de cero que no pertenecen a la diagonal (Compru�balo),
\begin{equation*}
 \lambda_j^{-1}=\begin{pmatrix}
1& \cdots & 0& 0& 0& \cdots & 0\\
 \vdots &  &  \vdots & \vdots &  \vdots & & \vdots\\
0& \cdots & 1& 0& 0& \cdots & 0\\
0& \cdots & 0& 1& 0& \cdots & 0\\
0& \cdots &0 & a_{j+1j}/a_{jj}& 0& \cdots & 0\\
0& \cdots &0 & a_{j+2j}/a_{jj}& 1& \cdots & 0\\
 \vdots &  &  \vdots & \vdots &  \vdots & & \vdots\\
0& \cdots & 0& a_{nj}/a_{jj}& 0&\cdots & 1\\
\end{pmatrix}
\end{equation*}

La segunda propiedad es que el producto $L=\lambda_1^{-1} \cdot  \lambda_2^{-1}\cdot  \lambda_3^{-1} \cdots  \lambda_n ^{-1}$, se puede obtener progresivamente, a la vez que se construye $U$, sin m�s que ir juntando en una �nica matriz $L$ las columnas de $\lambda_1^{-1}$, $\lambda_2^{-1}$, etc. que contienen elementos no nulos, en nuestro ejemplo,
\begin{equation*}
 L =\begin{pmatrix}
1& 0& 0&0\\
2/3& 1& 0& 0\\
3/3& 2/2.66& 1& 0\\
5/3& 4.66/2.66& -0.25/0.75 & 1
\end{pmatrix}
\end{equation*}

y, en general,

\begin{equation*}
 L=\begin{pmatrix}
1& \cdots & 0& 0& 0& \cdots & 0\\
 \vdots &  &  \vdots & \vdots &  \vdots & & \vdots\\
a_{j-11}/a_{11}& \cdots & 1& 0& 0& \cdots & 0\\
a_{j1}/a_{11}& \cdots & a_{jj-1}/a_{j-1 j-1}& 1& 0& \cdots & 0\\
a_{j+11}/a_{11}& \cdots &a_{j+1j-1}/a_{j-1 j-1} & a_{j+1j}/a_{jj}& 0& \cdots & 0\\
a_{j+11}/a_{11}& \cdots &a_{j+2j-1}/a_{j-1 j-1} & a_{j+2j}/a_{jj}& 1& \cdots & 0\\
 \vdots &  &  \vdots & \vdots &  \vdots & & \vdots\\
a_{n1}/a_{11}& \cdots & a_{nj-1}/a_{j-1j-1}& a_{nj}/a_{jj}& a_{nj+1}/a_{j+1j+1}&\cdots & 1\\
\end{pmatrix}
\end{equation*}

Por construcci�n,  la matriz $L$ es una matriz inferior, Con lo que quedar�a completo el c�lculo de la factorizaci�n LU,

\begin{equation*}
A=\overbrace{\lambda_1^{-1} \cdot  \lambda_2^{-1}\cdot  \lambda_3^{-1} \cdots  \lambda_n ^{-1}}^{L}\cdot \overbrace{\lambda_n \cdots \lambda_3 \cdot \lambda_2\cdot \lambda_1  \cdot A}^{U}=L\cdot U
\end{equation*}

La factorizaci�n que acabamos de describir, puede presentar problemas num�ricos dependiendo del c�mo sea la matriz que se desea factorizar. El primer problema se produce cuando el elemento de la diagonal de $u_{jj}$ por el que toca dividir para eliminar los elementos de la columna $j$, situados por debajo de la diagonal, es $0$. En ese caso el ordenador dar� un error de desbordamiento y no se podr� seguir factorizando. El segundo problema surge cuando los elementos de la matriz son dispares en magnitud; las operaciones matem�ticas realizadas durante el proceso de factorizaci�n pueden dar lugar a errores de redondeo importantes que hagan incorrecto el resultado de la factorizaci�n. Veamos un ejemplo un tanto extremo,

\begin{equation*}
\begin{pmatrix}
10^{-20}& 1\\
1& 1 
\end{pmatrix}=\overbrace{\begin{pmatrix}
1& 0\\
10^{20}& 1 
\end{pmatrix}}^{L}\cdot \overbrace{\begin{pmatrix}
1& 0\\
-10^{20}& 1
\end{pmatrix}\cdot \begin{pmatrix}
10^{-20}& 1\\
1& 1
\end{pmatrix}}^{U}= \overbrace{\begin{pmatrix}
1& 0\\
10^{20}& 1 
\end{pmatrix}}^{L} \cdot \overbrace{\begin{pmatrix}
10^{-20}& 1\\
0& 1 -10^{20} 
\end{pmatrix}}^{U} 
\end{equation*} 

Como el \emph{eps} del ordenador es del orden de $10^{-16}$, $1$ es despreciado frente $10^{20}$. Es decir, $(1-10^{20})\approx 10^{20}$, con lo cual el ordenador tendr� una versi�n aproximada de $U$

\begin{equation*}
U \approx \hat{U}= \begin{pmatrix}
10^{-20}& 1\\
0& -10^{20} 
\end{pmatrix} 
\end{equation*}

Pero si ahora volvemos a multiplicar $L\cdot U$  para recuperar $A$,
\begin{equation*}
L \cdot \hat{U}=  \begin{pmatrix}
1& 0\\
10^{20}& 1 
\end{pmatrix} \cdot \begin{pmatrix}
10^{-20}& 1\\
0& -10^{20} 
\end{pmatrix}= \begin{pmatrix}
10^{-20}& 1\\
1& 0
\end{pmatrix} \neq A
\end{equation*}

Una manera de paliar los efectos de redondeo, es reordenar las filas de la matriz de modo que el elemento $a_{jj}$ por el que se va a dividir los elementos de la fila $j$ en el proceso de eliminaci�n sea lo mayor posible. Este procedimiento se conoce con el nombre de pivoteo de filas.
Para el ejemplo que acabamos de examinar, supongamos que cambiamos de orden las dos filas de    la matriz,

\begin{equation*}
\begin{pmatrix}
10^{-20}& 1\\
1& 1 
\end{pmatrix} \rightarrow \begin{pmatrix}
1& 1\\
10^{-20}& 1 
\end{pmatrix}
\end{equation*}

Si recalculamos la factorizaci�n LU, para la nueva matriz con la filas intercambiadas,

\begin{equation*}
\begin{pmatrix}
1& 1\\
10^{-20}& 1 
\end{pmatrix}=\overbrace{\begin{pmatrix}
1& 0\\
10^{-20}& 1 
\end{pmatrix}}^{L}\cdot \overbrace{\begin{pmatrix}
1& 0\\
-10^{-20}& 1
\end{pmatrix}\cdot \begin{pmatrix}
1& 1\\
10^{-20}& 1 
\end{pmatrix}}^{U}= \overbrace{\begin{pmatrix}
1& 0\\
10^{-20}& 1 
\end{pmatrix}}^{L} \cdot \overbrace{\begin{pmatrix}
1& 1\\
0&  1-10^{-20}
\end{pmatrix}}^{U} 
\end{equation*}

De nuevo, por errores de redondeo el ordenador tendr� una version aproximada de $U$.

\begin{equation*}
U \approx \hat{U}= \begin{pmatrix}
1& 1\\
0& 1 
\end{pmatrix} 
\end{equation*}

Sin embargo si recalculamos el producto $L\cdot \hat{U}$ y volvemos a reordenar las filas del resultado,
\begin{equation*}
L \cdot \hat{U}=  \begin{pmatrix}
1& 0\\
10^{-20}& 1 
\end{pmatrix} \cdot \begin{pmatrix}
1& 1\\
0& 1 
\end{pmatrix}= \begin{pmatrix}
1& 1\\
10^{-20}& 1+10^{-20}
\end{pmatrix} \rightarrow \begin{pmatrix}
10^{-20}& 1+10^{-20}\\
1& 1
\end{pmatrix} \approx A
\end{equation*}

La permutaci�n de las filas de una matriz $A$ de orden $n\times m$, se puede definir a partir del producto con las matrices de permutaci�n de orden $n \times n$. �stas se obtienen permutando directamente las filas de la matriz identidad $I_{n \times n}$. Si una matriz de permutaci�n multiplica a otra matriz por la izquierda, permuta el orden de sus filas. Si la multiplica por la derecha. permuta el orden de sus columnas. As� por ejemplo, para matrices de orden $3 \times n$,

\begin{equation*}
I_{n \times n}=\begin{pmatrix}
1& 0& 0\\
0& 1& 0\\
0& 0& 1\\
\end{pmatrix} \rightarrow P_{1 \leftrightarrow 3}= \begin{pmatrix}
0& 0& 1\\
0& 1& 0\\
1& 0& 0\\
\end{pmatrix}
\end{equation*}

Si multiplicamos $P_{1 \leftrightarrow 3}$ con cualquier otra matriz $A$ de orden $3\times n$, El resultado es equivalente a intercambiar en la matriz $A$ la fila $1$ con la $3$. Por ejemplo
   
\begin{equation*}
P_{1\leftrightarrow 3}\cdot A=\begin{pmatrix}
0& 0& 1\\
0& 1& 0\\
1& 0& 0\\
\end{pmatrix} \cdot \begin{pmatrix}
1& 2& 5& 3\\
4& 2& 3& 0\\
3& 6& 2& 1
\end{pmatrix} = \begin{pmatrix}
3& 6& 2& 1\\
4& 2& 3& 0\\
1& 2& 5& 3
\end{pmatrix}= A_{1\leftrightarrow 3}
\end{equation*}

Volvamos al c�lculo de la factorizaci�n LU, pero ahora empleando pivoteo de filas. Supongamos una matriz $A$ de orden $n\times n$

El primer paso, es buscar el elemento mayor en valor absoluto de la primera columna en intercambia la primera fila de la matriz de $A$, con la fila que contiene dicho elemento. Si utilizamos la matriz de permutaci�n adecuada, esto puede expresarse como,

\begin{equation*}
A \rightarrow P_1\cdot A
\end{equation*}
A continuaci�n eliminamos los elementos de la primera fila situados por debajo de la diagonal,
\begin{equation*}
A \rightarrow P_1\cdot A \rightarrow \lambda_1 \cdot P_1 \cdot A
\end{equation*}
Volvemos a buscar el mayor elemento en valor absoluto para la segunda columna (Solo desde la diagonal hasta el ultimo elemento de la columna).  
\begin{equation*}
A \rightarrow P_1\cdot A \rightarrow \lambda_1 \cdot P_1 \cdot A \rightarrow  P_2 \cdot \lambda_1 \cdot P_1 \cdot A 
\end{equation*}

Eliminamos los elementos de la segunda fila situados por debajo de la diagonal,

\begin{equation*}
A \rightarrow P_1\cdot A \rightarrow \lambda_1 \cdot P_1 \cdot A \rightarrow  P_2 \cdot \lambda_1 \cdot P_1 \cdot A \rightarrow \lambda_2 \cdot  P_2 \cdot \lambda_1 \cdot P_1 \cdot A
\end{equation*}

Si seguimos todo el proceso hasta eliminar los elementos situados por debajo de la diagonal en la fila $n-1$, obtendremos la expresi�n de la matriz triangular superior $U$ resultante,

\begin{equation*}
\lambda_{n-1}P_{n-1}\lambda_{n-2}P_{n-2} \cdots  \lambda_2  P_2 \lambda_1  P_1 A = U
\end{equation*}

Aunque hemos obtenido $U$, necesitamos una manera de obtener $L$, ahora no es inmediato como en el caso de la factorizaci�n sin pivoteo, no basta con invertir las matrices $\lambda_i$ ya que tenemos por medio todas las matrices de permutaci�n utilizadas. Para obtenerla realizaremos la siguiente transformaci�n, 

\begin{equation*}
\lambda_{n-1}P_{n-1}\lambda_{n-2}P_{n-2} \cdots  \lambda_2  P_2 \lambda_1  P_1 A = \lambda_{n-1}'\lambda_{n-2}'\cdots \lambda_2'   \lambda_1' P_{n-1}P_{n-2} \cdots P_2  P_1 A
\end{equation*}

Donde las matrices $\lambda_i'$ pueden obtenerse a partir de las matrices $\lambda_i$ y de las matrices de permutaci�n de la manera siguiente,

\begin{align*}
\lambda_{n-1}' &= \lambda_{n-1} \\
\lambda_{n-2}' &= P_{n-1}\lambda_{n-2} P_{n-1} ^{-1}\\
\lambda_{n-3}' &= P_{n-1}P_{n-2}\lambda_{n-3}P_{n-2}^{-1} P_{n-1}^{-1} \\
 & \vdots \\
 \lambda_{k\ \ }'  &= P_{n-1}P_{n-2}\cdots P_{k+1} \lambda_{k}P_{k+1}^{-1} \cdots P_{n-2}^{1} P_{n-1}^{-1} \\
 & \vdots \\
\lambda_{2\ \ }'  &= P_{n-1}P_{n-2}\cdots P_3 \lambda_{2}P_3^{-1} \cdots P_{n-2}^{1} P_{n-1}^{-1} \\
\lambda_{1\ \ }'  &= P_{n-1}P_{n-2}\cdots P_3P_2 \lambda_{1}P_2^{-1}P_3^{-1} \cdots P_{n-2}^{1} P_{n-1}^{-1} \\
\end{align*}


Matem�ticamente el c�lculo de las matrices $\lambda_i'$ requiere calcular el producto de un gran n�mero de matrices.  Sin embargo dicho c�lculo es equivalente a permutar los elementos de $\lambda_i$ situados por debajo de la diagonal.   As� Por ejemplo para,

\begin{equation*}
\lambda_2=
\begin{pmatrix}
1& 0& 0& 0\\
0& 1& 0& 0\\
0& 2& 1& 0\\
0& 3& 0& 1
\end{pmatrix}, \ 
P_3=
\begin{pmatrix}
1& 0& 0& 0\\
0& 1& 0& 0\\
0& 0& 0& 1\\
0& 0& 1& 0
\end{pmatrix}
\end{equation*}

\begin{equation*}
\lambda_2'=P_3\cdot \lambda_2 P_3^{-1}
\begin{pmatrix}
1& 0& 0& 0\\
0& 1& 0& 0\\
0& 0& 0& 1\\
0& 0& 1& 0
\end{pmatrix}\cdot \begin{pmatrix}
1& 0& 0& 0\\
0& 1& 0& 0\\
0& 2& 1& 0\\
0& 3& 0& 1
\end{pmatrix} \cdot \begin{pmatrix}
1& 0& 0& 0\\
0& 1& 0& 0\\
0& 0& 0& 1\\
0& 0& 1& 0
\end{pmatrix} = \begin{pmatrix}
1& 0& 0& 0\\
0& 1& 0& 0\\
0& 3& 1& 0\\
0& 2& 0& 1
\end{pmatrix}
\end{equation*}

Por �ltimo podemos representar el producto de todas las matrices de permutaci�n como una sola matriz, $P_{n-1}P_{n-2} \cdots P_2  P_1=P$. (El producto de matrices de permutaci�n entre s�, da como resultado una nueva matriz de permutaci�n.

 
De esta manera, la factorizaci�n LU con pivoteo de filas podr�amos representarla como $P\cdot A=L\cdot U$,

\begin{equation*}
\overbrace{P_{n-1}P_{n-2} \cdots P_2  P_1}^{P}\cdot  A= \overbrace{\lambda_1'^{-1} \lambda_2'^{-2} \cdots \lambda_{n-2}'^{-1} \lambda_{n-1}'^{-1}}^{L}\cdot \overbrace{\lambda_{n-1}P_{n-1}\lambda_{n-2}P_{n-2} \cdots  \lambda_2  P_2 \lambda_1  P_1 A }^{U}   
\end{equation*}

Como ejemplo, vamos a volver a calcular la factorizaci�n LU de la matriz,
 \begin{equation*}
A=\begin{pmatrix}
3& 4& 2&5\\
2& 0& 1& -2\\
3& 2& 1& 8\\
5& 2& 3& 2
\end{pmatrix} 
\end{equation*}
 Pero empleando ahora pivoteo de filas.
 
En primer lugar buscamos  el elemento de la primera columna mayor en valor absoluto. En esta caso es el �ltimo elemento de la columna por lo que intercambiamos la primera fila con la cuarta,
 
 
  \begin{equation*}
P_1\cdot A=\begin{pmatrix}
0& 0& 0& 1\\
0& 1& 0& 0\\
0& 0&1& 0\\
1& 0& 0& 0
\end{pmatrix}\begin{pmatrix}
3& 4& 2&5\\
2& 0& 1& -2\\
3& 2& 1& 8\\
5& 2& 3& 2
\end{pmatrix}= \begin{pmatrix}
5& 2& 3& 2\\
2& 0& 1& -2\\
3& 2& 1& 8\\
3& 4& 2&5
\end{pmatrix}
\end{equation*}

Eliminamos ahora los elementos de la primera columna situados por debajo de la diagonal,

\begin{equation*}
\overbrace{\lambda_1 \cdot P_1\cdot A}^{U_1}=\begin{pmatrix}
1& 0& 0& 0\\
-2/5& 1& 0& 0\\
-3/5& 0& 1& 0\\
-3/5& 0& 0&1
\end{pmatrix} \cdot \begin{pmatrix}
5& 2& 3& 2\\
2& 0& 1& -2\\
3& 2& 1& 8\\
3& 4& 0&5
\end{pmatrix}= \begin{pmatrix}
5& 2& 3& 2\\
0& -0.8& -0.2& -2.8\\
0& 0.8& -0.8& -6.8\\
0& 2.8& 0.2&3.8
\end{pmatrix}
\end{equation*}

Para eliminar los elementos de la segunda columna, repetimos los mismos pasos. El elemento mayor de la segunda columna de $U_1$ es ahora $2.8$, por tanto intercambiamos entre s� la segunda y la cuarta fila,

\begin{equation*}
P_2\cdot \overbrace{\lambda_1 \cdot P_1\cdot A}^{U_1}=\begin{pmatrix}
1& 0& 0& 0\\
0& 0& 0& 1\\
0& 0& 1& 0\\
0& 1& 0&0
\end{pmatrix} \cdot \begin{pmatrix}
5& 2& 3& 2\\
0& -0.8& -0.2& -2.8\\
0& 0.8& -0.8& -6.8\\
0& 2.8& 0.2&3.8
\end{pmatrix}= \begin{pmatrix}
5& 2& 3& 2\\
0& 2.8& 0.2&3.8\\
0& 0.8& -0.8& -6.8\\
0& -0.8& -0.2& -2.8
\end{pmatrix}
\end{equation*}
A continuaci�n, eliminamos los elementos de la segunda columna situados por debajo de la diagonal,

\begin{equation*}
\overbrace{\lambda_2\cdot P_2\cdot \lambda_1 \cdot P_1\cdot A}^{U_2}=\begin{pmatrix}
1& 0& 0& 0\\
0& 1& 0& 0\\
0& -0.8/2.8& 1& 0\\
0& 0.8/2.8& 0&1
\end{pmatrix} \cdot \begin{pmatrix}
5& 2& 3& 2\\
0& 2.8& 0.2&3.8\\
0& 0.8& -0.8& -6.8\\
0& -0.8& -0.2& -2.8
\end{pmatrix} = \begin{pmatrix}
5& 2& 3& 2\\
0& 2.8& 0.2&3.8\\
0& 0& -0.85& -5.71\\
0& 0& -0.14& -1.71
\end{pmatrix}
\end{equation*}

Para eliminar el elemento que queda debajo de la diagonal en la tercera columna, procedemos igual que para las columnas anteriores. Como en este caso, el elemento situado en la diagonal es mayor en valor absoluto no es preciso permutar. (Desde un punto de vista formal, podr�amos decir que en este caso la matriz de permutaciones aplicada es la matriz identidad).

\begin{align*}
\overbrace{\lambda_3\cdot\lambda_2\cdot P_2\cdot \lambda_1 \cdot P_1\cdot A}^{U}&=\begin{pmatrix}
1& 0& 0& 0\\
0& 1& 0& 0\\
0& 0& 1& 0\\
0& 0& -0.14/0.85&1
\end{pmatrix} \cdot \begin{pmatrix}
5& 2& 3& 2\\
0& 2.8& 0.2&3.8\\
0& 0& -0.85& -5.71\\
0& 0& -0.14& -1.71
\end{pmatrix}\\ 
&= \begin{pmatrix}
5& 2& 3& 2\\
0& 2.8& 0.2&3.8\\
0& 0& -0.85& -5.71\\
0& 0& 0& -2.66
\end{pmatrix}= U 
\end{align*}

Como hemos visto, la matriz $\lambda_i'$ las obtenemos a  partir de $\lambda_i$, permutando los elementos situados por debajo de la diagonal, distintos de cero. En nuestro ejemplo solo hay matriz $\lambda_1$ afectada por una �nica permutaci�n $P_2$,
\begin{equation*}
\lambda'_1=\begin{pmatrix}
1& 0& 0& 0\\
3/5& 1& 0& 0\\
3/5& 0& 1& 0\\
2/5& 0& 0& 1
\end{pmatrix}
\end{equation*}

Para el resto, $\lambda_2'=\lambda_2$ y $\lambda'_3=\lambda_3$. Para obtener la matriz L cambiando los signos a los elementos situados por debajo de la diagonal de $\lambda'_1$, $\lambda_2$ y $\lambda_3$, y los agrupamos en una �nica matriz,

\begin{equation*}
L=\begin{pmatrix}
1& 0& 0& 0\\
3/5& 1& 0& 0\\
3/5& 0.8/2.8& 1& 0\\
2/5& -0.8/2.8& 0.14/0.85& 1
\end{pmatrix}
\end{equation*}

Por �ltimo debemos multiplicar las matrices de permutaciones para agruparlas en una sola,

\begin{equation*}
P=P_2\cdot P_1= \begin{pmatrix}
0& 0& 0& 1\\
1& 0& 0& 0\\
0& 0& 1& 0\\
0& 1& 0& 0
\end{pmatrix}
\end{equation*}

Es trivial comprobar que las matrices obtenidas cumplen \footnote{Los valores num�ricos empleados en el texto, est�n redondeados, por tanto dan una soluci�n incorrecta. Si se reproducen las operaciones descritas en Matlab el resultado es mucho m�s preciso.} $P\cdot A = L\cdot U$.

El siguiente c�digo calcula la factorizaci�n LU de una matriz de orden arbitrario con pivoteo de filas. \label{lufact} 

\begin{lstlisting}
function [L,U,P]=lufact(A)
% este programa calcula una factorizacionLU, de la matrix A,  empleando 
% el m�todo de eliminacion gaussiana con pivoteo de filas...
% L es una matriz triangular inferior, U es una matriz triangular superior y
% P es una matriz de permutaciones tal que P*A=L*U

% vamos a aprovechar la potencia de Matlab para evitar algunos buques en
% los calculos..

% primero definimos la matriz de permutaciones como la matriz identidad de
% orden igual al n�mero de filas de la matriz A, (si no hay pivoteo de filas 
% la matriz de permutaciones es precisamente la identidad)

t=size(A,1); % obtenemos el numero de filas de la matriz A
P=eye(t); % Construimos la matriz identidad 
% Ademas, inicializamos las matrices L y U
L=P;
U=A;


% iniciamos un bucle para ir recorriendo las columnas (solo tenemos que
% recorrer tantas columnas como fila - 1 tenga la matriz A
for j=1:t-1
    LA=zeros(t); %Matriz auxiliar (Lambda) de cada iteraci�n.
    % Buscamos el elemento mas grande de la columna j
    maxcol=abs(U(j,j)); 
    index=j;
    for i=j:t
        if abs(U(i,j))>maxcol
            maxcol=abs(U(i,j));
            index=i;
        end
    end
    
    % reordenamos las filas de P, L y U de modo que el valor mas grande de
    % j pase a acupar la diagonal (Esto es equivalente a multiplicar U por
    % la matriz de permetuciones correspondiente, ir calculado
    % iterativamente el valor de la matriz de permutaciones P final, y
    % reordenar las filas ya calculadas de L (aportadas por los valores de
    % las matrices lambdas anteriores... a lambda_j)
    
    Aux=U(j,:);
    Aux2=P(j,:);
    Aux3=L(j,1:j-1);
    
    % Reordenamos U (Toda la fila)
    U(j,:)=U(index,:);
    U(index,:)=Aux;
    
    % Reordenamos L (Solo los elementos de la columnas anteriores...situados
    % por debajo de la diagonal).
    L(j,1:j-1)=L(index,1:j-1);
    L(index,1:j-1)=Aux3;
    
    % modificamos la matriz de permutaciones
    P(j,:)=P(index,:);
    P(index,:)=Aux2;
    
   % Calculamos una matriz auxiliar con los factores de los elementos
   % a eliminar. 
   LA(j+1:t,j)=U(j+1:t,j)/U(j,j); % estos elementos son directamente los que 
   % se a�aden a la matriz L
   % Modificamos L y U
   L(j+1:t,j)=U(j+1:t,j)/U(j,j);
   U=(eye(t)-LA)*U; % la expresi�n eye(t)-LA nos permite cambiar de signo los
   % elementos situados por debajo de la diagonal principal...
end
\end{lstlisting}


Desde el punto de vista de la implementaci�n, el c�digo anterior simplifica el c�lculo de la factorizaci�n LU,

\begin{enumerate}
\item  no se calculan la inversa de $\lambda_i$. En realidad lo que se hace es ir construyendo iterativamente la matriz L: en cada iteraci�n, primero se permutan los elementos de las columnas de la matriz L construida en la iteraci�n anterior, y se a�aden los elementos de la columna correspondiente.
\item la matriz de permutaci�n se va calculando tambi�n iterativamente, intercambiando las filas de la matriz de permutaci�n obtenida en la iteraci�n anterior.
\end{enumerate}

Matlab posee un comando propio para calcular la factorizaci�n LU de una Matriz, \texttt{[L,U,P]=lu(A)}. Es importante pedir siempre que devuelva la matriz \texttt{P}, en otro caso, la matriz \texttt{L} devuelta por Matlab no tiene por que ser triangular inferior. 

El comando \texttt{lu} de Matlab, permite calcular la factorizaci�n LU, por otros m�todos distintos a la eliminaci�n gaussiana con pivoteo de filas. La explicaci�n de estos otros m�todos queda fuera del alcance de estos apuntes. Simplemente indicar que la factorizaci�n LU de una matriz no es �nica --- es posible encontrar distintos pares L y U que factorizan a la misma matriz A.

\subsection{Factorizaci�n de Cholesky}\label{chol}
Dada una matriz cuadrada, sim�trica y definida positiva, es siempre posible factorizarla como $A=L\cdot L^T$. Donde $L$ es una matriz triangular inferior,

\begin{equation*}
\begin{pmatrix}
a_{11}& a_{12}& \cdots & a_{1n}\\
a_{12}& a_{22}& \cdots & a_{2n}\\
\vdots & \vdots & \ddots & \vdots\\
a_{1n}& a_{2n}& \cdots & a_{nn}
\end{pmatrix}=\begin{pmatrix}
L_{11}& 0& \cdots & 0\\
L_{21}& L_{22}& \cdots & 0\\
\vdots & \vdots & \ddots & \vdots\\
L_{n1}& L_{n2}& \cdots & L_{nn}
\end{pmatrix} \cdot \begin{pmatrix}
L_{11}& L_{21}& \cdots & L_{n1}\\
0& L_{22}& \cdots & L_{n2}\\
\vdots & \vdots & \ddots & \vdots\\
0& 0& \cdots & L_{nn}
\end{pmatrix}
\end{equation*}

Para obtener la factorizaci�n de Cholesky podemos emplear directamente la definici�n. As� si calculamos $L\cdot L^T$, obtendr�amos $a_{11}=L_{11}\cdot L_{11}=L_{11}^2$,  $a_{22}=L_{21}\cdot L_{21}+L_{22}\cdot L_{22}=L_{21}^2+L_{22}^2$  etc. Es f�cil comprobar que los elementos de la diagonal $a_{ii}$ de la matriz $A$ cumplen,

\begin{equation*}
a_{ii}=\sum_{k=1}^{j}L_{ki}^2=L_{ii}+\sum_{k=1}^{j-1}L_{ki}^2
\end{equation*}
A partir de esta expresi�n podemos despejar $L_{ii}$,
\begin{equation*}
L_{ii}=\sqrt{a_{ii}-\sum_{k=1}^{j-1}L_{ki}^2}
\end{equation*}

Del mismo modo, para los elementos que no pertenecen a la diagonal,

\begin{equation*}
a_{ij}=\sum_{k=1}^{j}L_{ik}\cdot L_{kj}=L_{ij}\cdot L_{jj}+\sum_{k=1}^{j-1}L_{ik}\cdot L_{kj}
\end{equation*}
A partir de esta expresi�n podemos despejar $L_{ii}$,
\begin{equation*}
L_{ij}=\frac{a_{ij}-\sum_{k=1}^{j-1}L_{ik}\cdot L_{kj}}{L_{jj}}
\end{equation*}

Los elementos de la matriz L, se pueden calcular de las expresiones anteriores iterativamente por columnas: para obtener $L_{11}$ basta emplear $a_{11}$, para obtener los restantes elementos de la primera columna, basta conocer $L_{11}$. Conocida la primera columna de L, se puede calcular $L_{22}$ y as� sucesivamente hasta completar toda la matriz. El siguiente c�digo calcula la factorizaci�n de Cholesky de una matriz, siempre que cumpla las condiciones requeridas,

\begin{lstlisting}
function L=cholesky(A)
% hacemos que devuelva la matriz triangular inferior.
% Comprobamos que es cuadrada, simetrica, y definida positiva.
[a,b]=size(A);
if a-b~=0
    disp('La matriz no es cuadrada')
else
    % ha resultado cuadrada comprobamos si es simetrica
    c=A-A';
    if any(c)
        disp('LA matriz no es simetrica')
    else
        % ha resultado simetrica, comprobamos si es definida positiva
        auto=eig(A);
        if any(auto<=0)
            disp('la matriz no es definida positiva')
        else
            % una vez que la matriz cumple las condiciones necesarias,
            % factorizamos por cholesky

            for k=1:a
                L(k,k)=A(k,k);
                for i=1:k-1
                    L(k,k)=L(k,k)-L(k,i)^2;
                end
                L(k,k)=sqrt(L(k,k));
                for j=k+1:a
                    L(j,k)=A(j,k);
                    for i=1:k-1
                        L(j,k)=L(j,k)-L(j,i)*L(k,i);
                    end
                    L(j,k)=L(j,k)/L(k,k);
                end
            end
        end
    end
end

\end{lstlisting} 

Matlab tiene una funci�n interna \texttt{chol}, que permite obtener la factorizaci�n de Cholesky, Por defecto devuelve una matriz triangular superior U, de modo que la factorizaci�n calculada cumple $A=U^T\cdot U$. (En realidad se trata tan solo de una forma alternativa de definir la factorizaci�n de Cholesky), As� por ejemplo,

\begin{verbatim}
>> A
A =
     6     3     4
     3    14     3
     4     3     5
>> U=chol(A)
U =

    2.4495    1.2247    1.6330
         0    3.5355    0.2828
         0         0    1.5011
>> R=U'*U
R =
    6.0000    3.0000    4.0000
    3.0000   14.0000    3.0000
    4.0000    3.0000    5.0000
\end{verbatim}
Si se quiere obtener directamente la matriz L, hay que indicarlo expresamente: \texttt{L=chol(A,'lower')}.
Para obtener la factorizaci�n Matlab emplea SOLO la parte triangular superior o la triangular inferior de la matriz A. Supone que A es sim�trica pero NO lo comprueba.
Si la matriz no es definida positiva, la funci�n chol da un mensaje de error.

\subsection{Diagonalizaci�n}\label{sec:diag}
\paragraph{Autovectores y autovalores.} Dada una matriz $A$ de orden $n\times n$, se define como autovector, o vector propio de dicha matriz al vector $x$ que cumple,
\begin{equation*}
A\cdot x =\lambda \cdot x, \  x\neq 0, \ \lambda \in \mathbb{C}
\end{equation*}

Es decir, el efecto de multiplicar la matriz $A$ por el vector $x$ es equivalente a multiplicar el vector $x$  por un n�mero $\lambda$ que, en general, ser� un n�mero complejo.

$\lambda$ recibe el nombre de autovalor, o valor propio de la matriz $A$. As� por ejemplo,
\begin{equation*}
A= \begin{pmatrix}
-2& 0& 0\\
0& 2 & -1\\
0& 0& 3
\end{pmatrix}
\end{equation*}
tiene un autovalor $\lambda=3$ para el vector propio, $x=[0,\  -3,\  3]^T$ , 
\begin{equation*}
\begin{pmatrix}
-2& 0& 0\\
0& 2 & -1\\
0& 0& 3
\end{pmatrix}\cdot \begin{pmatrix}
0\\
-3\\
3
\end{pmatrix}=3\cdot \begin{pmatrix}
0\\
-3\\
3
\end{pmatrix}=\begin{pmatrix}
0\\
-9\\
9
\end{pmatrix}
\end{equation*}

El vector propio asociado a un valor propio no es �nico, si $x$ es un vector propio de una matriz $A$, asociado a un valor propio $\lambda$, Es trivial comprobar que cualquier vector de la forma $\alpha\cdot x, \ \alpha \in \mathbf{C}$ tambi�n es u vector propio asociado a $\lambda$,
\begin{equation}
A\cdot x= \lambda x \Rightarrow A \cdot \alpha x = \lambda  \alpha x
\end{equation}

En realidad, cada vector propio expande un subespacio $E_{\lambda}S$ de $\mathbf{R}^n$. Aplicar la matriz $A$, a un vector de $E_{\lambda}S$ es equivalente a multiplicar el vector por $\lambda$. Cada subespacio asociado a un autovalor recibe el nombre de autosubespacio o subespacio propio. 

El conjunto de todos los autovalores de una matriz $A$ recibe el nombre de espectro de $A$ y se representa como $\Lambda(A)$.
Una descomposici�n en autovalores de una matriz $A$ se define como.
\begin{equation*}
A=X\cdot D \cdot X^{-1}
\end{equation*}

Donde $D$ es una matriz diagonal formada por los autovalores de $A$. 

\begin{equation*}
\begin{pmatrix}
\lambda_1& 0 & \cdots & 0\\
0& \lambda_2& \cdots & 0\\
\vdots & \vdots& \cdots & 0\\
0& 0& \cdots & \lambda_n
\end{pmatrix}
\end{equation*}

Esta descomposici�n no siempre existe. En general, si dos matrices $A$ y $B$ cumplen,
\begin{equation*}
A=X\cdot B \cdot X^{-1}
\end{equation*}
se dice de ellas que son similares. A la transformaci�n que lleva a convertir $B$ en $A$ se le llama una transformaci�n de semejanza. Una matriz es diagonalizable cuando admite una transformaci�n de semejanza con una matriz diagonal de autovalores. 

Volviendo al ejemplo anterior podemos factorizar la matriz $A$ como,

\begin{equation*}
A= \begin{pmatrix}
-2& 0& 0\\
0& 2 & -1\\
0& 0& 3
\end{pmatrix}=\begin{pmatrix}
1& 0& 0\\
0& -1& -3\\
0& 0& 3
\end{pmatrix}\cdot \begin{pmatrix}
-2& 0& 0\\
0& 2& 0\\
0& 0& 3
\end{pmatrix} \cdot \begin{pmatrix}
1& 0& 0\\
0& -1& -1\\
0& 0& 1/3
\end{pmatrix}
\end{equation*}

Por tanto en este ejemplo, los autovalores de $A$ ser�an $\lambda_1=-2$, $\lambda_2=2$ y $\lambda_3=3$. Para estudiar la composici�n de la matriz $X$ Reescribimos la expresi�n de la relaci�n de semejanza de la siguiente manera,
\begin{equation*}
 A=X \cdot D \cdot X^{-1}\rightarrow A\cdot X= X\cdot D
\end{equation*}

Si consideramos ahora cada columna de la matriz $X$ como un vector,
\begin{equation*}
A\cdot X= X\cdot D \rightarrow A\cdot \left( x_1 | x_2 | \cdots | x_n \right)= \left( x_1 | x_2 | \cdots | x_n \right) \cdot\begin{pmatrix}
\lambda_1& 0 & \cdots & 0\\
0& \lambda_2& \cdots & 0\\
\vdots & \vdots& \cdots & 0\\
0& 0& \cdots & \lambda_n
\end{pmatrix}
\end{equation*}
Es f�cil comprobar que si consideramos el producto de la matriz $A$, por cada una de las columnas de la matriz $X$ se cumple,

\begin{align*}
A\cdot x_1 &=\lambda_1 \cdot x_1 \\
A\cdot x_2 &=\lambda_2 \cdot x_2 \\
\vdots \\
A\cdot x_n &=\lambda_n \cdot x_n \\
\end{align*}
Lo que nos lleva a que cada columna de la matriz $X$ tiene que ser un autovector de $A$ puesto que cumple,

\begin{equation*}
A\cdot x_i =\lambda_i \cdot x_i \\
\end{equation*}

\paragraph{Polinomio caracter�stico.} El polinomio caracter�stico de una matriz $A$ de dimensi�n $n\cdot n$, se define como,
\begin{equation*}
P_A(z)=det(zI-A)
\end{equation*}
Donde $I$ es a matriz identidad de dimensi�n $n \cdot n$. As� por ejemplo para una matriz de dimensi�n $2\cdot 2$,

\begin{align*}
P_A(z)=&det\left(z\cdot \begin{pmatrix}
1& 0\\
0& 1
\end{pmatrix}- \begin{pmatrix}
a_{11}& a_{12}\\
a_{21}& a_{22}
\end{pmatrix} \right)=\left\vert \begin{matrix}
z-a_{11}& -a_{12}\\
-a_{21}& z-a_{22}
\end{matrix} \right\vert=\\
\\
=&(z-a_{11})\cdot(z-a_{22})-a_{12}\cdot a_{21}=z^2-(a_{11}+a_{22})\cdot z+a_{11}\cdot a_{22}-a_{12}\cdot a_{21}
\end{align*}

Los autovalores $\lambda$ de una matriz $A$ son las ra�ces del polinomio caracter�stico de la matriz $A$,
\begin{equation*}
p_A(\lambda)=0
\end{equation*}

Las ra�ces de un polinomio pueden ser simples o m�ltiples, es decir una mima ra�z puede repetirse varias veces. Por tanto, los autovalores de una matriz, pueden tambi�n repetirse. Se llama multiplicidad algebraica de un autovalor al n�mero de veces que se repite.

Adem�s las ra�ces de un polinomio pueden ser reales o complejas. Para una matriz cuyos elementos son todos reales, si tiene autovalores complejos estos se dan siempre como pares de n�meros complejos conjugados. Es decir si $\lambda =a+bi$ es un autovalor entonces $\lambda^*=a-bi$ tambi�n lo es.

Veamos algunos ejemplos:

La matriz,
\begin{equation*}
A=\begin{pmatrix}
3& 2\\
-2& -2
\end{pmatrix}
\end{equation*}

Tiene como polinomio caracter�stico,

\begin{equation*}
P_A(z)=\left\vert\begin{matrix}
z-3& -2\\
2& z+2 
\end{matrix} \right\vert=z^2-z-2
\end{equation*}

Igualando el polinomio caracter�stico a cero y obteniendo las ra�ces de la ecuaci�n de segundo grado resultante, obtenemos los autovalores de la matriz $A$,
\begin{equation*}
\lambda^2-\lambda-2=0 \rightarrow \left\{ 
\begin{aligned}
\lambda_1&=2\\
\lambda_2&=-1
\end{aligned}
\right.
\end{equation*}

Un vector propio asociado a $\lambda_1=2$ ser�a $x_1=[2,\  -1]^T$,
\begin{equation*}
\begin{pmatrix}
3& 2\\
-2& -2
\end{pmatrix}\cdot \begin{pmatrix}
2\\
-1
\end{pmatrix}=2\cdot \begin{pmatrix}
2\\
-1
\end{pmatrix} =\begin{pmatrix}
4\\
-2
\end{pmatrix} 
\end{equation*}

y un vector propio asociado a $\lambda_2=-1$ ser�a $x_2=[1\ -2]^T$,

\begin{equation*}
\begin{pmatrix}
3& 2\\
-2& -2
\end{pmatrix}\cdot \begin{pmatrix}
1\\
-2
\end{pmatrix}=-1\cdot \begin{pmatrix}
1\\
-2
\end{pmatrix} =\begin{pmatrix}
-1\\
2
\end{pmatrix} 
\end{equation*}

La matriz,
\begin{equation*}
B=\begin{pmatrix}
4& -1\\
1& 2
\end{pmatrix}
\end{equation*}

Tiene como polinomio caracter�stico,

\begin{equation*}
P_B(z)=\left\vert\begin{matrix}
z-4& -1\\
1& z-2 
\end{matrix} \right\vert=z^2-6z+9
\end{equation*}

procediendo igual que en el caso anterior, obtenemos los autovalores de la matriz $B$,
\begin{equation*}
\lambda^2-6\lambda+9=0 \rightarrow \left\{ 
\begin{aligned}
\lambda_1&=3\\
\lambda_2&=3
\end{aligned}
\right.
\end{equation*}
En este caso, hemos obtenido para el polinomio caracter�stico una ra�z doble, con lo que obtenemos un �nico autovalor $\lambda=3$ de multiplicidad algebraica $2$.

La matriz,

\begin{equation*}
C=\begin{pmatrix}
2& -1\\
1& 2
\end{pmatrix}
\end{equation*}

tiene como polinomio caracter�stico,
\begin{equation*}
P_C(z)=\left\vert\begin{matrix}
z-2& 1\\
-1& z-2 
\end{matrix} \right\vert=z^2-4z+5
\end{equation*}
Con lo que sus autovalores resultan ser dos n�meros complejos conjugados,
\begin{equation*}
\lambda^2-4\lambda+5=0 \rightarrow \left\{ 
\begin{aligned}
\lambda_1&=2+i\\
\lambda_2&=2-i
\end{aligned}
\right.
\end{equation*}

Para que una matriz $A$ de orden $n\times n$ sea diagonalizable es preciso que cada autovalor tenga asociado un n�mero de autovectores linealmente independientes, igual a su multiplicidad algebraica. La matriz $B$ del ejemplo mostrado m�s arriba tiene tan solo un autovector, $x=[1,\ 1]^T$ asociado a su �nico autovalor de multiplicidad $2$. No es posible encontrar otro linealmente independiente; por lo tanto $B$ no es diagonalizable. 

La matriz, 
\begin{equation*}
G=\begin{pmatrix}
3& 0& 1\\
0& 3& 0\\
0& 0& 1
\end{pmatrix}
\end{equation*}

Tiene un autovalor $\lambda_1=3$, de multiplicidad $2$, y un autovalor $\lambda_2=1$ de multiplicidad $1$. Para el autovalor de multiplicidad $2$ es posible encontrar dos autovectores linealmente independientes, por ejemplo: $x_1=[1,\ 0,\ 0]^T$ y $x_2=[0,\ 1,\ 0]^T$. Para el segundo autovalor un posible autovector ser�a, $x_3=[-2,\ 0,\ 1]^T$. Es posible por tanto diagonalizar la matriz $G$,

\begin{equation*}
G=\begin{pmatrix}
3& 0& 1\\
0& 3& 0\\
0& 0& 1
\end{pmatrix}=X\cdot D \cdot X^{-1}=\begin{pmatrix}
1& 0& -2\\
0& 1& 0\\
0& 1& 1
\end{pmatrix}\cdot\begin{pmatrix}
3& 0& 0\\
0& 3& 0\\
0& 0& 1
\end{pmatrix}\cdot \begin{pmatrix}
1& 0& 2\\
0& 1& 0\\
0& 1& 1
\end{pmatrix}
\end{equation*}

\paragraph{Propiedades asociadas a los autovalores.} \label{resp} Damos a continuaci�n, sin demostraci�n, algunas propiedades de los autovalores de una matriz,

\begin{enumerate}
\item La suma de los autovalores de una matriz, coincide con su traza,
\begin{equation*}
\text{tr}(A)=\sum_{i=1}^n \lambda_i
\end{equation*}

\item El producto de los autovalores de una matriz coincide con su determinante,
\begin{equation*}
\left\vert A \right\vert = \prod_{i=1}^n \lambda_i
\end{equation*} 

\item Cuando los autovectores de una matriz $A$ son ortogonales entre s�, entonces la matriz $A$ es diagonalizable ortogonalmente,
\begin{equation*}
A=Q\cdot D \cdot Q^{-1}; \ Q^{-1}=Q^T \Rightarrow A=Q\cdot A \cdot Q^T
\end{equation*}
Cualquier matriz sim�trica posee autovalores reales y es diagonalizable ortogonalmente. En general, una matriz es diagonalizable ortogonalmente si es normal: $A\cdot A^T=A^T\cdot A$

\item El mayor de los autovalores en valor absoluto, de una matriz $A$ de orden $n$.  recibe el nombre de \emph{radio espectral} de dicha matriz,
\begin{equation*}
\rho(A)=\max_{i=1}^n \vert \lambda_i \vert
\end{equation*} 
\end{enumerate}

No vamos a ver ning�n  algoritmo concreto para diagonalizar matrices. Quiz� el mas utilizado, por su robustez num�rica es el basado en la factorizaci�n de Schur: cualquier matriz $A$ puede factorizarse como,
\begin{equation*}
A=Q\cdot T \cdot Q^T
\end{equation*}
Donde $Q$ es una matriz ortogonal y $T$ es una matriz triangular superior. Los elementos de la diagonal de $T$ son precisamente los autovalores de $A$.

Matlab incluye la funci�n \texttt{eig} para calcular los autovectores y autovalores de una matriz. esa funci�n admite como variable de entrada una matriz cuadrada de dimensi�n arbitraria y devuelve como variable de salida un vector columna con los autovalores de la matriz de entrada,

\begin{verbatim}
>> A=[1 2 3;2 3 -2;3 0 1]

A =

     1     2     3
     2     3    -2
     3     0     1

>> Lambda=eig(A)

Lambda =

   -2.7016
    4.0000
    3.7016

\end{verbatim}  
Tambi�n puede llamarse a la funci�n \texttt{eig} con dos variables de salida. En este caso la primera variable de salida es una matriz en la que cada columna representa un autovector de la matriz de entrada normalizado ---de m�dulo $1$---. La segunda variable de salida es una matriz diagonal cuyos elementos son los autovalores de la matriz de entrada.

\begin{verbatim}
>> A=[1 2 3;2 3 -2;3 0 1]

A =

     1     2     3
     2     3    -2
     3     0     1

>> [X,D]=eig(A)

X =

   -0.6967    0.7071    0.6548
    0.4425    0.0000   -0.2062
    0.5647    0.7071    0.7271


D =

   -2.7016         0         0
         0    4.0000         0
         0         0    3.7016
\end{verbatim}

Por �ltimo, Matlab tambi�n incluye una funci�n para calcular la factorizaci�n de Schur. Si lo aplicamos a la misma matriz \texttt{A}, para la que acabamos de calcular los autovalores,

\begin{verbatim}
>> [Q,T]=schur(A)
Q =

   -0.6967    0.6449   -0.3142
    0.4425    0.0415   -0.8958
    0.5647    0.7632    0.3142
\end{verbatim}

\begin{verbatim}
T =
   -2.7016   -0.6285   -1.2897
         0    4.0000   -1.3934
         0         0    3.7016

>> Q*Q'
ans =

    1.0000    0.0000    0.0000
    0.0000    1.0000    0.0000
    0.0000    0.0000    1.0000

\end{verbatim}

Se observa como los elementos de la diagonal de \texttt{T} son los autovalores de \texttt{A}  como la matriz \texttt{Q} es ortogonal.
  
\subsection{Factorizaci�n QR} \label{QR}
La idea es factorizar una matriz $A$ en el producto de dos matrices; una matriz ortogonal $Q$ y una matriz triangular superior R.
\begin{equation*}
A=Q\cdot R, \leftarrow Q\cdot Q^T=I
\end{equation*}
Supongamos que consideramos cada columna de $A$ y cada columna de $Q$ como un vector,
\begin{equation*}
A=\left( a_1\vert a_2\vert \cdots \vert a_n\right), \ Q=\left( q_1\vert q_2\vert \cdots \vert q_n \right)
\end{equation*}

Podemos ahora considerar el producto $Q\cdot R$ como combinaciones lineales de las columnas de $Q$, que dan como resultado las columnas de $A$,
\begin{align*}
&\begin{pmatrix}
a_{11}& a_{12}& \cdots & a_{1n}\\
a_{21}& a_{22}& \cdots & a_{2n}\\
\vdots & \vdots & \cdots & \vdots\\
a_{n1}& a_{n2}& \cdots & a_{nn}
\end{pmatrix}=\begin{pmatrix}
q_{11}& q_{12}& \cdots & q_{1n}\\
q_{21}& q_{22}& \cdots & q_{2n}\\
\vdots & \vdots & \cdots & \vdots\\
q_{n1}& q_{n2}& \cdots & q_{nn}
\end{pmatrix}\cdot \begin{pmatrix}
r_{11}& r_{12}& \cdots & r_{1n}\\
0 & r_{22}& \cdots & r_{2n}\\
\vdots & \vdots & \cdots & \vdots\\
0& 0& \cdots & a_{nn}
\end{pmatrix} \Rightarrow\\
&a_1=q_1\cdot r_{11} \rightarrow \begin{pmatrix}
a_{11}\\
a_{21}\\
\vdots \\
a_{n1}
\end{pmatrix}=\begin{pmatrix}
q_{11}\\
q_{21}\\
\vdots\\
q_{n1}
\end{pmatrix} \cdot r_{11},\\
&a_2=q_1\cdot r_{12}+ q_2\cdot r_{22} \rightarrow \begin{pmatrix}
a_{12}\\
a_{22}\\
\vdots \\
a_{n2}
\end{pmatrix}=\begin{pmatrix}
q_{11}\\
q_{21}\\
\vdots\\
q_{n1}
\end{pmatrix}\cdot r_{12}+\begin{pmatrix}
q_{12}\\
q_{22}\\
\vdots\\
q_{n2}
\end{pmatrix} \cdot r_{22}\\
\vdots \\
&a_n=q_1\cdot r_{1n}+ q_2\cdot r_{2n}+\cdots +r_{nn} \rightarrow \begin{pmatrix}
a_{1n}\\
a_{2n}\\
\vdots \\
a_{nn}
\end{pmatrix}=\begin{pmatrix}
q_{11}\\
q_{21}\\
\vdots\\
q_{n1}
\end{pmatrix}\cdot r_{1n}+\begin{pmatrix}
q_{12}\\
q_{22}\\
\vdots\\
q_{n2}
\end{pmatrix} \cdot r_{2n}+ \cdots + \begin{pmatrix}
q_{1n}\\
q_{2n}\\
\vdots\\
q_{nn} 
\end{pmatrix}\cdot r_{nn}
\end{align*}

Para que la matriz $Q$ sea ortogonal, basta hacer que sus columnas, consideradas cada una como un vector, sean ortonormales entre s�. Podemos obtener la matriz $Q$ a partir de la matriz $A$, aplicando a sus columnas alg�n m�todo de ortogonalizaci�n de vectores. En particular, emplearemos el m�todo conocido como ortogonalizaci�n de Grand-Schmidt.

\paragraph{Ortogonalizaci�n de Grand-Schmidt} Este m�todo permite transformar un conjunto de vectores linealmente independientes arbitrarios en un conjunto de vectores ortonormales. Lo describiremos a la vez que estudiamos su uso para obtener la factorizaci�n QR de una matriz $A$.

En primer lugar, consideremos el vector $a_1$, formado por los elementos de la primera columna de la matriz $A$. Si lo dividimos por su m�dulo, obtendremos un vector unitario, Este vector, ser� la primera columna de la matriz $Q$,
\begin{equation*}
q_1=\frac{a_1}{\lVert a_1 \rVert}
\end{equation*}

De la expresi�n anterior es f�cil deducir quien ser�a el elemento $r_{11}$ de la matriz $R$,

\begin{equation*}
a_1=q_1\cdot \lVert a_1 \rVert \Rightarrow r_{11}= \lVert a_1 \rVert
\end{equation*}

Supongamos por ejemplo que queremos factorizar la siguiente matriz,

\begin{equation*}
A=\begin{pmatrix}
2& 3& 1\\
2& 0& 2\\
1& 4& 3
\end{pmatrix}
\end{equation*}

La primera columna de la matriz $Q$ ser�a,
\begin{equation*}
q_1=\frac{a_1}{\lVert a_1 \rVert}=\frac{1}{3}\cdot \begin{pmatrix}
2\\
2\\
1
\end{pmatrix}=\begin{pmatrix}
2/3\\
2/3\\
1/3
\end{pmatrix}
\end{equation*}

y el elemento $r_{11}$ de la matriz $R$,
\begin{equation*}
r_{11}=\lVert a_1 \rVert=3
\end{equation*}

Para obtener la segunda columna $Q$, Calculamos en primer lugar la proyecci�n de la segunda columna de $A$, $a_2$ sobre el vector $q_1$. En general, la proyecci�n de un vector $a$, sobre un vector $b$ se calcula obteniendo el producto escalar de los dos vectores y dividiendo el resultado por el m�dulo del vector $b$. En nuestro casos el vector $q_1$, sobre el que se calcula la proyecci�n, es unitario, por lo que calcular la proyecci�n de $a_2$ sobre $q_1$ se reduce a calcular el producto escalar de los dos vectores,

\begin{equation*}
\phi_{a_2/q_1}=q_1^T\cdot a_2
\end{equation*}

Siguiendo con nuestro ejemplo,
\begin{equation*}
\phi_{a_2/q_1}=q_1^T\cdot a_2=
\begin{pmatrix}
2/3& 2/3& 1/3
\end{pmatrix}\cdot \begin{pmatrix}
3\\
0\\
4
\end{pmatrix}=\frac{10}{3}
\end{equation*}

Podemos interpretar la proyecci�n del vector $a_2$ sobre el vector $q_1$ como la \emph{componente} del vector $a_2$ en la direcci�n de $q_1$. Por eso, si restamos al vector $a_2$ su proyecci�n sobre $q_1$ multiplicada por $q_1$ el resultado ser� un nuevo vector $q_2$ ortogonal a $q_1$,

\begin{equation*}v_2=a_2-\phi_{a_2/q_1}\cdot q_1=a_2-(q_1^T\cdot a_2)\cdot q_1
\end{equation*}

La figura \ref{fig:qr} muestra gr�ficamente el proceso para dos vectores en el plano.

\begin{figure}[h]
\centering
\includegraphics[width=12cm]{qr.eps}
\caption{Obtenci�n de un vector ortogonal}
\label{fig:qr}
\end{figure}

Como $q_2$ tiene que ser un vector unitario para que la matriz $Q$ sea ortogonal, dividimos el vector obtenido, $q`_2$, por su m�dulo,

\begin{equation*}
q_2=\frac{V_2}{\lVert v_2 \rVert}=\frac{a_2-(q_1^T\cdot a_2)\cdot q_1}{\lVert a_2-(q_1^T\cdot a_2)\cdot q_1 \rVert}
\end{equation*}

Por �ltimo despejando $a_2$ podemos identificar los valores de $r_{12}$ y $r_{22}$, tal y como se describieron anteriormente,

\begin{align*}
a_2&= (q_1^T\cdot a_2)\cdot q_1+\lVert a_2-(q_1^T\cdot a_2)\cdot q_2 \rVert \cdot q_2\\
r_{12}&=q_1^T\cdot a_2\\
r_{22}&=\lVert a_2-(q_1^T\cdot a_2)\cdot q_2 \rVert 
 \end{align*}

Volviendo a la matriz $A$ de nuestro ejemplo la segunda columna de la matriz $Q$ y los valores de $r_{12}$ y $r_{22}$ quedar�an,


\begin{align*}
q_2&=\frac{v_2}{\lVert v_2 \rVert}=\frac{a_2-(q_1^T\cdot a_2)\cdot q_1}{\lVert a_2-(q_1^T\cdot a_2)\cdot q_1 \rVert}=\left[\begin{pmatrix}
3\\
0\\
4
\end{pmatrix}-\frac{10}{3}\cdot \begin{pmatrix}
2/3\\
2/3\\
1/3
\end{pmatrix}\right]\cdot \frac{3}{5\sqrt{5}}=\frac{1}{5\sqrt{5}}\begin{pmatrix}
7/3\\
-20/3\\
26/3
\end{pmatrix}\\
r_{12}&=\frac{10}{3}\\
r_{22}&=\frac{5\sqrt{5}}{3}
\end{align*}
Se puede comprobar que los vectores $q_1$ y $q_2$ son ortogonales,
\begin{equation*}
q_1^T\cdot q_2= \begin{pmatrix}
2/3& 2/3& 273
\end{pmatrix}\cdot {5\sqrt{5}}\begin{pmatrix}
7/3\\
-20/3\\
26/3
\end{pmatrix}={5\sqrt{5}}\cdot \left(\frac{14}{3}-\frac{40}{3}+\frac{26}{3}\right)=0
\end{equation*}

Para obtener la tercera columna de la matriz $Q$, $q_3$ proceder�amos de modo an�logo: calcular�amos la proyecci�n de la tercera columna de $A$, $a_3$ sobre los vectores formados por la los dos primeras columnas de $Q$, restamos de $a_3$ las dos proyecciones y normalizamos el vector resultante dividi�ndolo por su m�dulo,

\begin{equation*}
q_3=\frac{v_3}{\lVert v_3 \rVert}=\frac{a_3-(q_1^T\cdot a_3)\cdot q_1-(q_2^T\cdot a_3)\cdot q_2}{\lVert a_3-(q_1^T\cdot a_3)\cdot q_1-(q_2^T\cdot a_3)\cdot q_2 \rVert}
\end{equation*} 

y de modo an�logo al caso de la segunda columna,

\begin{align*}
r_{13}&=q_1^T\cdot a_3\\
r_{23}&=q_2^T\cdot a_3\\
r_{33}&=\lVert a_3-(q_1^T\cdot a_3)\cdot q_1-(q_2^T\cdot a_3)\cdot q_2 \rVert
\end{align*}

Es f�cil observar como vamos obteniendo las sucesivas columnas de la matriz $Q$, iterativamente. Podemos generalizar los resultas anteriores para cualquier columna arbitraria de la matriz $Q$,

\begin{equation*}
q_i=\frac{a_i-\sum_{j=1}^i (q_j^T\cdot a_i)\cdot q_i}{\lVert a_i-\sum_{j=1}^i (q_j^T\cdot a_i)\cdot q_i \rVert}
\end{equation*}

y para los valores de $r_{1i}, r_{2i}, \cdots r{ii}$,
\begin{align*}
r_{ji}&=q_j^T\cdot a_i, \ j<i\\
r_{ii}&= \lVert a_i-\sum_{j=1}^i (q_j^T\cdot a_i)\cdot q_i \rVert
\end{align*}

El siguiente c�digo de Matlab, calcula la factorizaci�n QR de una matriz, empleando el m�todo descrito,

\begin{lstlisting}
function [Q,R]=QRF1(A)
%%%%%%%%%%%%%%%%%%%%%%%%%%%%%%%%%%%%
% Factorizaci�n QR obtenida directamente por ortogonalizaci�n de grand-schmidt
% Ojo, el algoritmo es inestable... Con lo que la bondad de las soluciones
% dependera de la matriz que se quiera factorizar.
%%%%%%%%%%%%%%%%%%%%%%%%%%%%%%%%%%%
% En primer lugar obtenemos las dimensiones de la matriz
[m,n]=size(A);

% fatorizamos columna a columna
for j=1:n
    % Construimos un vector auxiliar v, nos servira para ir obteniendo las
    % columnas de la matriz Q.
    for i=1:m %Solo llegamos hasta m factorizaci�n incompleta si m>n
        v(i)=A(i,j)
    end
    for i=1:j-1
        % obtenemos los elementos de la matriz R, correspondientes a la
        % columna j, solo podemos construir hasta una fila antes de la
        % diagonal i=j-1. cada fila es el producto escalar de la columna i
        % de la matriz Q for la columna j de la Matriz A.
        R(i,j)=0
        for k=1:m
        R(i,j)=R(i,j)+Q(k,i)*A(k,j)
        end
        
        % obtenemos las componentes del vector auxiliar que nos permitira
        % construir la columna j de la matriz Q
        for k=1:m
            v(k)=v(k)-R(i,j)*Q(k,i)
        end
    end
    % Obtenemos el valor del elemento de la diagonal R(j,j) de la matriz R

    R(j,j)=0
    for k=1:m
        R(j,j)=R(j,j)+v(k)^2
    end
    R(j,j)=sqrt(R(j,j))
    
    % Y por �ltimo, divimos los elementos del vector v por R(j,j), para
    % obtener la columna j de la matriz Q
    for k=1:m
        Q(k,j)=v(k)/R(j,j)
    end
end
\end{lstlisting}
En general, el algoritmo que acabamos de describir para obtener la ortogonalizaci�n de Grand-Schmidt, es num�ricamente inestable. La estabilidad puede mejorarse, si vamos modificando progresivamente la matriz $A$, a medida que calculamos las columnas de $Q$. 

Cada vez que obtenemos una nueva columna de $Q$, modificamos las columnas de $A$ de modo que sean ortogonales a la columna de $Q$ obtenida. Para ello, lo m�s sencillo es crear una matriz auxiliar $V$ que hacemos, inicialmente, igual a $A$, $V^{(0)}=A$

para obtener la primera columna de $Q$, procedemos igual que antes, normalizando la primera columna de $V^{(0)}$
\begin{align*}
q_1=\frac{v_1^{(0)}}{\lVert v_1^{(0)}\rVert}\\
r_{11}=\lVert v_1^{(0)}\rVert
\end{align*}

A continuaci�n calculamos la proyecci�n de todas las dem�s columnas de la matriz $V^{(0)}$ con respecto a $q_1$, esto es equivalente a calcular los restantes elementos de la primera fila de la matriz $R$: $r_{21}, r_{31}, \cdots r_{n1}$,
\begin{equation*}
r_{1j}=q_1^T\cdot v_j^{(0)}
\end{equation*}

Una vez calculadas las proyecciones, modificamos todas las columnas de $V^{(0)}$, excepto la primera restando a cada una su proyecci�n con respecto a $q_1$.

\begin{equation*}
v_j^{(1)}=v_j^{(0)}-r_{1j}\cdot v_j^{(0)}, \ j=2,3,\cdot n
\end{equation*}

La nueva matriz $V^{(1)}$ cumple que todas sus columnas a partir de la segunda son ortogonales a $q_1$. Para obtener $q_2$ es suficiente dividir $v_2^{1}$ por su m�dulo,

\begin{align*}
q_2&=\frac{v_2^{(1)}}{\lVert v_2^{(1)} \rVert}\\
r_{22}&=\lVert v_2^{(1)} \rVert
\end{align*}

Podemos ahora calcular el resto de los elementos de la segunda fila de la matriz $R$ de modo an�logo a como hemos calculado los de la primera,

\begin{equation*}
r_{2j}=q_2^T\cdot v_j^{(1)}
\end{equation*}

y, de nuevo actualizar�amos todos las columnas de $V^{1}$, a partir de la tercera, para que fueran ortogonales a $q_2$, 

\begin{equation*}
v_j^{(2)}=v_j^{(1)}-r_{2j}\cdot v_j^{(1)}, \ j=3,4,\cdot n
\end{equation*}

Las columnas de $V^{(2)}$ ser�an ahora ortogonales a $q_1$ y $q_2$. Si seguimos el mismo procedimiento n veces, calculando cada vez una nueva columna de $Q$ y una fila de $R$, obtenemos finalmente la factorizaci�n QR de la matriz inicial $A$. En general, para obtener la columna $q_i$ y los elementos de la fila $i$ de la matriz $R$ tendr�amos,
 
\begin{align*}
q_i&=\frac{v_i^{(i-1)}}{\lVert v_i^{(i-1)} \rVert}\\
r_{ii}&=\lVert v_i^{(i-1)} \rVert
\end{align*}

\begin{equation*}
r_{ij}=q_i^T\cdot v_j^{(i-1)}
\end{equation*}

y la actualizaci�n correspondiente de la matriz $V$ ser�a,

\begin{equation*}
v_j^{(i)}=v_j^{(i)}-r_{ij}\cdot v_j^{(i)}, \ j=i+1,i+2,\cdot n
\end{equation*}

El resultado es el mismo que el que se obtiene por el primer m�todo descrito. La ventaja es que num�ricamente es m�s estable. 



El siguiente c�digo de Matlab, calcula la factorizaci�n QR de una matriz de orden $n\times n$, empleando el m�todo descrito,

\begin{lstlisting}
function [Q,R]=QRF2(A)
%%%%%%%%%%%%%%%%%%%%%%%%%%%%%%%%
% Calculo de factorizacion QR de la matriz A, mediante la ortogonaliaci�n de
% grand.schmidt modificada. Este algoritmo si que es estable...
% Obtenemos las dimensiones de A
%%%%%%%%%%%%%%%%%%%%%%%%%%%%%%%%%
[m,n]=size(A);

% creamos una matriz auxiliar v sobre la que vamos a realizar la
% factorizaci�n. Se podr�a realizar directamente sobre A Machacando sus
% columnas seg�n la factorizaci�n progresa... Ser�a lo correcto para ahorrar
% espacio de almacenamiento. Pero en fin, quiz� as� queda m�s claro aunque
% sea menos eficiente.
v=A;
% Como siempre, vamos factorizando por columnas de la matriz A


for i=1:m %la matriz Q tiene que ser mXm, aunque el numero de columnas de A sea n)
    % calculamos cada R(i,i) como el modulo del vector auxiliar v(1:m,i)
    R(i,i)=0;
    for k=1:m
        R(i,i)=R(i,i)+v(k,i)^2;
    end
    R(i,i)=sqrt(R(i,i));
    % calculamos el la columna i de la matriz Q, normalizando la columna i
    % de la matriz v
    for k=1:m
        Q(k,i)=v(k,i)/R(i,i);
    end
    % Modificamos todos los R(i,j) con ij>i, en cuanto tenemos la columna j
    % de la matriz Q, nos basta calcular el producto escalar con las
    % columnas de A (En nuestro caso de v porque est�n copiadas, de las
    % filas siguientes
    for j=i+1:n
        R(i,j)=0;
        for k=1:m
            R(i,j)=R(i,j)+Q(k,i)*v(k,j);
        end
        % i por �ltimo modificamos todas las columnas de la matriz v desde
        % i+1 hasta el final de la matriz. Aqui es donde cambia el algoritmo
        % ya que estamos modificando la matriz A, y las sucesivas matrices V
        % cada vez que obtenemos una nueva fila de valores de R
        for k=1:m
            v(k,j)=v(k,j)-R(i,j)*Q(k,i);
        end
    end
end
\end{lstlisting}

Para terminar, indicar que Matlab tiene su propia funci�n, \texttt{[Q,R]=qr(A)}, para calcular la factorizaci�n QR de una matriz. Para calcularla, emplea el m�todo de ortogonalizaci�n de Householder. Este m�todo es a�n m�s robusto que la ortogonalizaci�n de Grand-Schmidt modificada. Pero no lo veremos en este curso. Damos a continuaci�n un ejemplo de uso de la funci�n \texttt{qr},
\begin{verbatim}
>> A=[2 3 1;2 0 2;1 4 3]

A =

     2     3     1
     2     0     2
     1     4     3

>> [q,r]=qr(A)

q =

   -0.6667    0.2087   -0.7155
   -0.6667   -0.5963    0.4472
   -0.3333    0.7752    0.5367


r =

   -3.0000   -3.3333   -3.0000
         0    3.7268    1.3416
         0         0    1.7889

>> q*r

ans =

    2.0000    3.0000    1.0000
    2.0000    0.0000    2.0000
    1.0000    4.0000    3.0000

\end{verbatim}
  
\subsection{Factorizaci�n SVD}\label{sec:SVD}
Dada una matriz cualquiera $A$ de orden $m\times n$ es posible factorizarla en el producto de tres matrices,
\begin{equation*}
A=U\cdot S \cdot V^T 
\end{equation*}

Donde $U$ es una matriz de orden $m\times m$ ortogonal, $V$ es una matriz de orden $n\times n$ ortogonal y $S$ es una matriz diagonal de orden $m\times n$. Adem�s los elementos de $S$ son positivos o cero y est�n ordenados en orden no creciente,

\begin{equation*}
s=\begin{pmatrix}
\sigma_1& 0& \cdots & 0\\
0 & \sigma_2& \cdots & 0\\
\vdots & \vdots & \vdots & \vdots \\
0& 0& \cdots & \sigma_i\\ 
\end{pmatrix}; \ \sigma_1 \geq \sigma_2 \geq \cdots \geq \sigma_i; \ i=\min(m,n)
\end{equation*}

Los elementos de la diagonal de la matriz $S$, ($\sigma_1, \ \sigma_2, \  \cdots \ \sigma_i$), reciben el nombre de \emph{valores singulares} de la matriz $A$. De ah� el nombre que recibe esta factorizaci�n; SVD son las siglas en ingl�s de \emph{Singular Value Decomposition}.

No vamos a describir ning�n algoritmo para obtener la factorizaci�n SVD de una matriz. En Matlab existe la funci�n \texttt{[U,S,V]=svd(A)} que permite obtener directamente la factorizaci�n SDV de una matriz $A$ de dimensi�n arbitraria. A continuaci�n se incluyen unos ejemplos de uso para matrices no cuadradas,

\begin{verbatim}
>> A=[1 3 4;2 3 2;2 4 5;3 2 3]

A =

     1     3     4
     2     3     2
     2     4     5
     3     2     3

>> [U,S,V]=svd(A)

U =

   -0.4877    0.5175    0.1164   -0.6934
   -0.3860   -0.3612   -0.8375   -0.1387
   -0.6517    0.3024    0.0552    0.6934
   -0.4340   -0.7144    0.5311   -0.1387


S =

   10.2545         0         0
         0    1.9011         0
         0         0    1.1097
         0         0         0


V =

   -0.3769   -0.9170    0.1307
   -0.5945    0.1313   -0.7933
   -0.7103    0.3767    0.5946
\end{verbatim}

Como la matriz $A$ tiene m�s filas que columnas, la matriz $S$ resultante termina con una fila de ceros.

\begin{verbatim}
>> B=A'

B =

     1     2     2     3
     3     3     4     2
     4     2     5     3

>> [U,S,V]=svd(B)

U =

   -0.3769   -0.9170    0.1307
   -0.5945    0.1313   -0.7933
   -0.7103    0.3767    0.5946


S =

   10.2545         0         0         0
         0    1.9011         0         0
         0         0    1.1097         0


V =

   -0.4877    0.5175    0.1164   -0.6934
   -0.3860   -0.3612   -0.8375   -0.1387
   -0.6517    0.3024    0.0552    0.6934
   -0.4340   -0.7144    0.5311   -0.1387
   
\end{verbatim}

Como la matriz $B$, transpuesta de la matriz $A$ del ejemplo anterior, tiene m�s columnas que filas, la matriz $S$ termina con una columna de ceros.

A continuaci�n enunciamos sin demostraci�n algunas propiedades de la factorizaci�n SVD.

\begin{enumerate}
\item El rango de una matriz $A$ coincide con el n�mero de sus valores singulares distintos de cero.
\item La norma-2 inducida de una matriz $A$ coincide con su valor singular mayor $\sigma_1$.
\item La norma de Frobenius de una matriz $A$ cumple:
\begin{equation*}
\lVert A \rVert_{F}=\sqrt{\sigma_1^2+\sigma_2^2+\cdots +\sigma_r^2}
\end{equation*}
\item Los valores singulares de una matriz $A$ distintos de cero son iguales a la ra�z cuadrada positiva de los autovalores distintos de cero de las matrices $A\cdot A^T$ � $A^T\cdot A$. (los autovalores distintos de cero de estas dos matrices son iguales),
\begin{equation*}
\sigma_i^2=\lambda_i(A\cdot A^T)=\lambda_i(A^T\cdot A)
\end{equation*}
\item El valor absoluto del determinante de una matriz cuadrada $A$,$n\times n$, coincide con el producto de sus valores singulares,
\begin{equation*}
\vert \det(A) \vert = \prod_{i=1}^n \sigma_i
\end{equation*}
\item El n�mero de condici�n de una matriz cuadrada $A$ $n\cdot n$, que se define como el producto de la norma-2 inducida de $A$ por la norma-2 inducida de la inversa de $A$, puede expresarse como el cociente entre el el valor singular mayor de $A$ y su valor singular m�s peque�o,
\begin{equation*}
k(A)=\lVert A \rVert_2 \cdot \lVert A^{-1} \rVert_2 = \sigma_1 \cdot \frac{1}{\sigma_n}=\frac{\sigma_1}{\sigma_n}
\end{equation*}
El n�mero de condici�n de una matriz, es una propiedad importante que permite estimar c�mo de estables ser�n los c�lculos realizados empleando dicha matriz, en particular aquellos que involucran directa o indirectamente el c�lculo de su inversa.

\end{enumerate}


\chapter{Sistemas de ecuaciones lineales}\label{sistemas}

\section{Introducci�n}

Una ecuaci�n lineal es aquella que establece una relaci�n \emph{lineal} entre dos o m�s variables, por ejemplo,

\begin{equation*}
3x_1-2x_2=12
\end{equation*}

Se dice que es una relaci�n lineal, porque las variables est�n relacionadas entre s� tan solo mediante sumas y productos por coeficientes constantes. En particular, el ejemplo anterior puede representarse geom�tricamente mediante una l�nea recta.

El n�mero de variables relacionadas en una ecuaci�n lineal determina la dimensi�n de la ecuaci�n. La del ejemplo anterior es una ecuaci�n bidimensional, puesto que hay dos variables. El n�mero puede ser arbitrariamente grande en general,
\begin{equation*}
a_1x_1+a_2x_2+\cdots +a_nx_n=b
\end{equation*} 
ser� una ecuaci�n n-dimensional.

Como ya hemos se�alado m�s arriba, una ecuaci�n bidimensional admite una l�nea recta como representaci�n geom�trica, una ecuaci�n tridimensional admitir� un plano y para dimensiones mayores que tres cada ecuaci�n representar� un hiperplano de dimension n. Por supuesto, para dimensiones mayores que tres, no es posible obtener una representaci�n gr�fica de la ecuaci�n.

Las ecuaciones lineales juegan un papel muy importante en la f�sica y, en general en la ciencia y la tecnolog�a. La raz�n es que constituyen la aproximaci�n matem�tica m�s sencilla a la relaci�n entre magnitudes f�sicas. Por ejemplo cuando decimos que la fuerza aplicada a un resorte y la elongaci�n  que sufre est�n relacionadas por la ley de Hooke, $F=Kx$ estamos estableciendo una relaci�n lineal entre las magnitudes fuerza y elongaci�n. �Se cumple siempre dicha relaci�n? Desde luego que no. Pero es razonablemente cierta para elongaciones peque�as y, apoyados en ese sencillo modelo \emph{lineal} de la realidad, se puede aprender mucha f�sica.

Un sistema de ecuaciones lineales est� constituido por varias ecuaciones lineales, que expresan relaciones lineales distintas sobre las mismas variables. Por ejemplo,

\begin{align*}
a_{11}x_1+a_{12}x_2&=b_1\\
a_{21}x_1+a_{22}x_2&=b_2
\end{align*}

Se llaman soluciones del sistema de ecuaciones a los valores de las variables que satisfacen simult�neamente a todas las ecuaciones que componen el sistema. Desde el punto de vista de la obtenci�n de las soluciones a las variables se les suele denominar inc�gnitas, es decir valores no conocidos que deseamos obtener o calcular.

Un sistema de ecuaciones puede tener infinitas soluciones, puede tener una �nica soluci�n o puede no tener soluci�n. En lo que sigue, nos centraremos en sistemas de ecuaciones que tienen una �nica soluci�n. 

Una primera condici�n para que un sistema de ecuaciones tengan una �nica soluci�n es que el n�mero de inc�gnitas presentes en el sistema coincida con el n�mero de ecuaciones. 

De modo general podemos decir que vamos a estudiar m�todos num�ricos para resolver con un computador sistemas de $n$ ecuaciones con $n$ inc�gnitas,

\begin{align*}
a_{11}&x_1+a_{12}x_2+\cdots +a_{1n}x_n=b_1\\
a_{21}&x_1+a_{22}x_2+\cdots +a_{2n}x_n=b_2\\
\cdots & \\
a_{n1}&x_1+a_{n2}x_2+\cdots +a_{nn}x_n=b_n
\end{align*}  

Una de las grandes ventajas de los sistemas de ecuaciones lineales es que puede expresarse en forma de producto matricial,


\begin{equation*}
\left. \begin{aligned}
a_{11}&x_1+a_{12}x_2+\cdots +a_{1n}x_n=b_1\\
a_{21}&x_1+a_{22}x_2+\cdots +a_{2n}x_n=b_2\\
\cdots & \\
a_{n1}&x_1+a_{n2}x_2+\cdots +a_{nn}x_n=b_n
\end{aligned}\right\} \Rightarrow	\overbrace{\begin{pmatrix}
a_{11}& a_{12}& \cdots & a_{1n}\\
a_{21}& a_{22}& \cdots & a_{2n}\\
\vdots & \vdots & \ddots & \vdots\\
a_{n1}& a_{n2}& \cdots & a_{nn}
\end{pmatrix}}^A \cdot \overbrace{\begin{pmatrix}
x_1\\
x_2\\
\vdots \\
x_n
\end{pmatrix}}^x=\overbrace{\begin{pmatrix}
b_1\\
b_2\\
\vdots \\
b_n
\end{pmatrix}}^b
\end{equation*}

La matriz $A$ recibe el nombre de matriz de coeficientes del sistema de ecuaciones, el vector $x$ es el vector de inc�gnitas y el vector $b$ es el vector de t�rminos independientes. Para resolver un sistema de ecuaciones podr�amos aplicar lo aprendido en el cap�tulo anterior sobre �lgebra de matrices,
\begin{equation*}
A\cdot x=b \Rightarrow x=A^{-1}\cdot b
\end{equation*}

Es decir, bastar�a invertir la matriz de coeficientes y multiplicarla por la izquierda por el vector de coeficientes para obtener el vector de  t�rminos independientes. De aqu� podemos deducir una segunda condici�n para que un sistema de ecuaciones tenga una soluci�n �nica; Su matriz de coeficientes tiene que tener inversa. Veamos algunos ejemplos sencillos.

Tomaremos en primer lugar un sistema de dos ecuaciones con dos inc�gnitas,

\begin{align*}
4x_1+x_2&=6\\
3x_1-2x_2&=-1
\end{align*}

Si expresamos el sistema en forma de producto de matrices obtenemos,

\begin{equation*}
\begin{pmatrix}
4& 1\\
3& -2
\end{pmatrix} \cdot \begin{pmatrix}
x_1\\
x_2
\end{pmatrix}=\begin{pmatrix}
6\\
-1
\end{pmatrix}
\end{equation*}

e invirtiendo la matriz de coeficientes y multiplic�ndola por el vector de t�rminos independientes se llega al vector de soluciones del sistema,

\begin{equation*}
\begin{pmatrix}
x_1\\
x_2
\end{pmatrix}= \begin{pmatrix}
4& 1\\
3& -2
\end{pmatrix}^{-1} \cdot \begin{pmatrix}
6\\
-1
\end{pmatrix}=\begin{pmatrix}
2/11& 1/11\\
3/11& -4/11
\end{pmatrix}\begin{pmatrix}
6\\
-1
\end{pmatrix}=\begin{pmatrix}
1\\
2
\end{pmatrix}
\end{equation*}

En el ejemplo que acabamos de ver, cada ecuaci�n corresponde a una recta en el plano, en la figura \ref{fig:recta1} se han representado dichas rectas gr�ficamente. El punto en que se cortan es precisamente la soluci�n del sistema.

\begin{figure}[h]
\centering
\includegraphics[width=12cm]{recta1}
\caption{Sistema de ecuaciones con soluci�n �nica}
\label{fig:recta1}
\end{figure}

Supongamos ahora el siguiente sistema, tambi�n de dos ecuaciones con dos inc�gnitas,

\begin{align*}
4x_1+x_2&=6\\
2x_1+\frac{1}{2} x_2&=-1
\end{align*}

El sistema no tiene soluci�n. Su matriz de coeficientes tiene determinante cero, por lo que no es invertible,
\begin{equation*}
\vert A \vert =\begin{vmatrix}
4& 1\\
2& 1/2
\end{vmatrix} =0 \Rightarrow \nexists A^{-1}
\end{equation*}

Si representamos gr�ficamente las dos ecuaciones de este sistema (figura \ref{fig:recta2}) es f�cil entender lo que pasa, las rectas son paralelas, no existe ning�n punto $(x_1,x_2)$ que pertenezca a las dos rectas, y por tanto el sistema carece de soluci�n.

\begin{figure}[h]
\centering
\includegraphics[width=12cm]{recta2}
\caption{Sistema de ecuaciones sin soluci�n}
\label{fig:recta2}
\end{figure}

Dos rectas paralelas lo son, porque tienen la misma pendiente. Esto se refleja en la matriz de coeficientes, en que las filas son proporcionales; si multiplicamos la segunda fila por dos, obtenemos la primera. 

Por �ltimo, el sistema,

\begin{align*}
4x_1+x_2&=6\\
2x_1+\frac{1}{2} x_2&=3
\end{align*}

posee infinitas soluciones. la raz�n es que la segunda ecuaci�n es igual que la primera multiplicada por dos: es decir, representa exactamente la misma relaci�n lineal entre las variables $x_1$ y $x_2$, por tanto, todos los puntos de la recta son soluci�n del sistema. De nuevo, la matriz de coeficientes del sistema no tiene inversa ya que su determinante es cero.

\begin{figure}[h]
\centering
\includegraphics[width=12cm]{recta3.eps}
\caption{Sistema de ecuaciones con infinitas soluciones}
\label{recta3}
\end{figure}

Para sistemas de ecuaciones de dimensi�n mayor, se cumple tambi�n que que el sistema no tiene soluci�n �nica si el determinante de su matriz de coeficiente es cero. En todos los dem�s casos, es posible obtener la soluci�n del sistema invirtiendo la matriz de coeficientes y multiplicando el resultado por el vector de t�rminos independientes.

En cuanto un sistema de ecuaciones tiene una dimensi�n suficientemente grande, invertir su matriz de coeficientes se torna un problema costoso o sencillamente inabordable.

Desde un punto de vista num�rico, la inversi�n de una matriz, presenta frecuentemente problemas debido al error de redondeo en las operaciones. Por esto, casi nunca se resuelven los sistemas de ecuaciones invirtiendo su matriz de coeficientes. A lo largo de este cap�tulo estudiaremos dos tipos de m�todos de resoluci�n de sistemas de ecuaciones. El primero de ellos recibe el nombre gen�rico de m�todos directos, el segundo tipo lo constituyen los llamados m�todos iterativos.

\section{Condicionamiento}
En la introducci�n hemos visto que para que un sistema de ecuaciones tenga soluci�n, es preciso que su matriz de coeficientes sea invertible. Sin embargo cuando tratamos de resolver un sistema de ecuaciones num�ricamente, empleando un ordenador, debemos antes examinar cuidadosamente la matriz de coeficientes del sistema. Veamos un ejemplo: el sistema,

\begin{align*}
4x_1+x_2&=6\\
2x_1+0.4 x_2&=-1
\end{align*}

Tiene como soluciones,
\begin{equation*}
x=\begin{pmatrix}
-8.5\\
40
\end{pmatrix}
\end{equation*}

Si alteramos ligeramente uno de sus coeficientes,

\begin{align*}
4x_1+x_2&=6\\
2x_1+0.4{\color{red}9} x_2&=-1
\end{align*}

Las soluciones se alteran bastante;  se vuelven aproximadamente 10 veces m�s grande,
\begin{equation*}
x=\begin{pmatrix}
-98.5\\
400
\end{pmatrix}
\end{equation*}

y si volvemos a alterar el mismo coeficiente,

\begin{align*}
4x_1+x_2&=6\\
2x_1+0.4{\color{red}99} x_2&=-1
\end{align*}

La soluci�n es aproximadamente 100 veces m�s grande,
\begin{equation*}
x=\begin{pmatrix}
-998.5\\
4000
\end{pmatrix}
\end{equation*}

La raz�n para estos cambios es f�cil de comprender intuitivamente; a medida que aproximamos el coeficiente a $0.5$, estamos haciendo que las dos ecuaciones lineales sean cada vez mas paralelas, peque�as variaciones en la pendiente, modifican mucho la posici�n del punto de corte.

Cuando peque�as variaciones en la matriz de coeficientes generan grandes variaciones en las soluciones del sistema, se dice que el sistema est� mal condicionado, en otras palabras: que no es un sistema bueno para ser resuelto num�ricamente. Las soluciones obtenidas para un sistema mal condicionado, hay que tomarlas siempre con bastante escepticismo.

Para estimar el buen o mal condicionamiento de un sistema, se emplea el n�mero de condici�n, que definimos en el cap�tulo anterior, \ref{sec:SVD}, al hablar de la factorizaci�n SVD. El n�mero de condici�n de una matriz es el cociente entre sus valores singulares mayor y menor.  Cuanto m�s pr�ximo a $1$ sea el n�mero de condici�n, mejor condicionada estar� la matriz y cuanto mayor sea el n�mero de condici�n peor condicionada estar�.

Matlab tiene un comando espec�fico para obtener el n�mero de condici�n de una matriz, sin tener que calcular la factorizaci�n SVD, el comando \texttt{nc=cond(A)}. Si lo aplicamos a la matriz de coeficientes del �ltimo ejemplo mostrado,

\begin{verbatim}
>> A=[4 1; 2, 0.499]

A =

    4.0000    1.0000
    2.0000    0.4990

>> nc=cond(A)

nc =

  5.3123e+003
\end{verbatim}

El n�mero est� bastante alejado de uno, lo que, en principio, indica un mal condicionamiento del sistema. 

Incidentalmente, podemos calcular la factorizaci�n svd de la matriz de coeficientes y dividir el valor singular mayor entre el menor para comprobar que el resultado es el mismo que nos da la funci�n \texttt{cond},
\begin{verbatim}
>> [U,S,V]=svd(A)

U =

   -0.8944   -0.4472
   -0.4472    0.8944


S =

    4.6097         0
         0    0.0009


V =

   -0.9702    0.2424
   -0.2424   -0.9702

>> nc=S(1,1)/S(2,2)

nc =

  5.3123e+003
\end{verbatim}

Matlab emplea tambi�n la funci�n \texttt{rnc=rcond(A)} para estimar el condicionamiento de una matriz. No describiremos el m�todo, simplemente diremos que en lugar de obtener un n�mero de condici�n para la matriz, se utiliza un valor rec�proco. De este modo, cuanto m�s se aproxima a uno el resultado de \texttt{rcond}, mejor condicionada est� la matriz, y cuanto m�s se aproxime a cero peor. Para nuestro ejemplo anterior,

\begin{verbatim}
>> rnc=rcond(A)

rnc =

  1.3333e-004

\end{verbatim}

Pr�ximo a cero, lo que indica un mal condicionamiento de $A$. N�tese que el resultado de \texttt{rcond}, no es el inverso del valor de \texttt{cond}.

\section{M�todos directos}
\subsection{Sistemas triangulares}
Vamos a empezar el estudio de los m�todos directos por los algoritmos de resoluci�n de los sistemas m�s simples posibles, aquellos cuya matriz de coeficientes es una matriz diagonal, triangular superior o triangular inferior.
\paragraph{Sistemas diagonales.} Un sistema diagonal es aquel cuya matriz de coeficientes es una matriz diagonal. 

\begin{equation*}
\left. \begin{aligned}
a_{11}&x_1=b_1\\
a_{22}&x_2=b_2\\
\cdots & \\
a_{nn}&x_n=b_n
\end{aligned}\right\} \Rightarrow	\overbrace{\begin{pmatrix}
a_{11}& 0& \cdots & 0\\
0& a_{22}& \cdots & 0\\
\vdots & \vdots & \ddots & \vdots\\
0& 0& \cdots & a_{nn}
\end{pmatrix}}^A \cdot \overbrace{\begin{pmatrix}
x_1\\
x_2\\
\vdots \\
x_n
\end{pmatrix}}^x=\overbrace{\begin{pmatrix}
b_1\\
b_2\\
\vdots \\
b_n
\end{pmatrix}}^b
\end{equation*}

Su resoluci�n es trivial, basta dividir cada t�rmino independiente por el elemento correspondiente de la diagonal de la matriz de coeficientes,
\begin{equation*}
x_i=\frac{b_i}{a_{ii}}
\end{equation*}

Para obtener la soluci�n basta crear en Matlab un sencillo bucle \texttt{for},
\begin{verbatim}
n=size(A,1);
x=zeros(n,1);
for i=1:n
    x(i)=b(i)/A(i,i);
end
\end{verbatim} 

\paragraph{Sistemas triangulares inferiores: m�todo de sustituciones progresivas.} Un sistema triangular inferior de $n$ ecuaciones con $n$ inc�gnitas tendr� la forma general,

\begin{equation*}
\left. \begin{aligned}
a_{11}&x_1=b_1\\
a_{21}&x_1+a_{22}x_2=b_2\\
\cdots & \\
a_{n1}&x_1+a_{n2}x_2+\cdots +a_{nn}x_n=b_n
\end{aligned}\right\} \Rightarrow	\overbrace{\begin{pmatrix}
a_{11}& 0& \cdots & 0\\
a_{21}& a_{22}& \cdots & 0\\
\vdots & \vdots & \ddots & \vdots\\
a_{n1}& a_{n2}& \cdots & a_{nn}
\end{pmatrix}}^A \cdot \overbrace{\begin{pmatrix}
x_1\\
x_2\\
\vdots \\
x_n
\end{pmatrix}}^x=\overbrace{\begin{pmatrix}
b_1\\
b_2\\
\vdots \\
b_n
\end{pmatrix}}^b
\end{equation*}

El procedimiento para resolverlo a \emph{mano} es muy sencillo,
despejamos la primera la primera inc�gnita de la primera ecuaci�n,
\begin{equation*}
x_1=\frac{b_1}{a_{11}}
\end{equation*}

A continuaci�n sustituimos este resultado en la segunda ecuaci�n, y despejamos $x_2$,
\begin{equation*}
x_2=\frac{b_2-a_{21}x_1}{a_{22}}
\end{equation*}

De cada ecuaci�n vamos obteniendo una componente del vector soluci�n, sustituyendo las soluciones obtenidas en las ecuaciones anteriores, as� cuando llegamos a la ecuaci�n $i$,

\begin{equation*}
x_i=\frac{b_i-\sum_{j=1}^{i-1}a_{ij}x_j}{a_{ii}}
\end{equation*}

Si repetimos este mismo proceso hasta llegar a la �ltima ecuaci�n del sistema, $n$, habremos obtenido la soluci�n completa.

EL siguiente c�digo calcula la soluci�n de un sistema triangular inferior mediante sustituciones progresivas,

\begin{lstlisting}
function x = progresivas(A,b)
% Esta Funci�n permite obtener la solucion de un sistema triangular inferior
% empleando sustituciones progresivas. La variables de entrada son la matriz
% de coeficientes A  y el vector de terminos independientes b. la solucion se
% devuelve como un vector columna en la varible x

%%%%%%%%%%%%%%%%%%%%%%%%%%%%%%%%%%%%%%%%%%%

% Obtenemos el tama�o de la matriz de coeficientes y comprobamos que es
% cuadrada,
[f,c]=size(A);
if f~=c
    error('la matriz de coeficientes no es cuadrada')
end
% construimos un vector solucion, inicialmente formado por ceros,
x=zeros(f,1);
% construimos un bucle for para ir calculando progresivamente las soluciones
% del sistema
for i=1:f
    % primero igualamos la soluci�n al termino independiente de la ecuaci�n
    % que toque...
    x(i)=b(i)
    % y luego creamos un bucle para ir restando todas las soluciones
    % anteriores...
    for j=1:i-1
    x(i)=x(i)-A(i,j)*x(j)
    end
    % para terminar dividimos por el elemento de la diagonal de la matriz de
    % coeficientes...
    x(i)=x(i)/A(i,i)
end
\end{lstlisting}

\paragraph{Sistemas triangulares superiores: m�todo de sustituciones regresivas.} En este caso, el sistema general de $n$ ecuaciones con $n$ inc�gnitas tendr� la forma general,

\begin{equation*}
\left. \begin{aligned}
a_{11}x_1+a_{12}x_2+\cdots +a_{1n}x_n=b_1\\
a_{22}x_2+\cdots +a_{2n}x_n=b_2\\
\cdots  \\
a_{nn}x_n=b_n
\end{aligned}\right\} \Rightarrow	\overbrace{\begin{pmatrix}
a_{11}& a_{12}& \cdots & a_{1n}\\
0& a_{22}& \cdots & a_{2n}\\
\vdots & \vdots & \ddots & \vdots\\
0& 0& \cdots & a_{nn}
\end{pmatrix}}^A \cdot \overbrace{\begin{pmatrix}
x_1\\
x_2\\
\vdots \\
x_n
\end{pmatrix}}^x=\overbrace{\begin{pmatrix}
b_1\\
b_2\\
\vdots \\
b_n
\end{pmatrix}}^b
\end{equation*}


El m�todo de resoluci�n es id�ntico al de un sistema triangular inferior, simplemente que ahora, empezamos a resolver por la �ltima ecuaci�n,
\begin{equation*}
x_n=\frac{b_n}{a_{nn}}
\end{equation*}
Y seguimos sustituyendo hacia arriba,
\begin{equation*}
x_{n-1}=\frac{b_{n-1}-a_{(n-1)n}x_{n}}{a_{(n-1)(n-1)}}
\end{equation*}

\begin{equation*}
x_i=\frac{b_i-\sum_{j=i+1}^{n}a_{ij}x_j}{a_{ii}}
\end{equation*}

El c�digo para implementar este m�todo es similar al de las sustituciones progresivas. Se deja como ejercicio el construir una funci�n en Matlab que calcule la soluci�n de un sistema triangular superior por el m�todo de las sustituciones regresivas.


\subsection{M�todos basados en las factorizaciones}
Puesto que sabemos como resolver sistemas triangulares, una manera de resolver sistemas m�s complejos ser�a encontrar m�todos para reducirlos a sistemas triangulares. De este modo evitamos invertir la matriz de coeficientes y los posibles problemas de estabilidad num�rica derivados de dicha operaci�n. En el cap�tulo anterior \ref{sec:fact}, vimos varios m�todos de factorizar una matriz, que permiten una aplicaci�n directa a la resoluci�n de sistemas.
\paragraph{Factorizaci�n LU.} En el cap�tulo anterior \ref{sec:LU} vimos como factorizar una matriz en el producto de dos, una triangular inferior $L$ y una triangular superior $U$. La factorizaci�n pod�a incluir pivoteo de filas, para alcanzar una soluci�n num�ricamente estable. En este caso la factorizaci�n LU tomaba la forma,
\begin{equation*}
P\cdot A = L\cdot U
\end{equation*}

Donde $P$ representa una matriz de permutaciones.

Supongamos que queremos resolver un sistema de $n$ ecuaciones lineales con $n$ inc�gnitas que representamos gen�ricamente en forma matricial, como siempre,
\begin{equation*}
A\cdot x=b
\end{equation*}

Si calculamos la fatorizaci�n LU de su matriz de  coeficientes,

\begin{equation*}
A \rightarrow P\cdot A = L\cdot U
\end{equation*}

Podemos transformar nuestro sistema de ecuaciones en uno equivalente aplicando la matriz de permutaciones, por la izquierda, a ambos lados del igual. El efecto es equivalente a que cambi�ramos el orden en que se presentan las ecuaciones del sistema,

\begin{equation*}
A\cdot x=b\rightarrow P\cdot A \cdot x=P\cdot b
\end{equation*}

Si sustituimos ahora $P\cdot A$ por su factorizaci�n LU,

\begin{equation*}
P\cdot A \cdot x=P\cdot b \rightarrow L\cdot U \cdot x= P\cdot b
\end{equation*}

El nuevo sistema puede resolverse en dos pasos empleando sustituciones regresivas y sustituciones progresivas. Para ello, asociamos el producto $U\cdot x$, a un vector de inc�gnitas auxiliar al que llamaremos $z$,

\begin{equation*}
U\cdot x=z
\end{equation*}

Si sustituimos nuestro vector auxiliar $z$ en la expresi�n matricial de nuestro sistema de ecuaciones,

\begin{equation*}
L\cdot \overbrace{U\cdot x}^z=P\cdot b \rightarrow L\cdot z=P\cdot b
\end{equation*}

El sistema resultante es triangular inferior, por lo que podemos resolverlo por sustituciones progresivas, y obtener de este modo los valores de $z$. Podemos finalmente obtener la soluci�n del sistema a trav�s de la definici�n de $z$; $U\cdot x =z$, se trata de un sistema triangular superior, que podemos resolver mediante sustituciones regresivas.

Veamos un ejemplo. Supongamos que queremos resolver el sistema de ecuaciones lineales,

\begin{equation*}
\begin{pmatrix}
1& 3& 2\\
2& -1& 1\\
1& 4& 3
\end{pmatrix}\cdot \begin{pmatrix}
x_1\\
x_2\\
x_3
\end{pmatrix}=\begin{pmatrix}
13\\
3\\
18
\end{pmatrix}
\end{equation*}

En primer lugar deber�amos comprobar que la matriz de coeficiente esta bien condicionada,
\begin{verbatim}
>> A=[1 3 2;2 -1 1;1 4 3]
A =

     1     3     2
     2    -1     1
     1     4     3
>> nc=cond(A)
nc =

   24.3827
\end{verbatim}
No es un valor grande, con lo que podemos considerar que $A$ est� bien condicionada. 
Calculamos la factorizaci�n LU de la matriz de coeficientes, para ello podemos emplear el programa lufact.m, incluido en el cap�tulo anterior \ref{lufact}, o bien la funci�n \texttt{lu} de Matlab,

\begin{verbatim}
>> [L U P]=lufact(A)

L =

    1.0000         0         0
    0.5000    1.0000         0
    0.5000    0.7778    1.0000

U =

    2.0000   -1.0000    1.0000
         0    4.5000    2.5000
         0         0   -0.4444

P =

     0     1     0
     0     0     1
     1     0     0
\end{verbatim}

A continuaci�n debemos aplicar la matriz de permutaciones, al vector de t�rminos independientes del sistema, para poder construir el sistema equivalente $L\cdot U=P\cdot b$,
\begin{verbatim}
>> b=[13;3;18]

b =

    13
     3
    18

>> bp=P*b

bp =

     3
    18
    13

\end{verbatim}

Empleamos la matriz $L$ obtenida y el producto $bp=P\cdot b$ que acabamos de calcular, para obtener, por sustituciones progresivas, el vector auxiliar $z$ descrito m�s arriba. Empleamos para ello la funci�n \texttt{progresivas}, cuyo c�digo incluimos en la secci�n anterior, 
\begin{verbatim}
>> z=progresivas(L,bp)
z =

    3.0000
   16.5000
   -1.3333
\end{verbatim}

Finalmente, podemos obtener la soluci�n del sistema sin m�s que aplicar el m�todo de las sustituciones regresiva a la matriz $U$ y al vector auxiliar $z$ que acabamos de obtener,\footnote{La funci�n \texttt{regresivas} no se ha suministrado ni existe en Matlab. Su construcci�n se ha dejado como ejercicio en la secci�n anterior.}

\begin{verbatim}
>> x=regresivas(U,z)
x =

    1.0000
    2.0000
    3.0000


\end{verbatim}
Para comprobar que la soluci�n es correcta basta multiplicar la matriz de coeficientes del sistema original por el resultado obtenido para $x$ y comprobar que obtenemos como resultado el vector de t�rminos independientes.
\begin{verbatim}
>> A*x

ans =

    13
     3
    18
\end{verbatim}
 
\paragraph{Factorizaci�n de Cholesky.} Como vimos en el capitulo anterior \ref{chol}, la factorizaci�n de Cholesky permite descomponer una matriz $A$ en el producto de de una matriz triangular inferior, por su traspuesta. 
\begin{equation*}
A=L\cdot L^T
\end{equation*}
Para ello, es preciso que $A$ sea sim�trica y definida positiva (ver \ref{tiposm}). 

Por tanto, en el caso particular de un sistema cuya matriz de coeficientes fuera sim�trica y definida positiva, podr�amos descomponerla empleando la factorizaci�n de Cholesky y resolver el sistema de modo an�logo a como hicimos con con la factorizaci�n LU, sustituimos $A$ por el producto $L\cdot L^T$,
\begin{equation*}
A\cdot x = b \rightarrow L\cdot L^T\cdot x= b
\end{equation*}

Definimos el vector auxiliar $z$,
\begin{equation*}
L^T\cdot x= z
\end{equation*}

Resolvemos por sustituciones progresivas el sistema,

\begin{equation*}
L\cdot z=b
\end{equation*}

y por �ltimo obtenemos $x$ resolviendo por sustituciones regresivas el sistema,

\begin{equation*}
L^T\cdot x=z
\end{equation*}

La siguiente secuencia de comandos de Matlab muestra la resoluci�n del sistema,
\begin{equation*}
\begin{pmatrix}
2& 5& 1\\
5& 14& 2\\
1& 2& 6
\end{pmatrix}\cdot \begin{pmatrix}
x_1\\
x_2\\
x_3
\end{pmatrix}=\begin{pmatrix}
15\\
39\\
23
\end{pmatrix}
\end{equation*}

\begin{verbatim}
>> A=[2 5 1;5 14 2;1 2 6]
A =
     2     5     1
     5    14     2
     1     2     6

>> b=[15;39;23]
b =
    15
    39
    23

>> L=cholesky(A)
L =
    1.4142         0         0
    3.5355    1.2247         0
    0.7071   -0.4082    2.3094

>> z=progresivas(L,b)
z =
   10.6066
    1.2247
    6.9282

>> x=regresivas(L',z)
x =
    1.0000
    2.0000
    3.0000

>> A*x
ans =
    15
    39
    23
\end{verbatim}

En este ejemplo se ha empleado la funci�n \texttt{cholesky}, cuyo c�digo se incluy� en el cap�tulo anterior \ref{chol}, para factorizar la matriz de coeficientes del sistema. La factorizaci�n se podr�a haber llevado a cabo igualmente, empleando la funci�n de Matlab \texttt{chol}\footnote{Matlab, por defecto, devuelve una matriz triangular superior: \texttt{U = chol(A)}. La factorizaci�n es la misma que la descrita en este apartado. Simplemente $L^T = U$ y $L = U^T$. Por tanto, si se emplea directamente el comando \texttt{chol}   de matlab: $A\cdot x = b \rightarrow U^T\cdot U\cdot x= b$}

\paragraph{Factorizaci�n QR} Como vimos en el cap�tulo anterior \ref{QR}, la factorizaci�n QR, descompone una matriz en el producto de una matriz ortogonal  (ver \ref{tiposm}) $Q$  por una matriz triangular superior $R$. Si obtenemos la factorizaci�n QR de la matriz de coeficientes de un sistema,

\begin{equation*}
A\cdot x=b \rightarrow Q\cdot R\cdot x =b
\end{equation*}

Podemos resolver ahora el sistema en dos pasos. En primer lugar, como Q es ortogonal, $Q^{-1}=Q^T$, podemos multiplicar por $Q^T$ a ambos lados de la igualdad,

\begin{equation*}
Q\cdot R\cdot x =b \rightarrow Q^T\cdot Q\cdot R\cdot x = Q^T \cdot b \rightarrow R\cdot x= Q^T\cdot b
\end{equation*}

Pero el sistema resultante, es un sistema triangular superior, por lo que podemos resolverlo por sustituciones regresivas. Tomando el mismo ejemplo que resolvimos antes por factorizaci�n LU,

\begin{equation*}
\begin{pmatrix}
1& 3& 2\\
2& -1& 1\\
1& 4& 3
\end{pmatrix}\cdot \begin{pmatrix}
x_1\\
x_2\\
x_3
\end{pmatrix}=\begin{pmatrix}
13\\
3\\
18
\end{pmatrix}
\end{equation*}

podemos ahora resolverlo mediante factorizaci�n QR. Para ello aplicamos la funci�n \texttt{QRF2} incluida en el cap�tulo anterior \ref{QR}, o directamente la funci�n de Matlab \texttt{qr}, a la matriz de coeficientes del sistema,

\begin{verbatim}
>> A=[1 3 2;2 -1 1;1 4 3]
A =
     1     3     2
     2    -1     1
     1     4     3

>> [Q,R]=QRF2(A)
Q =
    0.4082    0.4637   -0.7863
    0.8165   -0.5707    0.0874
    0.4082    0.6777    0.6116

R =
    2.4495    2.0412    2.8577
         0    4.6726    2.3898
         0         0    0.3495
\end{verbatim}

A continuaci�n multiplicamos $Q^T$ por el vector de t�rminos independientes,

\begin{verbatim}
>> b=[13;3;18]
b =
    13
     3
    18

>> z=Q'*b
z =
   15.1052
   16.5147
    1.0484
\end{verbatim}

Por �ltimo resolvemos el sistema $R\cdot x= Q^T\cdot b$ mediante sustituciones regresivas y comprobamos el resultado,

\begin{verbatim}
>> x=regresivas(R,z)

x =

    1.0000
    2.0000
    3.0000

>> A*x

ans =

   13.0000
    3.0000
   18.0000
\end{verbatim}

\paragraph{Factorizaci�n SVD.} La  factorizaci�n svd \ref{sec:SVD}, descompone una matriz cualquiera en el producto de tres matrices,
\begin{equation*}
A=U\cdot S\cdot V^T
\end{equation*}
Donde $U$ y $V$ son matrices ortogonales y S es una matriz diagonal. Si calculamos la factorizaci�n svd de la matriz de coeficiente de un sistema,

\begin{equation*}
A\cdot x=b \rightarrow U\cdot S\cdot V^T\cdot x=b
\end{equation*}

Como en el caso de la factorizaci�n QR, podemos aprovechar la ortogonalidad de las matrices $U$ y $V$ para simplificar el sistema,

\begin{equation*}
U\cdot S\cdot V^T\cdot x=b \rightarrow U^T\cdot U\cdot S\cdot V^T \cdot x= U^T\cdot b \rightarrow S\cdot V^T \cdot x=U^T\cdot b
\end{equation*}

Como en casos anteriores, podemos crear un vector auxiliar $z$,
\begin{equation*}
V^T\cdot x=z
\end{equation*}

de modo que resolvemos primero el sistema,
\begin{equation*}
S\cdot z= U^T\cdot b
\end{equation*}

Como la matriz $S$ es diagonal, se trata de un sistema diagonal que, como hemos visto, es trivial de resolver.

Una vez conocido $Z$ podemos obtener la soluci�n del sistema original, haciendo ahora uso de la ortogonalidad de la matriz $V$,

\begin{equation*}
V^T\cdot x=z \rightarrow V\cdot V^T\cdot x=V\cdot z \rightarrow x=V\cdot z
\end{equation*}

Volvamos a nuestro ejemplo,
\begin{equation*}
\begin{pmatrix}
1& 3& 2\\
2& -1& 1\\
1& 4& 3
\end{pmatrix}\cdot \begin{pmatrix}
x_1\\
x_2\\
x_3
\end{pmatrix}=\begin{pmatrix}
13\\
3\\
18
\end{pmatrix}
\end{equation*}

En primer lugar factorizamos la matriz de coeficientes empleando el comando \texttt{svd} de Matlab,

\begin{verbatim}
>> A=[1 3 2;2 -1 1;1 4 3]
A =

     1     3     2
     2    -1     1
     1     4     3

>> [U,S,V]=svd(A)
U =

   -0.5908   -0.0053   -0.8068
   -0.0411   -0.9985    0.0366
   -0.8058    0.0548    0.5897

S =

    6.3232         0         0
         0    2.4393         0
         0         0    0.2593


V =

   -0.2339   -0.7983   -0.5549
   -0.7835    0.4927   -0.3786
   -0.5757   -0.3463    0.7408

\end{verbatim}

Calculamos a continuaci�n el valor del vector auxiliar $z$,

\begin{verbatim}
>> ba=U'*b
ba =
  -22.3076
   -2.0777
    0.2360

>> z=S^-1*ba
z =
   -3.5279
   -0.8518
    0.9101
\end{verbatim}
por �ltimo, calculamos $x$ y comprobamos la soluci�n obtenida,

\begin{verbatim}
>> x=V*z
x =
    1.0000
    2.0000
    3.0000

>> A*x
ans =
   13.0000
    3.0000
   18.0000
\end{verbatim}

\subsection{El m�todo de eliminaci�n de Gauss.}\index{Gauss! M�todo de eliminaci�n gaussiana} El m�todo de eliminaci�n gaussiana es el mismo descrito en el cap�tulo anterior \ref{sec:LU} para obtener la matriz triangular superior $U$ en la factorizaci�n LU de una matriz. Como ya se explic� en detalle, el m�todo consiste en \emph{hacer  cero} todos los elementos situados por debajo de la diagonal de una matriz. Para ello se sustituyen progresivamente las filas de la matriz, exceptuando la primera, por combinaciones adecuadas de dicha fila con las anteriores.

Toda la discusi�n incluida en el cap�tulo anterior sobre la eliminaci�n de Gauss, es v�lida tambi�n para la resoluci�n de sistemas, por lo que no volveremos a repetirla. Nos centraremos solo en su aplicaci�n al problema de resolver un sistema de ecuaciones lineales.

La idea fundamental, es sacar partido de las siguientes propiedades de todo sistema de ecuaciones lineales;
\begin{enumerate}
\item Un sistema de ecuaciones lineales no cambia aunque se altere el orden de sus ecuaciones.
\item Un sistema de ecuaciones lineales no cambia aunque se multiplique cualquiera de sus ecuaciones por una constante distinta de cero.
\item Un sistema de ecuaciones no cambia si se sustituye cualquiera de sus ecuaciones por una combinaci�n lineal de ella con otra ecuaci�n.
\end{enumerate}
 
Si usamos la representaci�n matricial de un sistema de ecuaciones, Cualquiera de los cambios descritos en las propiedades anteriores afecta tanto a la matriz de coeficientes como al vector de t�rminos independientes, por ejemplo dado el sistema,
 
\begin{equation*}
\begin{pmatrix}
1& 3& 2\\
2& -1& 1\\
1& 4& 3
\end{pmatrix}\cdot \begin{pmatrix}
x_1\\
x_2\\
x_3
\end{pmatrix}=\begin{pmatrix}
13\\
3\\
18
\end{pmatrix}
\end{equation*}
Si cambio de orden la segunda ecuaci�n con la primera obtengo el siguiente sistema equivalente,

\begin{equation*}
\begin{pmatrix}
2& -1& 1\\
1& 3& 2\\
1& 4& 3
\end{pmatrix}\cdot \begin{pmatrix}
x_1\\
x_2\\
x_3
\end{pmatrix}=\begin{pmatrix}
3\\
13\\
18
\end{pmatrix}
\end{equation*}

Es decir, se intercambia la primera fila de la matriz de coeficientes con la segunda, y el primer elemento del vector de t�rminos independientes con el segundo. El vector de inc�gnitas permanece inalterado. 

Si ahora sustituimos la segunda fila, por la diferencia entre ella y la primera multiplicada por $0.5$, obtenemos de nuevo un sistema equivalente, 

\begin{equation*}
\begin{pmatrix}
2& -1& 1\\
0& 3.5& 1.5\\
1& 4& 3
\end{pmatrix}\cdot \begin{pmatrix}
x_1\\
x_2\\
x_3
\end{pmatrix}=\begin{pmatrix}
3\\
11.5\\
18
\end{pmatrix}
\end{equation*}

Acabamos de dar los dos primeros pasos en el proceso de eliminaci�n de Gauss para convertir la matriz de coeficientes del sistema en una matriz triangular superior: hemos 'pivoteado' las dos primeras filas y despu�s hemos transformado en cero el primer elemento de la segunda fila, combin�ndola con la primera. Hemos aplicado tambi�n esta misma combinaci�n al segundo elemento del vector de t�rminos independientes, para que el sistema obtenido sea equivalente al original.

Para poder trabajar de una forma c�moda con el m�todo de eliminaci�n de Gauss, se suele construir una matriz, --conocida con el nombre de matriz ampliada--, a�adiendo a la matriz de coeficientes, el vector de t�rminos independientes como una columna m�s,

\begin{equation*}
A,\ b \rightarrow AM=(A\vert b)
\end{equation*} 

En nuestro ejemplo,

\begin{equation*}
\begin{pmatrix}
1& 3& 2\\
2& -1& 1\\
1& 4& 3
\end{pmatrix},\ \begin{pmatrix}
13\\
3\\
18
\end{pmatrix} \rightarrow
\begin{pmatrix}
1& 3& 2& {\color{red}13}\\
2& -1& 1& {\color{red}3}\\
1& 4& 3& {\color{red}18}
\end{pmatrix}
\end{equation*}

Podemos aplicar directamente a la matriz ampliada $AM$ el programa de eliminaci�n de Gauss, \texttt{eligauss}, que incluimos en el cap�tulo anterior \ref{elig}, en la secci�n dedicada a la factorizaci�n LU. Si aplicamos \texttt{eligauss} a la matriz ampliada del sistema del ejemplo, 

\begin{verbatim}
>> A=[1 3 2;2 -1 1;1 4 3]
A =

     1     3     2
     2    -1     1
     1     4     3

>> b=[13;3;18]
b =

    13
     3
    18

>> AM=[A b]
AM =
     1     3     2    13
     2    -1     1     3
     1     4     3    18
     

>> GA=eligauss(AM)
GA =

    1.0000    3.0000    2.0000   13.0000
         0   -7.0000   -3.0000  -23.0000
         0         0    0.5714    1.7143
\end{verbatim}

El programa ha obtenido como resultado una nueva matriz en la que los elementos situados por debajo de la diagonal son ahora cero. Podemos reconstruir, a partir del resultado obtenido un sistema equivalente actual Separando la �ltima columna del resultado,

\begin{equation*}
\begin{pmatrix}
1&    3&    2&   13\\
0&   -7&   -3&  -23\\
0&    0&    0.5714&    1.7143
\end{pmatrix}\rightarrow \begin{pmatrix}
1&    3&    2\\
0&   -7&   -3\\
0&    0&    0.571
\end{pmatrix}\cdot \begin{pmatrix}
x_1\\
x_2\\
x_3
\end{pmatrix}=\begin{pmatrix}
13\\
-23\\
1.7143
\end{pmatrix}
\end{equation*}

El sistema resultante de la eliminaci�n de Gauss es triangular superior, con lo que podemos resolverlo directamente mediante sustituciones regresivas,

\begin{verbatim}
>> G=GA(:,1:3)
G =

    1.0000    3.0000    2.0000
         0   -7.0000   -3.0000
         0         0    0.5714
>> bp=GA(:,4)
bp =

   13.0000
  -23.0000
    1.7143
>> x=regresivas(G,bp)
x =
    1.0000
    2.0000
    3.0000
\end{verbatim}

El programa \texttt{eligauss}, presenta el problema de que no incluye el pivoteo de filas. Como se explic� en el cap�tulo anterior este en necesario para evitar que un cero o un valor relativamente peque�o en la diagonal, impida completar el proceso de eliminaci�n o haga los c�lculos inestables. A continuaci�n, incluimos una versi�n modificada de \texttt{eligauss} que incluye el pivoteo de filas.

\begin{lstlisting}
function U=eligaussp(A)
% Esta funci�n obtiene una matriz triangular superior, a partir de una
% matriz dada, aplicando el m�todo de eliminaci�n gaussiana.
% realiza pivoteo de filas, as� que lo primero que hace es comprobar si el
% el elemento de la diagonal, de la fila que se va a emplear para eliminar
% el mayor que los que tiene por debajo en su columna, si no es as�,
% intercambia la fila con la que tenga el elemento mayor en dicha columna.

% Obtenemos el n�mero de filas de la matriz..
nf=size(A,1);
U=A;
%
for j=1:nf-1 % recorro todas la columnas menos la �ltima
%%%%%   pivoteo de filas%%%%%%%%%%%%%%%%%%%%%%%%%%    
    % Buscamos el elemento mayor de la columna j de la diagonal para abajo
    maxcol=abs(U(j,j));
    index=j;
    for l=j:nf
        if abs(U(l,j))>maxcol
            maxcol=abs(U(l,j));
            index=l;
        end
    end
    % si el mayor no era el elemento de la diagonal U(j,j), intecambiamos la
    % fila l con la fila j
    if index~=j
        aux=U(j,:);
        U(j,:)=U(index,:);
        U(index,:)=aux;
    end
%%%%%    fin del pivoteo de filas%%%%%%%%%%%%%%%%%%%%%    
    for i=j+1:nf % Recorro las filas desde debajo de la diagonal hasta la 
        % �ltima en Matlab tengo la suerte de poder manejar cada fila de un 
        % solo golpe.
        U(i,:)=U(i,:)-U(j,:)*U(i,j)/U(j,j);
    end
end
\end{lstlisting}

Si aplicamos esta funci�n a nuestro ejemplo de siempre,
\begin{verbatim}
>> GA=eligaussp(AM)
GA =

     1     4     3    18
     0    -1    -1    -5
     0     0     4    12
\end{verbatim}

y separando en la matriz ampliada la matriz de coeficientes y el vector de t�rminos independientes podemos resolver el sistema por sustituciones regresivas,

\begin{verbatim}
>> G=GA(:,1:3)
G =

     1     4     3
     0    -1    -1
     0     0     4

>> bp=GA(:,4)
bp =
    18
    -5
    12

>> x=regresivas(G,bp)
x =
     1
     2
     3

\end{verbatim}

\subsection{Gauss-Jordan  y matrices en forma reducida escalonada}\index{Gauss-Jordan! eliminaci�n}
El m�todo de eliminaci�n de Gauss, permite obtener a partir de una matriz arbitraria, una  matriz triangular superior. Una vez obtenida �sta, podr�amos seguir transformando la matriz de modo que hici�ramos cero todos los elementos situados por encima de su diagonal. Para ello bastar�a aplicar el mismo m�todo de eliminaci�n de Gauss, la diferencia es que ahora eliminar�amos --har�amos ceros-- los elementos situados por encima de la diagonal principal de la matriz, empezando por la �ltima columna y movi�ndonos de abajo a arriba y de derecha a izquierda. Este proceso se conoce con el nombre de eliminaci�n de Gauss-Jordan.
Por ejemplo supongamos que partimos de la matriz que acabamos de obtener por eliminaci�n de Gauss en ejemplo anterior,

\begin{equation*}
GA=\begin{pmatrix}
1&     4&     3&    18\\
 0&    -1&    -1&    -5\\
 0&     0&     4&    12
\end{pmatrix}
\end{equation*}  

Empezar�amos por hacer cero el elemento $ga_{23}$ que es el que est� situado encima de �ltimo elemento de la diagonal principal, para ello restar�amos a la segunda columna la tercera multiplicada por $-1$ y dividida por $4$,

\begin{equation*}
\begin{pmatrix}
1&     4&     3&    18\\
 0-0&    -1-0&    -1+4/4&    -5+12/4\\
 0&     0&     4&    12
\end{pmatrix}=\begin{pmatrix}
1&     4&     3&    18\\
 0&    -1&    0&    -2\\
 0&     0&     4&    12
 \end{pmatrix}
\end{equation*}  

A continuaci�n, eliminar�amos el elemento situado en la misma columna, una fila por encima. Para ello restamos a la primera la tercera multiplicada por $3$ y dividida por $4$,

\begin{equation*}
\begin{pmatrix}
1-0&     4-0&     3-3\cdot 4/4&    18-12\cdot 3/4\\
 0&    -1&    0&    -2\\
 0&     0&     4&    12
\end{pmatrix}=\begin{pmatrix}
1&     4&     0&    9\\
 0&    -1&    0&    -2\\
 0&     0&     4&    12
\end{pmatrix}
\end{equation*}

Como hemos llegado a la primera columna, pasamos a eliminar los elementos de la siguiente columna de la izquierda. En este ejemplo solo tenemos un elemento por encima de la diagonal. Para eliminarlo restamos de la primera fila la segunda multiplica por $4$ y dividida por $-1$,

\begin{equation*}
\begin{pmatrix}
1-0&     4-4\cdot(-1)/(-1)&     0&    9-(-2)\cdot4/(-1)\\
 0&    -1&    0&    -2\\
 0&     0&     4&    12
\end{pmatrix}=\begin{pmatrix}
1&     0&     0&    1\\
 0&    -1&    0&    -2\\
 0&     0&     4&    12
\end{pmatrix}
\end{equation*}

Si ahora separamos en la matriz ampliada resultante, la matriz de coeficientes y el vector de t�rminos independientes, obtenemos un sistema diagonal, cuya soluci�n es trivial,

\begin{equation*}
\begin{pmatrix}
1&     0&     0&    1\\
 0&    -1&    0&    -2\\
 0&     0&     4&    12
\end{pmatrix} \rightarrow \begin{pmatrix}
1&     0&     0\\
 0&    -1&    0\\
 0&     0&     4\\
\end{pmatrix}\cdot\begin{pmatrix}
x_1\\
x_2\\
x_3
\end{pmatrix}=\begin{pmatrix}
1\\
-2\\
12
\end{pmatrix} \Rightarrow x=\begin{pmatrix}
1\\
2\\
3\\
\end{pmatrix}
\end{equation*}

El siguiente programa \texttt{gauss-jordan}, a�ade las l�neas de c�digo necesarias a \texttt{eligaussp} para realizar la eliminaci�n de Gauss-Jordan completa de una matriz.

\begin{lstlisting}
function U=gauss_jordan(A)
% Esta funci�n realiza  la eliminaci�n de GAUSS-JORDAN, que permite obtener,
% a partir de una matriz dada, una matriz cuyos elementos por encima y 
% debajo de la diagonal son todos cero.

% Obtenemos el n�mero de filas de la matriz..
nf=size(A,1);
U=A;
%
% primera parte: reducci�n a una matriz triangular pro eliminaci�n
% progresiva
for j=1:nf-1 % recorro todas la columnas menos la �ltima
%%%%%%     pivoteo de filas%%%%%%%%%%%%%%%%%%%%%%%%    
    % Buscamos el elemento mayor de la columna j de la diagonal para abajo
    maxcol=abs(U(j,j));
    index=j;
    for l=j:nf
        if abs(U(l,j))>maxcol
            maxcol=abs(U(l,j));
            index=l;
        end
    end
    % si el mayor no era el elemento de la diagonal U(j,j), intecambiamos la
    % fila l con la fila j
    if index~=j
        aux=U(j,:);
        U(j,:)=U(index,:);
        U(index,:)=aux;
    end
%%%%    fin del pivoteo de filas%%%%%%%%%%%%%%%%%%%%%%    
    for i=j+1:nf % Recorro las filas desde debajo de la diagonal hasta la 
        % �ltima en Matlab tengo la suerte de poder manejar cada fila de un 
        % solo golpe.
        U(i,:)=U(i,:)-U(j,:)*U(i,j)/U(j,j);
    end
end


% segunda parte, obtenci�n de una matriz diagonal mediante eliminaci�n
% regresiva, Recorremos ahora las columnas en orden inverso empezando por el
% final... (Eliminaci�n de Gauss Jordan)

for j=nf:-1:2
    for i=j-1:-1:1 % Recorro las filas desde encima de la diagonal hasta la
        % primera, 
        U(i,:)=U(i,:)-U(j,:)*U(i,j)/U(j,j);
    end
end
\end{lstlisting}

Si tras aplicar la eliminaci�n de Gauss-Jordan a la matriz ampliada de un sistema, dividimos cada fila por el elemento que ocupa la diagonal principal, obtendr�amos en la �ltima columna las soluciones del sistema. En nuestro ejemplo,

\begin{equation*}
\begin{pmatrix}
1&     0&     0&    1\\
 0&    1&    0&    2\\
 0&     0&     1&    3
\end{pmatrix}
\end{equation*}
La matriz resultante se dice que esta en \emph{forma escalonada reducida por filas}\index{Matrices escalonadas}. Se deja como un ejercicio, a�adir el c�digo necesario al programa anterior para que d� como resultado la \emph{forma escalonada reducida por filas} de la matriz ampliada de un sistema. Como siempre, Matlab tiene su propia funci�n para obtener formas escalonadas reducidas por filas. Se trata de la funci�n \texttt{rref} (el nombre es la abreviatura en ingl�s de Row Reduced Echelon Form).

\begin{verbatim}
>> A
A =

     1     3     4     5    39
     2    -1     2     3    18
     4    -3     2     1     8
     2     3     4    -2    12

>> EF=rref(A)
EF =

     1     0     0     0     1
     0     1     0     0     2
     0     0     1     0     3
     0     0     0     1     4
\end{verbatim}
 
\section{M�todos iterativos}
Los m�todos iterativos se basan en una aproximaci�n distinta al problema de la resoluci�n de sistemas de ecuaciones lineales. La idea en todos ellos es buscar un m�todo que, a partir de un valor inicial para la soluci�n del sistema, vaya refin�ndolo progresivamente, acerc�ndose cada vez m�s a la soluci�n real. La figura \ref{fig:itera}, muestra un diagrama de flujo general para todos los m�todos iterativos de resoluci�n de sistemas.

\begin{figure}[h]
\centering
\begin{tikzpicture}
%\usetikzlibrary{shapes.geometric}
\path (5,0) node(a) [rectangle,draw=blue, very thick,align=center,rounded corners]{Soluci�n inicial $x^{(s)}=x^{(0)}$}
(5,-2) node(b)[rectangle,draw=blue, thick,rounded corners,align=center]{Calculamos\\ $x^{(s+1)}=M(x^{(s)})$}
(5,-4) node(c)[diamond,aspect=3,draw=red,thick]{es $\lVert  x^{(s+1)}-x^{(s)} \rVert \le \text{tol}$?}
(9.5,-4) node(d)[rectangle,draw=blue,align=center,very thick, rounded corners]{convergencia:\\ terminar}
(5,-6) node(g)[rectangle,draw=blue,thick,rounded corners,align=center]{$x^{(s)}=x^{(s+1)}$};
\draw[blue,-latex](a.south)--(b);
\draw[blue,-latex](b.south)--(c);
\draw[blue,-latex](c.east)--(d);
\draw (7.7,-4)node[above]{S�};
\draw[blue,-latex](c.south)--(g);
\draw (5,-5.1)node[right]{No};
\draw[blue,-latex](g.south)|-(2,-7)|-(b);

\end{tikzpicture}
\caption{Diagrama de flujo general de los m�todos iterativos para resolver sistemas de ecuaciones. La funci�n $M(x)$ es la que especifica en cada caso el m�todo.}
\label{fig:itera}
\end{figure}

Siguiendo el diagrama de flujo, el primer paso, es proponer un vector con soluciones del sistema. Si se conocen valores pr�ximos a las soluciones reales se eligen dichos valores como soluci�n inicial. Si, como es habitual, no se tiene idea alguna de cuales son las soluciones, lo m�s habitual es empezar con el vector $(0)$,
\begin{equation*}
x^{(0)}=\begin{pmatrix}
0\\
0\\
\vdots \\
0
\end{pmatrix}
\end{equation*}

A partir de la primera soluci�n, se calcula una segunda, siguiendo la especificaciones concretas del m�todo que se est� usando. En el diagrama de flujo se ha representado de modo gen�rico el m�todo mediante la funci�n $M(\cdot)$. Dedicaremos buena parte de esta secci�n a estudiar algunos de los m�todos m�s usuales.

Una vez que se tienen dos soluciones se comparan. Para ello, se ha utilizado el m�dulo del vector diferencia de las dos soluciones. Este m�dulo nos da una medida de cuanto se parecen entre s� las dos soluciones. Si la diferencia es suficientemente peque�a, --menor que un cierto valor de tolerancia $tol$-- damos la soluci�n por buena y el algoritmo termina. En caso contrario, copiamos la �ltima soluci�n obtenida en la pen�ltima y repetimos todo el proceso. El bucle se repite hasta que se cumpla la condici�n de que el m�dulo de la diferencia de dos soluciones sucesivas sea menor que la tolerancia establecida.

Una pregunta que surge de modo inmediato del esquema general que acabamos de introducir es si el proceso descrito converge; y, caso de hacerlo, si converge a la soluci�n correcta del sistema. 

La respuesta es que los sistemas iterativos no siempre convergen, es decir, no esta garantizado que tras un cierto n�mero de iteraciones la diferencia entre dos soluciones sucesivas sea menor que un valor de tolerancia arbitrario. La convergencia, como veremos m�s adelante, depender� tanto del sistema que se quiere resolver como del m�todo iterativo empleado. Por otro lado, lo que s� se cumple siempre es que, si el m�todo converge, las sucesivas soluciones obtenidas se van aproximando a la soluci�n real del sistema.

\subsection{El m�todo de Jacobi.} \index{M�todo de Jacobi}

\paragraph{Obtenci�n del algoritmo.}Empezaremos por introducir el m�todo iterativo de Jacobi, por ser el m�s intuitivo de todos. Para introducirlo, emplearemos un ejemplo sencillo. Supongamos que queremos resolver el siguiente sistema de ecuaciones,

\begin{align*}
3x_1+3x_2&=6\\
3x_1+4x_2&=7
\end{align*}

Supongamos que supi�ramos de antemano el valor de $x_2$, para obtener $x_1$, bastar�a entonces despejar $x_1$, por ejemplo de la primera ecuaci�n, y sustituir el valor conocido de $x_2$,

\begin{equation*}
x_1=\frac{6-3x_2}{3}
\end{equation*}

De modo an�logo, si conoci�ramos previamente $x_1$, podr�amos despejar $x_2$, ahora por ejemplo de la segunda ecuaci�n, y sustituir el valor conocido de $x_1$,

\begin{equation*}
x_2=\frac{7-3x_1}{4}
\end{equation*}

El m�todo de Jacobi lo que hace es \emph{suponer} conocido el valor de $x_2$, $x_2^{(0)}=0$, y con �l obtener un valor de $x_1^{(1)}$,
\begin{equation*}
x_1^{(1)}=\frac{6-3x_2^{(0)}}{3}=\frac{6-3\cdot 0}{3}=2
\end{equation*}

a continuaci�n \emph{supone} conocido el valor de $x_1$, $x_1^{(0)}=0$ y con el obtiene un nuevo valor para $x_2$,

\begin{equation*}
x_2^{(1)}=\frac{7-3x_1^{(0)}}{4}=\frac{7-3\cdot 0}{4}=1.75
\end{equation*}

En el siguiente paso, tomamos los valores obtenidos, $x_1^{(1)}$ y $x_2^{(1)}$ como punto de partida para calcular unos nuevos valores,
\begin{align*}
x_1^{(2)}&=\frac{5-3x_2^{(1)}}{3}=\frac{6-3\cdot 1.75}{3}=0.25\\
x_2^{(2)}&=\frac{7-3x_1^{(1)}}{4}=\frac{7-3\cdot 1.67}{4}=0.25
\end{align*}

y en general,

\begin{align*}
x_1^{(s+1)}&=\frac{6-3x_2^{(s)}}{3}\\
x_2^{(s+1)}&=\frac{7-3x_1^{(s)}}{4}
\end{align*}

Si repetimos el mismo c�lculo diez veces obtenemos,

\begin{align*}
x_1^{(10)}&=0.7627\\
x_2^{(10)}&=0.7627
\end{align*}

Si lo repetimos veinte veces,

\begin{align*}
x_1^{(20)}&=0.9437\\
x_2^{(20)}&=0.9437
\end{align*}

La soluci�n exacta del sistema es $x_1=1$, $x_2=1$. Seg�n aumentamos el n�mero de iteraciones, vemos c�mo las soluciones obtenidas se aproximan cada vez m�s a la soluci�n real.

A partir del ejemplo, podemos generalizar el m�todo de Jacobi para un sistema cualquiera de $n$ ecuaciones con $n$ inc�gnitas,

\begin{align*}
a_{11}&x_1+a_{12}x_2+\cdots +a_{1n}x_n=b_1\\
a_{21}&x_1+a_{22}x_2+\cdots +a_{2n}x_n=b_2\\
\cdots & \\
a_{n1}&x_1+a_{n2}x_2+\cdots +a_{nn}x_n=b_n
\end{align*}

La soluci�n para la inc�gnita $x_i$ en la iteraci�n, $s+1$ a partir de la solucione obtenida en la iteraci�n $s$ toma la forma,

\begin{equation*}
x_i^{(s+1)}=\frac{b_i-\sum_{j\neq i}a_{ij}x_j^{(s)}}{a_{ii}}
\end{equation*}

A continuaci�n, incluimos el c�digo correspondiente al c�lculo de una iteraci�n del algoritmo de Jacobi. Este c�digo es el que corresponde, para el m�todo de jacobi, a la funci�n $M(\cdot)$ del diagrama de flujo de la figura \ref{fig:itera}.

\begin{lstlisting}
 	xs=xs1;
    % volvemos a inicializar el vector de soluciones al valor de los
    % terminos independientes
    xs1=b; 
    for i=1:n % bucle para recorrer todas las ecuaciones
        for j=1:i-1 % restamos la contribucion de todas las incognitas
            xs1(i)=xs1(i)-A(i,j)*xs(j);       % por encima de x(i)
        end
        for j=i+1:n % restamos la contribuci�n de todas las incognitas
            % por debajo de x(i)
            xs1(i)=xs1(i)-A(i,j)*xs(j);
        end
        % dividimos por el elemento de la diagonal,
        xs1(i)=xs1(i)/A(i,i);
    end
\end{lstlisting}

\paragraph{Expresi�n matricial para el m�todo de Jacobi.} \index{M�todo de Jacobi! Expresi�n matricial} El m�todo de Jacobi que acabamos de exponer puede expresarse tambi�n en forma matricial. En Matlab, el empleo en forma matricial, tiene la ventaja de ahorrar bucles en el c�lculo de la nueva soluci�n a partir de la anterior. Si expresamos un sistema general de orden $n$ en forma matricial,

\begin{equation*}
\overbrace{\begin{pmatrix}
a_{11}& a_{12}& \cdots & a_{1n}\\
a_{21}& a_{22}& \cdots & a_{2n}\\
\vdots & \vdots & \ddots & \vdots\\
a_{n1}& a_{n2}& \cdots & a_{nn}
\end{pmatrix}}^A \cdot \overbrace{\begin{pmatrix}
x_1\\
x_2\\
\vdots \\
x_n
\end{pmatrix}}^x=\overbrace{\begin{pmatrix}
b_1\\
b_2\\
\vdots \\
b_n
\end{pmatrix}}^b
\end{equation*}

Podr�amos separar en tres sumandos la expresi�n de la izquierda del sistema: una matriz diagonal $D$, una matriz estrictamente diagonal superior $U$, y una matriz estrictamente triangular inferior $L$,

\begin{equation*}
\left[\overbrace{\begin{pmatrix}
a_{11}& 0& \cdots & 0\\
0& a_{22}& \cdots & 0\\
\vdots & \vdots & \ddots & \vdots\\
0& 0& \cdots & a_{nn}
\end{pmatrix}}^D+
\overbrace{\begin{pmatrix}
0& a_{12}& \cdots & a_{1n}\\
0& 0& \cdots & a_{2n}\\
\vdots & \vdots & \ddots & \vdots\\
0& 0& \cdots & 0
\end{pmatrix}}^U + \overbrace{\begin{pmatrix}
0& 0& \cdots & 0\\
a_{21}& 0& \cdots & 0\\
\vdots & \vdots & \ddots & \vdots\\
a_{n1}& a_{n2}& \cdots & 0
\end{pmatrix}}^L \right]\cdot \overbrace{\begin{pmatrix}
x_1\\
x_2\\
\vdots \\
x_n
\end{pmatrix}}^x=\overbrace{\begin{pmatrix}
b_1\\
b_2\\
\vdots \\
b_n
\end{pmatrix}}^b
\end{equation*}

Si pasamos los t�rminos correspondientes a las dos matrices triangulares al lado derecho de la igualdad,

\begin{align*}
\overbrace{\begin{pmatrix}
a_{11}& 0& \cdots & 0\\
0& a_{22}& \cdots & 0\\
\vdots & \vdots & \ddots & \vdots\\
0& 0& \cdots & a_{nn}
\end{pmatrix}}^D &\cdot \overbrace{\begin{pmatrix}
x_1\\
x_2\\
\vdots \\
x_n
\end{pmatrix}}^x=&\\
=&\overbrace{\begin{pmatrix}
b_1\\
b_2\\
\vdots \\
b_n
\end{pmatrix}}^b-\left[
\overbrace{\begin{pmatrix}
0& a_{12}& \cdots & a_{1n}\\
0& 0& \cdots & a_{2n}\\
\vdots & \vdots & \ddots & \vdots\\
0& 0& \cdots & 0
\end{pmatrix}}^U+
\overbrace{\begin{pmatrix}
0& 0& \cdots & 0\\
a_{21}& 0& \cdots & 0\\
\vdots & \vdots & \ddots & \vdots\\
a_{n1}& a_{n2}& \cdots & 0
\end{pmatrix}}^L \right] \cdot \overbrace{\begin{pmatrix}
x_1\\
x_2\\
\vdots \\
x_n
\end{pmatrix}}^x
\end{align*}

Si examinamos las filas de la matrices resultantes a uno y otro lado del igual, es f�cil ver que cada una de ellas coincide con la expresi�n correspondiente a una iteraci�n del m�todo de Jacobi, sin mas que cambiar los valores de las inc�gnitas a la izquierda del igual por $x^{(s+1)}$ y la derecha por $x^{(s)}$,

\begin{align*}
\overbrace{\begin{pmatrix}
a_{11}& 0& \cdots & 0\\
0& a_{22}& \cdots & 0\\
\vdots & \vdots & \ddots & \vdots\\
0& 0& \cdots & a_{nn}
\end{pmatrix}}^D &\cdot \overbrace{\begin{pmatrix}
x_1^{(s+1)}\\
x_2^{(s+1)}\\
\vdots \\
x_n^{(s+1)}
\end{pmatrix}}^{x^{s+1}}=&\\
=&\overbrace{\begin{pmatrix}
b_1\\
b_2\\
\vdots \\
b_n
\end{pmatrix}}^b-\left[
\overbrace{\begin{pmatrix}
0& a_{12}& \cdots & a_{1n}\\
0& 0& \cdots & a_{2n}\\
\vdots & \vdots & \ddots & \vdots\\
0& 0& \cdots & 0
\end{pmatrix}}^U+
\overbrace{\begin{pmatrix}
0& 0& \cdots & 0\\
a_{21}& 0& \cdots & 0\\
\vdots & \vdots & \ddots & \vdots\\
a_{n1}& a_{n2}& \cdots & 0
\end{pmatrix}}^L \right] \cdot \overbrace{\begin{pmatrix}
x_1^s\\
x_2^s\\
\vdots \\
x_n^s
\end{pmatrix}}^{x^s}
\end{align*}

 y multiplicar a ambos lados por la inversa de la matriz $D$\footnote{$D$ es una matriz diagonal su inversa se calcula trivialmente sin m�s cambiar cada elemento de la diagonal por su inverso.},
 
 \begin{align*}
 \overbrace{\begin{pmatrix}
x_1^{(s+1)}\\
x_2^{(s+1)}\\
\vdots \\
x_n^{(s+1)}
\end{pmatrix}}^{x^{s+1}}= &
\overbrace{\begin{pmatrix}
a_{11}& 0& \cdots & 0\\
0& a_{22}& \cdots & 0\\
\vdots & \vdots & \ddots & \vdots\\
0& 0& \cdots & a_{nn}
\end{pmatrix}^{-1}}^{D^{-1}} \cdot\\
&\cdot \left[ \overbrace{\begin{pmatrix}
b_1\\
b_2\\
\vdots \\
b_n
\end{pmatrix}}^b- \left[\overbrace{\begin{pmatrix}
0& a_{12}& \cdots & a_{1n}\\
0& 0& \cdots & a_{2n}\\
\vdots & \vdots & \ddots & \vdots\\
0& 0& \cdots & 0
\end{pmatrix}}^U+
\overbrace{\begin{pmatrix}
0& 0& \cdots & 0\\
a_{21}& 0& \cdots & 0\\
\vdots & \vdots & \ddots & \vdots\\
a_{n1}& a_{n2}& \cdots & 0
\end{pmatrix}}^L \right] \cdot \overbrace{\begin{pmatrix}
x_1^s\\
x_2^s\\
\vdots \\
x_n^s
\end{pmatrix}}^{x^s}\right]  
\end{align*}

El resultado final es una operaci�n matricial,

\begin{equation*}
x^{(s+1)}=D^{-1}\cdot b- D^{-1}\cdot\left(L+U\right)\cdot x^s
\end{equation*}

que equivale al c�digo para una iteraci�n y, por tanto a la funci�n $M(\cdot)$ del m�todo de Jacobi. 

Si analizamos la ecuaci�n anterior, observamos que el t�rmino, $D^{-1}\cdot b$ es fijo. Es decir, permanece igual en todas la iteraciones que necesitemos realizar para que la soluci�n converja. El segundo t�rmino, tiene tambi�n una parte fija, $-D^{-1}(L+U)$ Se trata de una matriz de orden $n$, igual al del sistema que tratamos de resolver, Esta matriz, recibe el nombre de matriz del m�todo, se suele representar por la letra $H$ y como veremos despu�s esta directamente relaciona con la convergencia del m�todo.

Si queremos implementar en Matlab el m�todo de Jacobi en forma matricial, lo mas eficiente es que calculemos en primer lugar las matrices $D, L, U, f=D^{-1}\cdot b$ y $H$. Una vez calculadas, empleamos el resultado para aproximar la soluci�n iterativamente. Veamos con con un ejemplo como construir la matrices en Matlab. Supongamos que tenemos el siguiente sistema,
\begin{equation*}
\begin{pmatrix}
4& 2& -1\\
3& -5& 1\\
1& -1& 6
\end{pmatrix}\cdot \begin{pmatrix}
x_1\\
x_2\\
x_3
\end{pmatrix}=\begin{pmatrix}
5\\
-4\\
17
\end{pmatrix}
\end{equation*}


En primer lugar, calculamos la matriz D, a partir de la matriz de coeficientes del sistema, empleando el comando de Matlab \texttt{diag}. Aplicando \texttt{diag} a una matriz extraemos en un vector los elementos de su diagonal, 

\begin{verbatim}
A =

     4     2    -1
     3    -5     1
     1    -1     6
>> d=diag(A)
d =
     4
    -5
     6
\end{verbatim}

Aplicando \texttt{diag} a un vector construimos una matriz con los elementos del vector colocados sobre la diagonal de la matriz. El resto de los elementos son cero.

\begin{verbatim}
>> D=diag(d)
D =

     4     0     0
     0    -5     0
     0     0     6
\end{verbatim}

Si aplicamos dos veces \texttt{diag} sobre la matriz de coeficientes de un sistema, obtenemos directamente la matriz $D$,

\begin{verbatim}
>> D=diag(diag(A))
D =

     4     0     0
     0    -5     0
     0     0     6
\end{verbatim}

Para calcular las matrices $L$ y $U$, podemos emplear los comandos de Matlab, \texttt{triu} y \texttt{tril} que extraen una matriz triangular superior y una matriz triangular inferior, respectivamente, a partir de una matriz dada.
\begin{verbatim}
>> TU=triu(A)
TU =

     4     2    -1
     0    -5     1
     0     0     6

>> TL=tril(A)
TL =

     4     0     0
     3    -5     0
     1    -1     6
\end{verbatim}

Las matrices $L$ y $U$ son estrictamente triangulares superior e inferior, Podemos obtenerlas restando a las matrices que acabamos de obtener la matriz $D$,

\begin{verbatim}
>> U=triu(A)-D
U =

     0     2    -1
     0     0     1
     0     0     0

>> L=tril(A)-D
L =

     0     0     0
     3     0     0
     1    -1     0
\end{verbatim}

Tambi�n podemos obtenerlas directamente, empleando solo la matriz de coeficientes,

\begin{verbatim}
>> U=A-tril(A)

U =

     0     2    -1
     0     0     1
     0     0     0

>> L=A-triu(A)

L =

     0     0     0
     3     0     0
     1    -1     0
\end{verbatim}

Construimos a continuaci�n el vector $f$,

\begin{verbatim}
>> f=D^-1*b
f =

    1.2500
    0.8000
    2.8333
\end{verbatim}

Construimos la matriz del sistema,

\begin{verbatim}
>> H=-D^-1*(L+U)
H =

         0    -0.5000   0.2500
    0.6000          0   0.2000
   -0.1667     0.1667        0
\end{verbatim}


A partir de las matrices construidas, el c�digo para calcular una iteraci�n por el m�todo de Jacobi ser�a simplemente,

\begin{verbatim}
xs1=f+H*xs
\end{verbatim}


En el siguiente fragmento de c�digo se re�nen todas las operaciones descritas,

\begin{lstlisting}
...
...
% obtenemos el tama�o del sistema,

n=size(A,1);
% creamos un vector de soluciones inicial,
xs=zeros(n,1);
% creamos las matrices del m�todo
D=diag(diag(A));
U=A-tril(A);
L=A-triu(A);
% ALternativamente para jacobi podemos crear una solo matriz equivalente a
% L+U, LpU=A-D
f=D^-1*b;
H=-D^-1*(L+U);

% calculamos la primera iteraci�n,
xs1=f+H*xs;
% calculamos la diferencia entre las dos soluciones,
tolf=norm(xs1-xs);
% ponemos a 1 el contador de iteraciones.
it=1;

% a partir de aqu� vendr�a el c�digo necesario para calcular las
% sucesivas iteraciones hasta que la soluci�n converja
...
...
\end{lstlisting}

\subsection{El m�todo de Gauss-Seidel.} \index{M�todo de Gauss-Seidel}
\paragraph{Obtenci�n del algoritmo.} Este m�todo aplica una mejora, sencilla y l�gica al m�todo de Jacobi que acabamos de estudiar. Si volvemos a escribir la expresi�n general para calcular una iteraci�n de una de las componentes de la soluci�n de un sistema por el m�todo de Jacobi,

\begin{equation*}
x_i^{(s+1)}=\frac{b_i-\sum_{j\neq i}a_{ij}x_j^{(s)}}{a_{ii}}
\end{equation*}

Observamos que para obtener el t�rmino $x_i^{(s+1)}$ empleamos todos los t�rminos $x_j^{(s)}, \ j\neq i$ de la iteraci�n anterior. Sin embargo, es f�cil darse cuenta que, como el algoritmo va calculando las componentes de cada iteraci�n por orden, cuando vamos a calcular $x_i^{(s+1)}$, disponemos ya del resultado de todas las componentes anteriores de la soluci�n para la iteraci�n $s+1$. Es decir, sabemos ya cu�l es el resultado de $x_l^{(s+1)}, \  l<i$. Si el algoritmo converge, esta soluciones ser�n mejores que las obtenidas en la iteraci�n anterior. Si las usamos para calcular $x_i^{(s+1)}$ el resultado que obtendremos, ser� m�s pr�ximo a la soluci�n exacta y por tanto el algoritmo converge m�s deprisa.

La idea, por tanto ser�a: 

Para calcular $x_1^{(s+1)}$ procedemos igual que en el m�todo de Jacobi. 
\begin{equation*}
x_1^{(s+1)}=\frac{b_1-\sum_{j=2}^n a_{1j}x_j^{(s)}}{a_{11}}
\end{equation*}
Para calcular $x_2^{(s+1)}$ usamos la soluci�n que acabamos de obtener  $x_1^{(s+1)}$ y todas las restantes soluciones de las iteraciones anteriores,
\begin{equation*}
x_2^{(s+1)}=\frac{b_2-a_{21}x_1^{s+1}-\sum_{j=3}^na_{2j}x_j^{(s)}}{a_{22}}
\end{equation*}

Para calcular $x_3^{(s+1)}$ usamos las dos soluciones que acabamos de obtener  $x_1^{(s+1)}$, $x_2^{(s+1)}$ y todas las restantes soluciones de las iteraciones anteriores,

\begin{equation*}
x_3^{(s+1)}=\frac{b_3-a_{31}x_1^{s+1}-a_{32}x_2^{s+1}-\sum_{j=4}^na_{3j}x_j^{(s)}}{a_{33}}
\end{equation*}

Y para una componente $i$ cualquiera de la soluci�n, obtenemos la expresi�n expresi�n general del m�todo de Gauss-Seidel,
  
\begin{equation*}
x_i^{(s+1)}=\frac{b_i-\sum_{j< i}a_{ij}x_j^{(s+1)}-\sum_{j> i}a_{ij}x_j^{(s)}}{a_{ii}}
\end{equation*}

El siguiente c�digo de Matlab implementa una iteraci�n del m�todo de Gauss-Seidel y puede considerarse como la funci�n $M(\cdot)$, incluida en el diagrama de flujo de la figura \ref{fig:itera}, para dicho m�todo.

\begin{lstlisting}
    xs=xs1;
    % volvemos a inicializar el vector de soluciones al valor de los
    % terminos independientes
    xs1=b; 
    for i=1:n % bucle para recorrer todas las ecuaciones
        for j=1:i-1 % restamos la contribucion de todas las incognitas
            xs1(i)=xs1(i)-A(i,j)*xs1(j);       % por encima de x(i)
        end
        for j=i+1:n % restamos la contribuci�n de todas las incognitas
            % por debajo de x(i)
            xs1(i)=xs1(i)-A(i,j)*xs(j);
        end
        % dividimos por el elemento de la diagonal,
        xs1(i)=xs1(i)/A(i,i);
    end
     % calculamos la diferencia entre las dos soluciones,
        tolf=norm(xs1-xs);
        % incrementamos el contador de iteraciones
        it=it+1;
\end{lstlisting}

Es interesante destacar, que el �nico cambio en todo el c�digo respecto al m�todo de Jacobi es la sustituci�n de la variable \texttt{xs(j)} por la variable \texttt{xs1(j)} al final del primer bucle for anidado.

\paragraph{Forma matricial del m�todo de Gauss-Seidel.} \index{M�todo de Gauss-Seidel! Forma matricial} De modo an�logo a c�mo hicimos para el m�todo de Jacobi, es posible obtener una soluci�n matricial para el m�todo de Gauss-Seidel. 

Supongamos que realizamos la misma descomposici�n en sumandos de la matriz de coeficientes que empleamos para el m�todo de Jacobi,

\begin{equation*}
A\cdot x =b \rightarrow (D+L+U)\cdot x=b
\end{equation*}  
En este caso, en c�lculo de cada componente de la soluci�n en una iteraci�n intervienen las componentes de la iteraci�n anterior que est�n por debajo de la que queremos calcular, por tanto solo deberemos pasar a la derecha de la igualdad, la matriz $U$, Que contiene los coeficientes que multiplican a dichas componentes,

\begin{equation*}
(D+L+U)\cdot x=b \rightarrow (D+L)\cdot x= b-U\cdot x
\end{equation*}

Sustituyendo $x$ a cada lado de la igualdad por $x^{(s+1)}$ y $x^{(s)}$ y multiplicando por la inversa de $D+U$ a ambos lados, obtenemos la expresi�n en forma matricial del c�lculo de una iteraci�n por el m�todo de Gauss-Seidel,

\begin{equation*}
x^{(s+1)}= (D+L)^{-1}\cdot b-(D+L)^{-1}\cdot U\cdot x^{(s)}
\end{equation*}

Igual que hemos hecho en con el m�todo de Jacobi, podemos identificar las partes fijas que no cambian al iterar: $f=(D+L)^{-1}\cdot b$ y la matriz del m�todo, que en este caso es, $H=-(D+L)^{-1}\cdot U$.

El siguiente fragmento de c�digo muestra la construcci�n de las matrices necesarias para implementar en Matlab el m�todo de Gauss-Seidel,
\begin{lstlisting}
...
...
% obtenemos el tama�o del sistema,

n=size(A,1);
% creamos un vector de soluciones inicial,
xs=zeros(n,1);
% calculamos las matrices necesarias

D=diag(diag(A));
U=A-tril(A);
L=A-triu(A);

f=(D+L)^(-1)*b;
H=-(D+L)^(-1)*U

% calculamos la primera iteraci�n
xs1=f+H*xs;
% calculamos la diferencia entre las dos soluciones,
tolf=norm(xs1-xs);
% ponemos a 1 el contador de iteraciones.
it=1;

% A partir de aqu� vendr�a el c�digo necesario para calcular las 
% Sucesivas iteraciones hasta que la soluci�n converja.
\end{lstlisting}

En general, el m�todo de Gauss-Seidel es capaz de obtener soluciones, para un sistema dado y un cierto valor de tolerancia, empleando menos iteraciones que el m�todo de Jacobi. Por ejemplo, si aplicamos ambos m�todos a la resoluci�n del sistema,

\begin{equation*}
\begin{pmatrix}
4& 2& -1\\
3& -5& 1\\
1& -1& 6
\end{pmatrix}\cdot \begin{pmatrix}
x_1\\
x_2\\
x_3
\end{pmatrix}=\begin{pmatrix}
5\\
-4\\
17
\end{pmatrix}
\end{equation*}

Empleando en ambos casos una tolerancia $tol=0.0000	1$, el m�todo de Jacobi necesita 23 iteraciones para lograr una soluci�n, mientras que el de Gauss-Seidel solo necesita 13. La figura \ref{fig:gjcom}, muestra la evoluci�n de la tolerancia $tol=\lVert x^{(s+1)}-x^{(s)} \rVert$, en funci�n del n�mero de iteraci�n. En ambos casos el valor inicial de la soluci�n se tom� como el vector $(0,0,0)^T$.

\begin{figure}[h]
\centering
\includegraphics[width=16cm]{gjcom.eps}
\caption{evoluci�n de la tolerancia (m�dulo de la diferencia entre dos soluciones sucesivas) para un mismo sistema resuelto mediante el m�todo de Gauss-Seidel y el m�todo de Jacobi}
\label{fig:gjcom}
\end{figure}

\subsection{Amortiguamiento.} \index{M�todos amortiguados} El amortiguamiento consiste en modificar un m�todo iterativo, de modo que en cada iteraci�n, se da como soluci�n la media ponderada de los resultados de dos iteraciones sucesivas,
\begin{equation*}
x^{(s+1)}=\omega \cdot x^{*}+(1-\omega) \cdot x^{(s)}
\end{equation*}

Donde $x^{(*)}$ representa el valor que se obtendr�a aplicando una iteraci�n del m�todo a $x^{(s)}$, es decir ser�a el valor de $x^{(s+1)}$ si no se aplica amortiguamiento.

El par�metro $\omega$ recibe el nombre de factor de relajamiento. Si $0<\omega<1$ se trata de un m�todo de subrelajaci�n. Su uso permite resolver sistemas que no convergen si se usa el mismo m�todo sin relajaci�n. Si $w>1$ el m�todo se llama de sobrerelajaci�n, permite acelerar la convergencia respecto al mismo m�todo sin relajaci�n. Por �ltimo, si hacemos $\omega=1$, recuperamos el m�todo original sin relajaci�n.

\paragraph{El m�todo de Jacobi amortiguado.} \index{M�todo de Jacobi amortiguado} Se obtiene aplicando el m�todo de relajaci�n que acabamos de describir, al m�todo de Jacobi. La expresi�n general de una iteraci�n del m�todo de Jacobi amortiguando ser�a,
\begin{equation*}
x_i^{(s+1)}=\omega\cdot \overbrace{\frac{b_i-\sum_{j\neq i}a_{ij}x_j^{(s)}}{a_{ii}}}^{x^{(*)}}+(1- \omega)\cdot x_i^{s}
\end{equation*}

Para implementarlo en Matlab, bastar�a a�adir al c�digo del m�todo de Jacobi una l�nea incluyendo el promedio entre las dos soluciones sucesivas,

\begin{lstlisting}
while norm(xs1-xs)>tol
 	xs=xs1;
    % volvemos a inicializar el vector de soluciones al valor de los
    % terminos independientes
    xs1=b; 
    for i=1:n % bucle para recorrer todas las ecuaciones
        for j=1:i-1 % restamos la contribucion de todas las incognitas
            xs1(i)=xs1(i)-A(i,j)*xs(j);       % por encima de x(i)
        end
        for j=i+1:n % restamos la contribuci�n de todas las incognitas
            % por debajo de x(i)
            xs1(i)=xs1(i)-A(i,j)*xs(j);
        end
        % dividimos por el elemento de la diagonal,
        xs1(i)=xs1(i)/A(i,i);
    end
    % promediamos la soluci�n obtenida con la anterior (amortiguamiento)
    xs1=w*xs1+(1-w)*xs
end
\end{lstlisting}

En forma matricial, la expresi�n general del m�todo de Jacobi amortiguado ser�a,

\begin{equation*}
x^{(s+1)}=\omega\cdot\overbrace{\left(D^{-1}\cdot b- D^{-1}\cdot\left(L+U\right)\cdot x^{(s)}\right)}^{x^{(*)}}+(1-w)x^{(s)}
\end{equation*}
Si reorganizamos esta expresi�n,

\begin{equation*}
x^{(s+1)}=\omega\cdot D^{-1}\cdot b+ \left[(1-w)\cdot I - w\cdot  D^{-1}\cdot  \left(L+U \right)\right]\cdot x^{(s)}
\end{equation*}

Podemos identificar f�cilmente el termino fijo, $f=\omega\cdot D^{-1}\cdot b$ y la matriz del m�todo $H=\left((1-w)\cdot I - w\cdot  D^{-1}\cdot  \left(L+U \right)\right)$.

Para implementar el c�digo del m�todo de Jacobi amortiguado en Matlab, debemos calcular la matriz identidad del tama�o de sistema y modificar las expresiones de $f$ y $H$,

\begin{lstlisting}
...
...
% obtenemos el tama�o del sistema,

n=size(A,1);
% creamos un vector de soluciones inicial,
xs=zeros(n,1);
% creamos las matrices del m�todo
D=diag(diag(A));
U=A-tril(A);
L=A-triu(A);
I=eye(n);
% Alternativamente para jacobi podemos crear una solo matriz equivalenta a
% L+U, LpU=A-D
f=w*D^-1*b;
H=-w*D^-1*(L+U)+(1-w)*I;

% calculamos la primera iteraci�n,
xs1=f+H*xs;
% calculamos la diferencia entre las dos soluciones,
tolf=norm(xs1-xs);
% ponemos a 1 el contador de iteraciones.
it=1;

% a partir de aqu� vendr�a el c�digo necesario para calcular las
% sucesivas iteraciones hasta que la soluci�n converja
...
...
\end{lstlisting}

\paragraph{El m�todo SOR.} \index{M�todo SOR}El m�todo SOR --\emph{Succesive OverRelaxation}-- se obtiene aplicando amortiguamiento al m�todo de Gauss-Seidel. Aplicando el mismo razonamiento que el caso de Jacobi amortiguado, la expresi�n general para una iteraci�n del m�todo SOR es,

\begin{equation*}
x_i^{(s+1)}=\omega\cdot \overbrace{\frac{b_i-\sum_{j< i}a_{ij}x_j^{(s+1)}-\sum_{j> i}a_{ij}x_j^{(s)}}{a_{ii}}}^{x^{(*)}}+(1-\omega)\cdot x_i^{(s)}
\end{equation*}

Al igual que en el caso de Jacobi amortiguado, para implementar en Matlab el m�todo SOR es suficiente a�adir una l�nea que calcule el promedio de dos soluciones sucesivas,

\begin{lstlisting}
    xs=xs1;
    % volvemos a inicializar el vector de soluciones al valor de los
    % terminos independientes
    xs1=b; 
    for i=1:n % bucle para recorrer todas las ecuaciones
        for j=1:i-1 % restamos la contribucion de todas las incognitas
            xs1(i)=xs1(i)-A(i,j)*xs1(j);       % por encima de x(i)
        end
        for j=i+1:n % restamos la contribuci�n de todas las incognitas
            % por debajo de x(i)
            xs1(i)=xs1(i)-A(i,j)*xs(j);
        end
        % dividimos por el elemento de la diagonal,
        xs1(i)=xs1(i)/A(i,i);
    end
     % amortiguamos la soluci�n
      xs1=w*xs1+(1-w)*xs;
     % calculamos la diferencia entre las dos soluciones,
        tolf=norm(xs1-xs);
        % incrementamos el contador de iteraciones
        it=it+1;
\end{lstlisting}

En forma matricial la expresi�n para una iteraci�n del m�todo SOR ser�a,
\begin{equation*}
x^{(s+1)}= \omega\cdot\overbrace{\left((D+L)^{-1}\cdot b-(D+L)^{-1}\cdot U\cdot x^{(s)}\right)}^{x^{(*)}}+(1-\omega)\cdot x^{(s)}
\end{equation*}

Y tras reordenar,

\begin{equation*}
x^{(s+1)}= \omega\cdot\left((D+L)^{-1}\cdot b\right)+\left[(1-\omega)\cdot I-\omega\cdot(D+L)^{-1}\cdot U\right]\cdot x^{(s)}
\end{equation*}

De nuevo, podemos identificar, el t�rmino fijo, $f=\omega\cdot\left((D+L)^{-1}\cdot b\right)$ y la matriz del m�todo, $H=\left((1-\omega)\cdot I-\omega\cdot(D+L)^{-1}\cdot U\right)$.

El siguiente fragmento de c�digo muestra la obtenci�n de $f$ y $H$ para el m�todo SOR,

\begin{lstlisting}
% obtenemos el tama�o del sistema,

n=size(A,1);
% creamos un vector de soluciones inicial,
xs=zeros(n,1);
% calculamos las matrices necesarias

D=diag(diag(A));
U=A-tril(A);
L=A-triu(A);
I=eye(n);


f=w*(D+L)^(-1)*b;
H=(1-w)*I-w*(D+L)^(-1)*U;


% calculamos la primera iteraci�n
xs1=f+H*xs;

% calculamos la diferencia entre las dos soluciones,
tolf=norm(x-x0);
% ponemos a 1 el contador de iteraciones.
it=1;
\end{lstlisting}

\subsection{An�lisis de convergencia}\index{An�lisis de convergencia}
En la introducci�n a los m�todos iterativos dejamos abierta la cuesti�n de su convergencia. Vamos a analizar en m�s detalle en qu� condiciones podemos asegurar que un m�todo iterativo, aplicado a un sistema de ecuaciones lineales concreto, converge. 

En primer lugar, tenemos que definir qu� entendemos por convergencia. Cuando un m�todo iterativo converge, lo hace en forma asint�tica. Es decir, har�a falta un n�mero infinito de iteraciones para alcanzar la soluci�n exacta.

\begin{equation*}
x^{(0)}\rightarrow x^{(1)} \rightarrow x^{(2)}\rightarrow \cdots \rightarrow x^{(\infty)}=x
\end{equation*}

L�gicamente, es inviable realizar un n�mero infinito de iteraciones. Por esta raz�n, las soluciones de los m�todos iterativos son siempre aproximadas; realizamos un n�mero finito de iteraciones hasta cumplir una determinada condici�n de convergencia. Como no conocemos la soluci�n exacta, imponemos dicha condici�n entre dos iteraciones sucesivas,
\begin{equation*}
 \lVert x^{(s+1)}-x^{(s)}\rVert \leq C \Rightarrow  \lVert x^{(s+1)}-x\rVert = \lvert e^{(s+1)} \rvert
\end{equation*}
Donde $e^{(s+1)}$ representar�a el error \emph{real} de convergencia cometido en la iteraci�n $s+1$.

Tomando como punto de partida la expresi�n general del c�lculo de una iteraci�n en forma matricial,
\begin{equation*}
x^{(s+1)}=f+H\cdot x^{(s)}
\end{equation*}

Podemos expresar el error de convergencia como,
\begin{equation*}
e^{(s+1)}=x^{(s+1)}-x=f+H\cdot x^{(s)}-x
\end{equation*}

Pero la soluci�n exacta, si pudiera alcanzarse, cumplir�a,

\begin{equation*}
x=f+H\cdot x
\end{equation*}

Y si sustituimos en la expresi�n del error de convergencia,

\begin{equation*}
e^{(s+1)}=f+H\cdot x^{(s)}-f-H\cdot x
\end{equation*}

Llegamos finalmente a la siguiente expresi�n, que relaciona los errores de convergencia de dos iteraciones sucesivas,

\begin{equation*}
e^{(s+1)}=H\cdot e^{(s)}
\end{equation*}

Para que el error disminuya de iteraci�n en iteraci�n y el m�todo converja, es necesario que la matriz del m�todo $H$ tenga norma-2  menor que la unidad.

Supongamos que un sistema de dimension $n$ su matriz del m�todo $H$, tiene un conjunto de $n$ autovectores linealmente independientes, $w_1, \ w_2, \cdots, \ w_n$, cada uno asociado a su correspondiente autovalor, $\lambda_1, \ \lambda_2, \ \cdots,\ \lambda_n$. El error de convergencia, es tambi�n un vector de dimensi�n $n$, por tanto podemos expresarlo como una combinaci�n lineal de los $n$ autovectores linealmente independientes de la matriz $H$. Supongamos que lo hacemos para el error de convergencia $e^{(0)}$correspondiente al valor inicial de la soluci�n $x^{(0)}$,

\begin{equation*}
e^{(0)}=\alpha_1\cdot w_1+\alpha_2\cdot w_2+\cdots+\alpha_n\cdot w_n
\end{equation*}

Si empleamos la ecuaci�n deducida antes para la relaci�n del error entre dos iteraciones sucesivas y recordando que aplicar una matriz a un autovector, es equivalente a multiplicarlo por el autovalor correspondientes: $H\cdot w_i=\lambda_i\cdot w_i$, obtenemos para el error de convergencia en la iteraci�n $s$,

\begin{align*}
e^{(1)}&=H\cdot e^{(0)}=\alpha_1\cdot \lambda_1 \cdot  w_1+\alpha_2\cdot \lambda_2 \cdot  w_2+\cdots+\alpha_n\cdot \lambda_n \cdot w_n\\
e^{(2)}&=H\cdot e^{(1)}=H^2\cdot e^{(0)}=\alpha_1\cdot \lambda_1^2 \cdot  w_1+\alpha_2\cdot \lambda_2^2 \cdot  w_2+\cdots+\alpha_n\cdot \lambda_n^2 \cdot w_n\\
\vdots \ \ \ & \\
e^{(s)}&=H\cdot e^{(s-1)}=H^s\cdot e^{(0)}=\alpha_1\cdot \lambda_1^s \cdot  w_1+\alpha_2\cdot \lambda_2^s \cdot  w_2+\cdots+\alpha_n\cdot \lambda_n^s \cdot w_n
\end{align*}

Para que el error tienda a cero, $e^{(s)}\rightarrow 0$ al aumentar $s$, para cualquier combinaci�n inicial de valores $\alpha_i$, esto es para cualquier aproximaci�n inicial $x^{(0)}$, es necesario que todos los autovalores de la matriz del m�todo cumplan,

\begin{equation*}
\vert \lambda_i \vert < 1
\end{equation*}

Por tanto, el sistema converge si el radio espectral de la matriz del m�todo es menor que la unidad. \footnote{El radio espectral de una matriz es el mayor de sus autovalores en valor absoluto. Ver cap�tulo \ref{resp}.}

\begin{equation*}
\rho(H)<1 \Rightarrow \lim_{s\rightarrow \infty}e^{(s)}=0
\end{equation*}

\paragraph{Velocidad de convergencia.} \index{Velocidad de convergencia}Para un n�mero de iteraciones suficientemente grande, el radio espectral de la matriz del m�todo nos da la velocidad de convergencia. Esto es debido a que el resto de los t�rminos del error, asociados a otros autovalores m�s peque�os tienden a cero m�s  deprisa. Por tanto podemos hacer la siguiente aproximaci�n, donde estamos suponiendo que el autovalor $\lambda_n$ es el radio espectral,

\begin{equation*}
e^{(s)}\approx c_n \rho(H)^s w_n= c_n \lambda_n^s w_n
\end{equation*}

Podemos ahora calcular cuantas iteraciones nos costar� reducir un error inicial por debajo de un determinado valor. Esto depender� de la soluci�n inicial y de radio espectral de la matriz del m�todo,
\begin{equation*}
e^{(s)}\approx  \rho(H)^s e^{(0)} \Rightarrow \frac{e^{(s)}}{e^{(0)}} \approx \rho(H)^s
\end{equation*}

as� por ejemplo si queremos reducir el error inicial en $m$ d�gitos,
\begin{equation*}
\frac{e^{(s)}}{e^{(0)}} \approx \rho(h)^s \leq 10^{-m} \Rightarrow s \geq \frac{-m}{\log_{10}\left(\rho(H)\right)}
\end{equation*}

La matriz del m�todo juega por tanto un papel fundamental tanto en la convergencia del m�todo, como en la velocidad (n�mero de iteraciones) con la que el m�todo converge. Los m�todos amortiguados, permiten modificar la matriz de convergencia, gracias al factor de amortiguamiento $\omega$, haciendo que sistemas para los que no es posible encontrar una soluci�n mediante un m�todo iterativo converjan.

Como ejemplo, el sistema,

\begin{equation*}
\begin{pmatrix}
1& 2& -1\\
2& -5& 1\\
1& -1& 3
\end{pmatrix}\cdot \begin{pmatrix}
x_1\\
x_2\\
x_3
\end{pmatrix}=\begin{pmatrix}
2\\
-5\\
8
\end{pmatrix}
\end{equation*}

No converge si tratamos de resolverlo por el m�todo de Jacobi. Sin embargo si es posible obtener su soluci�n empleando el m�todo de Jacobi Amortiguado. La figura \ref{fig:cjjam}  muestra la evoluci�n de la tolerancia para dicho sistema empleando ambos m�todos.

\begin{figure}[h]
\centering
\includegraphics[width=14cm]{comjjam.eps}
\caption{Evoluci�n de la tolerancia para un mismo sistema empleando el metodo de Jacobi (diverge) y el de Jacobi amortiguado (converge).}
\label{fig:cjjam}


\end{figure}

Si calculamos el radio espectral de la matriz del m�todo, para el m�todo de Jacobi tendr�amos,
\begin{verbatim}
>> A=[1 2 -1  ; 2 -5 1;1 -1 3]

A =

     1     2    -1
     2    -5     1
     1    -1     3

>> H=diag(diag(A))^-1*(A-diag(diag(A)))

H =

         0    2.0000   -1.0000
   -0.4000         0   -0.2000
    0.3333   -0.3333         0

>> l=eig(H)
l =

   0.1187 + 1.0531i
   0.1187 - 1.0531i
  -0.2374          

>> radio_espectral=max(abs(l))
radio_espectral =

    1.0597
\end{verbatim}

El radio espectral es mayor que la unidad y el m�todo no converge.

Si repetimos el c�lculo para el m�todo de Jacobi amortiguado, con $\omega=0.5$
\begin{verbatim}
>> H=(1-0.5)*eye(3)-0.5*diag(diag(A))^-1*(A-diag(diag(A)))

H =

    0.5000   -1.0000    0.5000
    0.2000    0.5000    0.1000
   -0.1667    0.1667    0.5000

>> l=eig(H)

l =

   0.4406 + 0.5265i
   0.4406 - 0.5265i
   0.6187          

>> radio_espectral=max(abs(l))

radio_espectral =

    0.6866
\end{verbatim}

El radio espectral es ahora menor que la unidad y el m�todo converge.

Por �ltimo indicar que cualquiera de los m�todos iterativos descrito converge para un sistema que cumpla que su matriz de coeficientes es estrictamente diagonal dominante.
\section{Ejercicios}
\begin{enumerate}
\item Crea una funci�n en matlab que calcule la soluci�n de un sistema de ecuaciones triangular inferior empleando el m�todo de sustituciones progresivas. La funci�n deber� tomar como valores de entrada una matriz triangular inferior de dimensi�n $(n\times n)$ arbitraria y un vector columna $(n\times 1)$ de t�rminos independientes. Deber� devolver como variable de salida un vector columna $(n\times 1)$ con las soluciones del sistema.

\item \label{eje62} Crea una funci�n en matlab que calcule la soluci�n de un sistema de ecuaciones triangular superior empleando el m�todo de sustituciones regresivas. La funci�n deber� tomar como valores de entrada una matriz triangular inferior de dimensi�n $(n\times n)$ arbitraria y un vector columna $(n\times 1)$ de t�rminos independientes. Deber� devolver un vector columna $(n\times 1)$ con las soluciones del sistema.

\item Crea una funci�n que dadas una matriz cuadrada  $ A,\ (n\times n)$ y un  vector columna $b,\ (n\times 1)$, construya la matriz ampliada $Ab$, Aplique el m�todo de eliminaci�n gausiana a la matriz ampliada y llame a la funci�n creada en el ejercicio \ref{eje62} Para resolver el sistema $Ax=b$. La funci�n deber� devolver como variables de salida, La matriz resultante de la aplicar la eliminaci�n gaussiana a la ampliada y un vector columna con las soluciones del sistema resuelto. 

\item Crea una funci�n que dadas una matriz cuadrada  $ A,\ (n\times n)$ y un  vector columna $b,\ (n\times 1)$, construya la matriz ampliada $Ab$,aAplique el m�todo de eliminaci�n Gauss-Jordan a la matriz ampliada y devuelva como variable de salida la matriz en forma escalonada reducida por filas, de modo que la �ltima fila sea la soluci�n del sistema $Ax=b$  

\item Construye un programa que resuelva un sistema de ecuaciones de dimensi�n arbitraria, empleando el m�todo de Jacobi simple (no en forma matricial). El programa deber� admitir como variables de entrada, una matriz de coeficientes $ A,\ (n\times n)$, un vector de t�rminos independientes $b,\ (n\times 1)$, una soluci�n inicial $x_0 ,\ (n\times 1)$, un valor para la tolerancia m�xima entre dos iteraciones sucesivas y un n�mero m�ximo de iteraciones permitido. El programa deber� devolver un vector columna con las soluciones del sistema, el n�mero de iteraciones empleado y el error relativo entre las dos �ltimas iteraciones realizadas.

\item Repite el ejercicio anterior empleando ahora el m�todo de Jacobi matricial. A�ade el c�digo necesario para que calcule en primer lugar el radio espectral de la matriz del m�todo y caso de no cumplirse  la condici�n de convergencia, el programa interrumpa su ejecuci�n y devuelva un mensaje de error indicando el valor del radio espectral.

\item Construye un programa que resuelva un sistema de ecuaciones de dimensi�n arbitraria, empleando el m�todo de Gauss-Seidel simple (no en forma matricial). El programa deber� admitir como variables de entrada, una matriz de coeficientes $ A,\ (n\times n)$, un vector de t�rminos independientes $b,\ (n\times 1)$, una soluci�n inicial $x_0 ,\ (n\times 1)$, un valor para la tolerancia m�xima entre dos iteraciones sucesivas y un n�mero m�ximo de iteraciones permitido. El programa deber� devolver un vector columna con las soluciones del sistema, el n�mero de iteraciones empleado y el error relativo entre las dos �ltimas iteraciones realizadas.

\item Repite el ejercicio anterior empleando ahora el m�todo de Gauss-Seidel matricial. A�ade el c�digo necesario para que calcule en primer lugar el radio espectral de la matriz del m�todo y caso de no cumplirse  la condici�n de convergencia, el programa interrumpa su ejecuci�n y devuelva un mensaje de error indicando el valor del radio espectral.

\item Construye un programa que resuelva un sistema de ecuaciones de dimensi�n arbitraria, empleando el m�todo de Jacobi amortiguado simple (no en forma matricial). El programa deber� admitir como variables de entrada, una matriz de coeficientes $ A,\ (n\times n)$, un vector de t�rminos independientes $b,\ (n\times 1)$, una soluci�n inicial $x_0 ,\ (n\times 1)$, un valor para el par�metro de amortiguamiento $\omega$, un valor para la tolerancia m�xima entre dos iteraciones sucesivas y un n�mero m�ximo de iteraciones permitido. El programa deber� devolver un vector columna con las soluciones del sistema, el n�mero de iteraciones empleado y el error relativo entre las dos �ltimas iteraciones realizadas.

\item Repite el ejercicio anterior empleando ahora el m�todo de Jacobi amortiguado matricial. A�ade el c�digo necesario para que calcule en primer lugar el radio espectral de la matriz del m�todo y caso de no cumplirse  la condici�n de convergencia, el programa interrumpa su ejecuci�n y devuelva un mensaje de error indicando el valor del radio espectral.

\item Construye un programa que resuelva un sistema de ecuaciones de dimensi�n arbitraria, empleando el m�todo SOR  simple (no en forma matricial). El programa deber� admitir como variables de entrada, una matriz de coeficientes $ A,\ (n\times n)$, un vector de t�rminos independientes $b,\ (n\times 1)$, una soluci�n inicial $x_0 ,\ (n\times 1)$, un valor para el par�metro de amortiguamiento $\omega$, un valor para la tolerancia m�xima entre dos iteraciones sucesivas y un n�mero m�ximo de iteraciones permitido. El programa deber� devolver un vector columna con las soluciones del sistema, el n�mero de iteraciones empleado y el error relativo entre las dos �ltimas iteraciones realizadas.

\item Repite el ejercicio anterior empleando ahora el m�todo SOR matricial. A�ade el c�digo necesario para que calcule en primer lugar el radio espectral de la matriz del m�todo y caso de no cumplirse  la condici�n de convergencia, el programa interrumpa su ejecuci�n y devuelva un mensaje de error indicando el valor del radio espectral.


\item Resuelve el siguiente sistema de ecuaciones, empleando las factorizaci�n LU de matlab. Indica todos los pasos empleados hasta obtener las soluci�n del sistema. \textbf{(1.5 puntos)}
\begin{align*}
3x_1&+3x_2-2x_3 +\ x_4= 13\\
 &+2x_2-\ x_3  \hspace{30pt} =\ 3\\
-2x_1&+\ x_2 + 5x_3 -4x_4=\ 1\\
2x_1& \hspace{30pt} -\ x_3 +2x_4 =\ 5
\end{align*}
Repite el ejercicio empleando las factorizaciones QR y SVD.

\item Para calcular las intensidades de un circuito de corriente continua como el de la figura, es suficiente emplear las leyes de Kirchoff. La primera ley --\emph{ley de nodos}-- establece que la suma de corrientes que llegan a un nodo del circuito debe ser igual a cero.
La segunda --\emph{ley de mallas}-- establece que la suma de las ca�das de tensi�n en un malla cerrada tiene que ser cero. La ca�da de voltaje en una resistencia se calcula empleando la ley de Ohm: $V = i\cdot R$
Si aplicamos las leyes de Kirchoff al circuito de la figura, obtenemos  el siguiente sistema de ecuaciones,

\begin{minipage}{0.3\textwidth}
\begin{align*}
i_1 - i_2 - i_3 &= 0\\
i_3 - i_4 - i_5 &= 0\\
i_5  + i_2 -i_6&= 0\\
5i_3 + 70 i_4 &= 180\\
25i_2 -5i_3-10i_5 &= 0\\
10i_5+8i_6 -70i_4 &= 0  
\end{align*}
\end{minipage}
\begin{minipage}{0.7\textwidth}
\includegraphics[scale=1.1]{circuito.eps}
\end{minipage}
\ \\
Las tres primeras ecuaciones corresponden a aplicar la \emph{ley de nodos} a los nodos marcados con un punto negro en la figura. 

Las tres �ltimas, a aplicar la \emph{ley de mallas} a las tres mallas considerando positivo recorrerlas en el sentido de las agujas del reloj.

El sistema de ecuaciones obtenido puede expresarse en forma matricial como,
\begin{equation}\label{cucu}
\begin{pmatrix}
1& -1& -1& 0 & 0 &0\\
0& 0& 1& -1& -1& 0\\
0& 1& 0&  0&  1& -1\\ 
0& 0& 5& 70& 0& 0\\
0& 25& -5& 0& -10& 0\\
0& 0& 0& -70& 10& 8
\end{pmatrix}\begin{pmatrix}
i_1\\ i_2\\ i_3\\ i_4\\ i_5\\ i_6
\end{pmatrix}= \begin{pmatrix}
0\\ 0\\ 0\\ 180\\ 0\\ 0
\end{pmatrix} 
\end{equation}



\begin{enumerate}
\item Define en Matlab{$^\copyright$} el sistema de ecuaciones (\ref{cucu}). Obt�n los valores de las intensidades en todas las ramas del circuito empleando directamente un solo comando o funci�n de Matlab{$^\copyright$}. Emplea el comando \texttt{cond} y d� si en tu opini�n el sistema est� bien o mal condicionado. (1 punto)

\item Emplea la factorizaci�n QR para obtener de nuevo la soluci�n del sistema.  (1 punto)

\item Utilizando la versi�n permutada del sistema, obtenida mediante la factorizaci�n LU, es decir, usando $P\cdot A$ como matriz del sistema y $P\cdot b$, como t�rmino independiente, calcula el radio espectral correspondiente al m�todo de Jacobi e indica si tiene sentido o no emplear este m�todo para obtener las intensidades del circuito de la figura. (2 puntos)

\item Emplea el m�todo de Jacobi amortiguado, con un amortiguamiento de $\omega =0.1$ para calcular las intensidades del circuito de la figura, con una tolerancia de $10^{-5}$. Indica el n�mero de iteraciones empleado hasta alcanzar la soluci�n.  (\emph{Nota importante:} Usa la versi�n permutada del sistema y ten en cuenta que va a necesitar bastantes iteraciones --m�s de $3000$-- para converger). (2 puntos)

\item Emplea el m�todo SOR, con un amortiguamiento de $\omega =0.2$ para calcular las intensidades del circuito de la figura, con una tolerancia de $10^{-5}$. Indica el n�mero de iteraciones empleado hasta alcanzar la soluci�n (\emph{Nota importante:} Usa la versi�n permutada del sistema).
Discute, a la vista de los resultados, c�al m�todo funciona mejor. (2 puntos)

\item Escribe un c�digo que permita dibujar el radio espectral de un m�todo amortiguado en funci�n de la matriz del m�todo, para distintos valores del amortiguamiento. Emplea el programa para los casos de las dos preguntas anteriores. �Ha sido razonable la elecci�n de los valores de $\omega$ realizada en dichas preguntas? (2 puntos) 
\end{enumerate}
\end{enumerate}

\section{Test del curso 2020/21}
\noindent \textbf{Problema 1}. El m�todo iterativo de Richardson para la resoluci�n de sistemas de ecuaciones lineales emplea la siguiente f�rmula de recurrencia
\begin{equation}\label{eq:1}
x^{(s+1)} = x^{(s)} + \omega\left( b - Ax^{(s)}\right),
\end{equation}  
		donde $A \in \mathbb{R}^{n\times n}$ y $b  \in \mathbb{R}^{n}$ representan respectivamente la matriz de coeficientes y el vector columna de t�rminos independientes del sistema de ecuaciones $Ax=b$ de orden $n\in\mathbb{N}$, y $x^{(s)}\in\mathbb{R}^n$ es el vector soluci�n en la iteraci�n $s\in\mathbb{N}$. El par�metro $\omega \in \mathbb{R}$  juega el mismo papel que el factor de relajaci�n en los m�todos amortiguados y debe ajustarse para asegurar la convergencia del m�todo.

 \begin{enumerate}
\item {\bf 1 punto.} Reescribe la f�rmula de recurrencia del m�todo Richardson (ecuaci�n \ref{eq:1})  en la forma matricial est�ndar
\begin{equation}\label{eq:2}
x^{(s+1)} = f + Hx^{(s)}.
\end{equation}

\item {\bf 2 puntos.} Construye una funci�n em Matlab que implemente el m�todo de Richardson enpleando la forma matricial est�ndar, es decir, en cada iteraci�n aplicamos la ecuaci�n (\ref{eq:2}). La funci�n deber� tomar como variables de entrada: 
\begin{enumerate}
	\item La matriz de coeficientes $A$.
	\item El vector de t�rminos independientes $b$.
	\item El valor inicial $x^{(0)}$.
	\item N�mero m�ximo de iteraciones a realizar.
	\item Tolerancia m�nima para el error entre iteraciones sucesivas.
	\item Un valor para el par�metro $\omega$.
\end{enumerate}
As� mismo, la funci�n deber� devolver como variables de salida: 
\begin{enumerate}
\item La soluci�n del sistema.
\item La tolerancia alcanzada.
\item El n�mero de iteraciones empleado en obtener la soluci�n.
\end{enumerate}

\item {\bf 2 puntos.} Emplea los m�todos de Richardson, Jacobi y Gauss-Seidel para obtener la soluci�n del sistema de ecuaciones
\begin{equation*}
\begin{pmatrix}
5&2&3&-1\\
2&6&3&0\\
1&-4&4&-1\\
2&0&3&7
\end{pmatrix}\begin{pmatrix}
x_1\\ x_2\\ x_3\\ x_4
\end{pmatrix}= \begin{pmatrix}
14\\ 23\\1\\ 39
\end{pmatrix}.
\end{equation*}
		 Emplea un valor $\omega = 0.16$ para el m�todo de Richardson y una tolerancia de $10^{-5}$ para los tres m�todos. Escoge el valor $x^{(0)}=\left[0,0,0,0\right]^T$ para los tres m�todos.

Clasifica los tres m�todos de mejor a peor, tomando como criterio el n�mero de iteraciones empleado por cada uno de ellos para alcanzar la soluci�n.

% \item {\bf 2 puntos.} \textcolor{red}{Define el error de la soluci�n propuesta en la iteraci�n $s$ con respecto al valor real de la soluci�n $x$, esto es, $e^{(s)} = x^{(s)} - x$. Halla la matriz $W\in\mathbb{R}^{n\times n}$ que satisface:}
%	\begin{equation}
%		e^{(s+1)} = We^{(s)}.
%		\label{eq: error}
%	\end{equation}
%	 Metodolog�a/pista: Sustituye $x^{(s)} = e^{(s)} + x$ en (\ref{eq:1}), y opera hasta alcanzar (\ref{eq: error}).

\item {\bf 2 punto.} Haz una gr�fica del radio espectral de $H$ en funci�n del par�metro $\omega$ para el m�todo de Richardson y el sistema de ecuaciones del apartado anterior. Emplea para ello valores de $\omega$ comprendidos en el intervalo $[0.01\ 0.9]$, toma una separaci�n entre valores de $0.01$.  Representa cada valor como un punto independiente.

Determina, a la vista de la gr�fica, si ser�a posible encontrar un valor de $\omega$ para el que se alcance la soluci�n del sistema en menos iteraciones, manteniendo la misma tolerancia. Indica, tambi�n de acuerdo con el gr�fico, cu�l es mayor valor de $\omega$ admisible. Razona las respuestas.
\end{enumerate}

\noindent \textbf{Problema 2.} El m�todo de Gauss-Jordan permite resolver sistemas de ecuaciones lineales de la forma $Ax =b, A \in \mathbb{R}^{n\times n}$ y $x,b  \in \mathbb{R}^{n}$. Si en lugar de emplear un vector columna de t�rminos independientes, sustituimos $b$ por una matriz $B$ de dimensi�n $n\times m$, el m�todo nos permite obtener como resultado una matriz de soluciones $X$ tambi�n de dimension  $n\times m$: $AX=B$. Cada columna, $x_j, \, j\in\{1,\dots,m\}$, de la matriz $X$ representa la soluci�n de sistema $Ax_j=b_j$, donde $b_j$ es la columna correspondiente de la matriz $B$. Es decir: hemos resuelto $m$ sistemas de ecuaciones simult�neamente.

\begin{enumerate}
\item {\bf 2 puntos.} Emplea el m�todo de Gauss-Jordan para obtener la soluci�n de $AX=B$, donde $A$ es la matriz de coeficientes del Problema 1 y B es la matriz identidad de dimension $4\times4$.
\item {\bf 1 punto.} �Sabr�as decir que relaci�n hay entre las matrices X y A?
\end{enumerate}




\chapter{Interpolaci�n y ajuste de funciones}\label{interpolacion}
En este cap�tulo vamos a estudiar distintos m�todos de aproximaci�n polin�mica. En t�rminos generales el problema consiste en sustituir una funci�n $f(x)$ por un polinomio,
\begin{equation*}
f(x)\approx p(x)=a_0+a_1\cdot x+a_2\cdot x^2+a_3\cdot x^3+\cdots +a_n \cdot x^n
\end{equation*}

Para obtener la aproximaci�n podemos partir de la ecuaci�n que define $f(x)$,  por ejemplo la funci�n error,

\begin{equation*}
erf(x)=\frac{2}{\sqrt{\pi}}\int_0^x e^{-t^2}dt
\end{equation*}

O bien, puede suceder que solo conozcamos algunos valores de la funci�n, por ejemplo a trav�s de una tabla de datos,

\begin{table}[h]
\caption{$f(x)=erf(x)$}
\centering
\begin{tabular}{c|c}
$x$&$f(x)$\\ 
\hline
$0.0$& $0.0000$\\
$0.1$&$0.1125$\\
$0.2$&$0.2227$\\
$0.3$&$0.3286$\\
$0.4$&$0.4284$\\
$0.5$&$0.5205$\\
\end{tabular}
\label{tpuntos2}
\end{table}
 
La aproximaci�n de una funci�n por un polinomio, tiene ventajas e inconvenientes. 

Probablemente la principal ventaja, desde el punto de vista del c�mputo, es que un polinomio es f�cil de evaluar  mediante un ordenador ya que solo involucra operaciones aritm�ticas sencillas. Adem�s, los polinomios son f�ciles de derivar e integrar, dando lugar a otros polinomios.

En cuanto a los inconvenientes hay que citar el crecimiento hacia infinito o menos infinito de cualquier polinomio para valores de la variable independiente alejados del origen. Esto puede dar lugar en algunos casos a errores de redondeo dif�ciles de manejar, haciendo muy dif�cil la aproximaci�n para funciones no crecientes.

Vamos a estudiar tres m�todos distintos; en primer lugar veremos la aproximaci�n mediante el polinomio de Taylor, �til para aproximar una funci�n en las inmediaciones de un punto. A continuaci�n,  veremos la interpolaci�n polin�mica y, por �ltimo, estudiaremos el ajuste polin�mico por m�nimos cuadrados.

El uso de uno u otro de estos m�todos esta asociado a la informaci�n disponible sobre la funci�n que se desea aproximar y al uso que se pretenda hacer de la aproximaci�n realizada.


\section{El polinomio de Taylor.}\index{Polinomio de Taylor}

Supongamos una funci�n infinitamente derivable en un entorno de un punto $x_0$. Su expansi�n en serie de Taylor se define como,

\begin{equation*}
f(x)=f(x_0)+f'(x_0)\cdot (x-x_0)+\frac{1}{2} f''(x_0)\cdot (x-x_0)^2+\cdots + \frac{1}{n!}f^{(n)}(x_0)\cdot (x-x_0)^n+ \frac{1}{(n+1)!}f^{(n+1)}(z)\cdot (x-x_0)^{n+1}
\end{equation*}

Donde $z$ es un punto sin determinar situado entre $x$  y $x_0$.   Si eliminamos el �ltimo t�rmino, la funci�n puede aproximarse por un polinomio de grado $n$																						\begin{equation*}
f(x)\approx f(x_0)+f'(x_0)\cdot (x-x_0)+\frac{1}{2}f''(x_0)\cdot (x-x_0)^2+\cdots + \frac{1}{n!}f^{(n)}(x_0)\cdot (x-x_0)^n
\end{equation*}

El error cometido al aproximar una funci�n por un polinomio de Taylor de grado $n$, viene dado por el t�rmino,
\index{Polinomio de Taylor! Error de la aproximaci�n}
\begin{equation*}
e(x)=\lvert f(x) -p(x)\rvert=\left\lvert\frac{1}{(n+1)!} f^{(n+1)}(z)\cdot (x-x_0)^{n+1}\right\rvert
\end{equation*}

Es f�cil deducir de la ecuaci�n que el error disminuye con el grado del polinomio empleado y aumenta con la distancia entre $x$ y $x_0$. Adem�s cuanto m�s suave es la funci�n (derivadas peque�as) mejor es la aproximaci�n.

Por ejemplo para la funci�n exponencial, el polinomio de Taylor de orden $n$ desarrollado en torno al punto $x_0=0$ es, \index{Polinomio de Taylor! Serie de la funci�n exponencial}

\begin{equation*}
e^x\approx 1+x+\frac{1}{2}x^2+\cdots+\frac{1}{n!}x^n=\sum_{i=0}^n\frac{1}{i!}x^i
\end{equation*}

y el del logaritmo natural, desarrollado en torno al punto $x_0=1$, \index{Polinomio de Taylor! Serie del logaritmo natural}

\begin{equation*}
\log(x)\approx (x-1)-\frac{1}{2}(x-1)^2+\cdots+\frac{(-1)^{n+1}}{n}(x-1)^n=\sum_{i=1}^n\frac{(-1)^{i+1}}{i}(x-1)^i
\end{equation*}

La existencia de un t�rmino general para los desarrollos de Taylor de muchas funciones elementales lo hace particularmente atractivo para aproximar funciones mediante un ordenador. As� por ejemplo, la siguiente funci�n de Matlab, aproxima el valor del logaritmo natural en un punto, empleando un polinomio de Taylor del grado que se desee,

\begin{lstlisting}
function y=taylorln(x,n)
% Esta funci�n aproxima el valor del logaritmo natural de un numero
% empleando para ello un polinomio de Taylor de grado n desarrollado en 
% torno a x=1. Las variables de entrada son: x, valor para el que se desea 
% calcular el logaritmo. n Grado del polinomio que se emplear� en el 
% c�lculo. La variable de salida y es el logaritmo de x. (nota si x
% es un vector, calcular� el logaritmo para todos los puntos del vector)

% iniciliazamos la variable de salida y
y=0;

% construimos un b�cle para ir a�adiendo t�rminos al desarrollo
for i=1:n
    y=y+(-1)^(i+1)*(x-1).^i/i;
end
\end{lstlisting}

La aproximaci�n funciona razonablemente bien para puntos comprendidos en el intervalo $0<x<2$. La figura \ref{fig:ln} muestra los resultados obtenidos en dicho intervalo para polinomios de Taylor del logaritmo natural de grados 2, 5, 10 20. La linea continua azul representa el valor del logaritmo obtenido con la funci�n de Matlab \texttt{log}.

\begin{figure}[h]
\centering
\includegraphics[width=12cm]{ln.eps}
\caption{Comparaci�n entre resultados obtenidos para polinomios de Taylor del logaritmo natural. (grados 2, 3, 5, 10, 20)}
\label{fig:ln}
\end{figure}

\index{Polinomio de Taylor! Series de las funciones seno y coseno}
Las funciones $\sin(x)$ y $\cos(x)$, son tambi�n simples de aproximar mediante polinomios de Taylor. Si desarrollamos en torno a $x_0=0$, la serie del coseno solo tendr� potencias pares mientras que la del seno solo tendr� potencias impares,

\begin{align*}
\cos(x)&\approx \sum_{i=0}^n \frac{(-1)^i}{(2i)!}x^{2i}\\
\sin(x)&\approx \sum_{i=0}^n \frac{(-1)^i}{(2i+1)!}x^{2i+1}
\end{align*}

\begin{figure}[h]
\centering
\subfigure[$\cos(x)$, polinomios 2, 4, 6 y 8 grados  \label{fig:cos}]{\includegraphics[width=10.5cm]{cos.eps}} \qquad 
\subfigure[$\sin(x)$, polinomios  3, 5, 7 y 9  grados \label{fig:sin}]{\includegraphics[width=10.5cm]{sin.eps}}\\
\caption{Polynomios de Taylor para las funciones coseno y seno  }
\end{figure}

En las figuras \ref{fig:cos} y \ref{fig:sin} Se muestran las aproximaciones mediante polinomios de Taylor de las funciones coseno y seno. Para el coseno se han empleado polinomios hasta grado 8 y para el seno hasta grado 9. En ambos casos se dan los resultados correspondientes a un periodo $(-\pi, \pi)$. Si se comparan los resultados con las funciones \texttt{cos} y \texttt{sin}, suministradas por Matlab, puede observarse que la aproximaci�n es bastante buena para los polinomios de mayor grado empleados en cada caso.

\section{Interpolaci�n polin�mica.}

Se entiende por interpolaci�n el proceso por el cual, dado un conjunto de pares de puntos $(x_0,y_0),(x_1,y_1),\cdots (x_n,y_n)$ se obtiene una funci�n $f(x)$, tal que, $y_i=f(x_i)$, para cada par de puntos $(x_i,y_i)$ del conjunto. Si, en particular, la funci�n empleada es un polinomio $f(x)\equiv p(x)$, entonces se trata de interpolaci�n polin�mica. \index{Interpolaci�n! Polin�mica}

\paragraph{Teorema de unicidad.} Dado un conjunto \index{Interpolaci�n! Teorema de unicidad}  $(x_0,y_0),(x_1,y_1),\cdots (x_n,y_n)$ de $n+1$ pares de puntos, tales que todos los valores $x_i$ de dicho conjuntos son diferentes entre s�, solo existe un polinomio $p(x)$ de grado $n$, tal que $y_i=p(x_i)$ para todos los pares de puntos del conjunto.

Si tratamos de interpolar los puntos con un polinomio de grado menor que $n$, es posible que no encontremos ninguno que pase por todos los puntos. Si, por el contrario empleamos un polinomio de grado mayor que $n$, nos encontramos con que no es �nico. Por �ltimo si el polinomio empleado es de grado $n$, entonces ser� siempre el mismo con independencia del m�todo que empleemos para construirlo.

\subsection{La matriz de Vandermonde} \index{Matriz de Vandermonde}
Supongamos que tenemos un conjunto de pares de puntos $\mathcal{A}$, 
\begin{table}[h]
%\caption{$f(x)=erf(x)$}
\centering
\begin{tabular}{c|c}
$x$&$f(x)$\\ 
\hline
$x_0$& $y_0$\\
$x_1$&$y_1$\\
$x_2$&$y_2$\\
$\vdots$&$\vdots$\\
$x_n$&$y_n$
\end{tabular}
\label{tpuntos3}
\end{table}

Para que un polinomio de orden $n$,
\begin{equation*}
p(x)=a_0+a_1x+a_2x^2+\cdots+a_nx^n
\end{equation*}

pase por todos los pares de $\mathcal{A}$ debe cumplir,

\begin{equation*}
y_i=a_0+a_1x_i+a_2x_i^2+\cdots+a_nx_i^n, \ \forall (x_i,y_i) \in \mathcal{A}
\end{equation*}

Es decir, obtendr�amos un sistema de $n$ ecuaciones lineales, una para cada par de valores, en la que las inc�gnitas son precisamente los $n$ coeficientes $a_i$ del polinomio.

Por ejemplo para los puntos,

\begin{table}[h]
%\caption{$f(x)=erf(x)$}
\centering 
\begin{tabular}{c|c}
$x$&$f(x)$\\ 
\hline
$1$&$\ 2$\\
$2$&$ \ 1$\\
$3$&$-2$
\end{tabular}
\label{tpuntos4}
\end{table}

Obtendr�amos,

\begin{align*}
a_0+a_1\cdot 1+ a_2\cdot 1^2&=2\\
a_0+a_1\cdot 2+ a_2\cdot 2^2&=1\\
a_0+a_1\cdot 3+ a_2\cdot 3^2&=-2
\end{align*}

que podr�amos expresar en forma matricial como,



\begin{equation*}
\begin{pmatrix}
1&1&1^2\\
1&2&2^2\\
1&3&3^2
\end{pmatrix}\cdot \begin{pmatrix}
a_0\\
a_1\\
a_2
\end{pmatrix}=\begin{pmatrix}
2\\
1\\
-2
\end{pmatrix}
\end{equation*}

Y en general, para $n$ pares de datos,

\begin{equation*}
\begin{pmatrix}
1&x_0&x_0^2&\cdots &x_0^n\\
1&x_1&x_1^2&\cdots &x_1^n\\
\vdots&\vdots&\vdots&\ddots&\vdots\\
1&x_n&x_n^2&\cdots &x_n^n
\end{pmatrix}\cdot \begin{pmatrix}
a_0\\
a_1\\
\vdots\\
a_n

\end{pmatrix}=\begin{pmatrix}
y_0\\
y_1\\
\vdots\\
y_n
\end{pmatrix}
\end{equation*}

La matriz de coeficientes del sistema resultante recibe el nombre de matriz de Vandermonde. Est� formada por la $n$ primeras potencias de cada uno de los valores de la variable independiente, colocados por filas. Es evidente que cuanto mayor es el n�mero de datos, mayor tender� a ser la diferencia de tama�o entre los elementos de cada fila. Por ello, en la mayor�a de los casos, resulta ser una matriz mal condicionada para resolver el sistema num�ricamente. En la pr�ctica, para obtener el polinomio interpolador, se emplean otros m�todos alternativos,

\subsection{El polinomio interpolador de Lagrange.} \label{sec:lagranje}\index{Interpolaci�n! Polinomio de Lagrange} \index{Polinomio de Lagrange}

A partir de los valores $x_0, x_1,\cdots, x_n$, se construye el siguiente conjunto de $n+1$ polinomios de grado $n$
\begin{equation*}
l_j(x)=\prod_{\substack{k=0\\
k\neq j}}^n\frac{x-x_k}{x_j-x_k}=\frac{(x-x_0)(x-x_1)\cdots(x-x_{j-1})(x-x_{j+1})\cdots(x-x_n)}{(x_j-x_0)(x_j-x_1)\cdots(x_j-x_{j-1})(x_j-x_{j+1})\cdots(x_j-x_n)}
\end{equation*}

Los polinomios as� definidos cumplen una interesante propiedad en  relaci�n con los valores $x_0, x_1,\cdots, x_n$, empleados para construirlos,

\begin{equation*}
l_j(x_i)= \left\{ 
\begin{aligned}
1,\ i=j\\
0,\ i\neq j
\end{aligned}
\right.
\end{equation*}

A partir de estos polinomios podemos construir ahora el siguiente polinomio de interpolaci�n empleando las im�genes $y_0,y_1\cdots, y_n$ correspondientes a los valores $x_0, x_1,\cdots, x_n$,

\begin{equation*}
p(x)=\sum_{j=0}^n l_j(x)\cdot y_j
\end{equation*}

Efectivamente, es f�cil comprobar que, tal y como se ha construido, este polinomio pasa por los pares de puntos $x_i,y_i$, puesto que $p(x_i)=y_i$.

El siguiente c�digo de Matlab calcula el valor en un punto x cualquiera del polinomio de interpolaci�n de Lagrange construido a partir un conjunto de puntos $\mathcal{A}\equiv \{(x_i,y_i)\}$ . 

\begin{lstlisting}
function y1=Lagrange(x,y,x1)
% este programa obtiene el valor interpolado y1 correspondiente al valor x1
% empleando el polinomio interpolador de Lagrange de grado n, obtenido a 
% partir de los vectores x e y (de longitud n)

% obtenemos el tama�o del conjunto de datos,

n=length(x);
% inicializamos la variable de salida
y1=0;
% construimos el valor a partir de los polinomios de Lagrange,
for j=1:n
    % inicializamos el polinomio de Lagrange correspondiente al dato i
    lj=1;
    % y lo calculamos...
    for i=1:j-1
        lj=lj*(x1-x(i))/(x(j)-x(i));
    end
    for i=j+1:n
       lj=lj*(x1-x(i))/(x(j)-x(i));
    end 
    
    % sumamos la contribuci�n del polinomio de Lagrange lj
    y1=y1+lj*y(j);
end
\end{lstlisting}

\section{Diferencias divididas.}\label{sec:difdiv} \index{Interpolaci�n! Diferencias Divididas} \index{Diferencias Divididas}
Tanto el m�todo de la matriz de Vandermonde como el de los polinomios de Lagrange, presentan el inconveniente de que si se a�ade un dato m�s $(x_{n+1}, y_{n+1})$ a la colecci�n de datos ya existentes, es preciso recalcular el polinomio de interpolaci�n desde el principio. 

El m�todo de las diferencias divididas, permite obtener el polinomio de interpolaci�n en un n�mero menor de operaciones que en el caso del polinomio de Lagrange y adem�s, el c�lculo se hace escalonadamente, aprovechando todos los resultados anteriores cuando se a�ade al polinomio la contribuci�n de un nuevo dato.

El polinomio de orden $n$ de diferencias divididas se construye de la siguiente manera,

\begin{equation*}
p_n(x)=a_0+(x-x_0)\cdot a_1+(x-x_0)\cdot (x-x_1)\cdot a_2+\cdots +(x-x_0)\cdot (x-x_1)\cdots (x-x_{n-2})\cdot(x-x_{n-1})\cdot a_n
\end{equation*}

Donde, como siempre, $(x_0, y_0), (x_1,y_1), \cdots (x_n, y_n)$, representan los datos para los que se quiere calcular el polinomio interpolador de grado $n$. Si sustituimos las datos en el polinomio, llegamos a un sistema de ecuaciones, triangular inferior, en el que las inc�gnitas son los coeficientes del polinomio.
\begin{align*}
a_0&&=y_0\\
a_0&+(x_1-x_0)a_1&=y_1\\
a_0&+(x_2-x_0)a_1+(x_2-x_0)(x_2-x_1)a_2&=y_2\\
\cdots\\
a_0&+(x_n-x_0)a_1+\cdots+(x_n-x_0)(x_n-x_1)\cdots (x_n-x_{n-2})(x_n-x_{n-1})a_n&=y_n
\end{align*}

Este sistema se resuelve expl�citamente empleando un esquema de diferencias divididas. 

La diferencia divida de primer orden entre dos puntos $(x_0,y_0)$ y $(x_1,y_1)$ se define como,
\begin{equation*}
f\left[x_0,x_1\right]=\frac{y_1-y_0}{x_1-x_0}
\end{equation*}														

Para tres puntos se define la diferencia dividida de segundo orden como, $(x_0,y_0)$, $(x_1,y_1)$ y $(x_2,y_2)$

\begin{equation*}
f\left[x_0,x_1,x_2\right]=\frac{f\left[x_1,x_2\right]-f\left[x_0,x_1\right]}{x_2-x_0}
\end{equation*}

y, en general definiremos la diferencia dividida de orden $i$ para $i+1$ puntos como,

\begin{equation*}
f\left[x_0,x_1,\cdots,x_i\right]=\frac{f\left[x_1,x_2,\cdots,x_i\right]-f\left[x_0,x_1,\cdots,x_{i-1}\right]}{x_i-x_0}
\end{equation*}																												
Si despejamos por sustituci�n progresiva los coeficientes del polinomio de interpolaci�n del sistema triangular inferior obtenido, cada coeficiente puede asociarse a una diferencia dividida,
\begin{align*}
a_0&=f\left[x_0\right]=y_0\\
a_1&=f\left[x_0,x_1\right]\\
\vdots\\
a_i&=f\left[x_0,x_1,\cdots,x_i\right]\\
\vdots\\
a_n&=f\left[x_0,x_1,\cdots,x_n\right]\\
\end{align*}																								
Por tanto, podemos obtener directamente los coeficientes del polinomio calculando las diferencias divididas. Veamos un ejemplo empleando el siguiente conjunto de cuatro datos,

\begin{table}[h]
\centering
\begin{tabular}{c|cccc}
x&0&1&3&4\\
\hline
y&1&-1&2&3
\end{tabular}
\end{table}

Habitualmente, se construye a partir de los datos una tabla, como la  \ref{tabdif}, de diferencias divididas. Las primera columna contiene los valores de la variable $x$, la siguiente los valores de las diferencias divididas de orden cero (valores de $y$). A partir de la segunda, las siguientes columnas contienen las diferencias dividas de los elementos de la columna anterior, calculados entre los elementos que ocupan filas consecutivas. La tabla va perdiendo cada vez una fila, hasta llegar a la diferencia dividida de orden $n$ de todos los datos iniciales.

\begin{table}[h]
\centering
\caption{Tabla de diferencia divididas para cuatro datos}
\begin{tabular}{ccccc}
$x_i$&$y_i$&$f\left[x_i,x_{i+1}\right]$&$f\left[x_i,x_{i+1},x_{i+2}\right]$&$f\left[x_i,x_{i+1},x_{i+2},x_{i+3}\right]$\\
\hline
$x_0=0$&$y_0=\ \  1$&$f\left[x_0,x_1\right]=-2$&$f\left[x_0,x_1,x_2\right]=\ \ 7/6$&$f\left[x_0,x_1,x_2,x_3\right]=-1/3$\\
$x_1=1$&$y_1=-1$&$f\left[x_1,x_2\right]=3/2$&$f\left[x_1,x_2,x_3\right]=-1/6$\\
$x_2=3$&$y_2=\ \ 2$&$f\left[x_2,x_3\right]=\ \ 1$\\
$x_3=4$&$y_3=\ \ 3$\\

\end{tabular}
\label{tabdif}
\end{table}

Los coeficientes del polinomio de diferencias divididas se corresponden con los elementos de la primera fila de la tabla. Por lo que en nuestro ejemplo el polinomio resultante ser�a,

\begin{equation*}
p_3(x)=1-2x+\frac{7}{6}x(x-1)-\frac{1}{3}x(x-1)(x-3)
\end{equation*}

Es importante hacer notar que el polinomio de interpolaci�n obtenido por diferencias divididas siempre aparece representado como suma de productos de binomios $(x-x_0)(x-x_1)\cdots$ y los coeficientes obtenidos corresponden a esta representaci�n y no a la representaci�n habitual de un polinomio como suma de potencias de la variable $x$. 

El siguiente c�digo permite calcular los coeficientes del polinomio de diferencias divididas a partir de un conjunto de $n$ datos.

\begin{lstlisting}
function a=difdiv(x,y)
% este polinomio permite obtener los coeficientes del polinomio de
% diferencias divididas que interpola los datos contenidos el los vectores x
% e y. Da como resultado un vector fila a con los coeficientes

% miramos cuantos datos tenemos
n=length(x);

% inicializamos el vector de coeficientes con las diferencias de orden 0, es
% decir los valores de y,

a=y;

% y ahora montamos un bucle, si tenemos n datos debemos calcular n
% diferencias, como ya tenenos la primera, iniciamos el bucle en 2,
for j=2:n
    % en cada iteraci�n calculamos las diferencias de un orden superior,
    % como solo nos vale la primera diferencia de cada orden empezamos el
    % bucle interior en el valor del exterior j
    for i=j:n
        a(i)=(a(i)-y(i-1))/(x(i)-x(i-j+1));
    end
    % volvemos a copiar en y las diferencias obtenidas para emplearlas en la
    % siguiente iteracion
    y=a;
    
end
\end{lstlisting}

Como el polinomio de diferencias divididas toma una forma especial, es preciso tenerlo en cuenta a la hora de calcular su valor en un punto $x$ determinado. Es siguiente c�digo permite evaluar un polinomio de diferencias divididas en un punto dado, conocidos sus coeficientes y los valores $x_1,\cdots, x_n$ a partir de los cuales se obtuvo el polinomio,

\begin{lstlisting}
function y=evdif(a,x,x1)
% esta funci�n obtiene el valor de un polinomio de diferencias divididas a
% partir de los coeficientes (a) del polinomio, los puntos (x) sobre los que
% se ha calculado el polinomio y el punto o vector de puntos (x1) para el
% que se quiere calcular el valor que toma el polinomio.

% obtenemos el tama�o del vector de coeficientes del polinomio,

n=length(a);

% Construimos un bucle para calcular el valor del polinomio,
y=a(1);
for k=1:n-1
    % calculamos el valor del producto de los binomios que multiplican al
    % coeficiente i
    binprod=1;
    for j=1:k
        binprod=binprod.*(x1-x(j));
    end
    y=y+a(k+1)*binprod;
end
\end{lstlisting}

Por �ltimo se podr�an reunir las dos funciones anteriores en una �nica funci�n que permitiera obtener directamente el valor que toma el polinomio de diferencias divididas en un punto x, partiendo de los datos interpolados. Una soluci�n sencilla, es crear una funci�n que llame a las dos anteriores,

\begin{lstlisting}
function y=intdifdiv(x,xp,yp)
% Esta funci�n calcula el valor del polinomio de diferencias divididas que 
% interpola los puntos (xp,yp) el el punto, o % los puntos contenidos en x. 
% Empleando las funciones, difdiv, para calcular los coeficientes del
% polinomio y evdif para evaluarlo

% llamamos a difdiv
a=difdiv(xp,yp);
 
 % y a continuaci�n llamamos a evdif
  y=evdif(a,xp,x);
 \end{lstlisting} 
 
 \subsection{El polinomio de Newton-Gregory} \label{sec:newgre} \index{Interpolaci�n!Polinomio de Newton-gregory} \index{Polinomio de Newton-Gregory}
 Supone una simplificaci�n al c�lculo del polinomio de diferencias divididas para el caso particular en que los datos se encuentran equiespaciados y dispuestos en orden creciente con respecto a los valores de la coordenada $x$.
 
En este caso, calcular los valores de las diferencias es mucho mas sencillo.  Si pensamos en las diferencias de primer orden, los denominadores de todas ellas son iguales, puesto que los datos est�n equiespaciados,

\begin{equation*}
\Delta x \equiv x_i-x_{i-1} =h
\end{equation*}

En cuanto a los numeradores, se calcular�an de modo an�logo al de las diferencias divididas normales,

\begin{equation*}
\Delta y_0= y_1-y_0, \Delta y_1=y_2-y_1, \cdots, \Delta y_i=y_{i+1}-y_i, \cdots, \Delta y_{n-1}=y_{n}-y_{n-1}
\end{equation*}

Las diferencias de orden superior para los numeradores se pueden obtener de modo recursivo, a partir de las de orden uno, puesto que los denominadores de todas ellas $h$, son iguales.


\begin{equation*}
\Delta^2 y_0=\Delta \left(\Delta y_0 \right) =(y_2-y_1)-(y_1-y_0)=(y_2-2y_1+y_0)
\end{equation*}

En este caso, el denominador de la diferencia ser�a $x_2-x_0=2h$, y la diferencia tomar�a la forma,

\begin{equation*}
f[x_0,x_1,x_2]=\frac{\Delta^2y_0}{2h^2}
\end{equation*}

En general, para la diferencias de orden n tendr�amos,
\begin{equation*}
\Delta^n y_0=y_n-\binom{n}{1}\cdot y_{n-1}+\binom{n}{2}\cdot y_{n-2}-\cdots+(-1)^n\cdot y_0
\end{equation*}

Donde se ha hecho uso de la expresi�n binomial,

\begin{equation*}
\binom{k}{l}=\frac{k!}{l!\cdot(k-l)!}
\end{equation*}

Para obtener la diferencia dividida de orden n, bastar�a ahora dividir por $n!\cdot h^n$.
\begin{equation*}
f\left[x_0,x_1,\cdots,x_n\right]=\frac{\Delta^n y_0}{n!\cdot h^n}
\end{equation*}

A partir de las diferencias, podemos representar el polinomio de diferencias divididas resultante como,

\begin{equation*}
p_n(x)=y_0+\frac{x-x_0}{h}\Delta y_0+\frac{(x-x_1)\cdot (x-x_0)}{2\cdot h^2}\Delta^2 y_0+\cdots +\frac{(x-x_{n-1}) \cdots (x-x_1)\cdot (x-x_0)}{n! \cdot h^n}\Delta^n y_0
\end{equation*}

Este polinomio se conoce como el polinomio de Newton-Gregory, y podr�a considerarse como una aproximaci�n  num�rica al polinomio de Taylor de orden n de la posible funci�n asociada a los datos empleados. 

En este caso, podr�amos construir la tabla para obtener los coeficientes de diferencias, calculando en cada columna simplemente las diferencias de los elementos de la columna anterior. Por ejemplo,

\begin{table}[h]
\centering
\caption{Tabla de diferencias para el polinomio de Newton -Gregory de cuatro datos}
\begin{tabular}{ccccc}
$x_i$&$y_i$&$\Delta y_i$&$\Delta^2 y_i$&$\Delta^3 y_i$\\
\hline
$x_0=0$&$y_0=\ \  1$&$-2$&$\ \ 5$&$-7$\\
$x_1=1$&$y_1=-1$&$ \ \ 3$&$ -2$\\
$x_2=2$&$y_2=\ \ 2$&$\ \ 1$\\
$x_3=3$&$y_3=\ \ 3$\\

\end{tabular}
\label{tabnewton}
\end{table}

Una vez calculadas las diferencias, basta dividir por $n!\cdot h^n$ los elementos de la primera fila de la tabla,

\begin{equation*}
a_0=1, a_1=\frac{-2}{1}, a_2=\frac{5}{2\cdot 1^2}, a_3=\frac{-7}{6\cdot 1^3}
\end{equation*}

El siguiente c�digo muestra un ejemplo de implementaci�n en Matlab del polinomio de Newton-Gregory

\begin{lstlisting}
function [a,y1]=newgre(x,y,x1)
% este polinomio permite obtener los coeficientes del polinomio de
% newton-gregory que interpola los datos contenidos en los vectores 
% x e y. Da como resultado un vector fila a con los coeficientes,  
% si se le da ademas un punto o vector de puntos calcula los  
% valoresque toma el polinomio en dichos puntos.

% miramos cuantos datos tenemos
n=length(x);

% inicializamos el vector de coeficientes con las diferencias de 
% orden 0, es decir los valores de y,
a=y;
h=x(2)-x(1);

% y ahora montamos un bucle, si tenemos n datos debemos calcular n
% diferencias, como ya tenenos la primera, iniciamos el bucle en 2,
for j=2:n
    % en cada iteraci�n calculamos las diferencias de un orden 
    % superior. Como solo nos vale la primera diferencia de cada 
    % orden empezamos el bucle interior en el valor del exterior j
    for i=j:n
        % ahora basta dividir en todos los casos por la distancia h
        % multiplicada por el orden de la diferencia
        a(i)=(a(i)-y(i-1))/((j-1)*h);
    end
    % volvemos acopiar en y las diferencias obtenidas para 
    % emplearlas en la siguiente iteracion
    y=a;
    
end
y1=[];
if nargin==3
    % Construimos un bucle para calcular el valor del polinomio,
    y1=a(1);
    for k=1:n-1
        % calculamos el valor del producto de los binomios que 
        % multiplican al coeficiente i
        binprod=1;
        for j=1:k
            binprod=binprod.*(x1-x(j));
        end
        y1=y1+a(k+1)*binprod;
    end
end
\end{lstlisting}


\section{Interpolaci�n por intervalos.}
\begin{figure}[h]
\centering
\includegraphics[width=10cm]{intpoli.eps}
\caption{Polinomio de interpolaci�n de grado nueve obtenido a partir de un conjunto de diez datos} 
\label{fig:intepol}
\end{figure}

Hasta ahora, hemos visto c�mo interpolar un conjunto de $n+1$ datos mediante un polinomio de grado $n$. En muchos casos, especialmente cuando el n�mero de datos es suficientemente alto, los resultados de dicha interpolaci�n pueden no ser satisfactorios.  La raz�n es que el grado del polinomio de interpolaci�n crece linealmente con el n�mero de puntos a interpolar, as� por ejemplo para interpolar 11 datos necesitamos un polinomio de grado 10. Desde un punto de vista num�rico, este tipo de polinomios pueden dar grandes errores debido al redondeo. Por otro lado, y dependiendo de la disposici�n de los datos para los que se realiza la interpolaci�n, puede resultar que el polinomio obtenido tome una forma demasiado complicada para los valores comprendidos entres los datos interpolados..  

La figura \ref{fig:intepol} muestra el polinomio de interpolaci�n de grado nueve para un conjunto de 10 datos. Es f�cil darse cuenta, simplemente observando los datos, que no hay ninguna raz�n que justifique las curvas que traza el polinomio entre los puntos $1$ y $2$  o los puntos $9$ y $10$, por ejemplo.

\begin{figure}
\centering
\subfigure[Interpolaci�n de orden cero  \label{fig:stepwise}]{\includegraphics[width=7cm]{steps.eps}} %\qquad 
\subfigure[Interpolaci�n lineal  \label{fig:lineal}]{\includegraphics[width=7cm]{lineal.eps}}\\
\caption{Interpolaciones de orden cero y lineal para los datos de la figura \ref{fig:intepol} }
\end{figure}


En muchos casos es preferible no emplear todos los datos disponibles para obtener un �nico polinomio de interpolaci�n. En su lugar, lo que se hace es dividir el conjunto de datos en varios grupos ---normalmente se agrupan formando intervalos de datos consecutivos--- y obtener varios polinomios de menor grado, de modo que cada uno interpole los datos de un grupo distinto. 

El grado de los polinomios empleados deber� estar, en principio, relacionado con los datos contenidos en cada tramo.


\paragraph{interpolaci�n de orden cero} \index{Interpolaci�n de orden cero} si hacemos que cada intervalo contenga un solo dato, obtendr�amos polinomios de interpolaci�n de grado cero, $a_{0i}=y_i$. El resultado, es un conjunto de escalones cuya valor var�a de un intervalo a otro de acuerdo con el dato representativo contenido en cada tramo. La figura \ref{fig:stepwise} muestra el resultado de la interpolaci�n de orden cero para los mismos diez datos de la figura \ref{fig:intepol}.

\paragraph{interpolaci�n lineal.} \index{Interpolaci�n lineal} En este caso, se dividen los datos en grupos de dos. Cada par de datos consecutivos se interpola calculando la recta que pasa por ellos. La interpolaci�n lineal se emplea en muchas aplicaciones debido a su sencillez de c�lculo. La figura \ref{fig:lineal}, muestra el resultado de aproximar linealmente los mismos datos contenidos en los ejemplos anteriores.

Siguiendo el mismo procedimiento, aumentando el n�mero de datos contenidos en cada intervalo, podr�amos definir una interpolaci�n cuadr�tica, con polinomios de segundo grado, tomando intervalos que contengan tres puntos, una interpolaci�n c�bica, para intervalos de cuatro puntos etc.

\subsection{Interpolaci�n mediante splines c�bicos} \index{Interpolaci�n!Splines}\index{Splines}
Hemos descrito antes c�mo el polinomio interpolador de orden $n$ para un conjunto de $n+1$ datos puede presentar el inconveniente de complicar excesivamente la forma de  la curva obtenida entre los puntos interpolados. La interpolaci�n a tramos que acabamos de describir, simplifica la forma de la curva entre los puntos pero presenta el problemas de la continuidad en las uniones entre tramos sucesivos. Ser�a deseable encontrar m�todos de interpolaci�n que fueran capaces de solucionar ambos problemas simult�neamente. Una buena aproximaci�n a dicha soluci�n la proporcionan los \emph{splines}.

Una funci�n \emph{spline} est� formado por un conjunto de polinomios, cada uno definido en un intervalo, que se unen entre s� obedeciendo a ciertas condiciones de continuidad.

Supongamos que tenemos una tabla de datos cualquiera,

\begin{table}[h]
\centering
\begin{tabular}{c|cccc}
x&$x_0$&$x_1$&$\cdots$&$x_n$\\
\hline
y&$y_0$&$y_1$&$\cdots$&$y_n$
\end{tabular}
\end{table}

Para construir una funci�n \emph{spline} $S$ de orden $m$, que interpole los datos de la tabla, se definen intervalos tomando como extremos dos puntos consecutivos de la tabla y un polinomio de grado $m$ para cada uno de los intervalos,

\begin{equation*}
S= \left\{ 
\begin{aligned}
S_0(x),& \ x\in [x_0,x_1]\\
S_1(x),& \ x\in [x_1,x_2]\\
\vdots \\
S_i(x),& \ x\in [x_i,x_{i+1}]\\
\vdots \\
S_{n-1}(x),& \ x\in [x_{n-1},x_n]
\end{aligned}
\right.
\end{equation*}

Para que $S$ sea una funci�n Spline de orden $m$ debe cumplir que sea continua y tenga $m-1$ derivadas continuas en el intervalo $[x_0,x_n]$ en que se desean interpolar los datos.
   
Para asegurar la continuidad, los polinomios que forman $S$ deben cumplir las siguientes condiciones en sus extremos;
\begin{align*}
S_i(x_{i+1})&=S_{i+1}(x_{i+1}),\ (1\leq i \leq n-1)\\
S'_i(x_{i+1})&=S'_{i+1}(x_{i+1}),\ (1\leq i \leq n-1)\\
S''_i(x_{i+1})&=S''_{i+1}(x_{i+1}),\ (1\leq i \leq n-1)\\
\vdots \\
S^{m-1}_i(x_{i+1})&=S^{m-1}_{i+1}(x_{i+1}),\ (1\leq i \leq n-1)\\
\end{align*}

Es decir, dos polinomios consecutivos del spline y sus $m-1$ primeras derivadas, deben tomar los mismos valores en el extremo com�n. 

Una consecuencia inmediata de las condiciones de continuidad exigidas a los splines es que sus derivadas sucesivas, $S',\ S'', \cdots$ son a su vez funciones spline de orden $m-1,\ m-2, \cdots$. Por otro lado, las condiciones de continuidad suministran  $(n-1)\cdot m$ ecuaciones que, unidas a las $n+1$ condiciones de interpolaci�n ---cada polinomio debe pasar por los datos que constituyen los extremos de su intervalo de definici�n---,  suministran un total de  $n\cdot (m+1)-(m-1)$ ecuaciones. Este n�mero es insuficiente para determinar los $(m+1)\cdot n$ par�metros correspondientes a los $n$ polinomios de grado $m$ empleados en la interpolaci�n. Las $m-1$ ecuaciones que faltan se obtienen imponiendo a los splines condiciones adicionales.


\paragraph{Splines c�bicos.} \index{Splines! Cubicos} Los splines m�s empleados son los formados por polinomios de tercer grado. En total, tendremos que determinar $(m+1)\cdot n=4\cdot n$ coeficientes para obtener todos los polinomios que componen el spline. Las condiciones de continuidad m�s la de interpolaci�n suministran en total $3\cdot (n-1)+n+1=4\cdot n-2$  ecuaciones. Necesitamos imponer al spline dos condiciones m�s. Algunas t�picas son,
\begin{enumerate}
\item Splines naturales $S''(x_0)=S''(x_n)=0$
\item Splines con valor conocido en la primera derivada de los extremos $S'(x_0)=y'_0, S'(x_n)=y'_n$
\item Splines peri�dicos,
\begin{equation*}
\left\{ 
\begin{aligned}
S(x_0)&=S(x_n)\\
S'(x_0)&=S'(x_n)\\
S''(x_0)&=S''(x_n)
\end{aligned}
\right.
\end{equation*}
\end{enumerate}

Intentar construir un sistema de ecuaciones para obtener a la vez todos los coeficientes de todos los polinomios es una tarea excesivamente compleja porque hay demasiados par�metros.  Para abordar el problema partimos del hecho de que $S''(x)$ es tambi�n un spline de orden 1 para los puntos interpolados. Si los definimos como,

\begin{equation*}
S''_i(x)=-M_i\frac{x-x_{i+1}}{h_i}+M_{i+1}\frac{x-x_i}{h_i},\   i=0,\cdots, n-1
\end{equation*}

donde $h_i=x_{i+1}-x_i$ representa el ancho de cada intervalo y donde cada valor $M_i=S''(x_i)$ ser� una de las inc�gnitas que deberemos resolver.

Si integramos dos veces la expresi�n anterior,
\begin{align*}
S'_i(x)&=-M_i\frac{(x-x_{i+1})^2}{2\cdot h_i}+M_{i+1}\frac{(x-x_i)^2}{2\cdot h_i}+A_i,\   i=0,\cdots, n-1\\
S_i(x)&=-M_i\frac{(x-x_{i+1})^3}{6\cdot h_i}+M_{i+1}\frac{(x-x_i)^3}{6\cdot h_i}+A_i(x-x_i)+B_i,\   i=0,\cdots, n-1\\
\end{align*}

Empezamos por imponer las condiciones de interpolaci�n: el polinomio $S_i$ debe pasar por el punto $(x_i,y_i)$,

\begin{equation*}
S_i(x_i)=-M_i\frac{(x_i-x_{i+1})^3}{6\cdot h_i}+B_i=y_i \Rightarrow B_i=y_i-\frac{M_i\cdot h_i^2}{6},\ i=0,\cdots, n-1
\end{equation*}

A continuaci�n imponemos continuidad del spline en los nodos comunes: El polinomio $S_{i-1}$ tambi�n debe pasar por el punto $(x_i, y_i)$,

\begin{align*}
S_{i-1}(x_i)&=M_i\frac{(x_i-x_{i-1})^3}{6\cdot h_i}+A_{i-1}(x_i-x_{i-1})+\overbrace{y_{i-1}-\frac{M_{i-1}\cdot h_{i-1}^2}{6}}^{B_{i-1}}=y_i \Rightarrow\\
\Rightarrow A_{i-1}&=\frac{y_i-y_{i-1}}{h_{i-1}}-\frac{M_i-M_{i-1}}{6}\cdot h_{i-1}, \ i=1,\cdots, n
\end{align*}

Y por tanto,

\begin{equation*}
A_i=\frac{y_{i+1}-y_i}{h_i}-\frac{M_{i+1}-M_i}{6}\cdot h_i, \ i=0,\cdots, n-1
\end{equation*}

En tercer lugar imponemos la condici�n de que las derivadas tambi�n sean continuas en los nodos comunes,

\begin{align*}
S'_i(x_i)&=-M_i\frac{(x_i-x_{i+1})^2}{2\cdot h_i}+M_{i+1}\frac{(x_i-x_i)^2}{2\cdot h_i}+\frac{y_{i+1}-y_i}{h_i}-\frac{M_{i+1}-M_i}{6}\cdot h_i,\   i=0,\cdots, n-1\\
S'_{i-1}(x_i)&=-M_{i-1}\frac{(x_i-x_i)^2}{2\cdot h_{i-1}}+M_{i}\frac{(x_i-x_{i-1})^2}{2\cdot h_{i-1 }}+\frac{y_i-y_{i-1}}{h_{i-1}}-\frac{M_i-M_{i-1}}{6}\cdot h_{i-1},\   i=1,\cdots, n\\
S'_i(x_i)&=S'_{i-1}(x_i) ,\   i=1,\cdots, n-1 \Rightarrow\\
&\Rightarrow -M_i\frac{h_i}{2}+\frac{y_{i+1}-y_i}{h_i}-\frac{M_{i+1}-M_i}{6}\cdot h_i=M_{i}\frac{h_{i-1}}{2}+\frac{y_i-y_{i-1}}{h_{i-1}}-\frac{M_i-M_{i-1}}{6}\cdot h_{i-1}
\end{align*}

Si agrupamos a un lado los valores $M_{i-1}, M_i, M_{i+1}$,

\begin{align*}
h_{i-1}\cdot M_{i-1}+2\cdot (h_{i-1}+h_i)\cdot M_i+h_i\cdot M_{i+1}=6\cdot \left(\frac{y_{i+1}-y_i}{h_i}-\frac{y_i-y_{i-1}}{h_{i-1}}\right)\\
i=1,\cdots ,n-1
\end{align*}

En total tenemos $M_0,\cdots, M_n$, $n+1$ inc�gnitas y la expresi�n anterior, solo nos suministra $n-1$ ecuaciones. Necesitamos dos ecuaciones m�s, Si imponemos la condici�n de splines naturales, Para el extremo de la izquierda del primer polinomio y para el extremo de la derecha del �ltimo,

\begin{align*}
M_0=S''(x_0)=0\\
M_n=S''(x_n)=0
\end{align*}

Con estas condiciones y la expresi�n obtenida para el resto de los $M_i$, podemos construir un sistema de ecuaciones tridiagonal

\begin{equation*}
\begin{pmatrix}
2(h_0+h_1) & h_1 & 0 &0&\cdots &0&0\\
 h_1 & 2(h_1+h_2) & h_2 &0& \cdots&0 & 0\\
0& h_2 & 2(h_2+h_3) & h_3 &\cdots &0& 0\\
\vdots & \vdots & \vdots &\vdots& \ddots & \vdots&\vdots \\
0 & 0 & 0&0&\cdots& 2(h_{n-3}+h_{n-2}) & h_{n-2} \\ 
0 & 0 & 0&0&\cdots&h_{n-2} & 2(h_{n-2}+h_{n-1})
\end{pmatrix}\cdot \begin{pmatrix}
M_1\\
M_2\\
M_3\\
\vdots \\
M_{n-1}
\end{pmatrix}=\begin{pmatrix}
b_1\\
b_2\\
b_3\\
\vdots \\
b_{n-1}
\end{pmatrix}
\end{equation*}

Donde hemos hecho,

\begin{equation*}
b_i=6\cdot \left(\frac{y_{i+1}-y_i}{h_i}-\frac{y_i-y_{i-1}}{h_{i-1}}\right)
\end{equation*}

Tenemos un sistema de ecuaciones en el que la matriz de coeficientes es tridiagonal y adem�s diagonal dominante, por lo que podr�amos emplear cualquiera de los m�todos vistos en cap�tulo  
\ref{sistemas}.  Una vez resuelto el sistema y obtenidos los valores de $M_i$, obtenemos los valores de $A_i$ y $B_i$ a Partir de las ecuaciones obtenidas m�s arriba.

Por �ltimo, la forma habitual de definir el polinomio de grado 3 $S_i$, empleado para interpolar los valores del intervalo $[x_i,x_{i+1}]$, mediante splines c�bicos se define como, 

\begin{equation*}
S_i(x)=\alpha_i+\beta_i(x-x_i)+\gamma_i(x-x_i)^2+\delta_i(x-x_i)^3, \ x\in [x_i,x_{i+1}],\ (i=0,1,\cdots,n-1)
\end{equation*}

Donde,

\begin{align*}
\alpha_i &=y_i\\
\beta_i &=\frac{y_{i+1}-y_i}{h_i}-\frac{M_i \cdot h_i}{3}-\frac{M_{i+1} \cdot h_i}{6}\\
\gamma_i &=\frac{M_i}{2}\\
\delta_i &=\frac{M_{i+1}-M_i}{6\cdot h_i}
\end{align*}

La siguiente funci�n permite obtener los coeficientes y el resultado de interpolar un conjunto de puntos mediante splines c�bicos,

\begin{lstlisting}
function [c,yi]=spcubic(x,y,xi)
% uso [c,yi]=spcubic(x,y,xi)
% Esta funci?n interpola los puntos contenidos en los vectores columna x e y
% empleando parae ello splines c?bicos naturales. devuelve los coeficientes
% de los polinomios en una matriz de dimension (n-1)*4. Cada fila contiene
% un splin desde So a Sn-1 los coeficiente est?n guardados el la fila en
% orden depotencia creciente: ejemplo c(i,1)+c(i,2)*x+c(i,3)*x^2+c(i,4)*x^3
% ademas devuelve los valores interpolados yi correspondientes a puntos xi
% contenidos en el intervalo definido por los valores de x

% obtenemos la longitud de los datos...
l=length(x);
% obtencion de los coeficientes M
% Construimos el vector de diferencias h y un vector de diferencias
% Dy=y(i+1)-y(i) Que nos ser? muy ?til a la hora de calcular el vector de
% t?rminos independientes del sistema

for i=1:l-1
    h(i)=x(i+1)-x(i);
    Dy(i)=y(i+1)-y(i);
end

% Construimos la matriz del sistema. (Lo ideal seria definirla como una
% matriz sparse pero en fin no sabeis, porque sois todav?a peque?os, etc...)

CSP(l-2,l-2)=2*(h(l-2)+h(l-1));
b(l-2,1)=6*(Dy(l-1)/h(l-1)-Dy(l-2)/h(l-2));
for i=1:l-3
    CSP(i,i)=2*(h(i)+h(i+1));
    CSP(i,i+1)=h(i+1);
    CSP(i+1,i)=h(i+1);
    b(i,1)=6*(Dy(i+1)/h(i+1)-Dy(i)/h(i));
end

% calculamos los coeficientes M,
M=CSP\b;

% A?adimos el primer y el ?ltimo valor como ceros (Splines naturales)
M=[0;M;0];

% calulamos los coeficientes A,
for i=1:l-1
    A(i)=Dy(i)/h(i)-(M(i+1)-M(i))*h(i)/6;
end
% Calculamos los coeficientes B,
for i=1:l-1
    B(i)=y(i)-M(i)*h(i)^2/6;
end
% Podemos ahora calcular el valor que toma el polinomio para los puntos
% que se desea interpolar

% miramos cuantos puntos tenemos
l2=length(xi);
for i=1:l2
    % miramos en que intervalo esta el punto xi(i)
    j=1;
    while xi(i)>x(j)
        j=j+1;
    end
    if j>l-1
        j=l-1; % aunque estamos extrapolando
    elseif j<2
        j=2; % estamos calculando el primer punto o estamos tanbien
        % extrapolando
    end
    
    yi(i)=-M(j-1)*(xi(i)-x(j))^3/(6*h(j-1))+...
        M(j)*(xi(i)-x(j-1))^3/(6*h(j-1))+A(j-1)*(xi(i)-x(j-1))+B(j-1);
end

% calcumos los coeficientes c del spline en forma 'normal'
% la primera columna, son los valores de i,
c=zeros(l-1,4);
for i=1:l-1
    c(i,1)=y(i);
    c(i,2)=Dy(i)/h(i)-M(i)*h(i)/3-M(i+1)*h(i)/6;
    c(i,3)=M(i)/2;
    c(i,4)=(M(i+1)-M(i))/(6*h(i));
end
\end{lstlisting}



 La figura, \ref{fig:splines} muestra el resultado de interpolar mediante un spline c�bico, los datos contenidos en la figura \ref{fig:intepol}. Es f�cil observar c�mo ahora los polinomios de interpolaci�n dan como resultado una curva suave en los datos interpolados y en la que adem�s las curvas son tambi�n suaves, sin presentar variaciones extra�as, para los puntos contenidos en cada intervalo entre dos datos.
 
\begin{figure}[h]
\centering
\includegraphics[width=12cm]{splines.eps}
\caption{Interpolaci�n mediante spline c�bico de los datos de la figura \ref{fig:intepol}} 
\label{fig:splines}
\end{figure} 
 
 
\subsection{Funciones propias de Matlab para interpolaci�n por intervalos} \index{interp1@\texttt{interp1}}

Para realizar una interpolaci�n por intervalos mediante cualquiera de los procedimientos descritos, Matlab incorpora la funci�n propia \texttt{interp1}. Esta funci�n admite como variables de entrada dos vectores con los valores de las coordenadas $x$ e $y$ de los datos que se desea interpolar y un tercer vector $x1$ con los puntos para los que se desea calcular el valor de la interpolaci�n. Adem�s, admite como variable  de entrada una cadena de caracteres que indica el m�todo con el que se quiere realizar la interpolaci�n. Dicha variable puede tomar los valores:
\begin{enumerate}
\item \texttt{'nearest'}. Interpola el intervalo empleando el valor $y_i$ correspondiente al valor  $x_i$ m�s cercano al punto que se quiere interpolar. El resultado es una interpolaci�n a escalones.
\item \texttt{'linear'} realiza una interpolaci�n lineal entre los puntos del conjunto de datos que se desea interpolar.
\item \texttt{'spline'}. Interpola empleando splines c�bicos naturales.
\item \texttt{'cubic'} o \texttt{'pchip} Emplea polinomios de Hermite c�bicos. Es un m�todo similar al de los splines que no describiremos en estos apuntes.
\end{enumerate} 

La funci�n devuelve como salida los valores interpolados correspondientes a los puntos de $x1$. El siguiente c�digo muestra el modo de usar el comando \texttt{interp1}. Para probarlo se han creado dos vectores \texttt{x} e \texttt{y} que contienen el conjunto de datos que se emplear� para calcular la interpolaci�n. Adem�s, se ha creado otro vector \texttt{x1} que contiene los puntos para los que se quiere calcular el resultado de la interpolaci�n.

\begin{verbatim}
>> x=[1:2:16]
x =
     1     3     5     7     9    11    13    15
>> y=[1 3 4 2 -1 4 -5 3]
y =
     1     3     4     2    -1     4    -5     3
>> x1=[3.5 7.5]
x1 =
    3.5000    7.5000
>> y1=interp1(x,y,x1,'spline')
y1 =
    3.4110    0.7575
>> x=[1:2:16]
x =
     1     3     5     7     9    11    13    15
>> y=[1 3 4 2 -1 4 -5 3]
y =
     1     3     4     2    -1     4    -5     3
>> x1=[3.5 7.5]
x1 =
    3.5000    7.5000
>> y1=interp1(x,y,x1,'nearest')
y1 =
     3     2
>> y1=interp1(x,y,x1,'linear')
y1 =
    3.2500    1.2500
>> y1=interp1(x,y,x1,'cubic')
y1 =
    3.3438    1.1938
\end{verbatim}

\section{Ajuste polin�mico por el m�todo de m�nimos cuadrados}\label{sec:mc}\index{M�nimos cuadrados} \index{Ajuste polin�mico}
Los m�todos de interpolaci�n que hemos descrito en las secciones anteriores pretenden encontrar un polinomio o una funci�n definida a partir de polinomios que pase por un conjunto de datos. En el caso del ajuste por m�nimos cuadrados, lo que se pretende es buscar el polinomio, de un grado dado, que mejor se aproxime a un conjunto de datos.

Supongamos que tenemos un conjunto de $m$ datos, 

\begin{table}[h]
\centering
\begin{tabular}{c|cccc}
x&$x_1$&$x_2$&$\cdots$&$x_m$\\
\hline
y&$y_1$&$y_2$&$\cdots$&$y_m$
\end{tabular}
\end{table} 
 
Queremos construir un polinomio $p(x)$  de grado $n < m-1$, de modo que los valores que toma el polinomio para los datos $p(x_i)$ sean lo m�s cercanos posibles a los correspondientes valores $y_i$. 

En primer lugar, necesitamos clarificar qu� entendemos por \emph{lo m�s cercano posible}.  Una posibilidad, es medir la diferencia, $y_i-p(x_i)$ para cada par de datos del conjunto. Sin embargo, es m�s frecuente emplear el cuadrado de dicha diferencia, $\left(y_i-p(x_i)\right)^2$. Esta cantidad tiene, entre otras, la ventaja de que su valor es siempre positivo con  independencia de que la diferencia sea positiva o negativa. Adem�s, representa el cuadrado de la distancia entre $p(x_i)$ e $y_i$. Podemos tomar la suma de dichas distancias al cuadrado, obtenidas por el polinomio para todos los pares de puntos, 

\begin{equation*}
\sum_{i=1}^m \left(y_i-p(x_i)\right)^2
\end{equation*}

como una medida de la distancia del polinomio a los datos.  De este modo,  el polinomio \emph{lo m�s cercano posible}  a los datos ser�a aquel que minimice la suma de diferencias al cuadrado que acabamos de definir. De ah� el nombre del m�todo.

En muchos casos, los datos a los que se pretende ajustar un polinomio por m�nimos cuadrados son datos experimentales. En funci�n del entorno experimental y del m�todo con que se han adquirido los datos, puede resultar que algunos resulten m�s fiables que otros. En este caso, ser�a deseable hacer que el polinomio se aproxime m�s a los datos m�s fiables. Una forma de hacerlo es a�adir unos \emph{pesos}, $\omega_i$, a las diferencias al  cuadrado en funci�n de la confianza que nos merece cada dato,

\begin{equation*}
\sum_{i=1}^m \omega_i \left(y_i-p(x_i)\right)^2
\end{equation*}

Los datos fiables se multiplican por valores de $\omega$ grandes y los poco fiables por valores peque�os.

Para ver c�mo obtener los coeficientes de un polinomio de m�nimos cuadrados, empezaremos con el caso m�s sencillo; un polinomio de grado $0$. En este caso, el polinomio es una constante, definida por su t�rmino independiente $p(x)=a_0$. El objetivo a minimizar ser�a entonces,

\begin{equation*}
g(a_0)=\sum_{i=1}^m \omega_i \left(y_i-a_0\right)^2
\end{equation*}

El valor m�nimo de esta funci�n debe cumplir que su derivada primera $g'(a_0)=0$ y que su derivada segunda  $g''(a_0)\geq 0$,

\begin{align*}
g'(a_0)&=-2\sum_{i=1}^m \omega_i \left(y_i-a_0\right)=0 \Rightarrow a_0=\frac{\sum_{i=1}^m \omega_i\cdot y_i}{ \sum_{i=1}^m \omega_i}\\
g''(a_0)&=2\sum_{i=1}^m \omega_i \Rightarrow  g''(a_0) \geq 0
\end{align*}

El resultado obtenido para el valor de $a_0$ es una media, ponderada con los pesos $w_i$ de los datos. Si hacemos $w_i=1 \ \forall w_i$ obtendr�amos exactamente la media de los datos. Este resultado resulta bastante razonable. Aproximar un conjunto de valores por un polinomio de grado cero, es tanto como suponer que la variable $y$ permanece constante para cualquier valor de $x$. Las diferencias observadas deber�an deberse entonces a errores aleatorios experimentales, y la mejor estima del valor de $y$ ser� precisamente el valor medio de los valores disponibles. La figura \ref{fig:mc0} muestra el resultado de  calcular el polinomio de m�nimos cuadrados de grado cero para un conjunto de datos.

\begin{figure}[h]
\centering
\includegraphics[width=12cm]{mc0.eps}
\caption{Polinomio de m�nimos cuadrados de grado 0} 
\label{fig:mc0}
\end{figure} 

El siguiente paso en dificultad ser�a tratar de aproximar un conjunto de datos por un polinomio de grado 1, es decir, por una linea recta, $p(x)=a_0+a_1 x$. En este caso, la suma de diferencias al cuadrado toma la forma,

\begin{equation*}
g(a_0,a_1)=\sum_{i=1}^m \omega_i\left(y_i-a_0-a_1 x_i \right)^2
\end{equation*}
 
En este caso, tenemos dos coeficientes sobre los que calcular el m�nimo. �ste se obtiene cuando las derivadas parciales de $g(a_0,a_1)$ respecto a ambos coeficientes son iguales a cero. 

\begin{align*}
\frac{\partial g}{\partial a_0}&=-2\sum_{i=1}^m \omega_i (y_i-a_0-a_1 x_i) = 0\\
\frac{\partial g}{\partial a_1}&=-2\sum_{i=1}^m \omega_i x_i(y_i-a_0-a_1 x_i) = 0
\end{align*} 
 
Si reordenamos las ecuaciones anteriores, 

\begin{align*}
&\left(\sum_{i=1}^m \omega_i\right)a_0+ \left(\sum_{i=1}^m \omega_ix_i\right)a_1 =\sum_{i=1}^m \omega_iy_i\\
&\left(\sum_{i=1}^m \omega_ix_i\right)a_0+ \left(\sum_{i=1}^m \omega_ix_i^2\right)a_1 =\sum_{i=1}^m \omega_ix_iy_i
\end{align*}

Obtenemos un sistema de dos ecuaciones lineales cuyas inc�gnitas son precisamente los coeficientes de la recta de m�nimos cuadrados.

Podemos ahora generalizar el resultado para un polinomio de grado $n$, $p(x)=a_0+a_1x+a_2x^2+\cdots +a_nx^n$. La funci�n $g$ toma la forma,

\begin{equation*}
g(a_0,a_1,\cdots, a_n)=\sum_{i=1}^m \omega_i \left (a_0+a_1x_i+\cdots+ a_nx_i^n-y_i\right)^2
\end{equation*}

De nuevo, para obtener los coeficientes del polinomio igualamos las derivadas parciales a cero,

\begin{equation*}
\frac{\partial g(a_0,a_1,\cdots, a_n)}{\partial a_j}=0 \Rightarrow \sum_{i=1}^m \omega_i x_i^j \left( a_0+a_1x_i+\cdots + a_nx_i^n-y_i\right)=0, \ j=0,1\cdots, n
\end{equation*} 

Si reordenamos las expresiones anteriores, llegamos a un sistema de $n+1$ ecuaciones lineales, cuyas inc�gnitas son los coeficientes del polinomio de m�nimos cuadrados,

\begin{equation*}
\begin{pmatrix}
s_0& s_1& \cdots s_n\\
s_1& s_2& \cdots s_{n+1}\\
\vdots & \vdots & \ddots \\
s_n& s_{n+1}& \cdots s_{2n}\\
\end{pmatrix}\cdot \begin{pmatrix}
a_0\\
a_1\\
\vdots \\
a_n
\end{pmatrix}=\begin{pmatrix}
c_0\\
c_1\\
\vdots \\
c_n
\end{pmatrix}
\end{equation*} 

Donde hemos definido $s_j$ y $c_j$ como,

\begin{align*}
s_j&=\sum_{i=1}^m \omega_ix_i^j\\
c_j&=\sum_{i=1}^m \omega_ix_i^jy_i
\end{align*}
 
El siguiente c�digo permite obtener el polinomio de m�nimos cuadrados que aproxima un conjunto de $n$ datos,

\begin{lstlisting}
function a=mc(x,y,n,w)
% uso: a=mc(x,y,n,w). Esta funci�n permite obtener los coeficientes del
% polinomio de m�nimos cuadrados que ajusta un conjunto de datos. Las
% variables de entrada son: x, vector con la componente x de los datos a
% ajustar. y, vector con la componente y de los datos a ajustar. n grado del
% polinomio de minimos cuadrados. w vector de pesos asociados a los datos. si no
% se sumininstra se toman todos los pesos como 1. Salidas: vector columna con
% los coeficientes del polinomio=> a(1)+a(2)*x+a(3)*x^2+a(n+1)*x^n

% comprobamos en primer lugar que tenemos datos suficientes,

m=length(x);
if m<n+1
    error('no hay datos sufiecientes para calcular el polinomio pedido')
end
% Si no se ha suministrado vector de pesos construimos uno formado por unos,
if nargin < 4
	w=ones(m,1);
end
% Montamos un bucle para crear los elementos s de la matriz de coeficientes,
for j=1:2*n+1
    s(j)=0;    
for i=1:m
    s(j)=s(j)+w(i)*x(i)^(j-1);
end
end


% y un segundo bucle para crear los t�rminos independientes...
for j=1:n+1
    c(j,1)=0;
    for i=1:m
        c(j,1)=c(j,1)+w(i)*x(i)^(j-1)*y(i)
    end
end

%  a partir de los valores de s, construimos la matriz del sistema,
for i=1:n+1
    for j=1:n+1
        A(i,j)=s(i+j-1)
    end
end

% solo nos queda resolver el sistema... Empleamos la division por la
% izquierda de matlab

a=A\c;
\end{lstlisting}

Una �ltima observaci�n importante es que si intentamos calcular el polinomio de m�nimos cuadrados de grado $m-1$ que aproxima un conjunto de $m$ datos, lo que obtendremos ser� el polinomio de interpolaci�n. En general, cuanto mayor sea el grado del polinomio m�s posibilidades hay de que la matriz de sistema empleado para obtener los coeficientes del polinomio est� mal condicionada.

\subsection{M�nimos cuadrados en Matlab.}

Matlab suministra dos maneras distintas de ajustar un polinomio a unos datos por m�nimos cuadrados. La primera es mediante el uso del comando \texttt{polyfit}. Este comando admite como entradas un vector de coordenadas $x$ y otro de coordenadas $y$, de los datos que se quieren aproximar, y una tercera variable con el grado del polinomio. Como resultado devuelve un vector con los coeficientes del polinomio de m�nimos cuadrados. Ordenados en orden de potencias decrecientes \texttt{a(1)*x\^\  n+a(2)*x\^\  (n-1)+...+a(n+1)}. La siguiente secuencia de comandos crea un par de vectores y muestra como manejar el comando,



\begin{verbatim}>> x=[1:10]
x =
     1     2     3     4     5     6     7     8     9    10
>> y=[-1 0 3 4 5.2 6 10 12 13 15.5];\index{Ra�ces de un polinomio}
>> a=polyfit(x,y,3)
a =
    0.0001    0.0411    1.3774   -2.4133
\end{verbatim}
A continuaci�n, podemos emplear el comando \texttt{polyval}, para obtener en valor del polinomio de m�nimos cuadrados obtenido en cualquier punto. En particular, si lo aplicamos a los datos $x$,

\begin{verbatim}
>> yhat=polyval(a,x)
yhat =
  Columns 1 through 7
   -0.9947    0.5067    2.0913    3.7596    5.5121    7.3491    9.2713
  Columns 8 through 10
   11.2790   13.3727   15.5529
\end{verbatim}

Por �ltimo, podemos calcular el error cometido por el polinomio $e_i=\vert p(x_i-y_i \vert$
\begin{verbatim}
>> error=abs(yhat-y)
error =
  Columns 1 through 7
    0.0053    0.5067    0.9087    0.2404    0.3121    1.3491    0.7287
  Columns 8 through 10
    0.7210    0.3727    0.0529
\end{verbatim} 
Adem�s del comando \texttt{polyfit} Matlab permite ajustar un polinomio por m�nimos cuadrados a un conjunto de datos a trav�s de la ventana gr�fica de Matlab. Para ello, es suficiente representar los datos con el comando \texttt{plot(x,y)}. Una vez que Matlab muestra la ventana gr�fica con los datos representados, se selecciona en el men� desplegable \texttt{tools} la opci�n \texttt{Basic Fitting}. Matlab abre una segunda ventana que permite seleccionar el polinomio de m�nimos cuadrados que se desea ajustar a los datos, as� como otras opciones que permiten obtener los coeficientes del polinomio y analizar la bondad del ajuste a partir de los residuos (ver secci�n siguiente). La figuras \ref{fig:minimos}, muestra un ejemplo de uso de la ventana gr�fica de Matlab para obtener un ajuste por m�nimos cuadrados 


\subsection{An�lisis de la bondad de un ajuste por m�nimos cuadrados.} \index{M�nimos cuadrados! Residuos}
Supongamos que tenemos un conjunto de datos obtenidos como resultado de un experimento. En muchos casos la finalidad de un ajuste por m�nimos cuadrados, es encontrar una ley que nos permita relacionar los datos de la variable independiente con la variable dependiente. Por ejemplo si aplicamos distintas fuerzas a muelle y medimos la elongaci�n sufrida por el muelle, esperamos obtener, en primera aproximaci�n una relaci�n lineal: $\Delta x\propto F$. (Ley de Hooke). 

Sin embargo, los resultados de un experimento no se ajustan nunca exactamente a una ley debido a errores aleatorios que no es posible corregir.

Cuando realizamos un ajuste por m�nimos cuadrados, podemos emplear cualquier polinomio desde grado $0$ hasta grado $m-1$. Desde el punto de vista del error cometido con respecto a los datos disponibles el mejor polinomio ser�a precisamente el de grado $m-1$ que da error cero para todos los datos, por tratarse del polinomio de interpolaci�n. Sin embargo, si los datos son experimentales estamos incluyendo los errores experimentales en el ajuste.

Por ello, para datos experimentales y suponiendo que los datos solo contienen errores aleatorios, el mejor ajuste lo dar� el polinomio de menor grado para el cual las diferencias entre los datos y el polinomio $y_i-p(x_i)$ se distribuyan aleatoriamente. Estas diferencias reciben habitualmente el nombre de residuos. \index{Residuos} 

\begin{figure}[h]
\centering
\subfigure[Menu desplegable para abrir ventana auxiliar  \label{fig:minimos1}]{\includegraphics[width=5cm]{minimos.pdf}} \qquad 
\subfigure[ventana auxiliar para realizar el ajuste y analizar resultados  \label{fig:minimos2}]{\includegraphics[width=10cm]{minimos2.pdf}}\\
\caption{Ejemplo de uso de la ventana gr�fica de Matlab para realizar un ajuste por m�nimos cuadrados}
\label{fig:minimos}
\end{figure}


Entre las herramientas que suministra la ventanas gr�ficas de Matlab para el ajuste por m�nimos cuadrados hay una que calcula y representa los residuos.  La figura \ref{fig:residuos} muestra un ejemplo de ajuste por m�nimos cuadrados empleando cada vez un polinomio de mayor grado. \index{Residuos!C�lculo con Matlab}
  
  
En la figura \ref{fig:residuos1} se observa claramente que el ajusto no es bueno, la recta de m�nimos cuadrados no es capaz de adaptarse a la forma que presentan los datos. Los residuos muestran claramente esta tendencia: no est�n distribuidos de forma aleatoria. En la figura \ref{fig:residuos2}, la par�bola aproxima mucho mejor el conjunto de datos, a simple vista parece un buen ajuste. Sin embargo, los residuos presenta una forma funcional clara que recuerda la forma de un polinomio de tercer grado. En la figura \ref{fig:residuos3}, los residuos est�n distribuidos de forma aleatoria. Si comparamos estos resultados con los de la figura \ref{fig:residuos4} vemos que en este �ltimo caso los residuos son m�s peque�os, pero conservan esencialmente la misma distribuci�n aleatoria que en la figura anterior. La aproximaci�n de los datos empleando un polinomio de cuarto grado no a�ade informaci�n sobre la forma de la funci�n que siguen los datos, y ha empezado a incluir en el ajuste los errores de los datos. 

\begin{figure}[h]
\centering
\subfigure[Recta de m�nimos cuadrados y residuos obtenidos \label{fig:residuos1}]{\includegraphics[width=6.5cm]{residuos1.eps}} \qquad 
\subfigure[Parabola de m�nimos cuadrados y residuos obtenidos  \label{fig:residuos2}]{\includegraphics[width=6.5cm]{residuos2.eps}}\\
\subfigure[polinomio de tercer grado  de m�nimos cuadrados y residuos obtenidos \label{fig:residuos3}]{\includegraphics[width=6.5cm]{residuos3.eps}} \qquad 
\subfigure[polinomio de cuarto grado  de m�nimos cuadrados y residuos obtenidos  \label{fig:residuos4}]{\includegraphics[width=6.5cm]{residuos4.eps}}
\caption{Comparaci�n entre los residuos obtenidos para los ajustes de m�nimos cuadrados de un conjunto de datos empleando polinomios de grados 1 a 4.} 
\label{fig:residuos}
\end{figure}
 
\section{Curvas de B�zier}\index{B�zier}\index{Curvas de B�zier}
Las curvas de B�zier permiten unir puntos mediante curvas de forma arbitraria.  Una curva de B�zier viene definida por un conjunto de $n+1$ puntos, $\left\lbrace\vec{p}_0, \vec{p}_2, \cdots, \vec{p}_n\right\rbrace$, que reciben el nombre de puntos de control. El orden en que se presentan los puntos de control es importante para la definici�n de la curva de B�zier correspondiente. �sta pasa siempre por los puntos $\vec{p_0}$ y $\vec{p_n}$. El resto de los puntos de control permiten dar forma a la curva, que tiene siempre la propiedad de ser tangente en el punto $\vec{p}_0$ a la recta que une $\vec{p}_0$ con $\vec{p}_1$ y tangente en el punto $\vec{p}_n$ a la recta que une $\vec{p}_{n-1}$ con $\vec{p}_n$.

Para definir la curva, se asocia a cada punto de control $\vec{p}_i$ una funci�n conocida con el nombre de funci�n de fusi�n. En el caso de las curvas de B�zier, las funciones de fusi�n empleadas son los polinomios de Bernstein, \index{Polinomios de Bernstein}

\begin{equation*}
B_i^n (t)=\binom{n}{i}\left(1-t\right)^{n-i}t^i, \quad i = 0, 1, \dots, n
\end{equation*}

El grado de los polinomios de Bernstein empleados est� asociado al n�mero de puntos de control; para un conjunto de $n+1$ puntos, los polinomios empleados son de grado $n$. La variable $t$ es un par�metro que var�a entre $0$ y $1$.

La ecuaci�n de la curva de B�zier que pasa por un conjunto de $n+1$ puntos de control, $\left\lbrace\vec{p}_0, \vec{p}_2, \cdots, \vec{p}_n\right\rbrace$, se define a partir de los polinomios de Bernstein como,

\begin{equation*}
\vec{p}(t) = \sum_{i = 0}^n B_i^n(t) \cdot \vec{p}_i = \sum_{i = 0}^n \binom{n}{i}\left(1-t\right)^{n-i}t^i  \cdot \vec{p}_i
\end{equation*}

La expresi�n anterior da como resultado $\vec{p}_0$ si hacemos $t = 0$ y $\vec{p}_n$ si hacemos $t = 1$. para los valores comprendidos entre $t=0$ y $t=1$, los puntos $\vec{p}(t)$  trazar�n una curva entre $\vec{p}_0$  y $\vec{p}_n$.

Veamos un ejemplo sencillo. Supongamos que queremos unir los puntos $\vec{p}_0 = (0,1)$ y $\vec{p}_n = (3,1)$ mediante curvas de B�zier. Si no a�adimos ning�n punto m�s de control, $\lbrace\vec{p}_0 = (0,1), \vec{p}_1 = (3,1)\rbrace$, el polinomio de B�zier que obtendremos ser� una recta que unir� los dos puntos,

\begin{equation*}
\vec{p}(t) = \binom{1}{0}\left(1-t\right)^{1}t^0  \cdot (0,1) +  \binom{1}{1}\left(1-t\right)^{0}t^1  \cdot (3,1) = \left(1-t\right)  \cdot (0,1) + t\cdot (3,1)
\end{equation*} 

Si separamos las componentes $x$ e $y$ del vector $\vec{p}(t)$,
\begin{align*}
x &= 3t\\
y &= 1
\end{align*}
Se trata de la ecuaci�n del segmento horizontal que une los puntos $\vec{p}_0 = (0,1)$ y $\vec{p}_n = (3,1)$ 

Si a�adimos un punto m�s de control, por ejemplo:  $\vec{p}_1 = (1,2) \rightarrow \lbrace\vec{p}_0 = (0,1), \vec{p}_1 = (1,2), \vec{p}_2 = (3,1)\rbrace$,  la curva resultante ser� ahora un segmento de un polinomio de segundo grado en la variable $t$,

\begin{align*}
\vec{p}(t) &= \binom{2}{0}\left(1-t\right)^{2}t^0  \cdot (0,1) +  \binom{2}{1}\left(1-t\right)t \cdot (1,2) + \binom{2}{2}\left(1-t\right)^{0}t^2  \cdot (3,1) =\\
&= \left(1-t\right)^{2} \cdot (0,1) + 2 \left(1-t\right)t \cdot (1,2) + t^2  \cdot (3,1) 
\end{align*} 

Seg�n vamos aumentando el n�mero de los puntos de control, iremos empleando polinomios de mayor grado. El segmento de polinomio empleado en cada caso para unir los puntos de control inicial y final variar� dependiendo de los puntos de control intermedios empleados.

La figura \ref{fig:bezier} muestra un ejemplo en el que se han construido varias curvas de B�zier sobre los mismos puntos de paso inicial y final. Es f�cil observar c�mo la forma de la curva depende del n�mero y la posici�n de los puntos de control. Si unimos �stos por orden mediante segmentos rectos, obtenemos un pol�gono que recibe el nombre de pol�gono de control. En la figura \ref{fig:bezier} se han representado los pol�gonos de control mediante l�neas de puntos.

El siguiente c�digo permite calcular y dibujar una curva de B�zier a partir de sus puntos de control.
\begin{lstlisting}
function ber = bezier(p,res)
% estas funcion pinta la curva de Bezier correspondiente a los puntos de 
% control p. 
% p debe ser una matriz de dimension n*2 donde n es el n�mero de puntos de 
% control empleados La primera columna contiene las coordenadas x 
% y la segunda las coordenas y.
% El codigo sigue directamente la definicion de los polinomios de Bernstain
% por tanto, NO es un codigo optimo. Calcula varias veces las mismas
% cantidades.
% Los coeficientes binomiales de los polinomios de Berstein se 
% pueden calcular directamente con el comando de matlab nchoosek. Sin 
% enbargo, en el programa se han calculado a partir de la definicion:
%  n!/((n-k)!k!)
% la variable de entrada res nos da el paso que emplear� el par�matro t
% para calcular y dibujar el polinomio

t = 0:res:1;
ber = zeros(2,length(t));
% calculamos un vector de coeficientes para los polinomios.

% coeficientes para la coordenada x.

n = size(p,1)-1;
num = factorial(n);
for i = 0:n    
    f=num/factorial(i)/factorial(n-i);
    ber(1,:) = ber(1,:) + f*p(i+1,1)*(1-t).^(n-i).*t.^i;
    ber(2,:) = ber(2,:) + f*p(i+1,2)*(1-t).^(n-i).*t.^i;
end
plot(p(:,1),p(:,2),'or')
hold on
plot(p(:,1),p(:,2),'r')
plot(ber(1,:),ber(2,:))
\end{lstlisting}

\begin{figure}[h]
\centering
\includegraphics[width=10cm]{bezier.eps} 
\caption{Curvas de B�zier trazadas entre los puntos $P_0 = (0,0)$ y $P_n = (2,1)$, variando el n�mero y posici�n de los puntos de control.} 
\label{fig:bezier}
\end{figure}

\paragraph{Curvas equivalentes de grado superior.} Dada una curva de B�zier, representada por un polinomio de Bernstein de grado $n$, es posible encontrar polinomios de grado superior que representan la misma curva de B�zier. L�gicamente esto supone un aumento del n�mero de puntos de control.

Supongamos que tenemos una curva de B�zier con cuatro puntos de control,

\begin{equation*}
\vec{p}(t) = \vec{p}_0B_0^3(t) + \vec{p}_1B_1^3(t) + \vec{p}_2B_2^3(t) + \vec{p}_3B_3^3(t)
\end{equation*}

Si multiplicamos este polinomio por $ t + (1 -t) \equiv 1$ El polinomio no var�a y por tanto representa la misma curva de B�zier. Sin embargo, tendr�amos ahora un polinomio un grado superior al inicial.  Si despu�s de la multiplicaci�n agrupamos t�rminos de la forma $(1-t)^{n-1}t^i$, Podr�amos obtener el valor de los nuevos puntos de control,

\begin{align*}
\vec{p}_0^+&= \vec{p}_0\\
\vec{p}_1^+&= \frac{1}{4}\vec{p}_0 + \frac{3}{4}\vec{p}_1\\
\vec{p}_2^+&= \frac{2}{4}\vec{p}_2 + \frac{2}{4}\vec{p}_3\\
\vec{p}_3^+&= \frac{3}{4}\vec{p}_2 + \frac{1}{4}\vec{p}_3\\
\vec{p}_4^+&= \vec{p}_3\\
\end{align*}

y, en general, los puntos de control de la curva de B�zier de grado $n+1$ equivalente  a una dada de grado $n$ puede obtenerse como,

\begin{equation*}
\vec{p}_i^+ = \alpha_i\vec{p}_{i-1} + (1-\alpha_i)\vec{p}_i, \qquad \alpha_i =\frac{i}{n+1}
\end{equation*}

Mediante la ecuaci�n anterior, es posible obtener iterativamente los puntos de control de la curva de B�zier equivalente a una dada de cualquier grado. Una propiedad interesante es que seg�n aumentamos el n�mero de puntos de control empleados, estos y el pol�gono de control correspondiente, van convergiendo a la curva de B�zier. 

La figura \ref{fig:bzgrad} muestra un ejemplo para una curva de B�zier construida a partir de tres puntos de control. Es f�cil ver c�mo a pesar de aumentar el n�mero de puntos de control, la curva resultante es siempre la misma.  Tambi�n es f�cil ver en la figura (\ref{fig:bz4}) la convergencia de los pol�gonos de control hacia la curva.

\begin{figure}[h]
\centering
\subfigure[Curva original (3 puntos) \label{fig:bz1}]{\includegraphics[width=6.8cm]{bezier3p.eps}} \qquad 
\subfigure[Curvas equivalentes de 4 a 6 puntos  \label{fig:bz2}]{\includegraphics[width=6.8cm]{bezier6p.eps}}\\
\subfigure[Curvas equivalentes de 4 a 12 puntos  \label{fig:bz3}]{\includegraphics[width=6.8cm]{bezier12p.eps}} \qquad 
\subfigure[Curvas equivalentes de de 4 a 30 puntos  \label{fig:bz4}]{\includegraphics[width=6.8cm]{bezier30p.eps}}
\caption{Curvas de B�zier equivalentes, construidas a partir de una curva con tres puntos de control} 
\label{fig:bzgrad}
\end{figure}

El siguiente c�digo permite calcular la curva equivalente de B�zier a una dada, para cualquier n�mero de puntos de control que se desee. \index{Curvas de B�zier equivalentes}

\begin{lstlisting}
function  pp = bzeq(p,n)
% Esta funcion calcula los puntos de control de una curva de Bezier
% equivalente de grado superior. p es matriz de tama�o mx2 que contiene 
% los puntos de control originales. n es el n�mero de puntos de paso de 
% la nueva curva de Bezier equivalente. n tiene que ser mayor que m.
% pp contiene los n puntos de control de la curva de B�zier resultante.

% Construimos un vector  con una fila mas para guardar un nuevo elemento
c = size(p,1);
% La siguiente linea sirve solo para ir pintando los resultados y ver 
%que  efectivamente todos los polinomios dan la misma curva de Bezier 
%  (si se tiene la funcion bezier).
bezier(p,0.01);
if c == n
    pp = p;
    return
else
    p = [p;zeros(1,2)];
    pp = p;
    for i = 2 : c+1
        pp(i,:) = p(i-1,:)*(i-1)/(c) + (1 -(i-1)/(c))*p(i,:);
        
    end
    % llamamos a la funcion recursivamente hasta tener n puntos de 
    % control el grado del polinomio sera n-1...
    pp = bzeq(pp,n);
end
\end{lstlisting}

\paragraph{Derivadas.} Las derivadas de una curva de B�zier con respecto al par�metro $t$ son particularmente f�ciles de obtener a partir de los puntos de control. Si tenemos una curva de B�zier de grado $n$, definida mediante puntos de control $\vec{p}_i$ su derivada primera con respecto a $t$ ser� una curva de B�zier de grado $n-1$ Cuyos puntos de control $\vec{d}_i$  puede obtenerse como:
\begin{equation*}
\vec{d_i} = n\left(\vec{p_{i+1}} -\vec{p_i}\right)
\end{equation*}

La nueva curva de B�zier obtenida de este modo, es una hod�grafa; representa el extremo de un vector tangente en cada punto a la curva de B�zier original  y guarda una relaci�n directa con la velocidad a la que se recorrer�a dicha curva. \index{hod�grafa}

La figura \ref{fig:bzc}, muestra una curva de B�zier sobre la que se ha trazado el vector derivada para algunos puntos. La figura \ref{fig:bzd} muestra la hod�grafa correspondiente y de nuevo los mismos vectores derivada de la figura \ref{fig:bzc}

El siguiente c�digo muestra como calcular los puntos de control de la derivada de una curva de B�zier.
\begin{lstlisting}
function d = dbez(p)
% esta funci�n obtiene los puntos de control de la derivada con 
% respecto al parametro de una curva de bezier cuyos puntos de  
% control est�n contenidos la matriz p. p de be ser una matriz de 
% nx2 donde n es el n�mero de puntos de control.
grado = size(p,1) -1
d = zeros(grado,2)
for i = 1:grado
    d(i,:) = (grado-1)*(p(i+1,:)-p(i,:))
end
% Codigo a�adido para dibujar y analizar los resultados
% pintamos la hodografa
v = bezier(d,0.01)
% le a�adimos algunos vectores para que se vea que tocan la hodografa
 hold on
 compass(v(1,1:10:size(v,2)),v(2,1:10:size(v,2)))
 figure
% las pintamos adem�s como vectores tangentes a la curva original
x = bezier(p,0.01)
hold on
quiver(x(1,1:10:size(v,2)),x(2,1:10:size(v,2)),v(1,1:10:size(v,2))...
    ,v(2,1:10:size(v,2)))
\end{lstlisting}

\begin{figure}
\centering
\subfigure[Curva de B�zier (4 puntos) \label{fig:bzc}]{\includegraphics[width=7cm]{bezierder.eps}} \qquad 
\subfigure[Derivada (hod�grafa)  \label{fig:bzd}]{\includegraphics[width=7cm]{bezierhodg.eps}}
\caption{Curva de B�zier y su derivada con respecto al par�metro del polinomio de Bernstein que la define: $t \in [0,1]$} 
\label{fig:bzder}
\end{figure} 

\paragraph{Interpolaci�n con curvas de B�zier} Podemos emplear curvas de B�zier para interpolar un conjunto de puntos $\lbrace \vec{p}_0, \cdots  \vec{p}_m\rbrace$. Si empleamos un curva para interpolar cada par de puntos, $\vec{p}_i, \vec{p}_{i+1}, \qquad i =1, \cdots m-1$ tenemos asegurada la continuidad en los puntos interpolados puesto que las curvas tienen que pasar por ellos. Como en el caso de la interpolaci�n mediante splines, podemos imponer continuidad en las derivadas para conseguir una curva de interpolaci�n suave. En el caso de las curvas de B�zier esto es particularmente simple. Si llamamos $B$ a la curva de B�zier de grado $n$ construida entre los puntos $\vec{p}_{i-1}, \vec{p}_{i}$  con puntos de control, $\vec{p}_{i-1}, b_1, \vec{b}_2,\cdots, \vec{b}_{n-1},\vec{p}_{i}$. Y  $C$ a la curva de B�zier de grado $s$ construida entre los puntos $\vec{p}_{i}, \vec{p}_{i+1}$  con puntos de control, $\vec{p}_{i}, \vec{c}_1, \vec{c}_2,\cdots, \vec{c}_{s-1},\vec{p}_{i+1}$. Para asegurar la continuidad en la primera derivada en el punto $\vec{p_i}$ basta imponer,

\begin{equation*}
n\cdot\left(\vec{p}_i-\vec{b}_{n-1}\right) = s\cdot\left(\vec{c}_1-\vec{p}_i\right)
\end{equation*}

Esta condici�n impone una relaci�n entre el pen�ltimo punto de control de la curva $B$ y el segundo punto de control de la curva $C$. Pero deja completa libertad sobre el resto de los puntos de control elegidos para construir las curvas.

Podemos, por ejemplo, elegir libremente todos los puntos de control de la curva $B$ y obtener a partir de ella el punto $\vec{c}_1$,

\begin{equation*}
\vec{c}_1 = \frac{n+s}{s}\vec{p}_i - \frac{n}{s}\vec{b}_{n-1}
\end{equation*}

\begin{figure}[h]
\centering
\subfigure[Interpolaci�n mediante curvas de B�zier de 3 puntos) \label{fig:ibz}]{\includegraphics[width=7cm]{bezierint.eps}} \qquad 
\subfigure[Interpolaci�n mediante curvas de B�zier de 3 y 4 puntos  \label{fig:ibz2}]{\includegraphics[width=7cm]{bezierint2.eps}}
\caption{Interpolaci�n de tres puntos mediante dos curvas de B�zier} 
\label{fig:ibz3}
\end{figure}
 

La figura \ref{fig:ibz3} muestra un ejemplo de interpolaci�n en la que se ha aplicado la condici�n de continuidad en la derivada que acabamos de describir.

\section{ejercicios}
\begin{enumerate}
\item Carga en Matlab los datos del fichero \texttt{datos.txt}\footnote{disponible en \url{https://github.com/UCM-237/LCC/tree/master/manual\_spa/datos}} y realiza las siguientes tareas:
\begin{enumerate}
\item \label{ej1a} Crea una funci�n que a partir de dos vectores de datos $x,y$ de igual longitud $n+1$, calcula la matriz de Vandermonde necesaria   para obtener el polinomio de interpolaci�n asociado a los puntos. Empleando la primera columna de datos contenida en \texttt{datos.txt} como datos $x$ y la segunda como datos $y$, genera el polinomio de interpolaci�n,
\begin{equation*}
p(x)=a_0+a_1x+a_2x^2+\cdots+a_nx^n
\end{equation*}

Calcula el valor que toma el polinomio de interpolaci�n en 100 puntos equiespaciados entre los valores $x_0$ y $x_{n}$ de los datos del fichero. Dibuja en una misma gr�fica los resultados obtenidos, empleando una l�nea continua, y los valores del fichero, mediante puntos separados empleando el s�mbolo que prefieras. Comprueba que el polinomio pasa por los puntos contenidos en el fichero.

\item Reproduce en matlab la funci�n 'Lagrange' de la secci�n \ref{sec:lagranje} para calcular el polinomio interpolador de Lagrange. Emplea la funci�n que acabas de crear para recalcular los valores del polinomio de interpolaci�n realizado en el ejercicio anterior y comprueba que los resultados obtenidos son los mismos que empleando la matriz de Vandermonde.

\item A partir de los ejemplos de la secci�n  secci�n \ref{sec:difdiv}, crea un programa que calcule los coeficientes del polinomio interpolador de diferencias divididas a partir de dos vectores de datos  $x,y$ de igual longitud $n+1$ y un segundo programa que calcule el valor del polinomio en un punto cualquiera a partir de los coeficientes obtenidos con el primer programa. Vuelve a calcular, empelando ahora el polinomio de diferencias dividas, los valores del polinomio de interpolaci�n sobre los mismos datos empleados en los ejercicios anteriores y comprueba que da los mismos resultados.

\item Por �ltimo, crea una funci�n que calcule el polinomio de interpolaci�n de Newton-Gregory (secci�n \ref{sec:newgre}). �Es posible usarlos para interpolar los datos del archivo \texttt{datos.txt}? Si la respuesta el afirmativa, repite el c�lculo del polinomio de interpolaci�n, empleando Newton-Gregory y comprueba si coincide con lo obtenido en los ejercicios anteriores.

\item Usa el comando \texttt{help} para conocer los distintos m�todos de interpolaci�n por intervalos disponibles  para la funci�n \texttt{interp1}. Empleando de nuevo los datos del fichero \texttt{datos.txt}, obt�n el resultado de interpolar los valores en 100 puntos equiespaciados entre los valores $x_0$ y $x_{n}$ de los datos del fichero mediante \texttt{interp1}.  Emplea para ello los m�todos \texttt{'nearest','linear'} y \texttt{'spline'}. Dibuja los resultados en la misma gr�fica empleada en el ejercicio  \ref{ej1a})
\end{enumerate}
\item Construye a partir del c�digo de ejemplo de la secci�n \ref{sec:mc}, una funci�n que calcule el polinomio de grado $n$ que ajusta por m�nimos cuadrados un conjunto de pares de datos datos $(x, y)$. Pru�balo sobre los datos del fichero \texttt{datos.txt}, sin emplear pesos. Compara los coeficientes del polinomio obtenido con los que se obtienen empleando el comando \texttt{polyfit} de Matlab. �Qu� conclusi�n sacas?

\item A�ade el c�digo necesario al programa anterior para que, una vez obtenidos los coeficientes del polinomio de m�nimos cuadrados, la funci�n calcule y devuelva un vector $r$ con los valores de los residuos, $r = y -p(x)$, donde $x$ e $y$ son los vectores del conjunto de datos para los que se ha obtenido el polinomio de m�nimos cuadrados y $p$ los valores obtenidos aplicando el polinomio a los valores $x$ de la colecci�n de datos.
\end{enumerate}


\section{Test del curso 2020/21}

% El archivo con los datos lo puedes encontrar en
%\url{}

\noindent \textbf{Problema 1}. Una cuerda de escalada aumenta su longitud cuando est� sometida a una tensi�n estacionaria en uno de sus extremos. En particular, el aumento de longitud sigue la siguiente ley
\begin{equation}\label{eq:0}
T = b\tanh(ax),
\end{equation}
donde $T \in \mathbb{R}^+$ es la tensi�n aplicada en Newtons, $x \in \mathbb{R}^+$ es la elongaci�n de la cuerda en metros y $a, b \ \in \mathbb{R}^+$ son par�metros constantes que dependen de las caracter�sticas de la cuerda.

En funci�n del comportamiento f�sico de la cuerda, podemos distinguir tres reg�menes:

\begin{itemize}
	\item \emph{Comportamiento el�stico}: Se dice que el comportamiento de la cuerda es el�stico si al desaparecer la tensi�n la cuerda recupera su longitud original. Esto se cumple para tensiones estacionarias peque�as, y las elongaciones est�n acotadas por un valor positivo $x_{le}$, esto es, $0 \le x \le x_{le}$. En r�gimen el�stico, la ecuaci�n (\ref{eq:0}) puede aproximarse por la siguiente expresi�n
\begin{equation}\label{eq:1}
T = \kappa x - \gamma x^3,
\end{equation}
donde $\kappa,\gamma \in \mathbb{R}^+$ son tambi�n constantes.

\item \emph{Comportamiento pl�stico}: Se dice que el comportamiento de la cuerda es pl�stico si al desaparecer la tensi�n la cuerda se deforma y no recupera su longitud original. Esto ocurre para tensiones medias, y la elongaci�n alcanzada en este r�gimen est� tambi�n acotada por $x_{le} < x \le x_{max}$.

\item \emph{Rotura}: Para tensiones grandes la cuerda no admite elongaciones mayores que $x_{max}$ y se rompe.

\end{itemize}

La f�brica de cuerdas de escalada \textit{Pa'bennos Matao S.L.} ha realizado un estudio sobre un nuevo modelo de cuerda. En dicho estudio se fij� un extremo de la cuerda a una mesa abatible suficientemente grande (que har� la funci�n de un plano inclinado), y se at� el otro extremo de la cuerda a una pesa. De manera secuencial se fue incrementando el �ngulo formado por la mesa con la horizontal. En particular, se empez� con cero grados y aumentando el �ngulo hasta que finalmente la cuerda alcanz� $x_{max}$ y se rompi�. 

Del estudio se pudo registrar la elongaci�n sufrida por la cuerda por los distintos $i\in\{1,\dots,52\}$ �ngulos de inclinaci�n. Estos datos est�n en el archivo:  \texttt{cuerda.txt}\footnote{disponible en \url{https://github.com/UCM-237/LCC/tree/master/manual\_spa/datos}}, en donde:
\begin{itemize}
	\item La primera columna corresponde a los �ngulos $\theta_i$ medidos en radianes.
	\item La segunda columna corresponde a las elongaciones $x_i$ medidas en metros.
\end{itemize}

\begin{enumerate}
\item (\textbf{1 punto}) Estima el valor de la elongaci�n $x_{max}$ para el cual se produce la rotura de la cuerda.

\item (\textbf{1 punto}) Obt�n la tensi�n ejercida por la pesa sobre la cuerda para cada �ngulo de inclinaci�n $\theta_i$ de la mesa, esto es
\begin{equation} \label{eq:2}
T_i = mg\sin(\theta_i),
\end{equation}
donde $m =1000$ Kg y $g = 9.8$ m/s$^2$. Representa gr�ficamente los datos: $T_i$ frente a $x_i$.

\item A partir de los pares de datos $T_i$ y $x_i$ podemos estimar la ecuaci�n (\ref{eq:1}).
\begin{enumerate}
\item (\textbf{2 puntos}) Ajusta los datos por m�nimos cuadrados a un polinomio de grado tres. Dado que dicha aproximaci�n solo es v�lida para el r�gimen de comportamiento el�stico, es imprescindible realizar el ajuste del polinomio asignando pesos a cada par de datos. Para dar m�s valor a las elongaciones peque�as y menos a las grandes, utiliza la siguiente expresi�n para definir los pesos
\begin{equation}
\omega_i = e^{-10x_i^2}.
\end{equation}
\item (\textbf{1 punto})  Representa, sobre la gr�fica dibujada en el apartado 2a, el polinomio obtenido en el apartado 3a. Seg�n tu criterio, �es razonable el ajuste realizado? \\ \textbf{Nota:} Recuerda que el vector de coeficientes, obtenido con el algoritmo de m�nimos cuadrados del manual, est� ordenado al rev�s con respecto a lo que esperan las funciones de Matlab. Puedes reordenarlo a mano o con el comando \texttt{flipud(p)}.
\end{enumerate}

\item (\textbf{1 punto})  Los coeficientes de las ecuaciones (\ref{eq:0}) y (\ref{eq:1}) est�n relacionados por las siguientes expresiones
\begin{equation}
\kappa = b\cdot a, \quad \gamma = \frac{b\cdot a^3}{3}. \nonumber
\end{equation}
		Calcula los valores de $a$ y $b$ a partir de los coeficiente del polinomio obtenido en el apartado 3a. Representa, en la misma gr�fica de los apartados anteriores, la funci�n $T(x) = b\tanh(ax)$. Explica, a la vista del gr�fico, si los resultados obtenidos son razonables o no. 

\item (\textbf{1 punto}) Calcula el valor de los residuos $r_i = T_i - P_3(x_i)$, donde $P_3(x)$ es el polinomio de grado tres obtenido en el apartado 3a. Si la cota $x_{le}$ para el comportamiento el�stico de la cuerda viene definida cuando $r_i \approx 500 N$. Encuentra un valor aproximado para $x_{le}$. \\
\textbf{Nota:} Hay que buscar $x_{le}$ empleando c�digo. No vale dibujar los residuos y estimarlo a vista.
\end{enumerate}

\noindent \textbf{Problema 2.} Dados los siguiente valores de la funci�n $f(x)$:
\begin{equation}
f(0) = 1, \quad f \left(\frac{\pi}{4}\right)= \frac{\sqrt{2}}{2}, \quad f \left(\frac{\pi}{2}\right) = 0, \quad f \left(\frac{3\pi}{4}\right) = -\frac{\sqrt{2}}{2}, \quad f(\pi) = -1 \nonumber
\end{equation}
\begin{enumerate}
\item (\textbf{1.5 puntos}) Utilizando el comando de Matlab correspondiente. Obtener mediante interpolaci�n con splines c�bicos los valores de la funci�n $f(x)$ sobre cien puntos equiespaciados en el intervalo $[0,\pi]$.

\item (\textbf{1.5 puntos}) Dibuja mediante un diagrama de barras los errores cometidos entre los $f(x)$ y la funci�n $\cos(x)$ de Matlab en los mismos puntos del intervalo anterior. Considera que no necesitamos saber m�s de dos decimales para el c�lculo de la funci�n coseno; �podr�amos utilizar la interpolaci�n para $f(x)$?
\end{enumerate}




\chapter{Diferenciaci�n e Integraci�n num�rica}

La diferenciaci�n, y sobre todo la integraci�n son operaciones habituales en c�lculo num�rico. En muchos casos obtener la expresi�n anal�tica de la derivada o la integral de una funci�n puede ser muy complicado o incluso imposible. Adem�s en ocasiones no disponemos de una expresi�n anal�tica para la funci�n que necesitamos integrar o derivar, sino tan solo de un conjunto de valores num�ricos de la misma. Este es el caso, por ejemplo, cuando estamos trabajando con datos experimentales. Si solo disponemos de valores num�ricos, entonces solo podemos calcular la integral o la derivada num�ricamente.

Los sistemas f�sicos se describen generalmente mediante ecuaciones diferenciales.  La mayor�a de las ecuaciones diferenciales no poseen una soluci�n anal�tica, siendo posible �nicamente obtener soluciones num�ricas. 

En t�rminos generales la diferenciaci�n num�rica consiste en aproximar el valor que toma la derivada de una funci�n en un punto. De modo an�logo, la integraci�n num�rica aproxima el valor que toma la integral de una funci�n en un intervalo.

\section{Diferenciaci�n num�rica.}

Como punto de partida, supondremos que tenemos un conjunto de puntos $\{x_i,y_i\}$,
\begin{table}[h]
\centering
\begin{tabular}{c|cccc}
x&$x_0$&$x_1$&$\cdots$&$x_n$\\
\hline
y&$y_0$&$y_1$&$\cdots$&$y_n$
\end{tabular}
\end{table} 

Que pertenecen a una funci�n $y=f(x)$ que podemos o no conocer anal�ticamente. El objetivo de la diferenciaci�n num�rica es estimar el valor de la derivada $f'(x)$ de la funci�n, en alguno de los puntos $x_i$ en los que el valor de $f(x)$ es conocido. 

En general existen dos formas de aproximar la derivada:

\begin{enumerate}
\item Derivando el polinomio de interpolaci�n. De este modo, obtenemos un nuevo polinomio que aproxima la derivada.
\begin{equation*}
f(x)\approx P_n(x) \Rightarrow f'(x) \approx P'_n(x)
\end{equation*}

\item Estimando la derivada como una f�rmula de diferencias finitas obtenida a partir de la aproximaci�n del polinomio de Taylor. 

Si partimos de la definici�n de derivada, 
\begin{equation*}
f'(x_0)=\lim_{h\rightarrow 0}\frac{f(x_0+h)-f(x_0)}{h} \approx \frac{f(x_0+h)-f(x_0)}{h} 
\end{equation*}
\end{enumerate}

Podemos asociar esta aproximaci�n con el polinomio de Taylor de primer orden de la funci�n $f(x)$,

\begin{equation*}
f(x)\approx f(x_0)+f'(x_0)\cdot(x-x_0) \Rightarrow f'(x_0)\approx \frac{f(x)-f(x_0)}{x-x_0}
\end{equation*}

Si hacemos $x-x_0=h$, ambas expresiones coinciden.
 
En general, los algoritmos de diferenciaci�n num�rica son inestables. Los errores iniciales que puedan contener los datos debido a factores experimentales o al redondeo del ordenador, aumentan en el proceso de diferenciaci�n. Por eso no se pueden calcular derivadas de orden alto y, los resultados obtenidos de la diferenciaci�n num�rica deben tomarse siempre extremando la precauci�n.

\subsection{Diferenciaci�n num�rica basada en el polinomio de interpolaci�n.}\index{Diferenciaci�n! polinomio interpolador}

El m�todo consiste en derivar el polinomio $P_n(x)$ de interpolaci�n obtenido por alguno de los m�todos estudiados en el cap�tulo \ref{interpolacion} y evaluar el polinomio derivada $P'_n(x)$ en el punto deseado.

Un ejemplo particularmente sencillo, para la expresi�n del polinomio derivada se obtiene  en el caso de datos equidistantes interpolados mediante el polinomio de Newton-Gregory,

\begin{equation*}
p_n(x)=y_0+\frac{x-x_0}{h}\Delta y_0+\frac{(x-x_1)\cdot (x-x_0)}{2\cdot h^2}\Delta^2 y_0+\cdots +\frac{(x-x_{n-1}) \cdots (x-x_1)\cdot (x-x_0)}{n! \cdot h^n}\Delta^n y_0
\end{equation*}

Si lo derivamos, obtenemos un nuevo polinomio,

\begin{align*}
p'_n(x)&=\frac{\Delta y_0}{h}+\frac{\Delta^2 y_0}{2\cdot h^2}\left[(x-x_1)+(x-x_0) \right] +\\
&+\frac{\Delta^3 y_0}{3! \cdot h^3}\left[(x-x_1)(x-x_2)+(x-x_0)(x-x_1)+(x-x_0)(x-x_2)\right]+\cdots +\\
&+\frac{\Delta^n y_0}{n! \cdot h^n}\sum_{k=0}^{n-1}\frac{(x-x_0)(x-x_1)\cdots (x-x_{n-1})}{x-x_k}
\end{align*}

Este polinomio es especialmente simple de evaluar en el punto $x_0$,

\begin{align*}
p'_n(x_0)&=\frac{\Delta y_0}{h}+\frac{\Delta^2 y_0}{2\cdot h^2}\overbrace{(x_0-x_1)}^{-h}+\cdots +
\frac{\Delta^n y_0}{n! \cdot h^n}[\overbrace{(x_0-x_1)}^{-h}\overbrace{(x_0-x_2)}^{-2h}\cdots \overbrace{(x_0-x_{n-1})}^{-(n-1)h}]\\
p'_n(x_0)&=\frac{1}{h}\left(\Delta y_0-\frac{\Delta^2 y_0}{2}+\frac{\Delta^3 y_0}{3}+\cdots +
\frac{\Delta^n y_0}{n}(-1)^{n-1}\right)
\end{align*}

Es interesante remarcar como en la expresi�n anterior, el valor de la derivada se va haciendo m�s preciso a medida que vamos a�adiendo diferencias de orden superior. Si solo conoci�ramos dos datos, $(x_0, y_0)$ y $(x_1, y_1)$. Solo podr�amos calcular la diferencia dividida de primer orden. En esta caso nuestro c�lculo aproximado de la derivada de $x_0$ ser�a,
\begin{equation*}
p'_1(x_0)=\frac{1}{h}\Delta y_0
\end{equation*}

Si conocemos tres datos, podr�amos calcular $\Delta^2y_0$ y a�adir un segundo t�rmino a nuestra estima de la derivada,
\begin{equation*}
p'_2(x_0)=\frac{1}{h}\left(\Delta y_0-\frac{\Delta^2 y_0}{2}\right)
\end{equation*}

y as� sucesivamente, mejorando cada vez m�s la precisi�n.

Veamos como ejemplo el c�lculo la derivada en el punto $x_0=0.0$, a partir de la siguiente tabla de datos,

\begin{table}[h]
\centering
\begin{tabular}{cccccc}
$x_i$&$y_i$&$\Delta y_i$&$\Delta^2 y_i$&$\Delta^3 y_i$ & $\Delta^4 y_i$\\
\hline
$0.0$& $0.000$& $0.203$ &$0.017$ &$0.024$ &$0.020$\\
$0.2$ &$0.203$ &$0.220$ &$0.041$ &$0.044$\\
$0.4$ &$0.423$ &$0.261$ &$0.085$\\
$0.6$ &$0.684$ &$0.346$\\
$0.8$ &$1.030$\\
\end{tabular}
\end{table} 

\begin{itemize}
\item Empleando los dos primeros puntos,
\begin{equation*}
y'(0,0)=p_1^1(0.0)=\frac{1}{0.2}\cdot 0.203= 1.015
\end{equation*}
\item Empleando los tres primeros puntos,
\begin{equation*}
y'(0,0)=p_2^1(0.0)=\frac{1}{0.2}\left( 0.203-\frac{0.017}{2}\right)= 0.9725
\end{equation*}
\item Empleando los cuatro primeros puntos,
\begin{equation*}
y'(0,0)=p_3^1(0.0)=\frac{1}{0.2}\left( 0.203-\frac{0.017}{2}+\frac{0.024}{3}\right)= 1.0125
\end{equation*}
\item Empleando los cinco puntos disponibles
\begin{equation*}
y'(0,0)=p_4^1(0.0)=\frac{1}{0.2}\left( 0.203-\frac{0.017}{2}+\frac{0.024}{3}-\frac{0.020}{4}\right)= 0.9875
\end{equation*}
\end{itemize}

\subsection{Diferenciaci�n num�rica basada en diferencias finitas}
\index{Diferenciaci�n! diferencias finitas}
Como se explic� en la introducci�n, la idea es emplear el desarrollo de Taylor para aproximar la derivada de una funci�n en punto. Si empezamos con el ejemplo m�s sencillo, podemos aproximar la derivada suprimiendo de su definici�n el \emph{paso al l�mite}\index{Diferenciaci�n!Diferencia adelantada de dos puntos},

\begin{equation*}
f'(x_k)=\lim_{h\rightarrow 0}\frac{f(x_k+h)-f(x_k)}{h} \approx \frac{f(x_k+h)-f(x_k)}{h} 
\end{equation*}

La expresi�n obtenida, se conoce con el nombre de formula de diferenciaci�n adelantada de dos puntos. El error cometido debido a la elecci�n de un valor de $h$ finito, se conoce con el nombre de error de truncamiento\index{Error de Truncamiento}. Es evidente que desde un punto de vista puramente matem�tico, la aproximaci�n es mejor cuanto menor es $h$. Sin embargo, desde un punto de vista num�rico esto no es as�. A medida que hacemos m�s peque�o el valor de $h$, aumenta el error de redondeo debido a la aritm�tica finita del computador\index{Error de redondeo}. Por tanto, el error cometido es la suma de ambos errores,
\begin{equation*}
f'(x)=\frac{f(x+h)-f(x)}{h}+\overbrace{C\cdot h}^{\mathtt{e.\ de\  truncamiento}}+\underbrace{D\cdot \frac{1}{h}}_{\mathtt{e.\ de\  redondeo}}, \ C>>D 
\end{equation*}

El valor �ptimo de $h$ es aquel que hace m�nima la suma del error de redondeo y el error de truncamiento. La figura \ref{fig:errores} muestra de modo esquem�tico como domina un error u otro seg�n hacemos crecer o decrecer el valor de $h$ en torno a su valor �ptimo.

\begin{figure}[h]
\centering
\includegraphics[width=12cm]{errores.eps}
\caption{Variaci�n del error cometido al aproximar la derivada de una funci�n empleando una f�rmula de diferenciaci�n de dos puntos.} 
\label{fig:errores}
\end{figure}

Como vimos en la introducci�n a esta secci�n, partiendo de el desarrollo de Taylor de una funci�n es posible obtener f�rmulas de diferenciaci�n num�rica y poder estimar el error cometido. As� por ejemplo, a  partir del polinomio de Taylor de primer orden,
\begin{equation*}
f(x+h)=f(x)+hf'(x)+\frac{h^2}{2}f''(z) \Rightarrow f'(x)=\frac{f(x+h)-f(x)}{h}-\frac{h}{2}f''(z), \ x<z<x+h
\end{equation*}

El error que se comete debido a la aproximaci�n, es proporcional al tama�o del intervalo empleado $h$ La constante de proporcionalidad depende de la derivada segunda de la funci�n, $f''(z)$ en alg�n punto indeterminado ($z\in x,x+h$). Para indicar esta relaci�n lineal entre el error cometido y el valor de $h$, se dice que el error es del \emph{orden} de $h$ y se representa como $O(h)$.
\begin{equation*}
f'(x)=\frac{f(x+h)-f(x)}{h}+O(h)
\end{equation*}

\begin{figure}[h]
\centering
\subfigure[Diferencia de dos puntos adelantada \label{fig:diffin1}]{\includegraphics[width=7cm]{diffin1.eps}} \qquad 
\subfigure[Diferencia de dos puntos centrada \label{fig:diffin2}]{\includegraphics[width=7cm]{diffin2.eps}}
\caption{Comparaci�n entre las aproximaciones a la derivada de una funci�n obgtenidas mediante las diferencias de dos puntos adelantada y centrada} 
\label{fig:diffin}
\end{figure}

Podemos mejorar la aproximaci�n, calculando el valor de polinomio de Taylor de tercer orden para dos puntos equidistantes situados a la izquierda y la derecha del punto x, restando dichas expresiones y despejando la derivada primera del resultado,

\begin{equation*}
\left. \begin{aligned}
f(x+h)=f(x)+hf'(x)+\frac{h^2}{2}f''(x)+\frac{h^3}{3!}f'''(z)\\
f(x-h)=f(x)-hf'(x)+\frac{h^2}{2}f''(x)-\frac{h^3}{3!}f'''(z)
\end{aligned} \right\rbrace \Rightarrow f'(x)=\frac{f(x+h)-f(x-h)}{2h}-\frac{h^2}{6}f'''(z) 
\end{equation*}

En esta caso, el error es proporcional al cuadrado de $h$, por tanto,

\begin{equation*}
f'(x_0)=\frac{f(x_1)-f(x_{-1})}
{2h}+O(h^2)
\end{equation*}

Donde hemos hecho $x\equiv x_0$, $x+h\equiv x_1$ y $x-h\equiv x_{-1}$. Esta aproximaci�n recibe el nombre de diferencia de dos puntos centrada.\index{Diferenciaci�n! Diferencia de dos puntos centrada} La figura \ref{fig:diffin1} muestra una comparaci�n entre la derivada real de una funci�n y su aproximaci�n mediante una diferencia adelantada de dos puntos. La figura \ref{fig:diffin2} muestra la misma comparaci�n empleando esta vez la aproximaci�n de dos punto centrada. En este ejemplo es f�cil ver como la aproximaci�n centrada da un mejor resultado. No hay que olvidar que  la bondad del resultado, para un valor de $h$ dado, depende tambi�n del valor de las derivadas de orden superior de la funci�n, por lo que no es posible asegurar que el resultado de la diferencia centrada sea siempre mejor.

Empleando el desarrollo de Taylor y tres puntos podemos aproximar la derivada por la diferencia de tres puntos adelantada,

\begin{align*}
\left. \begin{aligned}
4\cdot\left(f(x_1)=f(x_0)+hf'(x_0)+\frac{h^2}{2}f''(x_0)+\frac{h^3}{3!}f'''(z)\right)\\
f(x_2)=f(x_0)+2hf'(x_0)+2{h^2}{2}f''(x_0)+\frac{4h^3}{3}f'''(z)
\end{aligned} \right\rbrace \Rightarrow \\
\Rightarrow f'(x_0)=\frac{-f(x_2)+4f(x_1)-3f(x_0)}{2h}-\frac{h^2}{3}f'''(z) 
\end{align*} 

En este caso e error es tambi�n de  orden $h^2$ pero vale el doble que para la diferencia de dos puntos centrada.

A partir del desarrollo de Taylor y mediante el uso del n�mero de puntos adecuado, es posible obtener aproximaciones a la derivada primera y a las sucesivas derivadas de una funci�n, procediendo de modo an�logo a como acaba de mostrarse para el caso de las diferencias de dos y tres puntos. La tabla \ref{tabdiff} muestra algunas de las f�rmulas de derivaci�n mediante diferencias finitas m�s empleadas. Para simplificar la notaci�n, en todos los casos se ha tomado $y_i=f(x_i), \ y_i^{(j)}=f^{(j)}(x_i)$.

\begin{table}[h]
\centering
\begin{tabular}{|c|c|}
\hline
F�rmulas primera derivada & F�rmulas segunda derivada\\
\hline
\ & \\
$y'_0=\frac{y_1-y_0}{h}+O(h)$ & $y''_0=\frac{y_2-2y_1+y_0}{h^2}+O(h)$\\
\ & \\
$y'_0=\frac{y_1-y_{-1}}{2h}+O(h^2)$ &$y''_0=\frac{y_1-2y_0+y_{-1}}{h^2}$\\
\ & \\
$y'_0=\frac{-y_2+4y_1-3y_0}{2h}+O(h^2)$ &$y''_0=\frac{-y_3+y_2-5y_1+2y_0}{h^2}+O(h^2)$\\
\ & \\
$y'_0=\frac{-y_2+8y_1-8y_{-1}+y_{-2}}{12h}+O(h^4)$ &$y''_0=\frac{-y_2+16y_1-30y_0+16y_{-1}-y_{-2}}{12h^2}+O(h^4)$\\
\ & \\
\hline
F�rmulas tercera derivada & F�rmulas cuarta derivada\\
\hline
\ & \\
$y'''_0=\frac{y_3-3y_2+3y_1-y_0}{h^3}+O(h)$ &$y^{iv}_0=\frac{y_4-4y_3+6y_2-4y_1+y_0}{h^4}+O(h)$\\
\ & \\
$y'''_0=\frac{y_2-2y_1+2yx_{-1}-y_{-2}}{2h^3}+O(h^2)$ &$y^{iv}_0=\frac{y_2-4y_1+6y_0-4y_{-1}+y_{-2}}{h^4}+O(h^2)$\\
\ &  \\
\hline
\end{tabular}
\caption{F�rmulas de derivaci�n basadas en diferencias finitas}
\label{tabdiff}
\end{table}


\section{Integraci�n num�rica.}\label{sec:inum}

Dada una funci�n arbitraria $f(x)$ es en muchos casos posible obtener de modo anal�tico su primitiva $F(x)$ de modo que $f(x)=F'(x)$. En estos casos, la integral definida de $f(x)$ en un intervalo $[a,b]$ puede obtenerse directamente a partir de su primitiva,
\begin{equation*}
I(f)=\int_a^bf(x)dx=F(x)\vert_a^b=F(b)-F(a)
\end{equation*}
Hay sin embargo muchos casos en los cuales se desconoce la funci�n $F(x)$ y otros en los que ni siquiera se conoce la expresi�n de la funci�n $f(x)$, como por ejemplo, cuando solo se dispone de una tabla de valores $\left\lbrace x_i,y_i=f(x_i) \right\rbrace$ para representar la funci�n. En estos casos se puede aproximar la integral definida de la funci�n $f(x)$ en un intervalo $[a,b]$, a partir de los puntos disponibles, mediante lo que se conoce con el nombre de una f�rmula de cuadratura,

\begin{equation*}
I(f)=\int_a^bf(x)dx \approx \sum_{i=0}^nA_if(x_i)
\end{equation*}

Una t�cnica habitual de obtener los coeficientes $A_i$, es hacerlo de modo impl�cito a partir de la integraci�n  de los polinomios de interpolaci�n,

 \begin{equation*}
I(f)=\int_a^bf(x)dx \approx \int_a^bP_n(x)dx
\end{equation*}

Para ello, se identifican los extremos del intervalo de integraci�n con el primer y el �ltimo de los datos disponibles, $[a,b]\equiv [x_0,x_n]$.

As� por ejemplo, a partir de los polinomios de Lagrange, definidos en la secci�n \ref{sec:lagranje},

\begin{equation*}
p(x)=\sum_{j=0}^n l_j(x)\cdot y_j
\end{equation*}

Podemos obtener los coeficientes $A_i$ como,

\begin{equation*}
I(f)=\int_a^bf(x)dx \approx \int_a^bP_n(x)dx=\int_{x_0}^{x_n}\left( \sum_{j=0}^n l_j(x)\cdot y_j \right) dx \Rightarrow A_j=\int_{x_0}^{x_n}l_j(x)dx
\end{equation*}

La familia de m�todos de integraci�n, conocidas como f�rmulas de Newton-Cotes, puede obtenerse a partir del polinomio de interpolaci�n de Newton-Gregory descrito en la secci�n \ref{sec:newgre}\index{Integraci�n! Formulas de Newton-Cotes}. Supongamos que tenemos la funci�n a integrar definida a partir de un conjunto de puntos equiespaciados a lo largo del intervalo de integraci�n $\{(x_i,y_i)\}_{0,\cdots n}$. Podemos aproximar la integral $I(y)$ como,

\begin{equation*}
\int_{x_0}^{x_n}ydx \approx \int_{x_0}^{x_n}\left(y_0+\frac{x-x_0}{h}\Delta y_0+\frac{(x-x_0)(x-x_1)}{2!h^2}\Delta ^2 y_0+\cdots +\frac{(x-x_0)\cdots(x-x_{n-1})}{n!h^n}\Delta^n y_0 \right)
\end{equation*}

Las f�rmulas de Newton-Cotes se asocian con el grado del polinomio de interpolaci�n empleado en su obtenci�n:

\begin{itemize}
\item Para $n=1$, se obtiene la regla del trapecio,
\begin{equation*}
I(y)=\int_{x_0}^{x_1}ydx \approx \frac{h}{2}(y_0+y_1)
\end{equation*}
\item Para $n=2$, se obtiene la regla de Simpson 
\begin{equation*}
I(y)=\int_{x_0}^{x_2}ydx \approx \frac{h}{3}(y_0+4y_1+y_2)
\end{equation*}
\item Para $n=3$, se obtiene la regla de 3/8 de Simpson
\begin{equation*}
I(y)=\int_{x_0}^{x_3}ydx \approx \frac{3h}{8}(y_0+3y_1+3y_2+y_3)
\end{equation*}
\end{itemize}

No se suelen emplear polinomios de interpolaci�n de mayor grado debido a los errores de redondeo y a las oscilaciones locales que dichos polinomios presentan.
 
\subsection{La f�rmula del trapecio.}

La f�rmula del trapecio emplea tan solo dos puntos para obtener la integral de la funci�n en el intervalo definido por ellos. \index{Integraci�n! F�rmula del trapecio}

\begin{equation*}
I(y)=\int_{x_0}^{x_1}ydx \approx \int_{x_0}^{x_1}\left(y_0+\frac{x-x_0}{h}\Delta y_0\right)dx =y_0x+\left. \frac{\Delta y_0}{h}\frac{(x-x_0)^2}{2} \right\rvert_{x_0}^{x_1}=\frac{h}{2}(y_0+y_1)
\end{equation*}

La figura \ref{fig:trapecio} muestra gr�ficamente el resultado de aproximar la integral definida de una funci�n $y=f(x)$ mediante la f�rmula del trapecio. Gr�ficamente la integral coincide con el �rea del \emph{trapecio} formado por los puntos $(x_0,0)$, $(x_0,y_0)$, $(x_1,y_1)$ y $(x_1,0)$.  De ah� su nombre y la expresi�n matem�tica obtenida,

\begin{equation*}
I(y)=\frac{h}{2}(y_0+y_1)
\end{equation*}

que coincide con el �rea del trapecio mostrado en la figura.
\begin{figure}[h]
\centering
\includegraphics[width=9cm]{trapecio.eps}
\caption{Interpretaci�n gr�fica de la f�rmula del trapecio.} 
\label{fig:trapecio}
\end{figure}

\paragraph*{Formula extendida (o compuesta) del trapecio.}\index{Integraci�n! F�rmula compuesta del trapecio} La figura \ref{fig:trapecio} permite observar la diferencia entre el �rea calculada y el �rea comprendida entre la curva real y el eje $x$. Como se ha aproximado  la curva en el intervalo de integraci�n por un l�nea recta (polinomio de grado 1 $p_1$), El error ser� tanto mayor cuando mayor sea el intervalo de integraci�n y/o la variaci�n de la funci�n en dicho intervalo. Una soluci�n a este problema, si se conoce la expresi�n anal�tica de la funci�n que  se desea integrar o se conocen suficientes puntos es subdividir el intervalo de integraci�n en intervalos m�s peque�os y aplicar a cada uno de ellos la f�rmula de trapecio,

\begin{equation*}
I(y)=\int_{x_0}^{x_n}ydx \approx \sum_{i=0}^{n-1}\int_{x_i}^{x_{i+1}}y(x)dx=\sum_{i=0}^{n-1}\frac{h}{2}(y_i+y_{i+1})=\frac{h}{2}\left(y_0+2y_1+2y_2+\cdots+2y_{n-1}+y_n \right)
\end{equation*}

La figura \ref{fig:trapecio2}, muestra el resultado de aplicar la f�rmula extendida del trapecio a la misma funci�n de la figura \ref{fig:trapecio}. En este caso, se ha dividido el intervalo de integraci�n en cuatro subintervalos. Es inmediato observar a partir de la figura, que la aproximaci�n mejorar� progresivamente si se aumenta el n�mero de subintervalos y se reduce el tama�o de los mismos.

\begin{figure}[h]
\centering
\includegraphics[width=9cm]{trapecio2.eps}
\caption{Interpretaci�n gr�fica de la f�rmula extendida del trapecio.} 
\label{fig:trapecio2}
\end{figure}

\section{Las f�rmulas de Simpson.}
Se conocen con el nombre de f�rmulas integrales de Simpson, a las aproximaciones a la integral definida obtenida a partir de los polinomios interpoladores de Newton-Gregory de grado dos (Simpson $1/3$) y de grado tres (Simpson $3/8$) . \index{Integraci�n! F�rmulas de Simpson}

En el primer caso, es preciso conocer tres valores equiespaciados de la funci�n en el intervalo de integraci�n y en el segundo es preciso conocer cuatro puntos. 

\paragraph*{F�rmula de Simpson $1/3$.} La f�rmula de Simpson, o Simpson $1/3$,  emplea un polinomio de interpolaci�n de Newton-Gregory de grado dos para  obtener la aproximaci�n a la integral,

\begin{equation*}
I(y)\approx \int_{x_0}^{x_2}P_2(x)dx=\int_{x_0}^{x_2}\left(y_0+\frac{x-x_0}{h}\Delta y_0+\frac{(x-x_0)\cdot(x-x_1)}{2h^2}\Delta^2 y_0 \right)dx= \frac{h}{3}(y_0+4y_1+y_2)
\end{equation*}

La figura \ref{fig:simpsom}, muestra gr�ficamente el resultado de aplicar el m�todo de Simpson a la misma funci�n de los ejemplos anteriores.  De nuevo, la bondad de la aproximaci�n depende de lo que var�e la funci�n en el intervalo.  La diferencia fundamental con el m�todo del trapecio es que ahora el �rea calculada esta limitada por el segmento de par�bola definido por el polinomio de interpolaci�n empleado.

\begin{figure}[h]
\centering
\includegraphics[width=8cm]{simpsom.eps}
\caption{Interpretaci�n gr�fica de la f�rmula $1/3$ de Simpson.} 
\label{fig:simpsom}
\end{figure}

\paragraph*{F�rmula de Simpson $3/8$.} En este caso, se emplea un polinomio de Newton-Gregory de grado 3 para obtener la aproximaci�n a la integral,

\begin{align*}
I(y)&\approx \int_{x_0}^{x_3}P_2(x)dx=\\
&=\int_{x_0}^{x_3}\left(y_0+\frac{x-x_0}{h}\Delta y_0+\frac{(x-x_0)\cdot(x-x_1)}{2h^2}\Delta^2 y_0 +\frac{(x-x_0)\cdot(x-x_1)\cdot (x-x_2)}{3!h^3}\Delta^3 y_0\right)dx\\
&= \frac{3h}{8}(y_0+3y_1+3y_2+y_3)
\end{align*}

La figura  \ref{fig:simpsom38} muestra el resultado de aplicar la f�rmula de Simpson $3/8$ a la misma funci�n de los ejemplos anteriores. En este caso, la integral ser�a exacta porque la funci�n de ejemplo elegida es un polinomio de tercer grado y coincide exactamente con el polinomio de interpolaci�n construido para obtener la integral.

\begin{figure}[h]
\centering
\includegraphics[width=9cm]{simpsom38.eps}
\caption{Interpretaci�n gr�fica de la f�rmula $3/8$ de Simpson.} 
\label{fig:simpsom38}
\end{figure}

Al igual que en el caso del m�todo del trapecio, lo normal no es aplicar los m�todos de Simpson a todo el intervalo de integraci�n, sino dividirlo en subintervalos m�s peque�os y aplicar el m�todo sobre dichos subintervalos. El resultado se conoce como m�todos extendidos de Simpson. Al igual que sucede con la f�rmula del trapecio, los m�todos extendidos de Simpson mejoran la aproximaci�n obtenida para la integral tanto m�s cuanto m�s peque�o es el tama�o de los subintervalos empleados. 

As�, la f�rmula extendida de Simpson $1/3$ toma la forma,
 
\begin{align*}
	I(y)&\approx \sum_{i=0}^{\frac{n}{2}-1}\int_{x_{2i}}^{x_{2i+2}}P_2(x)dx=\sum_{i=0}^{\frac{n}{2}-1}\frac{h}{3}(y_{2i}+4y_{2i+1}+y_{2i+2})\\
&= \frac{h}{3}(y_0+4y_1+2y_2+4y_3+2y_4+\cdots + 2y_{n-2}+4y_{n-1}+y_n)
\end{align*}

Donde se ha dividido el intervalo de integraci�n en $n$ subintervalos, la f�rmula de Simpson se ha calculado para cada dos subintevalos y se han sumado los resultados.

Por �ltimo para la f�rmula extendida de Simpson $3/8$ se puede emplear la expresi�n,

\begin{align*}
	I(y)&\approx \sum_{i=0}^{\frac{n}{3}-1}\int_{x_{3i}}^{x_{3i+3}}P_3(x)dx=\sum_{i=0}^{\frac{n}{3}-1}\frac{3h}{8}(y_{3i}+3y_{3i+1}+3y_{3i+2}+y_{3i+3})\\
&= \frac{3h}{8}(y_0+3y_1+3y_2+2y_3+3y_4+3y_5+2y_6+ \cdots + 2y_{n-3}+3y_{n-2}+3y_{n-1}+y_n)
\end{align*}

En este caso tambi�n se divide el intervalo en $n$ subintervalos pero ahora se ha aplicado la regla de Simpson $3/8$ a cada tres subintevalos.

A continuaci�n se incluye un c�digo que permite permite aproximar la integral definida de una funci�n en un intervalo por cualquiera de los tres m�todos descritos: Trapecio, Simpson o Simpson $3/8$,

\begin{lstlisting}
function int=integra(fun,met,inter,dib)
% implementa los metodos del trapecio, etc...
% Uso: int=integra(fun,inter,dib)
% fun, nombre de la funci�n que se desea intergrar, debe ir entre comillas
% met, m�todo que se desea emplear para calcular la integral. los nombres
% validos para los metodos son: 'trapecio', 'simpson' y simpsom38'
% inter, es un vector de dos elementos que contiene lo extremos de intervalo
% de integracion [a,b]
% dib, si se da un valor a esta variable de entrada, dibuja la funci�n en el
% intervalo de integraci�n y la integral obtenida. Si se omite no dibuja
% nada.
% int, valor de la integral obtenida.
% Este programa no divide el intervalo de integraci�n en subintervalos...
% para calcular la integral empleando subintervalos usar la funci�n trocea,

if strcmp(met,'trapecio')
    f0=feval(fun,inter(1));
    f1=feval(fun,inter(2));
    h=inter(2)-inter(1);
    int=h*(f0+f1)/2;
    if nargin==4
        x=inter(1):(inter(2)-inter(1))/100:inter(2);
        yfuncion=feval(fun,x);
        ypolinomio=f0+(x-inter(1))*(f1-f0)/h;
        plot(x,yfuncion,x,ypolinomio,'r')
        hold on
        fill([x(1) x x(length(x))],[0 ypolinomio 0],'r') 
    end
elseif strcmp(met,'simpson')
    pmedio=(inter(2)+inter(1))/2;
    f0=feval(fun,inter(1));
    f1=feval(fun,pmedio);
    f2=feval(fun,inter(2));
    h=(inter(2)-inter(1))./2;
    int=h*(f0+4*f1+f2)/3;
    if nargin==4
        x=inter(1):(inter(2)-inter(1))/100:inter(2);
        yfuncion=feval(fun,x);
        ypolinomio=f0+(x-inter(1)).*(f1-f0)/h+(x-inter(1)).*(x-pmedio).*(f2...
        -2*f1+f0)./(2*h^2);
        plot(x,yfuncion,x,ypolinomio,'r')
        hold on
        fill([x(1) x x(length(x))],[0 ypolinomio 0],'r') 
    end        
elseif strcmp(met,'simpsom38')
    inter=inter(1):(inter(2)-inter(1))/3:inter(2);

    f=feval(fun,inter);
    h=(inter(4)-inter(1))/3;
    int=3*h*(f(1)+3*f(2)+3*f(3)+f(4))/8;
    if nargin==4
        x=inter(1):(inter(4)-inter(1))/100:inter(4);
        yfuncion=feval(fun,x);
        ypolinomio=f(1)+(x-inter(1)).*(f(2)-f(1))/h+(x-inter(1)).*(x...
        -inter(2)).*(f(3)-2*f(2)+f(1))./(2*h^2)+(x-inter(1)).*(x...
        -inter(2)).*(x-inter(3)).*(f(4)-3*f(3)+3*f(2)-f(1))./(6*h^3);
        plot(x,yfuncion,x,ypolinomio,'r')
        hold on
        fill([x(1) x x(length(x))],[0 ypolinomio 0],'r') 
    end
else
    int='metodo desconocido, los metodos conocidos son ''trapecio'', ...
    ''simpson' y ''simpsom38'''
end
\end{lstlisting}

Este programa aplica directamente el m�todo deseado sobre el intervalo de integraci�n. Para obtener los m�todos extendidos, podemos emplear un segundo programa que divida el intervalo de integraci�n inicial en el n�mero de subintervalos que deseemos, aplique el programa anterior a cada subintervalo, y, por �ltimo, sume todo los resultados para obtener el valor de la integral en el intervalo deseado. El siguiente programa, muestra un ejemplo de como hacerlo,

\begin{lstlisting}
function total=trocea(fun,met,inter,div,dib)
% esta funcion lo unico que hace es trocear un intervalo en el numero de
% tramos indicados por div, y llamar a la funcion integra para que integre
% en cada intervalo.
% USO: total=trocea(fun,met,inter,div,dib)
%  las variables de emtrada son la mismas que las de integra, salvo div que
%  representa el numero de subintervalos en que se desea dividir el
%  intervalo de integraci�n.
tramos=inter(1):(inter(2)-inter(1))/div:inter(2);
total=0;
if nargin==5
    hold on
end
for i=1:size(tramos,2)-1
    int=integra(fun,met,[tramos(i) tramos(i+1)],dib);
    total=total+int;
end
hold off
\end{lstlisting}t

Por �ltimo indicar que Matlab posee un comando propio para calcular la integral definida de una funci�n, el comando \texttt{quad}. Este comando admite como variables de entrada el nombre de una funci�n, y dos valores que representan los l�mites de integraci�n. Como variable de salida devuelve el valor de la integral definida en el intervalo introducido. Por ejemplo,

\begin{verbatim}
>> i=integral(@sin,0,pi)

i =

    2.0000
\end{verbatim}


\section{Problemas de valor inicial en ecuaciones diferenciales}
Las leyes de la f�sica est�n escritas en forma de ecuaciones diferenciales. \index{Ecuaci�n diferencial} \index{Problemas de valor inicial}

Una ecuaci�n diferencial establece una relaci�n matem�tica entre una variable y sus derivadas respecto a otra u otras variables de las que depende. El ejemplo m�s sencillo lo encontramos en las ecuaciones de la din�mica en una sola dimensi�n que relacionan la derivada segunda de la posici�n de un cuerpo con respecto al tiempo, con la fuerza que act�a sobre el mismo.

\begin{equation*}
m\cdot \frac{d^2x}{dt^2}=F
\end{equation*}

Si la fuerza es constante, o conocemos expl�citamente como var�a con el tiempo, podemos integrar la ecuaci�n anterior para obtener la derivada primera de la posici�n con respecto al tiempo ---la velocidad--- de una forma directa,

\begin{equation*}
m\cdot \frac{d^2x}{dt^2}=F \rightarrow v(t)=\frac{dx}{dt}=\int \frac{F(t)}{m}dt +v(0)
\end{equation*}

Donde suponemos conocido el valor $v(0)$ de la velocidad del cuerpo en el instante inicial.

Si volvemos a integrar ahora la expresi�n obtenida para la velocidad, obtendr�amos la posici�n en funci�n del tiempo,

\begin{equation*}
v(t)=\frac{dx}{dt}=\int \frac{F(t)}{m}dt +v(0)\rightarrow x(t)=\int\left(\int \frac{F(t)}{m}dt +v(0)\right)dt+x(0)
\end{equation*}

Donde suponemos conocida la posici�n inicial $x(0)$.

Quiz� el sistema f�sico idealizado m�s conocido y estudiado es el oscilador arm�nico. En este caso, el sistema est� sometido a una fuerza que depende de la posici�n y, si existe disipaci�n, a una fuerza que depende de la velocidad,

\begin{equation*}
m\frac{d^2x}{dt^2}=-kx-\mu \frac{dx}{dt}
\end{equation*}

En este caso, la expresi�n obtenida constituye una ecuaci�n diferencial ordinaria y ya no es tan sencillo obtener una expresi�n anal�tica para x(t). Para obtener dicha expresi�n anal�tica, es preciso emplear m�todos de resoluci�n de ecuaciones diferenciales.

El problema del oscilador arm�nico, pertenece a una familia de problemas conocida como problemas de valores iniciales. En general, un problema de valores iniciales de primer orden consiste en obtener la funci�n $x(t)$, que satisface la ecuaci�n,
\begin{equation*}
x'(t)\equiv \frac{dx}{dt}=f(x(t),t), \ x(t_0)
\end{equation*}

Donde $x(t_0)$ representa un valor inicial conocido de la funci�n $x(t)$.

En muchos casos, las ecuaciones diferenciales que describen los fen�menos f�sicos no admiten una soluci�n anal�tica, es decir no permiten obtener una funci�n para $x(t)$. En estos caso, es posible obtener soluciones num�ricas empleando un computador. El problema de valores iniciales se reduce entonces a encontrar un aproximaci�n discretizada de la funci�n $x(t)$. 

El desarrollo de t�cnicas de integraci�n num�rica de ecuaciones diferenciales constituye uno de los campos de trabajo m�s importantes de los m�todos de computaci�n cient�fica. Aqu� nos limitaremos a ver los m�s sencillos.

Esencialmente, los m�todos que vamos a describir se basan en discretizar el dominio donde se quiere conocer el valor de la funci�n $x(t)$. As� por ejemplo si se quiere conocer el valor que toma la funci�n en el intervalo $t \in [a,b]$ Se divide el intervalo en $n$ subintervalos cada uno de tama�o $h_i$. Los m�todos que vamos a estudiar nos proporcionan una aproximaci�n de la funci�n $x(t)$, $x_0,x_2\cdots x_n$ en los $n+1$ puntos $t_0,t_1,\cdot, t_n$, donde $t_0=a$, $t_n=b$, y $t_{i+1}-t_i=h_i$. El valor de $h_i$ recibe el nombre de paso de integraci�n. Adem�s se supone conocido el valor que toma la funci�n $x(t)$ en el extremo inicial $a$, $x(a)=x_a$.
  
\subsection{El m�todo de Euler.}
El m�todo de Euler, puede obtenerse a partir del desarrollo de Taylor de la funci�n $x$, entorno al valor conocido $(a,x_a)$. La idea es empezar en el valor conocido e ir obteniendo iterativamente el resto de los valores $x_1,\cdots$ hasta llegar al extremos $b$ del intervalo en que queremos conocer el valor de la funci�n $x$. En general podemos expresar la relaci�n entre dos valores sucesivos a partir del desarrollo de Taylor como, \index{M�todo de Euler}

\begin{equation*}
x(t_{i+1})=x(t_i)+(t_{i+1}-t_{i})x'(t_i)+\frac{(t_{i+1}-t_{i})^2}{2}x''(t_i)+\cdots+ \frac{(t_{i+1}-t_i)^n}{n!}x^{(n)}(t_i)+\cdots
\end{equation*}

Como se trata de un problema de condiciones iniciales, conocemos la derivada primera de la funci�n $x(t)$, expl�citamente,$x'(t)=f(x(t),t)$. Por tanto podemos sustituir las derivadas de $x$ por la funci�n $f$ y sus derivadas,t

\begin{equation*}
x(t_{i+1})=x(t_i)+(t_{i+1}-t_{i})f(t_i,x_i)+\frac{(t_{i+1}-t_{i})^2}{2}f'(t_i,x_i)+\cdots+ \frac{(t_{i+1}-t_i)^n}{n!}f^{(n-1)}(t_i,x_i)+\cdots
\end{equation*}

Donde $x_i\equiv x(t_i)$. Si truncamos el polinomio de Taylor, qued�ndonos solo con el t�rmino de primer grado, y hacemos que el paso de integraci�n sea fijo, $h_i\equiv h=\mathrm{cte}$ obtenemos el m�todo de Euler,

\begin{equation*}
x_{i+1}=x_i+h\cdot f(t_i,x_i)
\end{equation*}

A partir de un valor inicial, $x_0$ es posible obtener valores sucesivos mediante un algoritmo iterativo simple,
\begin{align*}
x_0&=x(a)\\
x_1&=x_0+hf(a,x_0)\\
x_2&=x_1+hf(a+h,x_1)\\
\vdots \\
x_{i+1}&=x_i+hf(a+ih,x_i)
\end{align*}

El siguiente c�digo implementa el m�todo de Euler para resolver un problema de condiciones iniciales de primer orden a partir de una funci�n $f(t)$ y un valor inicial $x_0$.

\begin{lstlisting}
function y=euler(fun,ya,a,b,h)
% implimentacion del metodo de Euler para resolver el problema y'(t)=f(y,t)
% con condici�n inicial ya
% Admite funciones sobre un vector de variables, en ese caso las
% condiciones iniciales ser�n tambi�n un vector
%Uso: y=euler(fun,ya,a,b,h)
%variables de entrada: fun, debe contener el nombre de una funci�n (sera la
%funci�n f) entrecomillas. ya, condici�n inicial. a, instante de tiempo 
%inicial o valor inicial de la variable independiente (t). 
%b, instante de tiempo final o valor final de la variable independiente
%h paso de integraci�n
%podemos por elemental prudencia, comprobar que b>a
%lo elegante no es pasar el paso como he hecho aqui, sino calcularlo en
%funcion de cuantos puntos de la solucion queremos en el intervalo...
if a>=b
y='el intervalo debe ser creciente, para evitarse lios';
return
end
y=ya;
paso=1;
t(paso)=a;
while t(paso)<=b
t(paso+1)=t(paso)+h;
y(paso+1,:)=y(paso,:)+h*feval(fun,t(paso),y(paso,:));
paso=paso+1;
end
%dibuja la solucion,
plot(t,y)

\end{lstlisting}

Un ejemplo sencillo de problema de condiciones iniciales de primer orden, nos los suministra la ecuaci�n diferencial de la carga y descarga de un condensador el�ctrico. La figura \ref{fig:RC}, muestra un circuito el�ctrico elemental formado por una resistencia $R$ en serie con un condensador $C$. \index{Circuito RC}

\begin{figure}[h]
\centering
\includegraphics[width=9cm]{rc.pdf}
\caption{Circuito RC}
\label{fig:RC}
\end{figure}

La intensidad el�ctrica que atraviesa un condensador depende de su capacidad, y de la variaci�n con respecto al tiempo del voltaje entre sus extremos,

\begin{equation*}
I=c\frac{dV_o}{dt}
\end{equation*}

La intensidad que atraviesa la resistencia se puede obtener a partir de la ley de Ohm,

\begin{equation*}
V_R=I\cdot R
\end{equation*}

Por otro lado, la intensidad que recorre el circuito es com�n para la resistencia y el condensador. El voltaje suministrado tiene que ser la suma de las ca�das de voltaje en la resistencia y en el condensador, $V_i=V_o+V_R$. Sustituyendo y despejando,

\begin{equation*}
V_i=V_o+V_R \rightarrow V_i=V_o+I\cdot R \rightarrow V_i=V_o+R\cdot C\frac{dV_0}{dt}
\end{equation*}

Si reordenamos el resultado, 

\begin{equation*}
\frac{dV_o}{dt}=\frac{V_i-V_o}{R \cdot C}
\end{equation*}

Obtenemos una ecuaci�n diferencial para el valor del voltaje en los extremos del condensador que puede tratarse como un  problema de valor inicial. Para este problema, la funci�n $f(t,x)$ toma la forma,

\begin{equation*}
f(t,x) \equiv \frac{V_i-V_o}{R \cdot C}
\end{equation*}

adem�s necesitamos conocer un valor inicial $V_o(0)$ para el voltaje en el condensador. Si suponemos que el condensador se encuentra inicialmente descargado, entonces $V_o(0)=0$. Para este problema se conoce la soluci�n anal�tica. El voltaje del condensador en funci�n del tiempo viene dado por la funci�n,

\begin{equation*}
V_0(t)=V_i\left(1-e^{-t/R\cdot C}\right)
\end{equation*} 


Podemos resolver el problema de la carga del condensador, empleando el c�digo de la funci�n de Euler incluido m�s arriba. Lo �nico que necesitamos es definir una funci�n de Matlab  para representar la funci�n $f(x,t)$ de nuestro problema de valor inicial, t
\begin{lstlisting}
function s=condensador(~,Vo)
% t tiempo en este ejemplo la funci�n no depende del tiempo por lo que esta
% variable realmente no se usa... pero como el programa Euler se la pasa, 
% empleamos el simbolo ~ para que no de errores...
% Vo Valor del voltaje en cada iteraci�n

% hay que definir los parametros fijos del circuito...
C=0.0001; % Capacidad del condesador en Faradios
R=1000; % resitencia en Ohmios
% el voltaje de entrada podr�a Vi podria ser una funci�n del tiempo... Aqu�
% vamos a considerar que es constante...
Vi=10; % Voltaje suministrado al circuito en Voltios
% Aqui cosntruimos la funcion que representa el circuito RC,

s=(Vi-Vo)/R/C;
\end{lstlisting}

\begin{figure}[h]
\centering
\includegraphics[width=12cm]{solrc.eps}
\caption{Comparacion entre los resultados obtenidos mediante el m�todo de Euler para dos pasos de integraci�n $h=0.05$ y $h=0.001$ y la soluci�n anal�tica para el voltaje $V_o$ de un condensador durante su carga.}
\label{fig:solrc}
\end{figure}

La figura \ref{fig:solrc} compara gr�ficamente los resultados obtenido si empleamos la funci�n descrita m�s arriba, \texttt{euler('condensador',0,0,1,h)}, para obtener el resultado de aplicar al circuito un voltaje de entrada constante $V_i=10$V durante un segundo, empleando dos pasos de integraci�n distintos $h=0.05$s  y $h=0.01$s. Adem�s, se ha a�adido a la gr�fica el resultado anal�tico.

En primer lugar, es interesante observar como el voltaje $V_0$ crece hasta alcanzar el valor $V_i=10$ V del voltaje suministrado al circuito. El tiempo que tarda el condensador en alcanzar dicho voltaje y quedar completamente cargado depende de su capacidad y de la resistencia presente en el circuito.

Como era de esperar, al hacer menor el paso de integraci�n la soluci�n num�rica se aproxima m�s a la soluci�n anal�tica. Sin embargo, como pasaba en el caso de los m�todos de diferenciaci�n de funciones, hay un valor de $h$ �ptimo. Si disminuimos el paso de integraci�n por debajo de ese valor, los errores de redondeo empiezan a dominar haciendo que la soluci�n empeore.

\paragraph{Problema de segundo orden.} Vamos a considerar ahora un sistema con una masa colgada de un resorte con un dispositivo mec�nico (amortiguador) que ejerce una fuerza opuesta al movimiento proporcional a la velocidad. En este caso la ecuaci�n diferencial ser�a la siguiente:
\begin{equation*}
m\frac{d^2 y}{dt^2}=m\cdot g-k \cdot y-\mu \frac{dy}{dt}
\end{equation*}

Tenemos un problema de valor inicial de segundo orden, con una ecuaci�n en la que conviven la variable $y$ su derivada $\frac{dy}{dt}$ y su derivada segunda $\frac{d^2y}{dt^2}$. Para resolver este tipo de problemas lo que hacemos es reescribir la ecuaci�n como dos ecuaciones de primer orden, una para la posici�n $y$ y otra para la velocidad $v_y=\frac{dy}{dt}$ de manera que obtendr�amos un sistema de dos ecuaciones de primer orden acopladas:

\begin{align*}
\frac{dy}{dt}&=v_y \\
\frac{dv_y}{dt}&=g-\frac{k}{m}\cdot y - \frac{\mu}{m}\cdot v_y 
\end{align*}

Ese sistema de ecuaciones de primer orden se puede resolver usando el m�todo de Euler. Necesitamos definir una funci�n de Matlab para representar el sistema de ecuaciones diferenciales

\begin{lstlisting}
function [dydt] = amortiguador(~,y)
% y(1) posici�n
% y(2) velocidad
% k constante del muelle
% mu coeficiente de rozamiento
% g aceleraci�n de la gravedad
% m masa
g=9.8;
k=10.0;
m=2.0;
mu=0.5;
dydt(1)=y(2);
dydt(2)=g-(k/m)*y(1)-(mu/m)*y(2);
end
\end{lstlisting}

Empleando la funci�n descrita \texttt{euler('amortiguador',[0 0],0,50,0.01)} obtenemos la evoluci�n temporal de la posici�n y la velocidad de la masa unida al resorte como puede verse en la figura \ref{fig:amortiguador1}. En esta figura se aprecia como la masa oscila en torno a la posici�n de equilibrio hasta que finalmente se para.

\begin{figure}[h]
	\centering
	\includegraphics[width=12cm]{amortiguador_1.png}
	\caption{Resultado de aplicar el m�todo de Euler a una masa suspendida en vertical de un muelle con rozamiento.}
	\label{fig:amortiguador1}
\end{figure}

Si cambiamos el valor de la constante del muelle a $k=100$ ($kg/s^2$) y volvemos a aplicar el m�todo de Euler para resolver obtenemos el resultado mostrado en la figura \ref{fig:amortiguador2}. En la gr�fica puede observarse que la masa comienza a oscilar con oscilaciones de amplitud creciente. Para analizar este comportamiento an�malo vamos a repetir el m�todo de Euler reduciendo en un orden de magnitud el paso de integraci�n. Si lo hacemos as� obtenemos los resultados de la figura \ref{fig:amortiguador3}. Al reducir el paso vemos que la masa oscila en torno a la posici�n de equilibrio hasta que se estabiliza. El comportamiento observado en la figura \ref{fig:amortiguador2} se debe a errores de integraci�n del algoritmo de Euler.

\begin{figure}[h]
	\centering
	\includegraphics[width=12cm]{amortiguador_2.png}
	\caption{Resultado de aplicar el m�todo de Euler a una masa suspendida en vertical de un muelle con rozamiento $k=100$.}
	\label{fig:amortiguador2}
\end{figure}

\begin{figure}[h]
	\centering
	\includegraphics[width=12cm]{amortiguador_3.png}
	\caption{Resultado de aplicar el m�todo de Euler a una masa suspendida en vertical de un muelle con rozamiento $k=100$ y paso $h=0.001$.}
	\label{fig:amortiguador3}
\end{figure}

En la pr�ctica se emplean algoritmos m�s precisos (y complejos) que el de Euler para resolver problemas de valor inicial, por ejemplo los m�todos de Runge-Kutta.

\subsection{M�todos de Runge-Kutta} \index{M�todo de Runge-Kutta}
Si intentamos emplear polinomios de Taylor de grado superior, al empleado en el m�todo de Euler, nos encontramos con la dificultad de obtener las derivadas sucesivas, con respecto al tiempo, de la funci�n $f$. As�, si quisi�ramos emplear el polinomio de Taylor de segundo grado, para la funci�n x, tendr�amos,
\begin{equation*}
x(t_{i+1})=x(t_i)+h\cdot f(t_i,x_i)+\frac{h^2}{2}f'(t_i,y_i)
\end{equation*}
Pero,
\begin{equation*}
f'(t,x)=\frac{\partial f(t,x)}{\partial t}+\frac{\partial f(t,x)}{\partial x}\cdot f(t,x)
\end{equation*}
Sacando factor com�n a h, obtenemos,
\begin{equation*}
x(t_{i+1})=x(t_i)+h\left(f(t_i,x_i)+\frac{h}{2}\left(\frac{\partial f(t_i,x_i)}{\partial t}+\frac{\partial f(t_i,x_i)}{\partial x}\cdot f(t_i,x_i)\right)\right)
\end{equation*}

A partir de este resultado, es f�cil comprender que resulte complicado obtener m�todos de resoluci�n basados en el desarrollo de Taylor. De hecho, se vuelven cada vez m�s complicados seg�n vamos empleando polinomios de Taylor de mayor grado.

Desde un punto de vista pr�ctico, lo que se hace es buscar aproximaciones a los t�rminos sucesivos del desarrollo de Taylor, para evitar tener que calcularlos expl�citamente. Estas aproximaciones se basan a su vez, en el desarrollo de Taylor para funciones de dos variables.

Los m�todos de integraci�n resultantes, se conocen con el nombre gen�rico de m�todos de Runge-Kutta. Veamos c�mo se har�a para el caso que acabamos de mostrar del polinomio de Taylor de segundo grado.

En primer lugar obtenemos el desarrollo del polinomio de Taylor de primer grado, en dos variables de la funci�n $f(t,x)$, en un entorno del punto $t_i,x_i$,
\begin{equation*}
f(t,x)=f(t_i,x_i)+\left((t-t_i)\frac{\partial f(t_i,x_i)}{\partial t}+(x-x_i)\frac{\partial f(t_i,x_i)}{\partial x}\right)
\end{equation*}

Si ahora comparamos este resultado con la ecuaci�n anterior, podr�amos tratar de identificar entre s� los t�rminos que acompa�an las derivadas parciales, 

\begin{align*}
t-t_i&=\frac{h}{2} \rightarrow t=t_i+\frac{h}{2}\\
x-x_i&=\frac{h}{2}\cdot f(t_i,x_i) \rightarrow x=x_i+\frac{h}{2}\cdot f(t_i,x_i)
\end{align*}

Es decir,

\begin{equation*}
f(t_i,x_i)+\frac{h}{2}\left(\frac{\partial f(t_i,x_i)}{\partial t}+\frac{\partial f(t_i,x_i)}{\partial x}\cdot f(t_i,x_i)\right)=f(t_i+\frac{h}{2},x_i+\frac{h}{2}\cdot f(t_i,x_i))
\end{equation*}

Si ahora sustituimos este resultado en nuestra expresi�n del polinomio de Taylor de segundo grado de la funci�n $x(t)$,

\begin{equation*}
x(t_{i+1})=x(t_i)+h\cdot f(t_i+\frac{h}{2},x_i+\frac{h}{2}f(t_i,x_i))
\end{equation*}
Donde $x_i\equiv x(t_i)$. Esta aproximaci�n da lugar al primero y m�s sencillo de los m�todos de Runge-Kutta, conocido como m�todo del punto medio. El nombre es debido a que la funci�n $f$ se eval�a en un punto a mitad de camino entre $t_i$ y $t_{i+1}=t_i+h$. El c�lculo de la soluci�n de un problema de valor inicial mediante este m�todo, se puede expresar de un modo an�logo al del m�todo de Euler, 
\begin{align*}
x_0&=x(a)\\
x_1&=x_0+h\cdot f(a+\frac{h}{2},x_0+\frac{h}{2}f(a,x_0))\\
\vdots \\
x_{i+1}&=x_i+h\cdot f(t_i+\frac{h}{2},x_i+\frac{h}{2}f(t_i,x_i))\\
\end{align*}
La siguiente funci�n de Matlab implementa el m�todo del punto medio,

\begin{lstlisting}
function y=pmedio(fun,ya,a,b,h)
% metodo de rungekutta de segundo orden mas conocido como del punto medio
% uso: y=pmedio(fun,ya,a,b,h)
%  variables de entrada: fun, nombre entre comillas de la funci�n que define
%  el problema de valor inicial que se desea resolver. ya, valor inicial. a,
% instante de tiempo inicial o valor inicial de la variable independiente.
% b valor de tiempo final o valor final de la variable independiente. h,
% paso de integracion
% podemos por elemental prudencia, comprobar que b>a
% lo elegante no es pasar el paso como he hecho aqui, sino calcularlo en
% funcion de cuantos puntos de la solucion queremos en el intervalo...
if a>=b
    y='el intervalo debe ser creciente, para evitarse lios'
    return
end
y=ya;
paso=1;
t(paso)=a;
while t(paso)<=b
    t(paso+1)=t(paso)+h;
    y(paso+1)=y(paso)+h*feval(fun,t(paso)+h/2,y(paso)+h*feval(fun,t(paso),y(paso))/2);
    paso=paso+1;
end
% pintamos la solucion,
plot(t,y)

\end{lstlisting}


El resto de los m�todos de Runge-Kutta, los dejaremos para cuando se�is m�s mayores... Pero que conste que es de lo m�s interesante de los m�todos num�ricos.

\section{Ejercicios}
\begin{enumerate}
\item Crea una funci�n de matlab que tomo como variables de entrada en handle de otra funci�n \texttt{@f}, un valor $x_0$, un intervalo $h$ y una �ltima variable \texttt{metodo }que contenga el nombre de un m�todo y devuelva el valor de la derivada de f calculada en el punto $x_0$, empleando el m�todo indicado en la variable \texttt{metodo}.  Los m�todos pueden ser:
\begin{itemize}
\item \texttt{metodo = '2ad'}: diferencia de dos puntos adelantada
\item \texttt{metodo = '2ce'}: diferencia de dos puntos centrada
\item \texttt{metodo = '3ad'}: diferencia de tres puntos adelantada
\end{itemize}
Emplea la funci�n que acabas de crear para obtener la derivada de la funci�n $f(x) = 1/x$ en el punto $x_0 = 1$. Prueba para valores de $h$ $0.1$, $0.5$,$1.5$. Explica los resultados. 

\item Analiza los programas mostrados en la secci�n \ref{sec:inum} para el calculo de integrales empleando los m�todos deTrapecio, Simpson y Simpson 3/8 y la funci�n para calcular los m�todos extendidos. Comprueba la precisi�n de los m�todos calculando la integral,
\begin{equation*}
\int_0^{\pi} \sin(x) = 2
\end{equation*}

\item El programa del apartado anterior, solo es �til cuando se conoce la funci�n que se desea integrar. Sin embargo es posible aplicar los m�todos de integraci�n num�rica descritos cuando solo se dispone de una tabla de datos. 

\begin{enumerate}
\item Escribe una funci�n admita como entrada un vector de datos equiespaciados $[y= y_0,y_1,\cdots,y_n]$, un intervalo de integraci�n $h=x_i - x_{i-1}$, y una variable con el nombre de un m�todo (de modo an�logo a como se ha hecho en el ejercicio anterior). La funci�n deber� devolver la integral de la funci�n que representan los datos de la tabla, aplicando el correspondiente m�todo extendido, sobre los datos de $y$:
\begin{align*}
 &\text{Trapecio:}\\
I(y)&=\int_{x_0}^{x_n}ydx \approx \sum_{i=0}^{n-1}\int_{x_i}^{x_{i+1}}y(x)dx=\sum_{i=0}^{n-1}\frac{h}{2}(y_i+y_{i+1})\\
&=\frac{h}{2}\left(y_0+2y_1+2y_2+\cdots+2y_{n-1}+y_n \right)\\
 &\text{Simpson:} \\
I(y)&\approx \sum_{i=0}^{\frac{n}{2}-1}\int_{x_{2i}}^{x_{2i+2}}P_2(x)dx=\sum_{i=0}^{\frac{n}{2}-1}\frac{h}{3}(y_{2i}+4y_{2i+1}+y_{2i+2})\\
&= \frac{h}{3}(y_0+4y_1+2y_2+4y_3+2y_4+\cdots + 2y_{n-2}+4y_{n-1}+y_n)\\
&\text{Simpson 3/8}\\
I(y)&\approx \sum_{i=0}^{\frac{n}{3}-1}\int_{x_{3i}}^{x_{3i+3}}P_3(x)dx=\sum_{i=0}^{\frac{n}{3}-1}\frac{3h}{8}(y_{3i}+3y_{3i+1}+3y_{3i+2}+y_{3i+3})\\
&= \frac{3h}{8}(y_0+3y_1+3y_2+2y_3+3y_4+3y_5+2y_6+ \cdots + 2y_{n-3}+3y_{n-2}+3y_{n-1}+y_n)
\end{align*}

\item Es importante tener en cuenta que, para el caso de Simpson, $n$ --El n�mero de datos del vector $y$ menos 1-- debe ser divisible entre dos. A�ade el c�digo necesario al programa realizado integre el �ltimo intervalo empleando el m�todo del trapecio. Del mismo modo, para Simpson 3/8, $n$ debe ser divisible entre tres. A�ada el c�digo necesario para que, si no se cumple esta condici�n, integre por el m�todo del trapecio, si sobra un intervalo o por el m�todo de Simpson si sobran dos.

\item Aplica la funci�n de los apartados anteriores a la siguiente tabla de datos,

%\begin{table}[h]
%\centering
\begin{tabular}{r r r r r r r r r r r r}
\hline
x & 0. &0.314 & 	0.628 &	0.942 &	1.256 &	1.570 &	1.884 &	 2.199 &	2.513 &	 2.827 &	3.141\\
y & 0. &0.309 &	0.587 &	0.809 &	0.951 &	1.000 &	0.951 &    0.809 &	0.587 &	0.309 &	0.000\\
\hline
\end{tabular}
%\end{table}
calcula el resultado empleando los tres m�todos, y comp�ralos con el que da la funci�n de Matlab \texttt{trapz}
\end{enumerate}

\item La ecuaci�n diferencial de un p�ndulo f�sico toma la forma,
\begin{equation*}
\frac{d^2\theta}{dt} =-\frac{g}{l}\sin(\theta),
\end{equation*}
donde $\theta$ es el �ngulo que el p�ndulo forma con la vertical, $g\approx9.8 m/s^2$ es la aceleraci�n de la gravedad y $l$ es la longitud del p�ndulo.
\begin{enumerate}
\item Reescribe la ecuaci�n del p�ndulo en dos ecuaciones de primer orden, una para el angulo en funci�n de la velocidad angular $\omega = \frac{d\theta}{dt}$ y otra para la velocidad angular $\frac{d\omega}{dt} = \frac{d^2\theta}{dt}$

\item \label{ap:4b} crea un programa que permita obtener los valores de $\omega$ y $\theta$ en funci�n de tiempo, a partir de las ecuaciones obtenidas en el apartado anterior, empleando el m�todo de Euler. El programa deber� permitir dar valores iniciales $\theta_0$ y $\omega_o$ a la posici�n y velocidad angulares, definir un intervalo de tiempo de integraci�n $[t_0,t_f]$ y un paso de integraci�n $\Delta t$

\item Para desplazamientos  peque�os de $\theta$, tomando $\sin(\theta) \approx \theta$, es posible aproximar  la ecuaci�n exacta del p�ndulo por la siguiente ecuaci�n,
\begin{equation*}
\frac{d^2\theta}{dt} =-\frac{g}{l}\theta,
\end{equation*}
A�ade al programa realizado en el apartado \ref{ap:4b} el c�digo necesario para que integre simult�neamente las ecuaciones del p�ndulo exacta y la aproximada.

\item Compara gr�ficamente los resultados para la posici�n y velocidad obtenidas a partir de las ecuaciones exacta y aproximada para un p�ndulo de longitud $l=0.1m$. Si parte del reposo con �ngulos iniciales $\theta_0$, $\pi/10, pi/6, pi/4, pi/3$. Emplea para la integraci�n un intervalo de tiempo de $t_f-t_0 = 1s$ y un paso de integraci�n $\Delta t=1e-3$
\end{enumerate}
\item La din�mica de un planeta que gira en torno a una estrella viene dada por las siguiente ecuaciones del movimiento,
\begin{align*}
\frac{d^2x}{dt^2} &= -G\frac{M}{(x^2+y^2)^{3/2}}x\\
\frac{d^2y}{dt^2} &= -G\frac{M}{(x^2+y^2)^{3/2}}y
\end{align*}
Donde $G$ es la constante universal de la gravedad, $M$ la masa de la estrella, que se encuentra situada en el origen del coordenadas, y $(x,y)$ las coordenadas del vector posici�n del planeta.
\begin{enumerate}
\item transforma las ecuaciones del movimiento en un sistema de cuatro ecuaciones de primer orden, empleando la velocidad del planeta, $v_x = \frac{dx}{dt},\ v_y = \frac{dy}{dt} $. 
\item Escribe un programa que permita obtener la trayectoria (�rbita) de un planeta resolviendo mediante el m�todo de Euler las ecuaciones obtenidas en el apartado anterior.
\item Obt�n la trayectoria (�rbita) de un planeta que gira en torno a una estrella, empleando los siguientes datos, medidos en unidades arbitrarias: Masa de la estrella:$M=1$. Constante de la gravedad $G=1$. Posici�n inicial del planeta: $x_0 = 1$, $y_0=0$. Velocidad inicial $v_x = 0$, $v_y = 0.7$. Emplea un paso de integraci�n $\Delta t =0.01$. Calcula la soluci�n durante un tiempo total $t = 10$.
\end{enumerate}
\textbf{Nota:} El m�todo de Euler no es suficientemente preciso para resolver este tipo de problemas. Es f�cil obtener resultados muy alejados de la realidad en funci�n del valor que tomen las condiciones iniciales. 
\end{enumerate}

\section{Test del curso 2020/21}

\begin{figure}[h]
\begin{minipage}{0.5\textwidth} \ 
\includegraphics[width=\textwidth]{dibujillo.eps}
\end{minipage}
\begin{minipage}{0.5\textwidth}
\begin{align*}
g &= 10 \, \text{m}/\text{s}^2 \text{\ gravedad}\\
k &= 60 \, \text{N}/\text{m} \text{\ constante recuperadora del muelle}\\
m &= 2 \, \text{kg} \text{\ masa del bloque}\\
\alpha &= \frac{\pi}{6} \text{\ �ngulo del plano inclinado}\\
N &= \text{Reacci�n normal del plano}\\
p &= mg\\
	\mu &= 0.1 \, \text{kg}/\text{s} \text{\ coeficiente de rozamiento}\\
\end{align*}

\end{minipage}
\caption{Sistema masa-muelle-plano inclinado.}
\label{fig1}
\end{figure}

Sobre un plano inclinado se coloca un bloque sujeto por un muelle, tal y como muestra en la Figura \ref{fig1}. La superficie de contacto entre el bloque y el plano est� muy pulida; por lo que se puede considerar que el rozamiento entre ambos es proporcional a la velocidad $v_x(t(t))\in\mathbb{R}$ a lo largo del eje paralelo al plano.

El movimiento del bloque en la direcci�n del plano puede determinarse mediante la ecuaci�n
\begin{equation}
	ma_{x}(t) = -kx(t) -\mu \lvert N \lvert v_x(t) + p_x, \label{eq: f}
\end{equation}
donde $a(t)\in\mathbb{R}$ representa la aceleraci�n del bloque, $x(t)\in\mathbb{R}$ su posici�n y el resto constantes se describen en la Figura \ref{fig1}. El origen, $x=0$, se toma en el punto en donde la fuerza recuperadora del muelle es nula. La posici�n crece en el sentido de bajada a lo largo del plano.

\begin{enumerate}

	\item \label{p1} Emplea el m�todo de Euler para estimar las posici�n $x(t)$ y las velocidad $v_x(t)$ del bloque a partir de la ecuaci�n (\ref{eq: f}). Utiliza un paso de integraci�n $h =10^{-3}\text{s}$. Considera una posici�n inicial $x(0) =0\text{m}$, una velocidad inicial $v(0) = 0\text{m/s}$ y un tiempo final de integraci�n $t_{f}=11\text{s}$.

\item \label{p2} Representa gr�ficamente los resultados obtenidos: Posici�n frente a tiempo y velocidad frente a tiempo. Emplea una figura distinta para cada representaci�n. No olvides a�adir r�tulos a los ejes indicando las variables representadas con sus unidades.

\item La soluci�n anal�tica del problema, para velocidad inicial $v(0)=0$ y cualquier posici�n inicial $x(0)$ toma la siguiente forma,

%\begin{equation}
%x(t) = \frac{mp_x}{k}-\left(\frac{mp_x}{k}-x_0\right)\frac{e^{-\frac{Fr}{2}t}}{\sqrt{1-\frac{F^2_r m}{4k}}}\cos\left(\sqrt{\frac{k}{m}-\frac{F^2_r}{4}}\cdot t-\text{asin}\left(\frac{F_r}{2}\sqrt{\frac{m}{k}}\right)   \right)
%\end{equation}
%

\begin{equation}\label{eq2}
	x(t) = \frac{p_x}{k}-\left(\frac{p_x}{k}-x(0)\right)\frac{\omega_0}{\omega}e^{-\eta t}\cos\left(\omega t-\phi \right)
\end{equation}
donde,

\begin{equation}
\begin{matrix}
\omega_0 = \sqrt{\frac{k}{m}} & \eta = \frac{\mu N}{2m}\\
\ & \ \\
\omega = \sqrt{\omega_0^2 -\eta^2} &
\phi = \arcsin \left(\frac{\eta}{\omega_0}\right)
\end{matrix} \nonumber
\end{equation}
 
\begin{enumerate}
\item Calcula la posici�n del bloque mediante la ecuaci�n (\ref{eq2}). Emplea para ello el mismo intervalo y los mismos instantes de tiempo empleados en el apartado \ref{p1}. Representa los resultados sobre el gr�fico  para la posici�n calculada mediante el m�todo de Euler que has obtenido en el apartado \ref{p2}. 
\item Representa en un gr�fico de barras los residuos resultantes de comparar la soluci�n anal�tica con la obtenida por el m�todo de Euler.
\item Calcula el error cuadr�tico medio cometido al emplear el m�todo de Euler. Considera como exacta la soluci�n anal�tica (\ref{eq2}).
\end{enumerate}

\item Los instantes de tiempo para los que la posici�n del bloque alcanza un m�ximo o un m�nimo local puede obtenerse a partir de la frecuencia de oscilaci�n $\omega$,
\begin{equation}\label{eq4}
t_{max/min} = \frac{n \pi}{\omega}, n = 1,2,3,\cdots, \infty
\end{equation}
\begin{enumerate}
\item \label{p4}Emplea las ecuaci�nes (\ref{eq4}) y (\ref{eq2}) para obtener los primeros 20 puntos singulares (m�ximos o m�nimos) del movimiento del bloque. Repres�ntalos sobre el gr�fico de la posici�n, obtenido en el apartado \ref{p2}.
\item Utilizando los datos obtenidos en el apartado \ref{p4}, emplea el m�todo de diferencia de dos puntos centrada para obtener las derivadas de las posiciones en los m�ximos y m�nimos locales con respecto al tiempo. Si hay puntos para los que no es posible aplicar este m�todo, calcula su derivada empleando otra aproximaci�n razonable.
 
\item Representa los resultados obtenidos en el apartado anterior sobre el gr�fico de la velocidad obtenido en el ejercicio \ref{p2}. A la vista de los resultados, �C�mo valorar�as la precisi�n del m�todo empleado para obtener las derivadas? 
\end{enumerate}

\item A partir de los resultados obtenidos para la velocidad en el apartado \ref{p1}, calcula la posici�n del bloque en el instante de tiempo $t=1s$ mediante la siguiente integral
\begin{equation}
x(1) = \int_0^1v_x(t)dt,
\end{equation}
\begin{enumerate}
\item Emplea para ello el m�todo del trapecio.
\item Compara el resultado con los valores obtenidos tanto mediante el m�todo de Euler como a partir de la soluci�n anal�tica.\\ 
\textbf{Nota:} Ten en cuenta que el m�todo de Euler puede que no calcule $x(1)$ y $v(1)$ ya que por error de precisi�n el computador no obtiene exactamente $t=1$s. Por tanto, emplea para el c�lculo y la comparativa los valores de $x(t)$ y $v(t)$ correspondientes al tiempo m�s pr�ximo a $t=1\text{s}$.
\end{enumerate}
\end{enumerate}

\include{tratamientoest}
\include{simbolico}
\printindex
\end{document} 