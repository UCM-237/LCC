\documentclass[a4paper,10pt]{book}
\usepackage[pdftex]{graphicx}
\usepackage{epstopdf}
\usepackage{subfigure}
\usepackage{amsmath,amsthm}
\usepackage{tikz}
\usetikzlibrary{shapes.geometric}
\usetikzlibrary{shapes.multipart}
\textwidth= 15cm
\evensidemargin=0cm
\usepackage[spanish]{babel}
\usepackage[latin1]{inputenc}
\usepackage{textcomp}
\usepackage{amstext}
\usepackage{amsfonts}
\usepackage{amssymb}
\usepackage{multicol}
\usepackage[hyperindex=true,breaklinks=true,colorlinks=true,linkcolor=blue]{hyperref}
\renewcommand{\tablename}{Tabla}
\renewcommand{\listtablename}{\'Indice de Tablas}

\usepackage{listings}
\usepackage{color} %red, green, blue, yellow, cyan, magenta, black, white
\definecolor{mygreen}{RGB}{28,172,0} % color values Red, Green, Blue
\definecolor{mylilas}{RGB}{170,55,241}

\usepackage{multirow}
\usepackage{imakeidx}

%\usepackage{draftwatermark}
%\SetWatermarkText{Borrador,juan.jimenez@fis.ucm.es}
%\SetWatermarkScale{2}


\graphicspath{{./figuras/}}
\makeindex
\begin{document}
\title{
\begin{flushleft}
\includegraphics[width=2.5cm]{ucm2.pdf}
Universidad Complutense de Madrid\\
---------------------------------------------------------------------\
\end{flushleft}
Laboratorio de Computaci\'on Cient\'ifica}
\author{ Juan Jim\'enez\\ H\'ector Garc\'ia de Marina}
\maketitle
\vspace*{\fill}

\includegraphics[scale=1]{by-sa.eps}\\
El contenido de estos apuntes est\'e1 bajo licencia Creative Commons Atribution-ShareAlike 4.0\\
\href{http://creativecommons.org/licenses/by-sa/4.0/}{http://creativecommons.org/licenses/by-sa/4.0/}\\
\copyright Juan Jim\'enez

\bigskip
\tableofcontents
\listoffigures
\listoftables

%\section*{Matlab Code}
\lstset{language=Matlab,%
    %basicstyle=\color{red},
    breaklines=true,%
    morekeywords={matlab2tikz},
    keywordstyle=\color{blue},%
    morekeywords=[2]{1}, keywordstyle=[2]{\color{black}},
    identifierstyle=\color{black},%
    stringstyle=\color{mylilas},
    commentstyle=\color{mygreen},%
    showstringspaces=false,%without this there will be a symbol in the places where there is a space
    numbers=left,%
    numberstyle={\tiny \color{black}},% size of the numbers
    numbersep=9pt, % this defines how far the numbers are from the text
    emph=[1]{for,end,break},emphstyle=[1]\color{red}, %some words to emphasise
    %emph=[2]{word1,word2}, emphstyle=[2]{style},    
}
%\lstinputlisting{../codigo/matlab/1-intro/ejemplo1.m}
\chapter*{Prefacio}
Estos apuntes cubren de forma aproximada  el contenido del \emph{Laboratorio de computaci�n cient�fica} del primer curso del grado en f�sica.
La idea de esta asignatura es  introducir al estudiante a las estructuras elementales de programaci�n y al c�lculo num�rico, como herramientas imprescindibles para el trabajo de investigaci�n.

Casi todos los m�todos que se describen en estos apuntes fueron desarrollados hace siglos por los grandes: Newton, Gauss, Lagrange, etc.  M�todos que no han perdido su utilidad y que, con el advenimiento de los computadores digitales, han ganado todav�a m�s si cabe en atractivo e inter�s. Se cumple una vez m�s la famosa frase atribuida a Bernardo de Chartres:
\begin{quote}
``Somos como enanos a los hombros de gigantes. Podemos ver m�s, y m�s lejos que ellos, no por que nuestra vista sea m�s aguda, sino porque somos levantados sobre su gran altura."
\end{quote}

En cuanto a los contenidos, ejemplos, c�digo, etc. Estos apuntes deben mucho a muchas personas. En primer lugar a Manuel Prieto y Segundo Esteban que elaboraron las presentaciones de la asignatura \emph{Introducci�n al c�lculo cient�fico y programaci�n} de la antigua licenciatura en f�sicas, de la que el laboratorio de computaci�n cient�fica es heredera. 

En segundo lugar a mis compa�eros de los departamentos de  \emph{F�sica de la Tierra, Astronom�a y Astrof�sica I} y  \emph{Arquitectura de computadores y Autom�tica} que han impartido la asignatura durante estos a�os: 

Rosa Gonz�lez Barras, Bel�n Rodr�guez Fonseca, Maurizio Matessini, Pablo Zurita, Vicente Carlos Ru�z Mart�nez, Encarna Serrano, Carlos Garc�a S�nchez, Jose Antonio Mart�n, Victoria L�pez L�pez,  Alberto del Barrio, Blanca Ayarzag�ena, Javier G�mez Selles, Nacho G�mez P�rez, Marta �valos, I�aqui Hidalgo, Daviz s�nchez,  Juan Rodriguez, Mar�a Ramirez, �lvaro de la C�mara (Espero no haberme olvidado de nadie).

Muchas gracias a todos por tantas horas de trabajo compartidas.

Por �ltimo, los errores y erratas que encuentres en estas notas, esos s� que son de mi exclusiva responsabilidad.  Puedes ---si quieres--- ayudarme a corregirlos en futuras ediciones escribiendo a: juan.jimenez@fis.ucm.es 

\begin{flushright}
Juan Jim�nez.
\end{flushright}
\chapter{Introducci�n al software cient�fico}
En la actualidad, el ordenador se ha convertido en una herramienta imprescindible para el trabajo de cualquier investigador cient�fico. Su uso ha permitido realizar tareas que sin su ayuda resultar�an sencillamente imposibles de acometer. Entre otras, distinguiremos las tres siguientes:
\begin{itemize}
\item Adquisici�n de datos de dispositivos experimentales.
\item An�lisis y tratamiento de datos experimentales. \index{Datos, an�lisis}
\item C�lculo Ci�ntifico.
\end{itemize}

La primera de �stas tareas queda fuera de los contenidos de esta asignatura. Su objetivo es emplear el ordenador para recoger datos autom�ticamente de los sensores empleados en un dispositivo experimental. El procedimiento habitual es emplear dispositivos electr�nicos que traducen las lecturas de un sensor (un term�metro, un man�metro, un caudal�metro, una c�mara etc.) a un voltaje. El voltaje es digitalizado, es decir, convertido a una secuencia de ceros y unos, y almacenado en un ordenador para su posterior an�lisis o/y directamente monitorizado, es decir, mostrado en la pantalla del ordenador. En muchos casos el ordenador es a su vez capaz de interactuar con el dispositivo experimental: iniciar o detener un experimento, regular las condiciones en que se realiza, disparar alarmas si se producen errores, etc.

De este modo, el investigador cient�fico, queda dispensado de la tarea de adquirir por s� mismo los datos experimentales. Tarea que en algunos casos resultar�a imposible, por ejemplo si necesita medir muchas variables a la vez o si debe medirlas a gran ritmo; y en la que, en general, es relativamente f�cil cometer errores.

El an�lisis y tratamiento de datos experimentales, constituye una tarea fundamental dentro del trabajo de investigaci�n cient�fica. Los ordenadores permiten realizar dichas tareas, de una forma eficiente y segura con cantidades de datos que resultar�an imposibles de manejar hace 50 a�os. Como veremos m�s adelante, una simple hoja de c�lculo puede ahorrarnos una cuantas horas de c�lculos tediosos. El an�lisis estad�stico de un conjunto de datos experimentales, el c�lculo --la estimaci�n-- de los errores experimentales cometidos, la posterior regresi�n de los datos obtenidos a una funci�n matem�tica que permita establecer una ley o al menos una relaci�n entre los datos obtenidos, formar parte del trabajo cotidiano del investigador, virtualmente en todos los campos de la ciencia.

Por �ltimo el c�lculo. \index{C�lculo num�rico} Cabr�a decir que constituye el n�cleo del trabajo de investigaci�n. El cient�fico trata de explicar la realidad que le rodea, mediante el empleo de una descripci�n matem�tica. Dicha descripci�n suele tomar la forma de un modelo matem�tico m�s o menos complejo. La validez de un modelo est� ligada a que sea capaz de reproducir los resultados experimentales obtenidos del fen�meno que pretende explicar. Si el modelo es bueno ser� capaz de obtener mediante c�lculo unos resultados similares a los obtenido mediante el experimento. De este modo, el modelo queda validado y es posible emplearlo para predecir c�mo se comportar� el sistema objeto de estudio en otras condiciones.

\section{Introducci�n a los computadores} \index{Computador} \index{Ordenador}
M�s o menos todos estamos familiarizados con lo que es un computador, los encontramos a diario continuamente  y, de hecho, hay muchos aspectos de nuestra vida actual que ser�an inimaginables sin los computadores.  En t�rminos muy generales, podemos definir un computador como una m�quina que es capaz de recibir instrucciones y realizar operaciones (c�lculos) a partir de las instrucciones recibidas. Precisamente es la capacidad de recibir instrucciones lo que hace del ordenador una herramienta vers�til; seg�n las instrucciones recibidas y de acuerdo tambi�n a sus posibilidades como m�quina,  el ordenador puede realizar tareas muy distintas, entre las que cabe destacar como m�s generales, las siguientes:
\begin{itemize}
\item Procesamiento de datos 
\item Almacenamiento de datos
\item Transferencias de datos entre el computador y el exterior
\item Control de las anteriores operaciones
\end{itemize}

El computador se dise�a para realizar funciones generales que se especifican cuando se programa. La programaci�n es la que concreta las tareas que efectivamente realiza un ordenador concreto.

\subsection{Niveles de descripci�n de un ordenador}

La figura \ref{fig:nivel} muestra un modelo general de un computador descrito por niveles. Cada nivel, supone y se apoya en el nivel anterior. 
\begin{figure}[h]
	\centering
		\includegraphics[width=10cm]{nivel_descripcion.pdf}
	\caption{Descripci�n por niveles de un computador}
	\label{fig:nivel}
\end{figure}
\begin{enumerate}
\item \textbf{Nivel F�sico.} Constituye la base del \emph{hardware} del computador. Est� constituido por los componentes electr�nicos b�sicos, diodos, transistores, resistencias, etc.  En un computador moderno, no es posible separar o tan siquiera observar dichos componentes: Se han fabricado directamente sobre un cristal semiconductor, y forman parte de un dispositivo electr�nico conocido con el nombre de circuito integrado.

\item \textbf{Circuito Digital.}
Los componentes del nivel f�sico se agrupan formando circuitos digitales, (En nuestro caso circuitos digitales integrados). Los circuitos digitales trabajan solo con dos niveles de tensi�n ($V_1, V_0$) lo que permite emplearlos para establecer relaciones l�gicas: $V_1$=verdadero, $V_2$=falso. Estas relaciones l�gicas establecidas empleando los valores de la tensi�n de los circuitos digitales constituyen el soporte de todos los c�lculos que el computador puede realizar.

\item \textbf{Organizaci�n Hardware del sistema.}\index{Computador ! \emph{hardware}} 
Los circuitos digitales integrados se agrupan y organizan para formar el \emph{Hardware} del ordenador.  Los m�dulos b�sicos que constituyen el \emph{Hardware} son la unidad central de procesos (CPU), La unidad de memoria y las unidades de entrada y salida de datos. Dichos componentes est�n conectados entre s� mediante un bus, que transfiere datos de una unidad a otra.

\item \textbf{Arquitectura del computador.} \index{Computador ! arquitectura}
La arquitectura define c�mo trabaja el computador. Por tanto, est� estrechamente relacionada con la organizaci�n hardware del sistema, pero opera a un nivel de abstracci�n superior. Establece c�mo se accede a los registros de memoria, arbitra el uso de los buses que comunican unos componentes con otros, y regula el trabajo de la CPU.  

Sobre la arquitectura se establece el lenguaje b�sico en el que trabaja el ordenador, conocido c�mo lenguaje m�quina. Es un lenguaje que emplea todav�a niveles l�gicos binarios (ceros o unos) y por tanto no demasiado apto para ser interpretado por los seres humanos. Este lenguaje permite al ordenador realizar operaciones b�sicas como copiar el contenido de un registro de memoria en otro, sumar el contenido de dos registros de memoria, etc. 

El lenguaje m�quina es adecuado para los computadores, pero no para los humanos, por eso, los fabricantes suministran junto con el computador un repertorio b�sico de instrucciones que su m�quina puede entender y realizar en un lenguaje algo m�s asequible. Se trata del lenguaje ensamblador. Los comandos de �ste lenguaje son f�cilmente traducibles en una o varias instrucciones de lenguaje m�quina.   A�n as� se trata de un lenguaje en el que programar directamente resulta una tarea tediosa y proclive a cometer errores. 

\item \textbf{Compiladores y Sistemas Operativos} \index{Sistema operativo} \index{Compilador}
Los Compiladores constituyen un tipo de programas especiales que permiten convertir un conjunto de instrucciones, escritas en un lenguaje de alto nivel en lenguaje m�quina. El programador escribe sus instrucciones en un fichero de texto normal, perfectamente legible para el ser humano, y el compilador convierte las instrucciones contenidas en dicho fichero en secuencias binarias comprensibles por la m�quina.

Los computadores primitivos solo eran capaces de ejecutar un programa a la vez. A medida que se fueron fabricando ordenadores mas sofisticados, surgi� la idea de crear programas que se encargaran de las tareas b�sicas: gestionar el flujo de informaci�n, manejar perif�ricos, etc. Estos programas reciben el nombre de sistemas operativos. Los computadores modernos cargan al arrancar un sistema operativo que controla la ejecuci�n del resto de las aplicaciones. Ejemplos de sistemas operativos son DOS (Disk Operating System), Unix y su versi�n para ordenadores personales Linux.

\item \textbf{Lenguajes de alto nivel.} \index{Programaci�n! lenguajes}
Los lenguajes de alto nivel est�n pensados para facilitar la tarea del programador, desentendi�ndose de los detalles de implementaci�n del hardware del ordenador.  Est�n compuestos por un conjunto de comandos y unas reglas sint�cticas, que permiten describir las instrucciones para el computador en forma de texto.

De una manera muy general, se pueden dividir los lenguajes de alto nivel en lenguajes compilados y lenguajes interpretados. Los lenguajes compilados emplean un compilador para convertir los comandos del lenguaje de alto nivel en lenguaje m�quina. Ejemplos de lenguajes compilados son C , C++ y Fortran. Los lenguajes interpretados a diferencia de los anteriores no se traducen a lenguaje m�quina antes de ejecutarse. Si no que utilizan otro programa --el interprete-- que va leyendo los comandos del lenguaje y convirti�ndolos en instrucciones m�quina a la vez que el programa se va ejecutando. Ejemplos de programas interpretado son Basic, Python y Java.

\item \textbf{Aplicaciones.} \index{Programaci�n! aplicaciones} Se suele entender por aplicaciones programas orientados a tareas espec�ficas, disponibles para un usuario final. Habitualmente se trata de programas escritos en un lenguaje de alto nivel y presentados en un formato f�cilmente comprensible para quien los usa.

Existen multitud de aplicaciones, entre las m�s conocidas cabe incluir los navegadores para Internet, como Explorer, Mocilla o Google Crome, los editores de texto, como Word, las hojas de c�lculo como Excel o los clientes de correo como Outlook. En realidad, la lista de aplicaciones disponibles en el mercado ser�a interminable. 
\end{enumerate}
\subsection{El modelo de computador de Von Neumann} \index{Von Neumann}
Los computadores modernos siguen, en lineas generales, el modelo propuesto por Von Newmann.  La figura \ref{fig:vonn} muestra un esquema de dicho modelo. 

\begin{figure}[h]
	\centering
		\includegraphics[width=10cm]{von.pdf}
	\caption{Modelo de Von Neumann}
	\label{fig:vonn}
\end{figure}

En el modelo de Von Newman se pueden  distinguir tres m�dulos b�sicos y una serie de elementos de interconexi�n.  Los m�dulos b�sicos son: 

\begin{itemize}
\item \textbf{La Unidad Central de Procesos.} CPU \index{CPU} (\emph{Central process unit)}) , esta unidad constituye el n�cleo en el que el ordenador realiza las operaciones. 

Dentro de la CPU pueden a su vez distinguirse las siguientes partes

\begin{itemize}

\item La unidad de proceso � ruta de datos: Est� formada por La Unidad Aritm�tico L�gica (ALU), \index{ALU} capaz de realizar las operaciones aritm�ticas y l�gicas que indican las instrucciones del programa. En general las ALUs se construyen para realizar aritm�tica entre enteros, y realizar las operaciones l�gicas b�sicas del algebra de Boole (AND, OR, etc). Habitualmente, las operaciones para n�meros no enteros, representados en \emph{punto flotante} se suelen realizar empleando un procesador espec�fico que se conoce con el nombre de Coprocesador matem�tico. La velocidad de procesamiento suele medirse en millones de operaciones por segundo (MIPS) o millones de operaciones en punto flotante por segundo (MFLOPS).

\item El banco de registros: Conjunto de registros en los que se almacenan los datos con los que trabaja la ALU y los resultados obtenidos.
 
\item La unidad de control (UC) o ruta de control: se encarga de buscar las instrucciones en la memoria principal y guardarlas en el registro de instrucciones, las decodifica, las ejecuta empleando la ALU, guarda los resultados en el registro de datos, y guarda las condiciones derivadas de la operaci�n realizada en el registro de estado.  El registro de datos de memoria, contiene los datos que se est�n leyendo de la memoria principal o van a escribirse en la misma. El registro de direcciones de memoria, guarda la direcci�n de la memoria principal a las que esta accediendo la ALU, para leer o escribir. El contador del programa, tambi�n conocido como puntero de instrucciones, es un registro que guarda la posici�n en la que se encuentra la CPU dentro de la secuencia de instrucciones de un programa.
\end{itemize}
 

\item \textbf{La unidad de memoria.} Se trata de la memoria principal o primaria del computador.  Est� dividida en bloques de memoria que se identifican mediante una direcci�n. La CPU tiene acceso directo a dichos bloques de memoria.

La unidad elemental de informaci�n digital es el bit \index{bit} (0,1). La capacidad de almacenamiento de datos se mide en Bytes \index{Byte} y en sus m�ltiplos, calculados  como potencias de 2\footnote{Los prefijos $K$(Kilo),$M$(Mega),$G$ (Giga), etc., se reservan en el sistema internacional para indicar potencias de 10. Para su equivalente en potencias de 2 se deben emplear los t�rminos $Ki$ (Kibi), $Mi$ (Mebi),$Gi$ (Gibi),$Ti$,(Tebi). Por tanto, deber�a decirse  Kibibyte, Mebibyte, etc. Sin embargo, esta notaci�n no est� muy extendida y se habla de KB (KiloBytes), MB (Megabytes), etc aunque se realice el c�lculo en potencias de 2 }

\begin{align} \nonumber
1\  Byte = &\ 8\ bits\ &\  \\ \nonumber
1\  Word = &\ 16\ bits=2B &\  \\ \nonumber
1\  KiB  = &\ 2^{10}\ bits=1024\ B&\ \\  \nonumber
1\  MiB = &\ 2^{20}\ bits=1024\ KB&\ \\  \nonumber
1\  GiB = &\ 2^{30}\ bits &\ \\  \nonumber
1\  TiB  = &\ 2^{40}\ bits\ &\
\end{align} 

\item \textbf{Unidad de Entrada/Salida.} Transfiere informaci�n entre el computador y los dispositivos perif�ricos.
\end{itemize}

Los elementos de interconexi�n se conocen con el nombre de \emph{Buses}. Se pueden distinguir tres: En bus de datos, por el que se transfieren datos entre la CPU y la memoria � la unidad de entrada/salida. El bus de direcciones, par especificar una direcci�n de memoria o del registro de E/S. Y el bus de Control, por el que se env�an se�ales de control, tales como la se�al de reloj, la se�al de control de lectura/escrituras entre otras.  

\subsection{Representaci�n binaria} \index{Base 2}
Veamos con algo m�s de detalle, c�mo representa la informaci�n un computador. Como se explic� anteriormente, La electr�nica que constituye la parte f�sica del ordenador, trabaja con dos niveles de voltaje. Esto permite definir dos estados, --alto, bajo-- que pueden representarse dos s�mbolos  $0$ y $1$. Habitualmente, empleamos $10$ s�mbolos ${0,1,2,3,4,5,6,7,8,9}$, es decir, empleamos una representaci�n decimal. Cuando queremos representar n�meros mayores que nueve, dado que hemos agotado el n�mero de d�gitos disponibles, lo que hacemos es combinarlos, agrupando cantidades de diez en diez. As� por ejemplo, el numero $16$, representa seis unidades m�s un grupo de diez unidades y el n�mero $462$ representa dos unidades m�s seis grupos de diez unidades m�s cuatro grupos de 10 grupos de 10 unidades.  Matem�ticamente, esto es equivalentes a emplear sumas de d�gitos por potencias de diez:
\begin{equation*}
13024 = 1\times10^4+3\times10^3+0\times10^2+2\times10^1+4\times10^0 
\end{equation*}

Si recorremos los d�gitos que componen el n�mero de izquierda derecha, cada uno de ellos representa una potencia de diez superior, porque cada uno representa la cantidad de grupos de 10 grupos, de grupos ... de diez grupos de unidades. Esto hace que potencialmente podamos representar cantidades tan grandes como queramos, empleando tan solo diez s�mbolos. Esta representaci�n, a la que estamos habituados recibe el nombre de representaci�n en base 10 \index{Base 10}. Pero no es la �nica posible.

Volvamos a la representaci�n empleada por el computador. En este caso solo tenemos dos s�mbolos distintos el $0$ y el $1$. Si queremos emplear una representaci�n an�loga a la representaci�n en base diez, deberemos agrupar ahora las cantidad en grupos de dos. As� los �nicos n�meros que admiten ser representados con un solo d�gito son el uno y el cero. Para representar el n�mero dos, necesitamos agrupar: tendremos $0$ unidades y $1$ grupo de dos, con lo que la representaci�n del n�mero dos en base dos ser� $10$. Para representar el n�mero tres, tendremos una unidad m�s un grupo de dos, por lo que la representaci�n ser� $11$, y as� sucesivamente. Matem�ticamente esto es equivalente emplear sumas de d�gitos por potencias de 2:

\begin{equation*}
10110 = 1\times2^4+0\times2^3+1\times2^2+1\times2^1+0\times2^0 
\end{equation*}

Esta representaci�n recibe el nombre de representaci�n binaria o en base 2.
\index{Conversi�n! binario a decimal}La expansi�n de un n�mero representado en binario en potencias de 2, nos da un m�todo directo de obtener su representaci�n decimal. As�, para el ejemplo anterior, si calculamos las potencias de dos y sumamos los resultados obtenemos:

  \begin{equation*}
  1\times2^4+0\times2^3+1\times2^2+1\times2^1+0\times2^0=16+0+4+2+0=22 
\end{equation*}

que es la representaci�n en base 10 del n�mero binario $10110$.

Para n�meros no enteros, la representaci�n tanto en decimal como en binario, se extiende de modo natural empleando potencias negativas de 10 y de 2 respectivamente. As�,
 
\begin{equation} \nonumber
835.41 = 8\times10^2+3\times10^1+5\times10^0+4\times10^{-1}+1\times10^{-2} 
\end{equation}

y para un n�mero en binario,

\begin{equation} \nonumber
101.01 = 1\times2^2+0\times2^1+1\times2^0+0\times2^{-1}+1\times2^{-2} 
\end{equation}

De nuevo, basta calcular el t�rmino de la derecha de la expresi�n anterior para obtener la representaci�n decimal del n�mero $101.01$.

\index{Conversi�n! decimal a binario. n�meros enteros}�C�mo transformar la representaci�n de un n�mero de decimal a binario? De nuevo nos da la clave la representaci�n en sumas de productos de d�gitos por potencias de dos. Empecemos por el caso de un n�mero entero. Supongamos un n�mero D, representado en decimal. Queremos expandirlo en una suma de potencias de dos. Si dividimos el n�mero por 2, podr�amos representarlo c�mo:
 
\begin{equation*}
\label{eq:1}
D=2\cdot C_1+R_1
\end{equation*}

donde $C_1$ representa el cociente de la divisi�n y $R_1$ el resto. Como estamos dividiendo por dos, el resto solo puede valer cero o uno. Supongamos ahora que volvemos a dividir el cociente obtenido por dos,

 \begin{equation*}
 \label{eq:2}
C_1=2\cdot C_2+R_2 \
\end{equation*}

Si sustituimos el valor obtenido para $C_1$ en la ecuaci�n inicial obtenemos,   
\begin{equation*}
D=2\cdot(2\cdot C_2+R_2)+R_1= 2^2\cdot C_2+R_2\cdot 2^1+R_1\cdot 2^0 
\end{equation*}

Si volvemos a dividir el nuevo cociente obtenido $C_2$ por dos, y volvemos a sustituir,

 \begin{align*}
C_2&=2\cdot C_3+R_3 \\
D&=2^2\cdot(2\cdot C_3+R_3)+R_2\cdot 2^1+R_1\cdot 2^0=2^3\cdot C_3+R_3\cdot 2^2 +R_2\cdot 2^1+R_1\cdot 2^0
\end{align*}

Supongamos que tras repetir este proceso $n$ veces, obtenemos un conciente $C_n=1$. L�gicamente no tiene sentido seguir dividiendo ya que a partir de este punto, cualquier divisi�n posterior que hagamos nos dar� cociente $0$ y resto igual a $C_n$. Por tanto, 

 \begin{align*}
D&=1\cdot 2^n+R_n\cdot 2^{n-1}\cdots +R_3\cdot 2^2 +R_2\cdot 2^1+R_1\cdot 2^0
\end{align*}

La expresi�n obtenida, coincide precisamente con la expansi�n en potencias de dos del n�mero binario $1R_n \cdots R_3R_2R_1$.


Como ejemplo,podemos obtener la representaci�n en binario del n�mero $234$, empleando el m�todo descrito: vamos dividiendo el n�mero y los cocientes sucesivos entre dos, hasta obtener un cociente igual a uno y a continuaci�n, construimos la representaci�n binaria del n�mero colocando por orden, de derecha a izquierda.  los restos  obtenidos de las sucesivas divisiones y a�adiendo un uno m�s a la izquierda de la cifra construida con los restos:

\begin{table}[h]
\begin{tabular}{|r|r|r|r|}
Dividendo& &Cociente $\div 2$&Resto\\
\hline
234& &117&0\\
117& &58&1\\
58& &29&0\\
29& &14&1\\
14& &7&0\\
7& &3&1\\
3& &1&1
\end{tabular}
\end{table}
 
 Por tanto, la representaci�n en binario de 234 es 11101010.
 
\index{Conversi�n! decimal a binario, n�meros no entero}Supongamos ahora un n�mero no entero, representado en decimal, de la forma $0,d$ . Si lo multiplicamos por dos:

\begin{equation}
E_1,d_1=0,d\cdot 2
\end{equation}
Donde $E_1$ representa la parte entera y $d_1$ la parte decimal del n�mero calculado.
Podemos entonces representar $0,d$ como,
\begin{equation}
\label{eq:5}
0,d=(E_1,d_1)\cdot 2^{-1}=E_1\cdot 2^{-1}+0,d_1\cdot 2^{-1}
\end{equation}  

Si volvemos a multiplicar $0,d_1$ por dos,

\begin{equation}
E_2,d_2 = 0,d_1\cdot 2
\end{equation}

\begin{equation}
0,d_1=E_2\cdot 2^{-1}+0,d_2\cdot 2^{-1}
\end{equation}  

y sustituyendo en \ref{eq:5}

\begin{equation}
0,d=E_1\cdot 2^{-1}+E_2\cdot 2^{-2}+0,d_2\cdot 2^{-2}
\end{equation}

�Hasta cuando repetir el proceso? En principio hasta que obtengamos un valor cero para la parte decimal, $0,d_n=0$. Pero esta condici�n puede no cumplirse nunca. Puede darse el caso --de hecho es lo m�s probable-- de que un n�mero que tiene una representaci�n exacta en decimal, no la tenga en binario. El criterio para detener el proceso ser� entonces obtener un determinado n�mero de decimales o bien seguir el proceso hasta que la parte decimal obtenida vuelva a repetirse. Puesto que los ordenadores tienen un tama�o de registro limitado, tambi�n est� limitado el n�mero de d�gitos con el que pueden representar un n�mero decimal. Por eso, lo habitual ser� truncar el n�mero asumiendo el error que se comete al proceder as�.  De este modo, obtenemos la expansi�n del n�mero original en potencias de dos,

\begin{equation}
0,d\cdot 2=E_1\cdot 2^{-1}+E_2\cdot 2^{-2}+\cdots+ E_n\cdot 2^{-3}+\cdots
\end{equation} 

Donde los valores $E_1\cdots E_n$ son precisamente los d�gitos correspondientes a la representaci�n del n�mero en binario: $0.E_1E_2\cdots E_n$. (Es trivial comprobar que solo pueden valer $0$ � $1$).


Veamos un ejemplo de cada caso, obteniendo la representaci�n binaria del n�mero $0,625$, que tiene representaci�n exacta, y la del n�mero $0,626$, que no la tiene. En este segundo caso, calcularemos una representaci�n aproximada, tomando 8 decimales.

\begin{table}[h]
\begin{tabular}{|r|r|r|r|r r|r|r|r|r|}
P decimal& &$\times 2$& P entera& &&P decimal& &$\times 2$& P entera\\
\cline{1-4}
\cline{7-10}
0,625& &1,25&1& &&0,623& &1,246&1\\
0,25  & &0,5  &0& &&0,246& &0,492&0\\
0,5    & &1,0  &1& &&0,492& &0,984&0\\
         & &       &  & &&0,984& &1,968&1\\
         & &       &  & &&0,968& &1,936&1\\
         & &       &  & &&0,936& &1,872&1\\
         & &       &  & &&0.872& &1.744&1\\
         & &       &  & &&0.744& &1.488&1\\
\end{tabular}
\end{table}

Para construir la representaci�n binaria del primero de los n�meros, nos basta tomar las partes enteras obtenidas, por orden, escribirlas de de izquierda a derecha y a�adir un $0$ y la coma decimal a la izquierda. Por tanto  la representaci�n binaria de $0,625$ es $0,101$.  Si expandimos su valor en potencias de dos, volvemos a recuperar el n�mero original en su representaci�n decimal.

 En el segundo caso, la representaci�n binaria, tomando nueve decimales de $0,623$ es $0.10011111$. Podemos calcular el error que cometemos al despreciar el resto de los decimales, volviendo a convertir el resultado obtenido a su representaci�n en base diez,

 \begin{equation*}
0\cdot 2^{0}+1\cdot 2^{-1}+0\cdot 2^{-2}+ 0\cdot 2^{-3}+1\cdot 2^{-4}+1\cdot 2^{-5}+ 1\cdot 2^{-6}+1\cdot 2^{-7}+1\cdot 2^{-8}=0,62109375
\end{equation*} 

El error cometido es, en este caso: $\text{Error}=0,623-0,62109375=0,00190625$.
  
 \section{Aplicaciones de Software Cient�fico}
Dentro del mundo de las aplicaciones, merecen una menci�n aparte las dedicadas al c�lculo cient�fico, por su conexi�n con la asignatura. 

Es posible emplear lenguajes de alto nivel para construir rutinas y programas que permitan resolver directamente un determinado problema de c�lculo. En este sentido, el lenguaje FORTRAN se ha empleado durante a�os para ese fin, y todav�a sigue emple�ndose en muchas disciplinas cient�ficas y de la Ingenier�a.  Sin embargo, hay muchos aspectos no triviales del c�lculo con un computador, que obligar�an al cient�fico que tuviera que programar sus propios programas a ser a la vez un experto en computadores.  Por esta raz�n, se han ido desarrollando aplicaciones espec�ficas para c�lculo cient�fico que permiten al investigador centrarse en la resoluci�n de su problema y no en el desarrollo de la herramienta adecuada para resolverlo.  
 
En algunos casos, se trata de aplicaciones a medida, relacionadas directamente con alg�n �rea cient�fica concreta. En otros, consisten en paquetes de funciones espec�ficos para realizar de forma eficiente determinados c�lculos, como por ejemplo
el paquete SPSS para c�lculo estad�stico.

Un grupo especialmente interesante lo forman algunos paquetes de software que podr�amos situar a mitad de camino entre los lenguajes de alto nivel y las aplicaciones: Contienen extensas librer�as de funciones, que pueden ser empleadas de una forma directa para realizar c�lculos y adem�s permiten realizar programas espec�ficos empleando su propio lenguaje. Entre estos podemos destacar Mathematica, Maple , Matlab, Octave y Scilab y Python. El uso de estas herramientas se ha extendido enormemente en la comunidad cient�fica. Algunas como Matlab  constituyen casi un est�ndar en determinadas �reas de conocimiento.

 


    

 

  

\include{introprog}
\include{aritmetica}
\include{calraiz}
\include{algebra}
\include{sistemas}
\include{interpolacion}
\include{integracion}
\include{tratamientoest}
\include{simbolico}
\printindex
\end{document} 